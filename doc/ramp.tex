% generated by GAPDoc2LaTeX from XML source (Frank Luebeck)
\documentclass[a4paper,11pt]{report}

\usepackage[top=37mm,bottom=37mm,left=27mm,right=27mm]{geometry}
\sloppy
\pagestyle{myheadings}
\usepackage{amssymb}
\usepackage[utf8]{inputenc}
\usepackage{makeidx}
\makeindex
\usepackage{color}
\definecolor{FireBrick}{rgb}{0.5812,0.0074,0.0083}
\definecolor{RoyalBlue}{rgb}{0.0236,0.0894,0.6179}
\definecolor{RoyalGreen}{rgb}{0.0236,0.6179,0.0894}
\definecolor{RoyalRed}{rgb}{0.6179,0.0236,0.0894}
\definecolor{LightBlue}{rgb}{0.8544,0.9511,1.0000}
\definecolor{Black}{rgb}{0.0,0.0,0.0}

\definecolor{linkColor}{rgb}{0.0,0.0,0.554}
\definecolor{citeColor}{rgb}{0.0,0.0,0.554}
\definecolor{fileColor}{rgb}{0.0,0.0,0.554}
\definecolor{urlColor}{rgb}{0.0,0.0,0.554}
\definecolor{promptColor}{rgb}{0.0,0.0,0.589}
\definecolor{brkpromptColor}{rgb}{0.589,0.0,0.0}
\definecolor{gapinputColor}{rgb}{0.589,0.0,0.0}
\definecolor{gapoutputColor}{rgb}{0.0,0.0,0.0}

%%  for a long time these were red and blue by default,
%%  now black, but keep variables to overwrite
\definecolor{FuncColor}{rgb}{0.0,0.0,0.0}
%% strange name because of pdflatex bug:
\definecolor{Chapter }{rgb}{0.0,0.0,0.0}
\definecolor{DarkOlive}{rgb}{0.1047,0.2412,0.0064}


\usepackage{fancyvrb}

\usepackage{mathptmx,helvet}
\usepackage[T1]{fontenc}
\usepackage{textcomp}


\usepackage[
            pdftex=true,
            bookmarks=true,        
            a4paper=true,
            pdftitle={Written with GAPDoc},
            pdfcreator={LaTeX with hyperref package / GAPDoc},
            colorlinks=true,
            backref=page,
            breaklinks=true,
            linkcolor=linkColor,
            citecolor=citeColor,
            filecolor=fileColor,
            urlcolor=urlColor,
            pdfpagemode={UseNone}, 
           ]{hyperref}

\newcommand{\maintitlesize}{\fontsize{50}{55}\selectfont}

% write page numbers to a .pnr log file for online help
\newwrite\pagenrlog
\immediate\openout\pagenrlog =\jobname.pnr
\immediate\write\pagenrlog{PAGENRS := [}
\newcommand{\logpage}[1]{\protect\write\pagenrlog{#1, \thepage,}}
%% were never documented, give conflicts with some additional packages

\newcommand{\GAP}{\textsf{GAP}}

%% nicer description environments, allows long labels
\usepackage{enumitem}
\setdescription{style=nextline}

%% depth of toc
\setcounter{tocdepth}{1}





%% command for ColorPrompt style examples
\newcommand{\gapprompt}[1]{\color{promptColor}{\bfseries #1}}
\newcommand{\gapbrkprompt}[1]{\color{brkpromptColor}{\bfseries #1}}
\newcommand{\gapinput}[1]{\color{gapinputColor}{#1}}


\begin{document}

\logpage{[ 0, 0, 0 ]}
\begin{titlepage}
\mbox{}\vfill

\begin{center}{\maintitlesize \textbf{ RAMP \mbox{}}}\\
\vfill

\hypersetup{pdftitle= RAMP }
\markright{\scriptsize \mbox{}\hfill  RAMP  \hfill\mbox{}}
{\Huge \textbf{ The Research Assistant for Maniplexes and Polytopes \mbox{}}}\\
\vfill

{\Huge  0.3 \mbox{}}\\[1cm]
{ 28 July 2020 \mbox{}}\\[1cm]
\mbox{}\\[2cm]
{\Large \textbf{ Gabe Cunningham\\
    \mbox{}}}\\
\hypersetup{pdfauthor= Gabe Cunningham\\
    }
\end{center}\vfill

\mbox{}\\
{\mbox{}\\
\small \noindent \textbf{ Gabe Cunningham\\
    }  Email: \href{mailto://gabriel.cunningham@umb.edu} {\texttt{gabriel.cunningham@umb.edu}}\\
  Homepage: \href{http://www.gabrielcunningham.com} {\texttt{http://www.gabrielcunningham.com}}\\
  Address: \begin{minipage}[t]{8cm}\noindent
 Gabe Cunningham\\
 Department of Mathematics\\
 University of Massachusetts Boston\\
 100 William T. Morrissey Blvd.\\
 Boston MA 02125\\
 \end{minipage}
}\\
\end{titlepage}

\newpage\setcounter{page}{2}
{\small 
\section*{Copyright}
\logpage{[ 0, 0, 1 ]}
 \index{License} {\copyright} 1997-2020 by Gabe Cunningham

 \textsf{RAMP} package is free software; you can redistribute it and/or modify it under the
terms of the \href{http://www.fsf.org/licenses/gpl.html} {GNU General Public License} as published by the Free Software Foundation; either version 2 of the License,
or (at your option) any later version. \mbox{}}\\[1cm]
{\small 
\section*{Acknowledgements}
\logpage{[ 0, 0, 2 ]}
 We appreciate very much all past and future comments, suggestions and
contributions to this package and its documentation provided by \textsf{GAP} users and developers. \mbox{}}\\[1cm]
\newpage

\def\contentsname{Contents\logpage{[ 0, 0, 3 ]}}

\tableofcontents
\newpage

     
\chapter{\textcolor{Chapter }{Constructions}}\label{Chapter_Constructions}
\logpage{[ 1, 0, 0 ]}
\hyperdef{L}{X836F56D77AA04554}{}
{
  
\section{\textcolor{Chapter }{Extensions, amalgamations, and quotients}}\label{Chapter_Constructions_Section_Extensions_amalgamations_and_quotients}
\logpage{[ 1, 1, 0 ]}
\hyperdef{L}{X871C2D73829F1FC1}{}
{
  

\subsection{\textcolor{Chapter }{UniversalPolytope (for IsInt)}}
\logpage{[ 1, 1, 1 ]}\nobreak
\hyperdef{L}{X808AEA5281D52911}{}
{\noindent\textcolor{FuncColor}{$\triangleright$\enspace\texttt{UniversalPolytope({\mdseries\slshape n})\index{UniversalPolytope@\texttt{UniversalPolytope}!for IsInt}
\label{UniversalPolytope:for IsInt}
}\hfill{\scriptsize (operation)}}\\


 Returns the universal polytope of rank \mbox{\texttt{\mdseries\slshape n}}. }

 

\subsection{\textcolor{Chapter }{FlatRegularPolyhedron (for IsInt, IsInt, IsInt, IsInt)}}
\logpage{[ 1, 1, 2 ]}\nobreak
\hyperdef{L}{X800E07C686CF3EA7}{}
{\noindent\textcolor{FuncColor}{$\triangleright$\enspace\texttt{FlatRegularPolyhedron({\mdseries\slshape p, q, i, j})\index{FlatRegularPolyhedron@\texttt{FlatRegularPolyhedron}!for IsInt, IsInt, IsInt, IsInt}
\label{FlatRegularPolyhedron:for IsInt, IsInt, IsInt, IsInt}
}\hfill{\scriptsize (operation)}}\\


 Returns the flat regular polyhedron with automorphism group [p, q] / (r2 r1 r0
r1 = (r0 r1)\texttt{\symbol{94}}i (r1 r2)\texttt{\symbol{94}}j). This function
does not currently validate the inputs to make sure that the output makes
sense. }

 

\subsection{\textcolor{Chapter }{QuotientPolytope (for IsManiplex, IsList)}}
\logpage{[ 1, 1, 3 ]}\nobreak
\hyperdef{L}{X81DABA8680E94B18}{}
{\noindent\textcolor{FuncColor}{$\triangleright$\enspace\texttt{QuotientPolytope({\mdseries\slshape M, rels})\index{QuotientPolytope@\texttt{QuotientPolytope}!for IsManiplex, IsList}
\label{QuotientPolytope:for IsManiplex, IsList}
}\hfill{\scriptsize (operation)}}\\


 Returns the quotient of \mbox{\texttt{\mdseries\slshape M}} by \mbox{\texttt{\mdseries\slshape rels}}, which may be given as either a list of Tietze words, such as
[[1,2,1,0,1,2,1,0]] or as a string like "(r0 r1 r2 r1)\texttt{\symbol{94}}2,
(r0 r1 r2)\texttt{\symbol{94}}4". }

 

\subsection{\textcolor{Chapter }{UniversalExtension (for IsManiplex)}}
\logpage{[ 1, 1, 4 ]}\nobreak
\hyperdef{L}{X8582D0CF7EA1CB34}{}
{\noindent\textcolor{FuncColor}{$\triangleright$\enspace\texttt{UniversalExtension({\mdseries\slshape M})\index{UniversalExtension@\texttt{UniversalExtension}!for IsManiplex}
\label{UniversalExtension:for IsManiplex}
}\hfill{\scriptsize (operation)}}\\


 Returns the universal extension of \mbox{\texttt{\mdseries\slshape M}}, i.e. the maniplex with facets isomorphic to \mbox{\texttt{\mdseries\slshape M}} that covers all other maniplexes with facets isomorphic to \mbox{\texttt{\mdseries\slshape M}}. Currently only defined for reflexible maniplexes. }

 

\subsection{\textcolor{Chapter }{UniversalExtension (for IsManiplex, IsInt)}}
\logpage{[ 1, 1, 5 ]}\nobreak
\hyperdef{L}{X78B4610E7DE67E32}{}
{\noindent\textcolor{FuncColor}{$\triangleright$\enspace\texttt{UniversalExtension({\mdseries\slshape M, k})\index{UniversalExtension@\texttt{UniversalExtension}!for IsManiplex, IsInt}
\label{UniversalExtension:for IsManiplex, IsInt}
}\hfill{\scriptsize (operation)}}\\


 Returns the universal extension of \mbox{\texttt{\mdseries\slshape M}} with last entry of Schlafli symbol \mbox{\texttt{\mdseries\slshape k}}. Currently only defined for reflexible maniplexes. }

 

\subsection{\textcolor{Chapter }{TrivialExtension (for IsManiplex)}}
\logpage{[ 1, 1, 6 ]}\nobreak
\hyperdef{L}{X84BB6B21853173FD}{}
{\noindent\textcolor{FuncColor}{$\triangleright$\enspace\texttt{TrivialExtension({\mdseries\slshape M})\index{TrivialExtension@\texttt{TrivialExtension}!for IsManiplex}
\label{TrivialExtension:for IsManiplex}
}\hfill{\scriptsize (operation)}}\\


 Returns the trivial extension of \mbox{\texttt{\mdseries\slshape M}}, also known as \texttt{\symbol{123}}\mbox{\texttt{\mdseries\slshape M/}}, 2\texttt{\symbol{125}}. }

 

\subsection{\textcolor{Chapter }{FlatExtension (for IsManiplex, IsInt)}}
\logpage{[ 1, 1, 7 ]}\nobreak
\hyperdef{L}{X7A5853E278A83D74}{}
{\noindent\textcolor{FuncColor}{$\triangleright$\enspace\texttt{FlatExtension({\mdseries\slshape M, k})\index{FlatExtension@\texttt{FlatExtension}!for IsManiplex, IsInt}
\label{FlatExtension:for IsManiplex, IsInt}
}\hfill{\scriptsize (operation)}}\\


 Returns the flat extension of \mbox{\texttt{\mdseries\slshape M}} with last entry of Schlafli symbol \mbox{\texttt{\mdseries\slshape k}}. (As defined in "Flat Extensions of Abstract Polytopes".) Currently only
defined for reflexible maniplexes. }

 

\subsection{\textcolor{Chapter }{Amalgamate (for IsManiplex, IsManiplex)}}
\logpage{[ 1, 1, 8 ]}\nobreak
\hyperdef{L}{X790F22D47CD222EB}{}
{\noindent\textcolor{FuncColor}{$\triangleright$\enspace\texttt{Amalgamate({\mdseries\slshape M1, M2})\index{Amalgamate@\texttt{Amalgamate}!for IsManiplex, IsManiplex}
\label{Amalgamate:for IsManiplex, IsManiplex}
}\hfill{\scriptsize (operation)}}\\


 Returns the amalgamation of \mbox{\texttt{\mdseries\slshape M1}} and \mbox{\texttt{\mdseries\slshape M2}}. Implicitly assumes that \mbox{\texttt{\mdseries\slshape M1}} and \mbox{\texttt{\mdseries\slshape M2}} are compatible. Currently only defined for reflexible maniplexes. }

 

\subsection{\textcolor{Chapter }{Medial (for IsManiplex)}}
\logpage{[ 1, 1, 9 ]}\nobreak
\hyperdef{L}{X840BC19484E0E9CC}{}
{\noindent\textcolor{FuncColor}{$\triangleright$\enspace\texttt{Medial({\mdseries\slshape M})\index{Medial@\texttt{Medial}!for IsManiplex}
\label{Medial:for IsManiplex}
}\hfill{\scriptsize (operation)}}\\


 Given a 3-maniplex \mbox{\texttt{\mdseries\slshape M}}, returns its medial. }

 }

 
\section{\textcolor{Chapter }{Duality}}\label{Chapter_Constructions_Section_Duality}
\logpage{[ 1, 2, 0 ]}
\hyperdef{L}{X87FD993F7F6F2FAB}{}
{
  

\subsection{\textcolor{Chapter }{Dual (for IsManiplex)}}
\logpage{[ 1, 2, 1 ]}\nobreak
\hyperdef{L}{X7D62BD3E7F941F5B}{}
{\noindent\textcolor{FuncColor}{$\triangleright$\enspace\texttt{Dual({\mdseries\slshape M})\index{Dual@\texttt{Dual}!for IsManiplex}
\label{Dual:for IsManiplex}
}\hfill{\scriptsize (attribute)}}\\
\textbf{\indent Returns:\ }
The maniplex that is dual to \mbox{\texttt{\mdseries\slshape M}}. 



 

 }

 

\subsection{\textcolor{Chapter }{IsSelfDual (for IsManiplex)}}
\logpage{[ 1, 2, 2 ]}\nobreak
\hyperdef{L}{X7FF8B96C83DA60CB}{}
{\noindent\textcolor{FuncColor}{$\triangleright$\enspace\texttt{IsSelfDual({\mdseries\slshape P})\index{IsSelfDual@\texttt{IsSelfDual}!for IsManiplex}
\label{IsSelfDual:for IsManiplex}
}\hfill{\scriptsize (property)}}\\
\textbf{\indent Returns:\ }
Whether this polytope is isomorphic to its dual. 



 

 }

 

\subsection{\textcolor{Chapter }{Petrial (for IsManiplex)}}
\logpage{[ 1, 2, 3 ]}\nobreak
\hyperdef{L}{X7E6D0732862D6BA3}{}
{\noindent\textcolor{FuncColor}{$\triangleright$\enspace\texttt{Petrial({\mdseries\slshape P})\index{Petrial@\texttt{Petrial}!for IsManiplex}
\label{Petrial:for IsManiplex}
}\hfill{\scriptsize (attribute)}}\\
\textbf{\indent Returns:\ }
The Petrial (Petrie dual) of \mbox{\texttt{\mdseries\slshape P}}. Note that this is not necessarily a polytope. 



 

 }

 

\subsection{\textcolor{Chapter }{IsSelfPetrial (for IsManiplex)}}
\logpage{[ 1, 2, 4 ]}\nobreak
\hyperdef{L}{X78FDB0047E3388A5}{}
{\noindent\textcolor{FuncColor}{$\triangleright$\enspace\texttt{IsSelfPetrial({\mdseries\slshape P})\index{IsSelfPetrial@\texttt{IsSelfPetrial}!for IsManiplex}
\label{IsSelfPetrial:for IsManiplex}
}\hfill{\scriptsize (property)}}\\
\textbf{\indent Returns:\ }
Whether this polytope is isomorphic to its Petrial. 



 

 }

 }

 
\section{\textcolor{Chapter }{Products}}\label{Chapter_Constructions_Section_Products}
\logpage{[ 1, 3, 0 ]}
\hyperdef{L}{X86CE352C7851221F}{}
{
  

\subsection{\textcolor{Chapter }{PyramidOver (for IsManiplex)}}
\logpage{[ 1, 3, 1 ]}\nobreak
\hyperdef{L}{X80A4156A8423CAF8}{}
{\noindent\textcolor{FuncColor}{$\triangleright$\enspace\texttt{PyramidOver({\mdseries\slshape M})\index{PyramidOver@\texttt{PyramidOver}!for IsManiplex}
\label{PyramidOver:for IsManiplex}
}\hfill{\scriptsize (operation)}}\\


 Returns the pyramid over \mbox{\texttt{\mdseries\slshape M}}. Currently only works for finite maniplexes. }

 

\subsection{\textcolor{Chapter }{PrismOver (for IsManiplex)}}
\logpage{[ 1, 3, 2 ]}\nobreak
\hyperdef{L}{X7C7D6F558335CFD0}{}
{\noindent\textcolor{FuncColor}{$\triangleright$\enspace\texttt{PrismOver({\mdseries\slshape M})\index{PrismOver@\texttt{PrismOver}!for IsManiplex}
\label{PrismOver:for IsManiplex}
}\hfill{\scriptsize (operation)}}\\


 Returns the prism over \mbox{\texttt{\mdseries\slshape M}}. Currently only works for finite maniplexes. }

 }

 }

   
\chapter{\textcolor{Chapter }{Databases}}\label{Chapter_Databases}
\logpage{[ 2, 0, 0 ]}
\hyperdef{L}{X7EB183C3780A475B}{}
{
  
\section{\textcolor{Chapter }{Regular polyhedra}}\label{Chapter_Databases_Section_Regular_polyhedra}
\logpage{[ 2, 1, 0 ]}
\hyperdef{L}{X8062E376879531A7}{}
{
  

\subsection{\textcolor{Chapter }{DegeneratePolyhedra (for IsInt)}}
\logpage{[ 2, 1, 1 ]}\nobreak
\hyperdef{L}{X856D5CE8798775FE}{}
{\noindent\textcolor{FuncColor}{$\triangleright$\enspace\texttt{DegeneratePolyhedra({\mdseries\slshape maxsize})\index{DegeneratePolyhedra@\texttt{DegeneratePolyhedra}!for IsInt}
\label{DegeneratePolyhedra:for IsInt}
}\hfill{\scriptsize (operation)}}\\


 Returns all degenerate polyhedra (of type \texttt{\symbol{123}}2,
q\texttt{\symbol{125}} and \texttt{\symbol{123}}p, 2\texttt{\symbol{125}})
with up to \mbox{\texttt{\mdseries\slshape maxsize}} flags. }

 

\subsection{\textcolor{Chapter }{FlatRegularPolyhedra (for IsInt)}}
\logpage{[ 2, 1, 2 ]}\nobreak
\hyperdef{L}{X82DC20587F569351}{}
{\noindent\textcolor{FuncColor}{$\triangleright$\enspace\texttt{FlatRegularPolyhedra({\mdseries\slshape maxsize})\index{FlatRegularPolyhedra@\texttt{FlatRegularPolyhedra}!for IsInt}
\label{FlatRegularPolyhedra:for IsInt}
}\hfill{\scriptsize (operation)}}\\


 Returns all nondegenerate flat regular polyhedra with up to \mbox{\texttt{\mdseries\slshape maxsize}} flags. Currently supports a maxsize of 4000 or less. }

 

\subsection{\textcolor{Chapter }{SmallRegularPolyhedra (for IsInt)}}
\logpage{[ 2, 1, 3 ]}\nobreak
\hyperdef{L}{X7BF512FB7EA9F274}{}
{\noindent\textcolor{FuncColor}{$\triangleright$\enspace\texttt{SmallRegularPolyhedra({\mdseries\slshape maxsize})\index{SmallRegularPolyhedra@\texttt{SmallRegularPolyhedra}!for IsInt}
\label{SmallRegularPolyhedra:for IsInt}
}\hfill{\scriptsize (operation)}}\\


 Returns all regular polyhedra with up to \mbox{\texttt{\mdseries\slshape maxsize}} flags. Currently supports a maxsize of 4000 or less. You can also set options
"nondegenerate" and "nonflat". 
\begin{Verbatim}[commandchars=!@|,fontsize=\small,frame=single,label=Example]
  L1 := SmallRegularPolyhedra(500);;
  L2 := SmallRegularPolyhedra(1000 : nondegenerate);;
  L3 := SmallRegularPolyhedra(2000 : nondegenerate, nonflat);;
\end{Verbatim}
 }

 }

 }

   
\chapter{\textcolor{Chapter }{Combinatorics and Structure}}\label{Chapter_Combinatorics_and_Structure}
\logpage{[ 3, 0, 0 ]}
\hyperdef{L}{X8153E7658330D710}{}
{
  
\section{\textcolor{Chapter }{Faces}}\label{Chapter_Combinatorics_and_Structure_Section_Faces}
\logpage{[ 3, 1, 0 ]}
\hyperdef{L}{X872AD1E785C7EB03}{}
{
  

\subsection{\textcolor{Chapter }{NumberOfIFaces (for IsManiplex, IsInt)}}
\logpage{[ 3, 1, 1 ]}\nobreak
\hyperdef{L}{X86085A967E396035}{}
{\noindent\textcolor{FuncColor}{$\triangleright$\enspace\texttt{NumberOfIFaces({\mdseries\slshape M, i})\index{NumberOfIFaces@\texttt{NumberOfIFaces}!for IsManiplex, IsInt}
\label{NumberOfIFaces:for IsManiplex, IsInt}
}\hfill{\scriptsize (operation)}}\\


 Returns The number of \mbox{\texttt{\mdseries\slshape i}}-faces of \mbox{\texttt{\mdseries\slshape M}}. }

 

\subsection{\textcolor{Chapter }{NumberOfVertices (for IsManiplex)}}
\logpage{[ 3, 1, 2 ]}\nobreak
\hyperdef{L}{X7C126944873D2461}{}
{\noindent\textcolor{FuncColor}{$\triangleright$\enspace\texttt{NumberOfVertices({\mdseries\slshape M})\index{NumberOfVertices@\texttt{NumberOfVertices}!for IsManiplex}
\label{NumberOfVertices:for IsManiplex}
}\hfill{\scriptsize (attribute)}}\\


 Returns the number of vertices of \mbox{\texttt{\mdseries\slshape M}}. }

 

\subsection{\textcolor{Chapter }{NumberOfEdges (for IsManiplex)}}
\logpage{[ 3, 1, 3 ]}\nobreak
\hyperdef{L}{X86CDC9B38792B038}{}
{\noindent\textcolor{FuncColor}{$\triangleright$\enspace\texttt{NumberOfEdges({\mdseries\slshape M})\index{NumberOfEdges@\texttt{NumberOfEdges}!for IsManiplex}
\label{NumberOfEdges:for IsManiplex}
}\hfill{\scriptsize (attribute)}}\\


 Returns the number of edges of \mbox{\texttt{\mdseries\slshape M}}. }

 

\subsection{\textcolor{Chapter }{NumberOfFacets (for IsManiplex)}}
\logpage{[ 3, 1, 4 ]}\nobreak
\hyperdef{L}{X7DE9F1057B309554}{}
{\noindent\textcolor{FuncColor}{$\triangleright$\enspace\texttt{NumberOfFacets({\mdseries\slshape M})\index{NumberOfFacets@\texttt{NumberOfFacets}!for IsManiplex}
\label{NumberOfFacets:for IsManiplex}
}\hfill{\scriptsize (attribute)}}\\


 Returns the number of facets of \mbox{\texttt{\mdseries\slshape M}}. }

 

\subsection{\textcolor{Chapter }{NumberOfRidges (for IsManiplex)}}
\logpage{[ 3, 1, 5 ]}\nobreak
\hyperdef{L}{X7C5446247FD63F28}{}
{\noindent\textcolor{FuncColor}{$\triangleright$\enspace\texttt{NumberOfRidges({\mdseries\slshape M})\index{NumberOfRidges@\texttt{NumberOfRidges}!for IsManiplex}
\label{NumberOfRidges:for IsManiplex}
}\hfill{\scriptsize (attribute)}}\\


 Returns the number of ridges ((n-2)-faces) of \mbox{\texttt{\mdseries\slshape M}}. }

 

\subsection{\textcolor{Chapter }{Fvector (for IsManiplex)}}
\logpage{[ 3, 1, 6 ]}\nobreak
\hyperdef{L}{X816A47B879737629}{}
{\noindent\textcolor{FuncColor}{$\triangleright$\enspace\texttt{Fvector({\mdseries\slshape M})\index{Fvector@\texttt{Fvector}!for IsManiplex}
\label{Fvector:for IsManiplex}
}\hfill{\scriptsize (attribute)}}\\


 Returns the f-vector of \mbox{\texttt{\mdseries\slshape M}}. }

 

\subsection{\textcolor{Chapter }{Facets (for IsManiplex)}}
\logpage{[ 3, 1, 7 ]}\nobreak
\hyperdef{L}{X78D48A797A8A981C}{}
{\noindent\textcolor{FuncColor}{$\triangleright$\enspace\texttt{Facets({\mdseries\slshape M})\index{Facets@\texttt{Facets}!for IsManiplex}
\label{Facets:for IsManiplex}
}\hfill{\scriptsize (attribute)}}\\


 Returns the facet-types of \mbox{\texttt{\mdseries\slshape M}} (i.e. the maniplexes corresponding to the facets). Currently only works for
reflexible maniplexes. }

 

\subsection{\textcolor{Chapter }{VertexFigures (for IsManiplex)}}
\logpage{[ 3, 1, 8 ]}\nobreak
\hyperdef{L}{X7FB4AE2779695BB4}{}
{\noindent\textcolor{FuncColor}{$\triangleright$\enspace\texttt{VertexFigures({\mdseries\slshape M})\index{VertexFigures@\texttt{VertexFigures}!for IsManiplex}
\label{VertexFigures:for IsManiplex}
}\hfill{\scriptsize (attribute)}}\\


 Returns the types of vertex-figures of \mbox{\texttt{\mdseries\slshape M}} (i.e. the maniplexes corresponding to the vertex-figures). Currently only
works for reflexible maniplexes. }

 }

 
\section{\textcolor{Chapter }{Posets}}\label{Chapter_Combinatorics_and_Structure_Section_Posets}
\logpage{[ 3, 2, 0 ]}
\hyperdef{L}{X79540DAB85902432}{}
{
  

\subsection{\textcolor{Chapter }{PosetFromFaceListOfFlags (for IsList)}}
\logpage{[ 3, 2, 1 ]}\nobreak
\hyperdef{L}{X7D495B407CDFAEA2}{}
{\noindent\textcolor{FuncColor}{$\triangleright$\enspace\texttt{PosetFromFaceListOfFlags({\mdseries\slshape list})\index{PosetFromFaceListOfFlags@\texttt{PosetFromFaceListOfFlags}!for IsList}
\label{PosetFromFaceListOfFlags:for IsList}
}\hfill{\scriptsize (operation)}}\\


 Given a list of lists of faces in increasing rank, where each face is
described by the incident flags, gives you a IsPosetOfFlags object back. }

 

\subsection{\textcolor{Chapter }{RankOfPoset (for IsPosetOfFlags)}}
\logpage{[ 3, 2, 2 ]}\nobreak
\hyperdef{L}{X79E4595984054EC6}{}
{\noindent\textcolor{FuncColor}{$\triangleright$\enspace\texttt{RankOfPoset({\mdseries\slshape poset})\index{RankOfPoset@\texttt{RankOfPoset}!for IsPosetOfFlags}
\label{RankOfPoset:for IsPosetOfFlags}
}\hfill{\scriptsize (operation)}}\\


 Given a \mbox{\texttt{\mdseries\slshape poset}}, returns the rank of the poset. Note: There may be hidden assumptions here to
untangle later. }

 

\subsection{\textcolor{Chapter }{PosetOfConnectionGroup (for IsGroup)}}
\logpage{[ 3, 2, 3 ]}\nobreak
\hyperdef{L}{X7DACCD4D8665EC1B}{}
{\noindent\textcolor{FuncColor}{$\triangleright$\enspace\texttt{PosetOfConnectionGroup({\mdseries\slshape g})\index{PosetOfConnectionGroup@\texttt{PosetOfConnectionGroup}!for IsGroup}
\label{PosetOfConnectionGroup:for IsGroup}
}\hfill{\scriptsize (operation)}}\\


 Given a group, returns a poset as a list of faces ordered by rank, where each
face is represented as a list of the flags it contains. Note that this
function does not include the minimal (empty) face nor the maximal face of the
maniplex. Note that the \mbox{\texttt{\mdseries\slshape i}}-faces correspond to the \mbox{\texttt{\mdseries\slshape i+1}} item in the list because of how GAP indexes lists. }

 

\subsection{\textcolor{Chapter }{FullPosetOfConnectionGroup (for IsGroup)}}
\logpage{[ 3, 2, 4 ]}\nobreak
\hyperdef{L}{X8514F5857FC8CC54}{}
{\noindent\textcolor{FuncColor}{$\triangleright$\enspace\texttt{FullPosetOfConnectionGroup({\mdseries\slshape g})\index{FullPosetOfConnectionGroup@\texttt{FullPosetOfConnectionGroup}!for IsGroup}
\label{FullPosetOfConnectionGroup:for IsGroup}
}\hfill{\scriptsize (operation)}}\\


 Returns a full poset as a list of faces ordered by rank, where each face is
represented as a list of the flags it contains. This function does include the
minimal (empty) face nor the maximal face of the maniplex, so the list has \mbox{\texttt{\mdseries\slshape n+2}} ranks if the maniplex is of rank \mbox{\texttt{\mdseries\slshape n}}. Note that the \mbox{\texttt{\mdseries\slshape i}}-faces correspond to the \mbox{\texttt{\mdseries\slshape i+1}} item in the list because of how GAP indexes lists. }

 

\subsection{\textcolor{Chapter }{PosetOfManiplex (for IsManiplex)}}
\logpage{[ 3, 2, 5 ]}\nobreak
\hyperdef{L}{X7C15F3F8792500D7}{}
{\noindent\textcolor{FuncColor}{$\triangleright$\enspace\texttt{PosetOfManiplex({\mdseries\slshape mani})\index{PosetOfManiplex@\texttt{PosetOfManiplex}!for IsManiplex}
\label{PosetOfManiplex:for IsManiplex}
}\hfill{\scriptsize (operation)}}\\


 Given a maniplex, returns a poset as a list of faces ordered by rank, where
each face is represented as a list of the flags it contains. Note that this
function does not include the minimal (empty) face nor the maximal face of the
maniplex. Note that the \mbox{\texttt{\mdseries\slshape i}}-faces correspond to the \mbox{\texttt{\mdseries\slshape i+1}} item in the list because of how GAP indexes lists. }

 

\subsection{\textcolor{Chapter }{FullPosetOfManiplex (for IsManiplex)}}
\logpage{[ 3, 2, 6 ]}\nobreak
\hyperdef{L}{X7B71A5747E126D8E}{}
{\noindent\textcolor{FuncColor}{$\triangleright$\enspace\texttt{FullPosetOfManiplex({\mdseries\slshape mani})\index{FullPosetOfManiplex@\texttt{FullPosetOfManiplex}!for IsManiplex}
\label{FullPosetOfManiplex:for IsManiplex}
}\hfill{\scriptsize (operation)}}\\


 Given a maniplex, returns a poset as a list of faces ordered by rank, where
each face is represented as a list of the flags it contains. Note that this
function does include the minimal (empty) face and the maximal face of the
maniplex. Note that the \mbox{\texttt{\mdseries\slshape i}}-faces correspond to the \mbox{\texttt{\mdseries\slshape i+1}} item in the list because of how GAP indexes lists. }

 

\subsection{\textcolor{Chapter }{FlagsAsListOfFacesFromPoset (for IsPosetOfFlags)}}
\logpage{[ 3, 2, 7 ]}\nobreak
\hyperdef{L}{X7EDEF1508698F917}{}
{\noindent\textcolor{FuncColor}{$\triangleright$\enspace\texttt{FlagsAsListOfFacesFromPoset({\mdseries\slshape poset})\index{FlagsAsListOfFacesFromPoset@\texttt{FlagsAsListOfFacesFromPoset}!for IsPosetOfFlags}
\label{FlagsAsListOfFacesFromPoset:for IsPosetOfFlags}
}\hfill{\scriptsize (operation)}}\\


 Given a poset, (not a fullposet) this will give you a version of the list of
flags in terms of the faces described in the poset. }

 

\subsection{\textcolor{Chapter }{AdjacentFlag (for IsList,IsList,IsInt)}}
\logpage{[ 3, 2, 8 ]}\nobreak
\hyperdef{L}{X8795640C830E2602}{}
{\noindent\textcolor{FuncColor}{$\triangleright$\enspace\texttt{AdjacentFlag({\mdseries\slshape flag, poset, i})\index{AdjacentFlag@\texttt{AdjacentFlag}!for IsList,IsList,IsInt}
\label{AdjacentFlag:for IsList,IsList,IsInt}
}\hfill{\scriptsize (operation)}}\\


 Given a flag (represented as chains of faces comprised of lists of flags) and
a poset and a rank, this function will give you the \mbox{\texttt{\mdseries\slshape i}}-adjacent flag. Note that adjacencies are listed from ranks 0 to one less than
the dimension. You can replace \mbox{\texttt{\mdseries\slshape flag}} with the integer corresponding to that flag. }

 

\subsection{\textcolor{Chapter }{AdjacentFlag (for IsList,IsPosetOfFlags,IsInt)}}
\logpage{[ 3, 2, 9 ]}\nobreak
\hyperdef{L}{X84773D767ABA4355}{}
{\noindent\textcolor{FuncColor}{$\triangleright$\enspace\texttt{AdjacentFlag({\mdseries\slshape arg1, arg2, arg3})\index{AdjacentFlag@\texttt{AdjacentFlag}!for IsList,IsPosetOfFlags,IsInt}
\label{AdjacentFlag:for IsList,IsPosetOfFlags,IsInt}
}\hfill{\scriptsize (operation)}}\\


 

 }

 

\subsection{\textcolor{Chapter }{ConnectionGeneratorOfPoset (for IsList,IsInt)}}
\logpage{[ 3, 2, 10 ]}\nobreak
\hyperdef{L}{X7EE0B9D37852B3B8}{}
{\noindent\textcolor{FuncColor}{$\triangleright$\enspace\texttt{ConnectionGeneratorOfPoset({\mdseries\slshape poset, i})\index{ConnectionGeneratorOfPoset@\texttt{ConnectionGeneratorOfPoset}!for IsList,IsInt}
\label{ConnectionGeneratorOfPoset:for IsList,IsInt}
}\hfill{\scriptsize (operation)}}\\


 Given a \mbox{\texttt{\mdseries\slshape poset}} and an integer \mbox{\texttt{\mdseries\slshape i}}, this function will give you the associated permutation for the rank \mbox{\texttt{\mdseries\slshape i}}-connection. }

 

\subsection{\textcolor{Chapter }{ConnectionGeneratorOfPoset (for IsPosetOfFlags,IsInt)}}
\logpage{[ 3, 2, 11 ]}\nobreak
\hyperdef{L}{X7FC505DD78AF5ED5}{}
{\noindent\textcolor{FuncColor}{$\triangleright$\enspace\texttt{ConnectionGeneratorOfPoset({\mdseries\slshape arg1, arg2})\index{ConnectionGeneratorOfPoset@\texttt{ConnectionGeneratorOfPoset}!for IsPosetOfFlags,IsInt}
\label{ConnectionGeneratorOfPoset:for IsPosetOfFlags,IsInt}
}\hfill{\scriptsize (operation)}}\\


 

 }

 

\subsection{\textcolor{Chapter }{ConnectionGroupOfPoset (for IsPosetOfFlags)}}
\logpage{[ 3, 2, 12 ]}\nobreak
\hyperdef{L}{X853198097D1C0256}{}
{\noindent\textcolor{FuncColor}{$\triangleright$\enspace\texttt{ConnectionGroupOfPoset({\mdseries\slshape poset})\index{ConnectionGroupOfPoset@\texttt{ConnectionGroupOfPoset}!for IsPosetOfFlags}
\label{ConnectionGroupOfPoset:for IsPosetOfFlags}
}\hfill{\scriptsize (operation)}}\\


 Given a \mbox{\texttt{\mdseries\slshape poset}} corresponding to a maniplex, this function will give you the connection group.
Note that it assumes we are NOT using a full poset (currently). }

 

\subsection{\textcolor{Chapter }{IsFlaggablePoset (for IsPosetOfFlags)}}
\logpage{[ 3, 2, 13 ]}\nobreak
\hyperdef{L}{X84E3CFD97C8AE422}{}
{\noindent\textcolor{FuncColor}{$\triangleright$\enspace\texttt{IsFlaggablePoset({\mdseries\slshape poset})\index{IsFlaggablePoset@\texttt{IsFlaggablePoset}!for IsPosetOfFlags}
\label{IsFlaggablePoset:for IsPosetOfFlags}
}\hfill{\scriptsize (operation)}}\\


 Given a \mbox{\texttt{\mdseries\slshape poset}} (whose elements are lists of flags) corresponding to a maniplex, this function
will tell you if it is flaggable, i.e., if the flags can be recovered from the
poset or not. }

 

\subsection{\textcolor{Chapter }{ListIsFullPoset (for IsList)}}
\logpage{[ 3, 2, 14 ]}\nobreak
\hyperdef{L}{X85F7C2CE7D70A92D}{}
{\noindent\textcolor{FuncColor}{$\triangleright$\enspace\texttt{ListIsFullPoset({\mdseries\slshape list})\index{ListIsFullPoset@\texttt{ListIsFullPoset}!for IsList}
\label{ListIsFullPoset:for IsList}
}\hfill{\scriptsize (operation)}}\\


 Given \mbox{\texttt{\mdseries\slshape list}}, a poset as a list of faces ordered by rank, each face listing the flags on
the face, this function will tell you if the poset is full or not. }

 }

 }

   
\chapter{\textcolor{Chapter }{Families of Polytopes}}\label{Chapter_Families_of_Polytopes}
\logpage{[ 4, 0, 0 ]}
\hyperdef{L}{X7BF24A5D7B3386D7}{}
{
  
\section{\textcolor{Chapter }{Classical Polytopes}}\label{Chapter_Families_of_Polytopes_Section_Classical_Polytopes}
\logpage{[ 4, 1, 0 ]}
\hyperdef{L}{X7A281A1C7DCB2D96}{}
{
  

 

\subsection{\textcolor{Chapter }{Vertex}}
\logpage{[ 4, 1, 1 ]}\nobreak
\hyperdef{L}{X868FA75B794AE1AA}{}
{\noindent\textcolor{FuncColor}{$\triangleright$\enspace\texttt{Vertex({\mdseries\slshape })\index{Vertex@\texttt{Vertex}}
\label{Vertex}
}\hfill{\scriptsize (operation)}}\\


 

 }

 

 

\subsection{\textcolor{Chapter }{Edge}}
\logpage{[ 4, 1, 2 ]}\nobreak
\hyperdef{L}{X7DA6B54B7F300B92}{}
{\noindent\textcolor{FuncColor}{$\triangleright$\enspace\texttt{Edge({\mdseries\slshape })\index{Edge@\texttt{Edge}}
\label{Edge}
}\hfill{\scriptsize (operation)}}\\


 

 }

 

\subsection{\textcolor{Chapter }{Pgon (for IsInt)}}
\logpage{[ 4, 1, 3 ]}\nobreak
\hyperdef{L}{X8436B8097852EF9B}{}
{\noindent\textcolor{FuncColor}{$\triangleright$\enspace\texttt{Pgon({\mdseries\slshape p})\index{Pgon@\texttt{Pgon}!for IsInt}
\label{Pgon:for IsInt}
}\hfill{\scriptsize (operation)}}\\


 

 }

 

\subsection{\textcolor{Chapter }{Cube (for IsInt)}}
\logpage{[ 4, 1, 4 ]}\nobreak
\hyperdef{L}{X8306D0C17A6BDDCE}{}
{\noindent\textcolor{FuncColor}{$\triangleright$\enspace\texttt{Cube({\mdseries\slshape n})\index{Cube@\texttt{Cube}!for IsInt}
\label{Cube:for IsInt}
}\hfill{\scriptsize (operation)}}\\


 

 }

 

\subsection{\textcolor{Chapter }{HemiCube (for IsInt)}}
\logpage{[ 4, 1, 5 ]}\nobreak
\hyperdef{L}{X828C02F4861F0CD3}{}
{\noindent\textcolor{FuncColor}{$\triangleright$\enspace\texttt{HemiCube({\mdseries\slshape n})\index{HemiCube@\texttt{HemiCube}!for IsInt}
\label{HemiCube:for IsInt}
}\hfill{\scriptsize (operation)}}\\


 

 }

 

\subsection{\textcolor{Chapter }{CrossPolytope (for IsInt)}}
\logpage{[ 4, 1, 6 ]}\nobreak
\hyperdef{L}{X78DC5BA486C3288C}{}
{\noindent\textcolor{FuncColor}{$\triangleright$\enspace\texttt{CrossPolytope({\mdseries\slshape n})\index{CrossPolytope@\texttt{CrossPolytope}!for IsInt}
\label{CrossPolytope:for IsInt}
}\hfill{\scriptsize (operation)}}\\


 

 }

 

\subsection{\textcolor{Chapter }{HemiCrossPolytope (for IsInt)}}
\logpage{[ 4, 1, 7 ]}\nobreak
\hyperdef{L}{X7B0D84B87F00A73F}{}
{\noindent\textcolor{FuncColor}{$\triangleright$\enspace\texttt{HemiCrossPolytope({\mdseries\slshape n})\index{HemiCrossPolytope@\texttt{HemiCrossPolytope}!for IsInt}
\label{HemiCrossPolytope:for IsInt}
}\hfill{\scriptsize (operation)}}\\


 

 }

 

\subsection{\textcolor{Chapter }{Simplex (for IsInt)}}
\logpage{[ 4, 1, 8 ]}\nobreak
\hyperdef{L}{X82C75D12838D3FD0}{}
{\noindent\textcolor{FuncColor}{$\triangleright$\enspace\texttt{Simplex({\mdseries\slshape n})\index{Simplex@\texttt{Simplex}!for IsInt}
\label{Simplex:for IsInt}
}\hfill{\scriptsize (operation)}}\\


 

 }

 

\subsection{\textcolor{Chapter }{CubicTiling (for IsInt)}}
\logpage{[ 4, 1, 9 ]}\nobreak
\hyperdef{L}{X7CCDE7817EC0E5B5}{}
{\noindent\textcolor{FuncColor}{$\triangleright$\enspace\texttt{CubicTiling({\mdseries\slshape n})\index{CubicTiling@\texttt{CubicTiling}!for IsInt}
\label{CubicTiling:for IsInt}
}\hfill{\scriptsize (operation)}}\\


 

 }

 

 

\subsection{\textcolor{Chapter }{Dodecahedron}}
\logpage{[ 4, 1, 10 ]}\nobreak
\hyperdef{L}{X81A6D8FE876EB3BE}{}
{\noindent\textcolor{FuncColor}{$\triangleright$\enspace\texttt{Dodecahedron({\mdseries\slshape })\index{Dodecahedron@\texttt{Dodecahedron}}
\label{Dodecahedron}
}\hfill{\scriptsize (operation)}}\\


 

 }

 

 

\subsection{\textcolor{Chapter }{HemiDodecahedron}}
\logpage{[ 4, 1, 11 ]}\nobreak
\hyperdef{L}{X7A49325F782047CC}{}
{\noindent\textcolor{FuncColor}{$\triangleright$\enspace\texttt{HemiDodecahedron({\mdseries\slshape })\index{HemiDodecahedron@\texttt{HemiDodecahedron}}
\label{HemiDodecahedron}
}\hfill{\scriptsize (operation)}}\\


 

 }

 

 

\subsection{\textcolor{Chapter }{Icosahedron}}
\logpage{[ 4, 1, 12 ]}\nobreak
\hyperdef{L}{X83E0EF8F7CCD6979}{}
{\noindent\textcolor{FuncColor}{$\triangleright$\enspace\texttt{Icosahedron({\mdseries\slshape })\index{Icosahedron@\texttt{Icosahedron}}
\label{Icosahedron}
}\hfill{\scriptsize (operation)}}\\


 

 }

 

 

\subsection{\textcolor{Chapter }{HemiIcosahedron}}
\logpage{[ 4, 1, 13 ]}\nobreak
\hyperdef{L}{X7CBB4FC88050D25F}{}
{\noindent\textcolor{FuncColor}{$\triangleright$\enspace\texttt{HemiIcosahedron({\mdseries\slshape })\index{HemiIcosahedron@\texttt{HemiIcosahedron}}
\label{HemiIcosahedron}
}\hfill{\scriptsize (operation)}}\\


 

 }

 

 

\subsection{\textcolor{Chapter }{24Cell}}
\logpage{[ 4, 1, 14 ]}\nobreak
\hyperdef{L}{X7C7194D3826A9287}{}
{\noindent\textcolor{FuncColor}{$\triangleright$\enspace\texttt{24Cell({\mdseries\slshape })\index{24Cell@\texttt{24Cell}}
\label{24Cell}
}\hfill{\scriptsize (operation)}}\\


 

 }

 

 

\subsection{\textcolor{Chapter }{Hemi24Cell}}
\logpage{[ 4, 1, 15 ]}\nobreak
\hyperdef{L}{X863C4F1D85F017B3}{}
{\noindent\textcolor{FuncColor}{$\triangleright$\enspace\texttt{Hemi24Cell({\mdseries\slshape })\index{Hemi24Cell@\texttt{Hemi24Cell}}
\label{Hemi24Cell}
}\hfill{\scriptsize (operation)}}\\


 

 }

 

 

\subsection{\textcolor{Chapter }{120Cell}}
\logpage{[ 4, 1, 16 ]}\nobreak
\hyperdef{L}{X7A7A51CC878450C4}{}
{\noindent\textcolor{FuncColor}{$\triangleright$\enspace\texttt{120Cell({\mdseries\slshape })\index{120Cell@\texttt{120Cell}}
\label{120Cell}
}\hfill{\scriptsize (operation)}}\\


 

 }

 

 

\subsection{\textcolor{Chapter }{Hemi120Cell}}
\logpage{[ 4, 1, 17 ]}\nobreak
\hyperdef{L}{X80EBD6F28447F653}{}
{\noindent\textcolor{FuncColor}{$\triangleright$\enspace\texttt{Hemi120Cell({\mdseries\slshape })\index{Hemi120Cell@\texttt{Hemi120Cell}}
\label{Hemi120Cell}
}\hfill{\scriptsize (operation)}}\\


 

 }

 

 

\subsection{\textcolor{Chapter }{600Cell}}
\logpage{[ 4, 1, 18 ]}\nobreak
\hyperdef{L}{X82FCA8347D417FB6}{}
{\noindent\textcolor{FuncColor}{$\triangleright$\enspace\texttt{600Cell({\mdseries\slshape })\index{600Cell@\texttt{600Cell}}
\label{600Cell}
}\hfill{\scriptsize (operation)}}\\


 

 }

 

 

\subsection{\textcolor{Chapter }{Hemi600Cell}}
\logpage{[ 4, 1, 19 ]}\nobreak
\hyperdef{L}{X786D2F0A7BB97182}{}
{\noindent\textcolor{FuncColor}{$\triangleright$\enspace\texttt{Hemi600Cell({\mdseries\slshape })\index{Hemi600Cell@\texttt{Hemi600Cell}}
\label{Hemi600Cell}
}\hfill{\scriptsize (operation)}}\\


 

 }

 }

 }

   
\chapter{\textcolor{Chapter }{Groups}}\label{Chapter_Groups}
\logpage{[ 5, 0, 0 ]}
\hyperdef{L}{X8716635F7951801B}{}
{
  
\section{\textcolor{Chapter }{Groups}}\label{Chapter_Groups_Section_Groups}
\logpage{[ 5, 1, 0 ]}
\hyperdef{L}{X8716635F7951801B}{}
{
  

\subsection{\textcolor{Chapter }{AutomorphismGroup (for IsManiplex)}}
\logpage{[ 5, 1, 1 ]}\nobreak
\hyperdef{L}{X78746F0385216D57}{}
{\noindent\textcolor{FuncColor}{$\triangleright$\enspace\texttt{AutomorphismGroup({\mdseries\slshape M})\index{AutomorphismGroup@\texttt{AutomorphismGroup}!for IsManiplex}
\label{AutomorphismGroup:for IsManiplex}
}\hfill{\scriptsize (attribute)}}\\


 Returns the automorphism group of \mbox{\texttt{\mdseries\slshape M}}. This group is not guaranteed to be in any particular form. }

 

\subsection{\textcolor{Chapter }{AutomorphismGroupFpGroup (for IsManiplex)}}
\logpage{[ 5, 1, 2 ]}\nobreak
\hyperdef{L}{X837641127998923F}{}
{\noindent\textcolor{FuncColor}{$\triangleright$\enspace\texttt{AutomorphismGroupFpGroup({\mdseries\slshape M})\index{AutomorphismGroupFpGroup@\texttt{AutomorphismGroupFpGroup}!for IsManiplex}
\label{AutomorphismGroupFpGroup:for IsManiplex}
}\hfill{\scriptsize (attribute)}}\\


 Returns the automorphism group of \mbox{\texttt{\mdseries\slshape M}} as a finitely presented group. }

 

\subsection{\textcolor{Chapter }{AutomorphismGroupPermGroup (for IsManiplex)}}
\logpage{[ 5, 1, 3 ]}\nobreak
\hyperdef{L}{X7D6651587DF3790A}{}
{\noindent\textcolor{FuncColor}{$\triangleright$\enspace\texttt{AutomorphismGroupPermGroup({\mdseries\slshape M})\index{AutomorphismGroupPermGroup@\texttt{AutomorphismGroupPermGroup}!for IsManiplex}
\label{AutomorphismGroupPermGroup:for IsManiplex}
}\hfill{\scriptsize (attribute)}}\\


 Returns the automorphism group of \mbox{\texttt{\mdseries\slshape M}} as a permutation group. }

 

\subsection{\textcolor{Chapter }{ConnectionGroup (for IsManiplex)}}
\logpage{[ 5, 1, 4 ]}\nobreak
\hyperdef{L}{X7815931C7E926F4F}{}
{\noindent\textcolor{FuncColor}{$\triangleright$\enspace\texttt{ConnectionGroup({\mdseries\slshape M})\index{ConnectionGroup@\texttt{ConnectionGroup}!for IsManiplex}
\label{ConnectionGroup:for IsManiplex}
}\hfill{\scriptsize (attribute)}}\\


 Returns the connection group of \mbox{\texttt{\mdseries\slshape M}} as a permutation group. We may eventually allow other types of connection
groups. }

 

\subsection{\textcolor{Chapter }{RotationGroup (for IsManiplex)}}
\logpage{[ 5, 1, 5 ]}\nobreak
\hyperdef{L}{X7D194510834667C8}{}
{\noindent\textcolor{FuncColor}{$\triangleright$\enspace\texttt{RotationGroup({\mdseries\slshape M})\index{RotationGroup@\texttt{RotationGroup}!for IsManiplex}
\label{RotationGroup:for IsManiplex}
}\hfill{\scriptsize (attribute)}}\\


 Returns the rotation group of \mbox{\texttt{\mdseries\slshape M}}. This group is not guaranteed to be in any particular form. }

 

\subsection{\textcolor{Chapter }{ExtraRelators (for IsReflexibleManiplex)}}
\logpage{[ 5, 1, 6 ]}\nobreak
\hyperdef{L}{X7D2CE6707B452C0D}{}
{\noindent\textcolor{FuncColor}{$\triangleright$\enspace\texttt{ExtraRelators({\mdseries\slshape M})\index{ExtraRelators@\texttt{ExtraRelators}!for IsReflexibleManiplex}
\label{ExtraRelators:for IsReflexibleManiplex}
}\hfill{\scriptsize (attribute)}}\\


 For a reflexible maniplex \mbox{\texttt{\mdseries\slshape M}}, returns the relators needed to define its automorphism group as a quotient
of the string Coxeter group given by its Schlafli symbol. Not particularly
robust at the moment. }

 

\subsection{\textcolor{Chapter }{IsStringC (for IsGroup)}}
\logpage{[ 5, 1, 7 ]}\nobreak
\hyperdef{L}{X835FA8527E2442F4}{}
{\noindent\textcolor{FuncColor}{$\triangleright$\enspace\texttt{IsStringC({\mdseries\slshape G})\index{IsStringC@\texttt{IsStringC}!for IsGroup}
\label{IsStringC:for IsGroup}
}\hfill{\scriptsize (operation)}}\\


 For an sggi \mbox{\texttt{\mdseries\slshape G}}, returns whether the group is a string C group. It does not check whether \mbox{\texttt{\mdseries\slshape G}} is an sggi. }

 }

 }

   
\chapter{\textcolor{Chapter }{Properties}}\label{Chapter_Properties}
\logpage{[ 6, 0, 0 ]}
\hyperdef{L}{X871597447BB998A1}{}
{
  
\section{\textcolor{Chapter }{Orientability}}\label{Chapter_Properties_Section_Orientability}
\logpage{[ 6, 1, 0 ]}
\hyperdef{L}{X861E4BAD800B2785}{}
{
  

\subsection{\textcolor{Chapter }{IsOrientable (for IsManiplex)}}
\logpage{[ 6, 1, 1 ]}\nobreak
\hyperdef{L}{X7DCAF9D27F36EBD3}{}
{\noindent\textcolor{FuncColor}{$\triangleright$\enspace\texttt{IsOrientable({\mdseries\slshape p})\index{IsOrientable@\texttt{IsOrientable}!for IsManiplex}
\label{IsOrientable:for IsManiplex}
}\hfill{\scriptsize (property)}}\\
\textbf{\indent Returns:\ }
\texttt{true} or \texttt{false} 



 A polytope is orientable if its flag graph is bipartite. Currently only
implemented for regular polytopes. }

 

\subsection{\textcolor{Chapter }{IsIOrientable (for IsManiplex, IsList)}}
\logpage{[ 6, 1, 2 ]}\nobreak
\hyperdef{L}{X79F849DD7F778616}{}
{\noindent\textcolor{FuncColor}{$\triangleright$\enspace\texttt{IsIOrientable({\mdseries\slshape p, I})\index{IsIOrientable@\texttt{IsIOrientable}!for IsManiplex, IsList}
\label{IsIOrientable:for IsManiplex, IsList}
}\hfill{\scriptsize (operation)}}\\


 For a subset I of \texttt{\symbol{123}}0, ..., n-1\texttt{\symbol{125}}, a
polytope if I-orientable if every closed path in its flag graph contains an
even number of edges with colors in I. Currently only implemented for regular
polytopes. }

 

\subsection{\textcolor{Chapter }{IsVertexBipartite (for IsManiplex)}}
\logpage{[ 6, 1, 3 ]}\nobreak
\hyperdef{L}{X877609AB7ABD24AC}{}
{\noindent\textcolor{FuncColor}{$\triangleright$\enspace\texttt{IsVertexBipartite({\mdseries\slshape p})\index{IsVertexBipartite@\texttt{IsVertexBipartite}!for IsManiplex}
\label{IsVertexBipartite:for IsManiplex}
}\hfill{\scriptsize (property)}}\\
\textbf{\indent Returns:\ }
\texttt{true} or \texttt{false} 



 A polytope is vertex-bipartite if its 1-skeleton is bipartite. This is
equivalent to being I-orientable for I =
\texttt{\symbol{123}}0\texttt{\symbol{125}}. }

 

\subsection{\textcolor{Chapter }{IsFacetBipartite (for IsManiplex)}}
\logpage{[ 6, 1, 4 ]}\nobreak
\hyperdef{L}{X7BD9E8A87D6E8FAB}{}
{\noindent\textcolor{FuncColor}{$\triangleright$\enspace\texttt{IsFacetBipartite({\mdseries\slshape p})\index{IsFacetBipartite@\texttt{IsFacetBipartite}!for IsManiplex}
\label{IsFacetBipartite:for IsManiplex}
}\hfill{\scriptsize (property)}}\\
\textbf{\indent Returns:\ }
\texttt{true} or \texttt{false} 



 A polytope is facet-bipartite if the 1-skeleton of its dual is bipartite. This
is equivalent to being I-orientable for I =
\texttt{\symbol{123}}n-1\texttt{\symbol{125}}. }

 }

 }

   
\chapter{\textcolor{Chapter }{Basics}}\label{Chapter_Basics}
\logpage{[ 7, 0, 0 ]}
\hyperdef{L}{X868F7BAB7AC2EEBC}{}
{
  
\section{\textcolor{Chapter }{Constructors}}\label{Chapter_Basics_Section_Constructors}
\logpage{[ 7, 1, 0 ]}
\hyperdef{L}{X86EC0F0A78ECBC10}{}
{
  
\subsection{\textcolor{Chapter }{UniversalSggi}}\label{UniversalSggi}
\logpage{[ 7, 1, 1 ]}
\hyperdef{L}{X7AE843CE7878951F}{}
{
\noindent\textcolor{FuncColor}{$\triangleright$\enspace\texttt{UniversalSggi({\mdseries\slshape n})\index{UniversalSggi@\texttt{UniversalSggi}!for IsInt}
\label{UniversalSggi:for IsInt}
}\hfill{\scriptsize (operation)}}\\
\noindent\textcolor{FuncColor}{$\triangleright$\enspace\texttt{UniversalSggi({\mdseries\slshape sym})\index{UniversalSggi@\texttt{UniversalSggi}!for IsList}
\label{UniversalSggi:for IsList}
}\hfill{\scriptsize (operation)}}\\


 In the first form, returns the universal Coxeter Group of rank n. In the
second form, returns the Coxeter Group with Schlafli symbol sym. }

 

\subsection{\textcolor{Chapter }{ReflexibleManiplex (for IsGroup)}}
\logpage{[ 7, 1, 2 ]}\nobreak
\hyperdef{L}{X8494D0598240BD67}{}
{\noindent\textcolor{FuncColor}{$\triangleright$\enspace\texttt{ReflexibleManiplex({\mdseries\slshape g})\index{ReflexibleManiplex@\texttt{ReflexibleManiplex}!for IsGroup}
\label{ReflexibleManiplex:for IsGroup}
}\hfill{\scriptsize (operation)}}\\


 Given a group g (which should be a string C-group), returns the abstract
regular polytope with that automorphism group, where the privileged generators
are those returned by GeneratorsOfGroup(g). }

 

\subsection{\textcolor{Chapter }{ReflexibleManiplex (for IsList, IsList)}}
\logpage{[ 7, 1, 3 ]}\nobreak
\hyperdef{L}{X7A50E3AA7F5A9E55}{}
{\noindent\textcolor{FuncColor}{$\triangleright$\enspace\texttt{ReflexibleManiplex({\mdseries\slshape symbol, relations})\index{ReflexibleManiplex@\texttt{ReflexibleManiplex}!for IsList, IsList}
\label{ReflexibleManiplex:for IsList, IsList}
}\hfill{\scriptsize (operation)}}\\


 Returns an abstract regular polytope with the given Schlafli symbol and with
the given relations. The formatting of the relations is quite flexible. All of
the following work: 
\begin{Verbatim}[commandchars=!@|,fontsize=\small,frame=single,label=Example]
  q := ReflexibleManiplex([4,3,4], "(r0 r1 r2)^3, (r1 r2 r3)^3");
  q := ReflexibleManiplex([4,3,4], "(r0 r1 r2)^3 = (r1 r2 r3)^3 = 1");
  p := ReflexibleManiplex([infinity], "r0 r1 r0 = r1 r0 r1");
\end{Verbatim}
 }

 

\subsection{\textcolor{Chapter }{ReflexibleManiplex (for IsString)}}
\logpage{[ 7, 1, 4 ]}\nobreak
\hyperdef{L}{X8209E4587808FE97}{}
{\noindent\textcolor{FuncColor}{$\triangleright$\enspace\texttt{ReflexibleManiplex({\mdseries\slshape name})\index{ReflexibleManiplex@\texttt{ReflexibleManiplex}!for IsString}
\label{ReflexibleManiplex:for IsString}
}\hfill{\scriptsize (operation)}}\\


 Returns the regular polytope with the given symbolic name. Examples:
ReflexibleManiplex("\texttt{\symbol{123}}3,3,3\texttt{\symbol{125}}");
ReflexibleManiplex("\texttt{\symbol{123}}4,3\texttt{\symbol{125}}{\textunderscore}3"); }

 }

 }

   
\chapter{\textcolor{Chapter }{Comparing maniplexes}}\label{Chapter_Comparing_maniplexes}
\logpage{[ 8, 0, 0 ]}
\hyperdef{L}{X7A3CC7F9873E02BF}{}
{
  
\section{\textcolor{Chapter }{Quotients and covers}}\label{Chapter_Comparing_maniplexes_Section_Quotients_and_covers}
\logpage{[ 8, 1, 0 ]}
\hyperdef{L}{X7CD5138B85A97590}{}
{
  

\subsection{\textcolor{Chapter }{IsQuotientOf (for IsManiplex, IsManiplex)}}
\logpage{[ 8, 1, 1 ]}\nobreak
\hyperdef{L}{X7E83B53D8513D4A3}{}
{\noindent\textcolor{FuncColor}{$\triangleright$\enspace\texttt{IsQuotientOf({\mdseries\slshape M1, M2})\index{IsQuotientOf@\texttt{IsQuotientOf}!for IsManiplex, IsManiplex}
\label{IsQuotientOf:for IsManiplex, IsManiplex}
}\hfill{\scriptsize (operation)}}\\


 Returns whether \mbox{\texttt{\mdseries\slshape M1}} is a quotient of \mbox{\texttt{\mdseries\slshape M2}}. }

 

\subsection{\textcolor{Chapter }{IsCoverOf (for IsManiplex, IsManiplex)}}
\logpage{[ 8, 1, 2 ]}\nobreak
\hyperdef{L}{X7F320DDB834A0A09}{}
{\noindent\textcolor{FuncColor}{$\triangleright$\enspace\texttt{IsCoverOf({\mdseries\slshape M1, M2})\index{IsCoverOf@\texttt{IsCoverOf}!for IsManiplex, IsManiplex}
\label{IsCoverOf:for IsManiplex, IsManiplex}
}\hfill{\scriptsize (operation)}}\\


 Returns whether \mbox{\texttt{\mdseries\slshape M1}} is a cover of \mbox{\texttt{\mdseries\slshape M2}}. }

 

\subsection{\textcolor{Chapter }{IsIsomorphicTo (for IsManiplex, IsManiplex)}}
\logpage{[ 8, 1, 3 ]}\nobreak
\hyperdef{L}{X785EBE7B87A025B4}{}
{\noindent\textcolor{FuncColor}{$\triangleright$\enspace\texttt{IsIsomorphicTo({\mdseries\slshape M1, M2})\index{IsIsomorphicTo@\texttt{IsIsomorphicTo}!for IsManiplex, IsManiplex}
\label{IsIsomorphicTo:for IsManiplex, IsManiplex}
}\hfill{\scriptsize (operation)}}\\


 Returns whether \mbox{\texttt{\mdseries\slshape M1}} is isomorphic to \mbox{\texttt{\mdseries\slshape M2}}. }

 

\subsection{\textcolor{Chapter }{SmallestRegularCover (for IsManiplex)}}
\logpage{[ 8, 1, 4 ]}\nobreak
\hyperdef{L}{X879C71E784C41B77}{}
{\noindent\textcolor{FuncColor}{$\triangleright$\enspace\texttt{SmallestRegularCover({\mdseries\slshape M})\index{SmallestRegularCover@\texttt{SmallestRegularCover}!for IsManiplex}
\label{SmallestRegularCover:for IsManiplex}
}\hfill{\scriptsize (attribute)}}\\


 Returns the smallest regular cover of \mbox{\texttt{\mdseries\slshape M}}, which is the maniplex whose automorphism group is the connection group of \mbox{\texttt{\mdseries\slshape M}}. }

 }

 }

 \def\indexname{Index\logpage{[ "Ind", 0, 0 ]}
\hyperdef{L}{X83A0356F839C696F}{}
}

\cleardoublepage
\phantomsection
\addcontentsline{toc}{chapter}{Index}


\printindex

\immediate\write\pagenrlog{["Ind", 0, 0], \arabic{page},}
\immediate\write\pagenrlog{["Ind", 0, 0], \arabic{page},}
\newpage
\immediate\write\pagenrlog{["End"], \arabic{page}];}
\immediate\closeout\pagenrlog
\end{document}
