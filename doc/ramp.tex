% generated by GAPDoc2LaTeX from XML source (Frank Luebeck)
\documentclass[a4paper,11pt]{report}

\usepackage[top=37mm,bottom=37mm,left=27mm,right=27mm]{geometry}
\sloppy
\pagestyle{myheadings}
\usepackage{amssymb}
\usepackage[utf8]{inputenc}
\usepackage{makeidx}
\makeindex
\usepackage{color}
\definecolor{FireBrick}{rgb}{0.5812,0.0074,0.0083}
\definecolor{RoyalBlue}{rgb}{0.0236,0.0894,0.6179}
\definecolor{RoyalGreen}{rgb}{0.0236,0.6179,0.0894}
\definecolor{RoyalRed}{rgb}{0.6179,0.0236,0.0894}
\definecolor{LightBlue}{rgb}{0.8544,0.9511,1.0000}
\definecolor{Black}{rgb}{0.0,0.0,0.0}

\definecolor{linkColor}{rgb}{0.0,0.0,0.554}
\definecolor{citeColor}{rgb}{0.0,0.0,0.554}
\definecolor{fileColor}{rgb}{0.0,0.0,0.554}
\definecolor{urlColor}{rgb}{0.0,0.0,0.554}
\definecolor{promptColor}{rgb}{0.0,0.0,0.589}
\definecolor{brkpromptColor}{rgb}{0.589,0.0,0.0}
\definecolor{gapinputColor}{rgb}{0.589,0.0,0.0}
\definecolor{gapoutputColor}{rgb}{0.0,0.0,0.0}

%%  for a long time these were red and blue by default,
%%  now black, but keep variables to overwrite
\definecolor{FuncColor}{rgb}{0.0,0.0,0.0}
%% strange name because of pdflatex bug:
\definecolor{Chapter }{rgb}{0.0,0.0,0.0}
\definecolor{DarkOlive}{rgb}{0.1047,0.2412,0.0064}


\usepackage{fancyvrb}

\usepackage{mathptmx,helvet}
\usepackage[T1]{fontenc}
\usepackage{textcomp}


\usepackage[
            pdftex=true,
            bookmarks=true,        
            a4paper=true,
            pdftitle={Written with GAPDoc},
            pdfcreator={LaTeX with hyperref package / GAPDoc},
            colorlinks=true,
            backref=page,
            breaklinks=true,
            linkcolor=linkColor,
            citecolor=citeColor,
            filecolor=fileColor,
            urlcolor=urlColor,
            pdfpagemode={UseNone}, 
           ]{hyperref}

\newcommand{\maintitlesize}{\fontsize{50}{55}\selectfont}

% write page numbers to a .pnr log file for online help
\newwrite\pagenrlog
\immediate\openout\pagenrlog =\jobname.pnr
\immediate\write\pagenrlog{PAGENRS := [}
\newcommand{\logpage}[1]{\protect\write\pagenrlog{#1, \thepage,}}
%% were never documented, give conflicts with some additional packages

\newcommand{\GAP}{\textsf{GAP}}

%% nicer description environments, allows long labels
\usepackage{enumitem}
\setdescription{style=nextline}

%% depth of toc
\setcounter{tocdepth}{1}





%% command for ColorPrompt style examples
\newcommand{\gapprompt}[1]{\color{promptColor}{\bfseries #1}}
\newcommand{\gapbrkprompt}[1]{\color{brkpromptColor}{\bfseries #1}}
\newcommand{\gapinput}[1]{\color{gapinputColor}{#1}}


\begin{document}

\logpage{[ 0, 0, 0 ]}
\begin{titlepage}
\mbox{}\vfill

\begin{center}{\maintitlesize \textbf{ RAMP \mbox{}}}\\
\vfill

\hypersetup{pdftitle= RAMP }
\markright{\scriptsize \mbox{}\hfill  RAMP  \hfill\mbox{}}
{\Huge \textbf{ The Research Assistant for Maniplexes and Polytopes \mbox{}}}\\
\vfill

{\Huge  0.7.01 \mbox{}}\\[1cm]
{ 1 July 2022 \mbox{}}\\[1cm]
\mbox{}\\[2cm]
{\Large \textbf{ Gabe Cunningham\\
   \mbox{}}}\\
{\Large \textbf{ Mark Mixer\\
  \mbox{}}}\\
{\Large \textbf{ Gordon Williams\\
    \mbox{}}}\\
\hypersetup{pdfauthor= Gabe Cunningham\\
   ;  Mark Mixer\\
  ;  Gordon Williams\\
    }
\end{center}\vfill

\mbox{}\\
{\mbox{}\\
\small \noindent \textbf{ Gabe Cunningham\\
   }  Email: \href{mailto://gabriel.cunningham@gmail.com} {\texttt{gabriel.cunningham@gmail.com}}\\
  Homepage: \href{http://www.gabrielcunningham.com} {\texttt{http://www.gabrielcunningham.com}}}\\
{\mbox{}\\
\small \noindent \textbf{ Mark Mixer\\
  }  Email: \href{mailto://mixerm@wit.edu} {\texttt{mixerm@wit.edu}}}\\
{\mbox{}\\
\small \noindent \textbf{ Gordon Williams\\
    }  Email: \href{mailto://giwilliams@alaska.edu} {\texttt{giwilliams@alaska.edu}}\\
  Homepage: \href{http://williams.alaska.edu} {\texttt{http://williams.alaska.edu}}\\
  Address: \begin{minipage}[t]{8cm}\noindent
 PO Box 756660\\
 Department of Mathematics and Statistics\\
 University of Alaska Fairbanks\\
 Fairbanks, AK 99775-6660\\
 \end{minipage}
}\\
\end{titlepage}

\newpage\setcounter{page}{2}
{\small 
\section*{Copyright}
\logpage{[ 0, 0, 1 ]}
 \index{License} {\copyright} 1997-2022 by Gabe Cunningham, Mark Mixer, and Gordon Williams

 \textsf{RAMP} package is free software; you can redistribute it and/or modify it under the
terms of the \href{http://www.fsf.org/licenses/gpl.html} {GNU General Public License} as published by the Free Software Foundation; either version 2 of the License,
or (at your option) any later version. \mbox{}}\\[1cm]
{\small 
\section*{Acknowledgements}
\logpage{[ 0, 0, 2 ]}
 We appreciate very much all past and future comments, suggestions and
contributions to this package and its documentation provided by \textsf{GAP} users and developers. \mbox{}}\\[1cm]
\newpage

\def\contentsname{Contents\logpage{[ 0, 0, 3 ]}}

\tableofcontents
\newpage

     
\chapter{\textcolor{Chapter }{Installation}}\label{Chapter_Installation}
\logpage{[ 1, 0, 0 ]}
\hyperdef{L}{X8360C04082558A12}{}
{
  
\section{\textcolor{Chapter }{Basics}}\label{Chapter_Installation_Section_Basics}
\logpage{[ 1, 1, 0 ]}
\hyperdef{L}{X868F7BAB7AC2EEBC}{}
{
  Some quick notes on installation: 
\begin{itemize}
\item  RAMP is confirmed to work with version 4.11.1 of GAP, but is known not to work
with some earlier versions. 
\item  Copy the RAMP folder and its contents to your GAP \texttt{/pkg} folder. 
\begin{itemize}
\item  If using the GAP.app on macOS, this should be your user \texttt{Library/Preferences/GAP/pkg} folder. Probably the easiest way to do this if you have received RAMP as a \texttt{.zip} file is to copy the file into this location, and then unpack it. After that,
you can delete the \texttt{.zip} file. 
\end{itemize}
 
\end{itemize}
 

 }

 }

   
\chapter{\textcolor{Chapter }{Using RAMP}}\label{Chapter_Using_RAMP}
\logpage{[ 2, 0, 0 ]}
\hyperdef{L}{X80436AD27BDFFDEC}{}
{
  
\section{\textcolor{Chapter }{Assumptions}}\label{Chapter_Using_RAMP_Section_Assumptions}
\logpage{[ 2, 1, 0 ]}
\hyperdef{L}{X7C6D4C1681F6EA0B}{}
{
  There are a few assumptions that many methods make. 

 1. The connection group of a maniplex with N flags is often assumed to act on
[1..N]. This is gradually being rewritten to allow any set of integers as
flags, but use caution when working with such connection groups. 

 2. When working with a connection group $\langle r_0, \ldots, r_{n-1} \rangle$, some methods may have strange behavior if any $r_i$ or $r_i r_j$ has any fixed points. Indeed, Sggis of that type define pre-maniplexes rather
than maniplexes. Eventually, the methods that build maniplexes will verify
that no $r_i$ or $r_i r_j$ has fixed points. 

 }

 
\section{\textcolor{Chapter }{Extending RAMP}}\label{Chapter_Using_RAMP_Section_Extending_RAMP}
\logpage{[ 2, 2, 0 ]}
\hyperdef{L}{X879632E87BBAC866}{}
{
  Suppose you want to add a new operation on maniplexes to RAMP. We will see how
to accomplish that with a hypothetical example. Let's pretend that there's a
mathematical operation on maniplexes called "Stretch". 

 Our first step will be to create two new files in the lib/ directory:
stretch.gd and stretch.gi. The first file is for the declaration of the new
operation, and the second is for the implementation. 

 In stretch.gd, we want to add a line that declares the new operation,
something like this: 
\begin{Verbatim}[commandchars=!@|,fontsize=\small,frame=single,label=Example]
  DeclareOperation("Stretch", IsManiplex);
\end{Verbatim}
 Now we will write the implementation in stretch.gi: 
\begin{Verbatim}[commandchars=!@|,fontsize=\small,frame=single,label=Example]
  InstallMethod(Stretch, 
  		[IsManiplex],
  		function(M)
  		...actual code goes here...
  		end);
\end{Verbatim}
 Finally, we need to make sure that these new files are read when RAMP is
loaded up. Open up init.g (in the root RAMP directory) and add the line 
\begin{Verbatim}[commandchars=!@|,fontsize=\small,frame=single,label=Example]
  ReadPackage( "ramp", "lib/stretch.gd" ); 
\end{Verbatim}
 (We recommend that you put that line in alphabetical order with the rest.)
Similarly, open up read.g and add the line 
\begin{Verbatim}[commandchars=!@|,fontsize=\small,frame=single,label=Example]
  ReadPackage( "ramp", "lib/stretch.gi" ); 
\end{Verbatim}
 Now your code is available in your copy of RAMP! 

 Here are two more things you should do. First, test your code. Create a new
file called stretch.tst in the tst directory of RAMP. The format of the tests
is that you first write a line that starts with "gap{\textgreater} " and
continues with some input, as if you actually typed it in to the GAP prompt.
Then, on the following line, put the expected output. 

 To run your tests, run the following command in RAMP: 
\begin{Verbatim}[commandchars=!@|,fontsize=\small,frame=single,label=Example]
  !gapprompt@gap>| !gapinput@TestRamp("stretch.tst");|
\end{Verbatim}
 If any tests fail (that is, if the output from GAP does not match the expected
output from your test file), then GAP will alert you to the discrepancies.
Otherwise, when the tests are complete, there will be no output and you will
just see the gap{\textgreater} prompt again. 

 You can also call TEST{\textunderscore}RAMP() to run all of the tests in the
/tst directory. 

 Finally, you should document your operation! Take a look at one of the .gd
files included with RAMP to see what you should include. To actually build the
documentation, you will need the package AutoDoc. For example, the following
will rebuild the documentation: 
\begin{Verbatim}[commandchars=!@|,fontsize=\small,frame=single,label=Example]
  !gapprompt@gap>| !gapinput@LoadPackage("AutoDoc");|
  !gapprompt@gap>| !gapinput@AutoDoc("ramp", rec( scaffold := true, autodoc := true));|
\end{Verbatim}
 To see your updated documentation, you can either navigate to the html file in
the doc/ directory, or you can quit GAP and restart it, and then your
documentation will be available in the inline help. If you have LaTeX set up
properly, then it will also build a pdf manual. 

 }

 }

   
\chapter{\textcolor{Chapter }{Groups for Maps, Polytopes, and Maniplexes}}\label{Chapter_Groups_for_Maps_Polytopes_and_Maniplexes}
\logpage{[ 3, 0, 0 ]}
\hyperdef{L}{X84EB3BF4805FF034}{}
{
  
\section{\textcolor{Chapter }{Groups of Maps, Polytopes, and Maniplexes}}\label{Chapter_Groups_for_Maps_Polytopes_and_Maniplexes_Section_Groups_of_Maps_Polytopes_and_Maniplexes}
\logpage{[ 3, 1, 0 ]}
\hyperdef{L}{X7F53D040850C7587}{}
{
  
\subsection{\textcolor{Chapter }{Automorphism Groups}}\label{Automorphism_Group}
\logpage{[ 3, 1, 1 ]}
\hyperdef{L}{X7A007A0C80D26351}{}
{
\noindent\textcolor{FuncColor}{$\triangleright$\enspace\texttt{AutomorphismGroup({\mdseries\slshape M})\index{AutomorphismGroup@\texttt{AutomorphismGroup}!for IsPremaniplex}
\label{AutomorphismGroup:for IsPremaniplex}
}\hfill{\scriptsize (attribute)}}\\
\noindent\textcolor{FuncColor}{$\triangleright$\enspace\texttt{AutomorphismGroupFpGroup({\mdseries\slshape M})\index{AutomorphismGroupFpGroup@\texttt{AutomorphismGroupFpGroup}!for IsManiplex}
\label{AutomorphismGroupFpGroup:for IsManiplex}
}\hfill{\scriptsize (attribute)}}\\
\noindent\textcolor{FuncColor}{$\triangleright$\enspace\texttt{AutomorphismGroupPermGroup({\mdseries\slshape M})\index{AutomorphismGroupPermGroup@\texttt{AutomorphismGroupPermGroup}!for IsManiplex}
\label{AutomorphismGroupPermGroup:for IsManiplex}
}\hfill{\scriptsize (attribute)}}\\
\noindent\textcolor{FuncColor}{$\triangleright$\enspace\texttt{AutomorphismGroupOnFlags({\mdseries\slshape M})\index{AutomorphismGroupOnFlags@\texttt{AutomorphismGroupOnFlags}!for IsManiplex}
\label{AutomorphismGroupOnFlags:for IsManiplex}
}\hfill{\scriptsize (attribute)}}\\


 Returns the automorphism group of \mbox{\texttt{\mdseries\slshape M}}. This group is not guaranteed to be in any particular form. For particular
permutation representations you should consider the various
AutomorphismGroupOn... functions, or AutomorphismGroupFpGroup. Returns the
automorphism group of \mbox{\texttt{\mdseries\slshape M}} as a finitely presented group. If \mbox{\texttt{\mdseries\slshape M}} is reflexible, then this is guaranteed to be the standard presentation.
Returns the automorphism group of \mbox{\texttt{\mdseries\slshape M}} as a permutation group. This group is not guaranteed to be in any particular
form. For particular permutation representations you should consider the
various AutomorphismGroupOn... functions. Returns the automorphism group of \mbox{\texttt{\mdseries\slshape M}} as a permutation group action on the flags of \mbox{\texttt{\mdseries\slshape M}}. }

 
\begin{Verbatim}[commandchars=!@|,fontsize=\small,frame=single,label=Example]
  !gapprompt@gap>| !gapinput@s0 := (3,7)(4,8)(5,6);;|
  !gapprompt@gap>| !gapinput@s1 := (2,3)(4,6)(5,7);;|
  !gapprompt@gap>| !gapinput@s2 := (1,2)(3,6)(4,8)(5,7);;|
  !gapprompt@gap>| !gapinput@poly := Group([s0,s1,s2]);;|
  !gapprompt@gap>| !gapinput@p:=ARP(poly);|
  regular 3-polytope
  !gapprompt@gap>| !gapinput@AutomorphismGroup(p);|
  Group([ (3,7)(4,8)(5,6), (2,3)(4,6)(5,7), (1,2)(3,6)(4,8)(5,7) ])
  !gapprompt@gap>| !gapinput@AutomorphismGroupFpGroup(p);|
  <fp group on the generators [ r0, r1, r2 ]>
  !gapprompt@gap>| !gapinput@AutomorphismGroupPermGroup(Cube(3));|
  Group([ (3,4), (2,3)(4,5), (1,2)(5,6) ])
  !gapprompt@gap>| !gapinput@AutomorphismGroupOnFlags(Cube(3));|
  <permutation group with 3 generators>
  !gapprompt@gap>| !gapinput@GeneratorsOfGroup(last);|
  [ (1,20)(2,13)(3,10)(4,34)(5,35)(6,7)(8,27)(9,28)(11,23)(12,24)(14,44)(15,45)(16,18)(17,19)(21,40)(22,41)(25,37)(26,38)(29,48)(30,32)(31,33)(36,47)(39,46)(42,43), 
    (1,11)(2,18)(3,4)(5,26)(6,41)(7,8)(9,33)(10,45)(12,15)(13,31)(14,25)(16,28)(17,27)(19,22)(20,38)(21,32)(23,35)(24,34)(29,39)(30,47)(36,43)(37,48)(40,42)(44,46), 
    (1,3)(2,7)(4,25)(5,28)(6,13)(8,32)(9,35)(10,20)(11,14)(12,17)(15,47)(16,40)(18,21)(19,24)(22,48)(23,44)(26,42)(27,30)(29,41)(31,39)(33,46)(34,37)(36,45)(38,43) ]
\end{Verbatim}
 

\subsection{\textcolor{Chapter }{ConnectionGroup (for IsPremaniplex)}}
\logpage{[ 3, 1, 2 ]}\nobreak
\hyperdef{L}{X860DF96279DD28AB}{}
{\noindent\textcolor{FuncColor}{$\triangleright$\enspace\texttt{ConnectionGroup({\mdseries\slshape M})\index{ConnectionGroup@\texttt{ConnectionGroup}!for IsPremaniplex}
\label{ConnectionGroup:for IsPremaniplex}
}\hfill{\scriptsize (attribute)}}\\


 Returns the connection group of the premaniplex \mbox{\texttt{\mdseries\slshape M}} as a permutation group. We may eventually allow other types of connection
groups. Synonym: MonodromyGroup }

 
\begin{Verbatim}[commandchars=!@|,fontsize=\small,frame=single,label=Example]
  !gapprompt@gap>| !gapinput@ConnectionGroup(HemiCube(3));|
  Group([ (1,8)(2,7)(3,14)(4,13)(5,20)(6,19)(9,16)(10,15)(11,22)(12,21)(17,24)(18,23), (1,3)(2,5)
    (4,6)(7,9)(8,11)(10,12)(13,15)(14,17)(16,18)(19,21)(20,23)(22,24), (1,2)(3,4)(5,6)(7,8)(9,10)
    (11,12)(13,14)(15,16)(17,18)(19,20)(21,22)(23,24) ])
\end{Verbatim}
 

\subsection{\textcolor{Chapter }{EvenConnectionGroup (for IsManiplex)}}
\logpage{[ 3, 1, 3 ]}\nobreak
\hyperdef{L}{X7CE8CFB183A4FFEE}{}
{\noindent\textcolor{FuncColor}{$\triangleright$\enspace\texttt{EvenConnectionGroup({\mdseries\slshape M})\index{EvenConnectionGroup@\texttt{EvenConnectionGroup}!for IsManiplex}
\label{EvenConnectionGroup:for IsManiplex}
}\hfill{\scriptsize (attribute)}}\\


 Returns the even-word subgroup of the connection group of \mbox{\texttt{\mdseries\slshape M}} as a permutation group. }

 
\begin{Verbatim}[commandchars=!@|,fontsize=\small,frame=single,label=Example]
  !gapprompt@gap>| !gapinput@EvenConnectionGroup(HemiCube(3));|
  Group([ (1,11,24,14)(2,9,18,20)(3,17,22,8)(4,15,12,19)(5,23,16,7)(6,21,10,13), (1,4,5)(2,6,3)
    (7,10,11)(8,12,9)(13,16,17)(14,18,15)(19,22,23)(20,24,21) ])
\end{Verbatim}
 

\subsection{\textcolor{Chapter }{RotationGroup (for IsManiplex)}}
\logpage{[ 3, 1, 4 ]}\nobreak
\hyperdef{L}{X7D194510834667C8}{}
{\noindent\textcolor{FuncColor}{$\triangleright$\enspace\texttt{RotationGroup({\mdseries\slshape M})\index{RotationGroup@\texttt{RotationGroup}!for IsManiplex}
\label{RotationGroup:for IsManiplex}
}\hfill{\scriptsize (attribute)}}\\


 Returns the rotation group of \mbox{\texttt{\mdseries\slshape M}}. This group is not guaranteed to be in any particular form. }

 
\begin{Verbatim}[commandchars=!@|,fontsize=\small,frame=single,label=Example]
  !gapprompt@gap>| !gapinput@RotationGroup(HemiCube(3));|
  Group([ r0*r1, r1*r2 ])
\end{Verbatim}
 

\subsection{\textcolor{Chapter }{RotationGroupFpGroup (for IsManiplex)}}
\logpage{[ 3, 1, 5 ]}\nobreak
\hyperdef{L}{X831C36BB82F429C4}{}
{\noindent\textcolor{FuncColor}{$\triangleright$\enspace\texttt{RotationGroupFpGroup({\mdseries\slshape M})\index{RotationGroupFpGroup@\texttt{RotationGroupFpGroup}!for IsManiplex}
\label{RotationGroupFpGroup:for IsManiplex}
}\hfill{\scriptsize (attribute)}}\\


 Returns the rotation group of \mbox{\texttt{\mdseries\slshape M}}, as a finitely presented group on the standard generators. }

 
\begin{Verbatim}[commandchars=!@|,fontsize=\small,frame=single,label=Example]
  !gapprompt@gap>| !gapinput@RotationGroupFpGroup(ToroidalMap44([1,2]));|
  <fp group on the generators [ s1, s2 ]>
  !gapprompt@gap>| !gapinput@RelatorsOfFpGroup(last);|
  [ (s1*s2)^2, s1^4, s2^4, s2^-1*s1*(s2*s1^-1)^2 ]
\end{Verbatim}
 

\subsection{\textcolor{Chapter }{ChiralityGroup (for IsRotaryManiplex)}}
\logpage{[ 3, 1, 6 ]}\nobreak
\hyperdef{L}{X7FDCA212782DC0E3}{}
{\noindent\textcolor{FuncColor}{$\triangleright$\enspace\texttt{ChiralityGroup({\mdseries\slshape M})\index{ChiralityGroup@\texttt{ChiralityGroup}!for IsRotaryManiplex}
\label{ChiralityGroup:for IsRotaryManiplex}
}\hfill{\scriptsize (attribute)}}\\


 Returns the chirality group of the rotary maniplex \mbox{\texttt{\mdseries\slshape M}}. This is the kernel of the group epimorphism from the rotation group of \mbox{\texttt{\mdseries\slshape M}} to the rotation group of its maximal reflexible quotient. In particular, the
chirality group is trivial if and only if \mbox{\texttt{\mdseries\slshape M}} is reflexible. }

 
\begin{Verbatim}[commandchars=!@|,fontsize=\small,frame=single,label=Example]
  !gapprompt@gap>| !gapinput@M := ToroidalMap44([1,2]);|
  ToroidalMap44([ 1, 2 ])
  !gapprompt@gap>| !gapinput@G := ChiralityGroup(M);|
  Group([ s2^-1*s1^-1*s2*s1^3*s2*s1 ])
  !gapprompt@gap>| !gapinput@Size(G);|
  5
\end{Verbatim}
 

\subsection{\textcolor{Chapter }{ExtraRelators (for IsReflexibleManiplex)}}
\logpage{[ 3, 1, 7 ]}\nobreak
\hyperdef{L}{X7D2CE6707B452C0D}{}
{\noindent\textcolor{FuncColor}{$\triangleright$\enspace\texttt{ExtraRelators({\mdseries\slshape M})\index{ExtraRelators@\texttt{ExtraRelators}!for IsReflexibleManiplex}
\label{ExtraRelators:for IsReflexibleManiplex}
}\hfill{\scriptsize (attribute)}}\\


 For a reflexible maniplex \mbox{\texttt{\mdseries\slshape M}}, returns the relators needed to define its automorphism group as a quotient
of the string Coxeter group given by its Schlafli symbol. Not particularly
robust at the moment. }

 
\begin{Verbatim}[commandchars=!@|,fontsize=\small,frame=single,label=Example]
  !gapprompt@gap>| !gapinput@ExtraRelators(HemiCube(3));|
  [ (r0*r1*r2)^3 ]
\end{Verbatim}
 

\subsection{\textcolor{Chapter }{ExtraRotRelators (for IsRotaryManiplex)}}
\logpage{[ 3, 1, 8 ]}\nobreak
\hyperdef{L}{X797752A88078CF22}{}
{\noindent\textcolor{FuncColor}{$\triangleright$\enspace\texttt{ExtraRotRelators({\mdseries\slshape M})\index{ExtraRotRelators@\texttt{ExtraRotRelators}!for IsRotaryManiplex}
\label{ExtraRotRelators:for IsRotaryManiplex}
}\hfill{\scriptsize (attribute)}}\\


 For a reflexible maniplex \mbox{\texttt{\mdseries\slshape M}}, returns the relators needed to define its rotation group as a quotient of
the rotation group of a string Coxeter group given by its Schlafli symbol. Not
particularly robust at the moment. }

 
\begin{Verbatim}[commandchars=!@|,fontsize=\small,frame=single,label=Example]
  !gapprompt@gap>| !gapinput@ExtraRotRelators(HemiCube(3));|
  [ (F2^-1*F1^-1)^2, (F2*F1^2*F2^-1*F1^-1)^2 ]
\end{Verbatim}
 

\subsection{\textcolor{Chapter }{IsManiplexable (for IsPermGroup)}}
\logpage{[ 3, 1, 9 ]}\nobreak
\hyperdef{L}{X85C1BC3D85D78ECC}{}
{\noindent\textcolor{FuncColor}{$\triangleright$\enspace\texttt{IsManiplexable({\mdseries\slshape permgroup})\index{IsManiplexable@\texttt{IsManiplexable}!for IsPermGroup}
\label{IsManiplexable:for IsPermGroup}
}\hfill{\scriptsize (operation)}}\\
\textbf{\indent Returns:\ }
\texttt{Boolean}. 



 Given a permutation group, it asks if the generators could be the connection
group of a maniplex. That is to say, are each of the generators and their
products fixed point free. }

 }

 
\section{\textcolor{Chapter }{Sggis}}\label{Chapter_Groups_for_Maps_Polytopes_and_Maniplexes_Section_Sggis}
\logpage{[ 3, 2, 0 ]}
\hyperdef{L}{X8547ABE882A26E43}{}
{
  
\subsection{\textcolor{Chapter }{UniversalSggi}}\label{UniversalSggi}
\logpage{[ 3, 2, 1 ]}
\hyperdef{L}{X7AE843CE7878951F}{}
{
\noindent\textcolor{FuncColor}{$\triangleright$\enspace\texttt{UniversalSggi({\mdseries\slshape n})\index{UniversalSggi@\texttt{UniversalSggi}!for IsInt}
\label{UniversalSggi:for IsInt}
}\hfill{\scriptsize (operation)}}\\
\noindent\textcolor{FuncColor}{$\triangleright$\enspace\texttt{UniversalSggi({\mdseries\slshape sym})\index{UniversalSggi@\texttt{UniversalSggi}!for IsList}
\label{UniversalSggi:for IsList}
}\hfill{\scriptsize (operation)}}\\
\textbf{\indent Returns:\ }
\texttt{IsFpGroup} 



 In the first form, returns the universal Coxeter Group of rank n. In the
second form, returns the Coxeter Group with Schlafli symbol sym. }

 
\begin{Verbatim}[commandchars=!@|,fontsize=\small,frame=single,label=Example]
  !gapprompt@gap>| !gapinput@g:=UniversalSggi(3);|
  <fp group of size infinity on the generators [ r0, r1, r2 ]>
  !gapprompt@gap>| !gapinput@q:=UniversalSggi([3,4]);|
  <fp group of size 48 on the generators [ r0, r1, r2 ]>
  !gapprompt@gap>| !gapinput@IsQuotient(g,q);|
  true
\end{Verbatim}
 
\subsection{\textcolor{Chapter }{Sggi}}\label{Sggi}
\logpage{[ 3, 2, 2 ]}
\hyperdef{L}{X7A4E9AE88662FE00}{}
{
\noindent\textcolor{FuncColor}{$\triangleright$\enspace\texttt{Sggi({\mdseries\slshape symbol[, relations]})\index{Sggi@\texttt{Sggi}!for IsList, IsList}
\label{Sggi:for IsList, IsList}
}\hfill{\scriptsize (operation)}}\\
\noindent\textcolor{FuncColor}{$\triangleright$\enspace\texttt{Sggi({\mdseries\slshape sym, words, orders})\index{Sggi@\texttt{Sggi}!for IsList, IsList, IsList}
\label{Sggi:for IsList, IsList, IsList}
}\hfill{\scriptsize (operation)}}\\
\textbf{\indent Returns:\ }
\texttt{IsFpGroup} 



 Returns the sggi defined by the given Schlafli symbol and with the given
relations. The relations can be given by a list of Tietze words or as a string
of relators or relations that involve r0 etc. If no relations are given, then
returns the universal sggi with the given Schlafli symbol. This method
automatically calls \texttt{InterpolatedString} on the relations, so you may use \$variable in the relations, and it will be
replaced with the value of \texttt{variable} (but for global variables only). 
\begin{Verbatim}[commandchars=!@|,fontsize=\small,frame=single,label=Example]
  !gapprompt@gap>| !gapinput@g := Sggi([4,3,4], "(r0 r1 r2)^3, (r1 r2 r3)^3");;|
  !gapprompt@gap>| !gapinput@h := Sggi([4,4], "r0 = r2");;|
  !gapprompt@gap>| !gapinput@k := Sggi([infinity, infinity], [[1,2,1,2,1,2], [2,3,2,3,2,3]]);;|
  !gapprompt@gap>| !gapinput@k = Sggi([3,3]);|
  true
  !gapprompt@gap>| !gapinput@n := 3;;|
  !gapprompt@gap>| !gapinput@Size(Sggi([4,4], "(r0 r1 r2 r1)^$n"));|
  72
\end{Verbatim}
 The second form takes the Schlafli Symbol \mbox{\texttt{\mdseries\slshape sym}}, a list of \mbox{\texttt{\mdseries\slshape words}} in the generators r0 etc, and a list of \mbox{\texttt{\mdseries\slshape orders}}. It returns the reflexible maniplex that is the quotient of the universal
maniplex with that Schlalfi Symbol by the relations obtained by setting each \mbox{\texttt{\mdseries\slshape word[i]}} to have order \mbox{\texttt{\mdseries\slshape order[i]}}. This is primarily useful for quickly constructing a family of related Sggis. }

 
\begin{Verbatim}[commandchars=!@|,fontsize=\small,frame=single,label=Example]
  !gapprompt@gap>| !gapinput@L := List([1..5], k -> Sggi([4,4], ["r0 r1 r2"], [2*k]));;|
  !gapprompt@gap>| !gapinput@List(L, Size);|
  [ 16, 64, 144, 256, 400 ]
\end{Verbatim}
 

\subsection{\textcolor{Chapter }{IsGgi (for IsGroup)}}
\logpage{[ 3, 2, 3 ]}\nobreak
\hyperdef{L}{X81A3AE37837A4B8D}{}
{\noindent\textcolor{FuncColor}{$\triangleright$\enspace\texttt{IsGgi({\mdseries\slshape g})\index{IsGgi@\texttt{IsGgi}!for IsGroup}
\label{IsGgi:for IsGroup}
}\hfill{\scriptsize (property)}}\\
\textbf{\indent Returns:\ }
whether \mbox{\texttt{\mdseries\slshape g}} is generated by involutions. Or more specifically, whether GeneratorsOfGroup(\mbox{\texttt{\mdseries\slshape g}}) all have order 2 or less. 



 

 }

 
\begin{Verbatim}[commandchars=!@|,fontsize=\small,frame=single,label=Example]
  !gapprompt@gap>| !gapinput@IsGgi(SymmetricGroup(4));|
  false
  !gapprompt@gap>| !gapinput@IsGgi(Group([(1,2),(2,3)]));|
  true
\end{Verbatim}
 

\subsection{\textcolor{Chapter }{IsStringy (for IsGroup)}}
\logpage{[ 3, 2, 4 ]}\nobreak
\hyperdef{L}{X7BBA245D7DBAAE7A}{}
{\noindent\textcolor{FuncColor}{$\triangleright$\enspace\texttt{IsStringy({\mdseries\slshape g})\index{IsStringy@\texttt{IsStringy}!for IsGroup}
\label{IsStringy:for IsGroup}
}\hfill{\scriptsize (property)}}\\
\textbf{\indent Returns:\ }
whether every pair of non-adjacent generators of \mbox{\texttt{\mdseries\slshape g}} commute. 



 

 }

 
\begin{Verbatim}[commandchars=!@|,fontsize=\small,frame=single,label=Example]
  !gapprompt@gap>| !gapinput@IsStringy(Group((1,2),(2,3),(3,4)));|
  true
  !gapprompt@gap>| !gapinput@IsStringy(Group((1,2),(3,4),(2,3)));|
  false
\end{Verbatim}
 

\subsection{\textcolor{Chapter }{IsSggi (for IsGroup)}}
\logpage{[ 3, 2, 5 ]}\nobreak
\hyperdef{L}{X802333F880B11196}{}
{\noindent\textcolor{FuncColor}{$\triangleright$\enspace\texttt{IsSggi({\mdseries\slshape g})\index{IsSggi@\texttt{IsSggi}!for IsGroup}
\label{IsSggi:for IsGroup}
}\hfill{\scriptsize (property)}}\\
\textbf{\indent Returns:\ }
whether \mbox{\texttt{\mdseries\slshape g}} is a string group generated by involutions. Equivalent to \texttt{IsGgi(g) and IsStringy(g)}. 



 

 }

 
\begin{Verbatim}[commandchars=!@|,fontsize=\small,frame=single,label=Example]
  !gapprompt@gap>| !gapinput@IsSggi(SymmetricGroup(4));|
  false
  !gapprompt@gap>| !gapinput@IsSggi(Group((1,2),(3,4),(2,3)));|
  false
  !gapprompt@gap>| !gapinput@IsSggi(Group((1,2),(2,3),(3,4)));|
  true
\end{Verbatim}
 

\subsection{\textcolor{Chapter }{IsFixedPointFreeSggi (for IsGroup)}}
\logpage{[ 3, 2, 6 ]}\nobreak
\hyperdef{L}{X86A6626F7AD7BDF1}{}
{\noindent\textcolor{FuncColor}{$\triangleright$\enspace\texttt{IsFixedPointFreeSggi({\mdseries\slshape g})\index{IsFixedPointFreeSggi@\texttt{IsFixedPointFreeSggi}!for IsGroup}
\label{IsFixedPointFreeSggi:for IsGroup}
}\hfill{\scriptsize (property)}}\\
\textbf{\indent Returns:\ }
whether \mbox{\texttt{\mdseries\slshape g}} is a string group generated by involutions such that no generator and no
product of two generators has any fixed points. A premaniplex M is a maniplex
if and only if IsFixedPointFreeSggi(ConnectionGroup(M)). Equivalent to \texttt{IsGgi(g) and IsStringy(g)}. 



 

 }

 
\begin{Verbatim}[commandchars=!@|,fontsize=\small,frame=single,label=Example]
  !gapprompt@gap>| !gapinput@IsFixedPointFreeSggi(Group((1,2)(3,4), (1,3)(2,4) ,(1,4)(2,3)));|
  true
  !gapprompt@gap>| !gapinput@IsFixedPointFreeSggi(Group((1,2)(3,4), (1,2)(3,4), (1,4)(2,3)));|
  false
\end{Verbatim}
 

\subsection{\textcolor{Chapter }{IsStringRotationGroup (for IsGroup)}}
\logpage{[ 3, 2, 7 ]}\nobreak
\hyperdef{L}{X7800886185453C10}{}
{\noindent\textcolor{FuncColor}{$\triangleright$\enspace\texttt{IsStringRotationGroup({\mdseries\slshape g})\index{IsStringRotationGroup@\texttt{IsStringRotationGroup}!for IsGroup}
\label{IsStringRotationGroup:for IsGroup}
}\hfill{\scriptsize (property)}}\\
\textbf{\indent Returns:\ }
Whether \mbox{\texttt{\mdseries\slshape g}} is a string rotation group, i.e. the even word subgroup of an Sggi. This means
that the product of adjacent generators should be an involution. 



 

 }

 
\begin{Verbatim}[commandchars=!@|,fontsize=\small,frame=single,label=Example]
  !gapprompt@gap>| !gapinput@IsStringRotationGroup(Group((1,2)(3,4), (2,3,4)));|
  false
  !gapprompt@gap>| !gapinput@IsStringRotationGroup(Group((1,3,2), (2,4,3)));|
  true
\end{Verbatim}
 

\subsection{\textcolor{Chapter }{IsStringC (for IsGroup)}}
\logpage{[ 3, 2, 8 ]}\nobreak
\hyperdef{L}{X835FA8527E2442F4}{}
{\noindent\textcolor{FuncColor}{$\triangleright$\enspace\texttt{IsStringC({\mdseries\slshape G})\index{IsStringC@\texttt{IsStringC}!for IsGroup}
\label{IsStringC:for IsGroup}
}\hfill{\scriptsize (operation)}}\\
\textbf{\indent Returns:\ }
Whether \mbox{\texttt{\mdseries\slshape G}} is a string C group. Currently only works for finite groups. 



 

 }

 
\begin{Verbatim}[commandchars=!@|,fontsize=\small,frame=single,label=Example]
  !gapprompt@gap>| !gapinput@IsStringC(Sggi([4,4], "r0 r1 r2"));|
  false
  !gapprompt@gap>| !gapinput@IsStringC(Sggi([4,4], "(r0 r1 r2)^4"));|
  true
\end{Verbatim}
 

\subsection{\textcolor{Chapter }{IsStringCPlus (for IsGroup)}}
\logpage{[ 3, 2, 9 ]}\nobreak
\hyperdef{L}{X8445AFAB7CF3F319}{}
{\noindent\textcolor{FuncColor}{$\triangleright$\enspace\texttt{IsStringCPlus({\mdseries\slshape G})\index{IsStringCPlus@\texttt{IsStringCPlus}!for IsGroup}
\label{IsStringCPlus:for IsGroup}
}\hfill{\scriptsize (operation)}}\\
\textbf{\indent Returns:\ }
Whether \mbox{\texttt{\mdseries\slshape G}} is a string C+ group. Currently only works for finite groups. 



 

 }

 
\begin{Verbatim}[commandchars=!@|,fontsize=\small,frame=single,label=Example]
  !gapprompt@gap>| !gapinput@IsStringCPlus(Group((1,2)(3,4), (2,3,4)));|
  false
  !gapprompt@gap>| !gapinput@IsStringCPlus(Group((1,3,2), (2,4,3)));|
  true
  !gapprompt@gap>| !gapinput@IsStringCPlus(RotationGroup(ToroidalMap44([1,0])));|
  false
\end{Verbatim}
 

\subsection{\textcolor{Chapter }{SggiElement (for IsGroup, IsString)}}
\logpage{[ 3, 2, 10 ]}\nobreak
\hyperdef{L}{X815ADF0A7B5A456A}{}
{\noindent\textcolor{FuncColor}{$\triangleright$\enspace\texttt{SggiElement({\mdseries\slshape g, str})\index{SggiElement@\texttt{SggiElement}!for IsGroup, IsString}
\label{SggiElement:for IsGroup, IsString}
}\hfill{\scriptsize (operation)}}\\
\textbf{\indent Returns:\ }
the element of \mbox{\texttt{\mdseries\slshape g}} with underlying word \mbox{\texttt{\mdseries\slshape str}}. 



 This method automatically calls \texttt{InterpolatedString} on the relations, so you may use \$variable in the relations, and it will be
replaced with the value of \texttt{variable} (but for global variables only). 
\begin{Verbatim}[commandchars=!@|,fontsize=\small,frame=single,label=Example]
  !gapprompt@gap>| !gapinput@g := Group((1,2),(2,3),(3,4));;|
  !gapprompt@gap>| !gapinput@SggiElement(g, "r0 r1");|
  (1,3,2)
  !gapprompt@gap>| !gapinput@n := 2;;|
  !gapprompt@gap>| !gapinput@SggiElement(g, "(r0 r1)^$n");|
  (1,2,3)
\end{Verbatim}
 For convenience, you can also use a reflexible maniplex M in place of g, in
which case \texttt{AutomorphismGroup(M)} is used for g. }

 

\subsection{\textcolor{Chapter }{SimplifiedSggiElement (for IsGroup, IsString)}}
\logpage{[ 3, 2, 11 ]}\nobreak
\hyperdef{L}{X78DC846B7C44FFE6}{}
{\noindent\textcolor{FuncColor}{$\triangleright$\enspace\texttt{SimplifiedSggiElement({\mdseries\slshape g, str})\index{SimplifiedSggiElement@\texttt{SimplifiedSggiElement}!for IsGroup, IsString}
\label{SimplifiedSggiElement:for IsGroup, IsString}
}\hfill{\scriptsize (operation)}}\\
\textbf{\indent Returns:\ }
the element of \mbox{\texttt{\mdseries\slshape g}} with underlying word \mbox{\texttt{\mdseries\slshape str}}, in a reduced form. 



 This acts like SggiElement, except that the word is in reduced form. Note that
this is accomplished by calling SetReducedMultiplication on g. As a result,
further computations with g may be substantially slower. This method
automatically calls \texttt{InterpolatedString} on the relations, so you may use \$variable in the relations, and it will be
replaced with the value of \texttt{variable} (but for global variables only). For convenience, you can also use a
reflexible maniplex M in place of g, in which case \texttt{AutomorphismGroup(M)} is used for g. }

 
\begin{Verbatim}[commandchars=!@|,fontsize=\small,frame=single,label=Example]
  !gapprompt@gap>| !gapinput@g := AutomorphismGroup(Cube(3));;|
  !gapprompt@gap>| !gapinput@SimiplifiedSggiElement(g, "(r0 r1)^5");|
  r0*r1
\end{Verbatim}
 

\subsection{\textcolor{Chapter }{IsRelationOfReflexibleManiplex (for IsManiplex, IsString)}}
\logpage{[ 3, 2, 12 ]}\nobreak
\hyperdef{L}{X8390184C789CC336}{}
{\noindent\textcolor{FuncColor}{$\triangleright$\enspace\texttt{IsRelationOfReflexibleManiplex({\mdseries\slshape M, rel})\index{IsRelationOfReflexibleManiplex@\texttt{IsRelationOfReflexibleManiplex}!for IsManiplex, IsString}
\label{IsRelationOfReflexibleManiplex:for IsManiplex, IsString}
}\hfill{\scriptsize (operation)}}\\
\textbf{\indent Returns:\ }
IsBool 



 Determines whether the relation given by the string \mbox{\texttt{\mdseries\slshape rel}} holds in \texttt{AutomorphismGroup(M)}. This method automatically calls \texttt{InterpolatedString} on the relations, so you may use \$variable in the relations, and it will be
replaced with the value of \texttt{variable} (but for global variables only). }

 
\begin{Verbatim}[commandchars=!@|,fontsize=\small,frame=single,label=Example]
  !gapprompt@gap>| !gapinput@M := ReflexibleManiplex([8,6],"(r0 r1)^4 (r1 r2)^3");;|
  !gapprompt@gap>| !gapinput@IsRelationOfReflexibleManiplex(M, "(r0 r1 r2)^3");|
  false
  !gapprompt@gap>| !gapinput@IsRelationOfReflexibleManiplex(M, "(r0 r1 r2)^12");|
  true
\end{Verbatim}
 

\subsection{\textcolor{Chapter }{SggiFamily (for IsGroup, IsList)}}
\logpage{[ 3, 2, 13 ]}\nobreak
\hyperdef{L}{X87C1FF7481A1D9B2}{}
{\noindent\textcolor{FuncColor}{$\triangleright$\enspace\texttt{SggiFamily({\mdseries\slshape parent, words})\index{SggiFamily@\texttt{SggiFamily}!for IsGroup, IsList}
\label{SggiFamily:for IsGroup, IsList}
}\hfill{\scriptsize (operation)}}\\


 Given a \mbox{\texttt{\mdseries\slshape parent}} group and a list of strings that represent words in r0, r1, etc, returns a
function. That function accepts a list of positive integers L, and returns the
quotient of \mbox{\texttt{\mdseries\slshape parent}} by the relations that set the order of each \mbox{\texttt{\mdseries\slshape words[i]}} to L[i]. 
\begin{Verbatim}[commandchars=!@|,fontsize=\small,frame=single,label=Example]
  !gapprompt@gap>| !gapinput@f := SggiFamily(Sggi([4,4]), ["r0 r1 r2 r1"]);|
  function( orders ) ... end
  !gapprompt@gap>| !gapinput@g := f([3]);|
  <fp group on the generators [ r0, r1, r2 ]>
  !gapprompt@gap>| !gapinput@Size(g);|
  72
  !gapprompt@gap>| !gapinput@h := f([6]);|
  <fp group on the generators [ r0, r1, r2 ]>
  !gapprompt@gap>| !gapinput@IsQuotient(h,g);|
  true
\end{Verbatim}
 One of the advantages of building an SggiFamily is that testing whether one
member of the family is a quotient of another member can be done quite
quickly. }

 

\subsection{\textcolor{Chapter }{IsCConnected (for IsManiplex)}}
\logpage{[ 3, 2, 14 ]}\nobreak
\hyperdef{L}{X786B2C407A3E9EBB}{}
{\noindent\textcolor{FuncColor}{$\triangleright$\enspace\texttt{IsCConnected({\mdseries\slshape m})\index{IsCConnected@\texttt{IsCConnected}!for IsManiplex}
\label{IsCConnected:for IsManiplex}
}\hfill{\scriptsize (property)}}\\
\textbf{\indent Returns:\ }
IsBool 



 Determines whether a given maniplex is C-connected (i.e., is the connection
group a string C-group). }

 
\begin{Verbatim}[commandchars=!@|,fontsize=\small,frame=single,label=Example]
  !gapprompt@gap>| !gapinput@IsCConnected(ToroidalMap44([1,0]));|
  false
  !gapprompt@gap>| !gapinput@IsCConnected(Prism(ToroidalMap44([1,0])));|
  true
\end{Verbatim}
 }

 }

   
\chapter{\textcolor{Chapter }{Families of Polytopes}}\label{Chapter_Families_of_Polytopes}
\logpage{[ 4, 0, 0 ]}
\hyperdef{L}{X7BF24A5D7B3386D7}{}
{
  
\section{\textcolor{Chapter }{Classical polytopes}}\label{Chapter_Families_of_Polytopes_Section_Classical_polytopes}
\logpage{[ 4, 1, 0 ]}
\hyperdef{L}{X7A281A1C7DCB2D96}{}
{
  

\subsection{\textcolor{Chapter }{Vertex}}
\logpage{[ 4, 1, 1 ]}\nobreak
\hyperdef{L}{X868FA75B794AE1AA}{}
{\noindent\textcolor{FuncColor}{$\triangleright$\enspace\texttt{Vertex({\mdseries\slshape })\index{Vertex@\texttt{Vertex}}
\label{Vertex}
}\hfill{\scriptsize (operation)}}\\
\textbf{\indent Returns:\ }
IsPolytope 



 Returns the universal 0-polytope. }

 
\begin{Verbatim}[commandchars=!@|,fontsize=\small,frame=single,label=Example]
  !gapprompt@gap>| !gapinput@Vertex();|
  UniversalPolytope(0)
\end{Verbatim}
 

\subsection{\textcolor{Chapter }{Edge}}
\logpage{[ 4, 1, 2 ]}\nobreak
\hyperdef{L}{X7DA6B54B7F300B92}{}
{\noindent\textcolor{FuncColor}{$\triangleright$\enspace\texttt{Edge({\mdseries\slshape })\index{Edge@\texttt{Edge}}
\label{Edge}
}\hfill{\scriptsize (operation)}}\\
\textbf{\indent Returns:\ }
IsPolytope 



 Returns the universal 1-polytope. }

 
\begin{Verbatim}[commandchars=!@|,fontsize=\small,frame=single,label=Example]
  !gapprompt@gap>| !gapinput@Edge();|
  UniversalPolytope(1)
\end{Verbatim}
 

\subsection{\textcolor{Chapter }{Pgon (for IsInt)}}
\logpage{[ 4, 1, 3 ]}\nobreak
\hyperdef{L}{X8436B8097852EF9B}{}
{\noindent\textcolor{FuncColor}{$\triangleright$\enspace\texttt{Pgon({\mdseries\slshape p})\index{Pgon@\texttt{Pgon}!for IsInt}
\label{Pgon:for IsInt}
}\hfill{\scriptsize (operation)}}\\
\textbf{\indent Returns:\ }
IsPolytope 



 Returns the p-gon. }

 
\begin{Verbatim}[commandchars=!@|,fontsize=\small,frame=single,label=Example]
  !gapprompt@gap>| !gapinput@Facets(Pgon(5));|
  [ UniversalPolytope(1) ]
\end{Verbatim}
 

\subsection{\textcolor{Chapter }{Cube (for IsInt)}}
\logpage{[ 4, 1, 4 ]}\nobreak
\hyperdef{L}{X8306D0C17A6BDDCE}{}
{\noindent\textcolor{FuncColor}{$\triangleright$\enspace\texttt{Cube({\mdseries\slshape n})\index{Cube@\texttt{Cube}!for IsInt}
\label{Cube:for IsInt}
}\hfill{\scriptsize (operation)}}\\
\textbf{\indent Returns:\ }
IsPolytope 



 Returns the n-cube. }

 
\begin{Verbatim}[commandchars=!@|,fontsize=\small,frame=single,label=Example]
  !gapprompt@gap>| !gapinput@Fvector(Cube(4));|
  [ 16, 32, 24, 8 ]
\end{Verbatim}
 

\subsection{\textcolor{Chapter }{HemiCube (for IsInt)}}
\logpage{[ 4, 1, 5 ]}\nobreak
\hyperdef{L}{X828C02F4861F0CD3}{}
{\noindent\textcolor{FuncColor}{$\triangleright$\enspace\texttt{HemiCube({\mdseries\slshape n})\index{HemiCube@\texttt{HemiCube}!for IsInt}
\label{HemiCube:for IsInt}
}\hfill{\scriptsize (operation)}}\\
\textbf{\indent Returns:\ }
IsPolytope 



 Returns the n-hemi-cube. }

 
\begin{Verbatim}[commandchars=!@|,fontsize=\small,frame=single,label=Example]
  !gapprompt@gap>| !gapinput@Fvector(HemiCube(4));|
  [ 8, 16, 12, 4 ]
\end{Verbatim}
 

\subsection{\textcolor{Chapter }{CrossPolytope (for IsInt)}}
\logpage{[ 4, 1, 6 ]}\nobreak
\hyperdef{L}{X78DC5BA486C3288C}{}
{\noindent\textcolor{FuncColor}{$\triangleright$\enspace\texttt{CrossPolytope({\mdseries\slshape n})\index{CrossPolytope@\texttt{CrossPolytope}!for IsInt}
\label{CrossPolytope:for IsInt}
}\hfill{\scriptsize (operation)}}\\
\textbf{\indent Returns:\ }
IsPolytope 



 Returns the n-cross-polytope. }

 
\begin{Verbatim}[commandchars=!@|,fontsize=\small,frame=single,label=Example]
  !gapprompt@gap>| !gapinput@NumberOfVertices(CrossPolytope(5));|
  10
\end{Verbatim}
 

\subsection{\textcolor{Chapter }{Octahedron}}
\logpage{[ 4, 1, 7 ]}\nobreak
\hyperdef{L}{X84BE285087AAC1F7}{}
{\noindent\textcolor{FuncColor}{$\triangleright$\enspace\texttt{Octahedron({\mdseries\slshape })\index{Octahedron@\texttt{Octahedron}}
\label{Octahedron}
}\hfill{\scriptsize (operation)}}\\
\textbf{\indent Returns:\ }
IsPolytope 



 Returns the octahedron (3-cross-polytope). }

 
\begin{Verbatim}[commandchars=!@|,fontsize=\small,frame=single,label=Example]
  !gapprompt@gap>| !gapinput@Octahedron() = CrossPolytope(3)|
  true
\end{Verbatim}
 

\subsection{\textcolor{Chapter }{HemiCrossPolytope (for IsInt)}}
\logpage{[ 4, 1, 8 ]}\nobreak
\hyperdef{L}{X7B0D84B87F00A73F}{}
{\noindent\textcolor{FuncColor}{$\triangleright$\enspace\texttt{HemiCrossPolytope({\mdseries\slshape n})\index{HemiCrossPolytope@\texttt{HemiCrossPolytope}!for IsInt}
\label{HemiCrossPolytope:for IsInt}
}\hfill{\scriptsize (operation)}}\\
\textbf{\indent Returns:\ }
IsPolytope 



 Returns the n-hemi-cross-polytope. }

 
\begin{Verbatim}[commandchars=!@|,fontsize=\small,frame=single,label=Example]
  !gapprompt@gap>| !gapinput@NumberOfVertices(HemiCrossPolytope(5));|
  5
\end{Verbatim}
 

\subsection{\textcolor{Chapter }{Simplex (for IsInt)}}
\logpage{[ 4, 1, 9 ]}\nobreak
\hyperdef{L}{X82C75D12838D3FD0}{}
{\noindent\textcolor{FuncColor}{$\triangleright$\enspace\texttt{Simplex({\mdseries\slshape n})\index{Simplex@\texttt{Simplex}!for IsInt}
\label{Simplex:for IsInt}
}\hfill{\scriptsize (operation)}}\\
\textbf{\indent Returns:\ }
IsPolytope 



 Returns the n-simplex. }

 
\begin{Verbatim}[commandchars=!@|,fontsize=\small,frame=single,label=Example]
  !gapprompt@gap>| !gapinput@Petrial(Simplex(3))=HemiCube(3);|
  true
\end{Verbatim}
 

\subsection{\textcolor{Chapter }{Tetrahedron}}
\logpage{[ 4, 1, 10 ]}\nobreak
\hyperdef{L}{X7B44DDD485145773}{}
{\noindent\textcolor{FuncColor}{$\triangleright$\enspace\texttt{Tetrahedron({\mdseries\slshape })\index{Tetrahedron@\texttt{Tetrahedron}}
\label{Tetrahedron}
}\hfill{\scriptsize (operation)}}\\
\textbf{\indent Returns:\ }
IsPolytope 



 Returns the tetrahedron (3-simplex). }

 
\begin{Verbatim}[commandchars=!@|,fontsize=\small,frame=single,label=Example]
  !gapprompt@gap>| !gapinput@Tetrahedron() = Simplex(3)|
  true
\end{Verbatim}
 

\subsection{\textcolor{Chapter }{CubicTiling (for IsInt)}}
\logpage{[ 4, 1, 11 ]}\nobreak
\hyperdef{L}{X7CCDE7817EC0E5B5}{}
{\noindent\textcolor{FuncColor}{$\triangleright$\enspace\texttt{CubicTiling({\mdseries\slshape n})\index{CubicTiling@\texttt{CubicTiling}!for IsInt}
\label{CubicTiling:for IsInt}
}\hfill{\scriptsize (operation)}}\\
\textbf{\indent Returns:\ }
IsPolytope 



 Returns the rank $n+1$ polytope; the tiling of $E^n$ by $n$-cubes. }

 
\begin{Verbatim}[commandchars=!@|,fontsize=\small,frame=single,label=Example]
  !gapprompt@gap>| !gapinput@SchlafliSymbol(CubicTiling(3));|
  [ 4, 3, 4 ]
\end{Verbatim}
 

\subsection{\textcolor{Chapter }{Dodecahedron}}
\logpage{[ 4, 1, 12 ]}\nobreak
\hyperdef{L}{X81A6D8FE876EB3BE}{}
{\noindent\textcolor{FuncColor}{$\triangleright$\enspace\texttt{Dodecahedron({\mdseries\slshape })\index{Dodecahedron@\texttt{Dodecahedron}}
\label{Dodecahedron}
}\hfill{\scriptsize (operation)}}\\
\textbf{\indent Returns:\ }
IsPolytope 



 Returns the dodecahedron, \texttt{\texttt{\symbol{123}}5, 3\texttt{\symbol{125}}}. }

 
\begin{Verbatim}[commandchars=!@|,fontsize=\small,frame=single,label=Example]
  !gapprompt@gap>| !gapinput@Dual(Dodecahedron());|
  Icosahedron()
\end{Verbatim}
 

\subsection{\textcolor{Chapter }{HemiDodecahedron}}
\logpage{[ 4, 1, 13 ]}\nobreak
\hyperdef{L}{X7A49325F782047CC}{}
{\noindent\textcolor{FuncColor}{$\triangleright$\enspace\texttt{HemiDodecahedron({\mdseries\slshape })\index{HemiDodecahedron@\texttt{HemiDodecahedron}}
\label{HemiDodecahedron}
}\hfill{\scriptsize (operation)}}\\
\textbf{\indent Returns:\ }
IsPolytope 



 Returns the hemi-dodecahedron, \texttt{\texttt{\symbol{123}}5, 3\texttt{\symbol{125}}{\textunderscore}5}. }

 
\begin{Verbatim}[commandchars=!@|,fontsize=\small,frame=single,label=Example]
  !gapprompt@gap>| !gapinput@Dual(HemiDodecahedron());|
  ReflexibleManiplex([ 3, 5 ], "(r2*r1*r0)^5")
\end{Verbatim}
 

\subsection{\textcolor{Chapter }{Icosahedron}}
\logpage{[ 4, 1, 14 ]}\nobreak
\hyperdef{L}{X83E0EF8F7CCD6979}{}
{\noindent\textcolor{FuncColor}{$\triangleright$\enspace\texttt{Icosahedron({\mdseries\slshape })\index{Icosahedron@\texttt{Icosahedron}}
\label{Icosahedron}
}\hfill{\scriptsize (operation)}}\\
\textbf{\indent Returns:\ }
IsPolytope 



 Returns the icosahedron, \texttt{\texttt{\symbol{123}}3, 5\texttt{\symbol{125}}}. }

 
\begin{Verbatim}[commandchars=!@|,fontsize=\small,frame=single,label=Example]
  !gapprompt@gap>| !gapinput@Dual(Icosahedron());|
  Dodecahedron()
\end{Verbatim}
 

\subsection{\textcolor{Chapter }{HemiIcosahedron}}
\logpage{[ 4, 1, 15 ]}\nobreak
\hyperdef{L}{X7CBB4FC88050D25F}{}
{\noindent\textcolor{FuncColor}{$\triangleright$\enspace\texttt{HemiIcosahedron({\mdseries\slshape })\index{HemiIcosahedron@\texttt{HemiIcosahedron}}
\label{HemiIcosahedron}
}\hfill{\scriptsize (operation)}}\\
\textbf{\indent Returns:\ }
IsPolytope 



 Returns the hemi-icosahedron, \texttt{\texttt{\symbol{123}}3, 5\texttt{\symbol{125}}{\textunderscore}5}. }

 
\begin{Verbatim}[commandchars=!@|,fontsize=\small,frame=single,label=Example]
  !gapprompt@gap>| !gapinput@Fvector(HemiIcosahedron());|
  [ 6, 15, 10 ]
\end{Verbatim}
 

\subsection{\textcolor{Chapter }{SmallStellatedDodecahedron}}
\logpage{[ 4, 1, 16 ]}\nobreak
\hyperdef{L}{X7A72E50F8099929E}{}
{\noindent\textcolor{FuncColor}{$\triangleright$\enspace\texttt{SmallStellatedDodecahedron({\mdseries\slshape })\index{SmallStellatedDodecahedron@\texttt{SmallStellatedDodecahedron}}
\label{SmallStellatedDodecahedron}
}\hfill{\scriptsize (operation)}}\\
\textbf{\indent Returns:\ }
IsPolytope 



 Constructs the small stellated dodecahedron combinatorially. This is the same
combinatorial object as the great dodecahedron. You may also use the command \texttt{GreatDodecahedron();}. }

 
\begin{Verbatim}[commandchars=!@|,fontsize=\small,frame=single,label=Example]
  !gapprompt@gap>| !gapinput@SmallStellatedDodecahedron()=GreatDodecahedron();|
  true
  !gapprompt@gap>| !gapinput@Size(GreatDodecahedron());|
  120
\end{Verbatim}
 

\subsection{\textcolor{Chapter }{24Cell}}
\logpage{[ 4, 1, 17 ]}\nobreak
\hyperdef{L}{X7C7194D3826A9287}{}
{\noindent\textcolor{FuncColor}{$\triangleright$\enspace\texttt{24Cell({\mdseries\slshape })\index{24Cell@\texttt{24Cell}}
\label{24Cell}
}\hfill{\scriptsize (operation)}}\\
\textbf{\indent Returns:\ }
IsPolytope 



 Returns the 24-cell, \texttt{\texttt{\symbol{123}}3, 4, 3\texttt{\symbol{125}}}. }

 
\begin{Verbatim}[commandchars=!@|,fontsize=\small,frame=single,label=Example]
  !gapprompt@gap>| !gapinput@SchlafliSymbol(24Cell());|
  [ 3, 4, 3 ]
\end{Verbatim}
 

\subsection{\textcolor{Chapter }{Hemi24Cell}}
\logpage{[ 4, 1, 18 ]}\nobreak
\hyperdef{L}{X863C4F1D85F017B3}{}
{\noindent\textcolor{FuncColor}{$\triangleright$\enspace\texttt{Hemi24Cell({\mdseries\slshape })\index{Hemi24Cell@\texttt{Hemi24Cell}}
\label{Hemi24Cell}
}\hfill{\scriptsize (operation)}}\\
\textbf{\indent Returns:\ }
IsPolytope 



 Returns the hemi-24-cell, \texttt{\texttt{\symbol{123}}3, 4, 3\texttt{\symbol{125}}{\textunderscore}6}. }

 
\begin{Verbatim}[commandchars=!@|,fontsize=\small,frame=single,label=Example]
  !gapprompt@gap>| !gapinput@SchlafliSymbol(Hemi24Cell());|
  [ 3, 4, 3 ]
\end{Verbatim}
 

\subsection{\textcolor{Chapter }{120Cell}}
\logpage{[ 4, 1, 19 ]}\nobreak
\hyperdef{L}{X7A7A51CC878450C4}{}
{\noindent\textcolor{FuncColor}{$\triangleright$\enspace\texttt{120Cell({\mdseries\slshape })\index{120Cell@\texttt{120Cell}}
\label{120Cell}
}\hfill{\scriptsize (operation)}}\\
\textbf{\indent Returns:\ }
IsPolytope 



 Returns the 120-cell, \texttt{\symbol{123}}5, 3, 3\texttt{\symbol{125}}. }

 
\begin{Verbatim}[commandchars=!@|,fontsize=\small,frame=single,label=Example]
  !gapprompt@gap>| !gapinput@NumberOfIFaces(120Cell(),3);|
  120
\end{Verbatim}
 

\subsection{\textcolor{Chapter }{Hemi120Cell}}
\logpage{[ 4, 1, 20 ]}\nobreak
\hyperdef{L}{X80EBD6F28447F653}{}
{\noindent\textcolor{FuncColor}{$\triangleright$\enspace\texttt{Hemi120Cell({\mdseries\slshape })\index{Hemi120Cell@\texttt{Hemi120Cell}}
\label{Hemi120Cell}
}\hfill{\scriptsize (operation)}}\\
\textbf{\indent Returns:\ }
IsPolytope 



 Returns the hemi-120-cell, \texttt{\texttt{\symbol{123}}5, 3, 3\texttt{\symbol{125}}{\textunderscore}15}. }

 
\begin{Verbatim}[commandchars=!@|,fontsize=\small,frame=single,label=Example]
  !gapprompt@gap>| !gapinput@NumberOfIFaces(Hemi120Cell(),3);|
  60
\end{Verbatim}
 

\subsection{\textcolor{Chapter }{600Cell}}
\logpage{[ 4, 1, 21 ]}\nobreak
\hyperdef{L}{X82FCA8347D417FB6}{}
{\noindent\textcolor{FuncColor}{$\triangleright$\enspace\texttt{600Cell({\mdseries\slshape })\index{600Cell@\texttt{600Cell}}
\label{600Cell}
}\hfill{\scriptsize (operation)}}\\
\textbf{\indent Returns:\ }
IsPolytope 



 Returns the 600-cell, \texttt{\texttt{\symbol{123}}3, 3, 5\texttt{\symbol{125}}}. }

 
\begin{Verbatim}[commandchars=!@|,fontsize=\small,frame=single,label=Example]
  !gapprompt@gap>| !gapinput@Dual(600Cell());|
  120Cell()
\end{Verbatim}
 

\subsection{\textcolor{Chapter }{Hemi600Cell}}
\logpage{[ 4, 1, 22 ]}\nobreak
\hyperdef{L}{X786D2F0A7BB97182}{}
{\noindent\textcolor{FuncColor}{$\triangleright$\enspace\texttt{Hemi600Cell({\mdseries\slshape })\index{Hemi600Cell@\texttt{Hemi600Cell}}
\label{Hemi600Cell}
}\hfill{\scriptsize (operation)}}\\
\textbf{\indent Returns:\ }
IsPolytope 



 Returns the hemi-600-cell, \texttt{\texttt{\symbol{123}}3, 3, 5\texttt{\symbol{125}}{\textunderscore}15}. }

 
\begin{Verbatim}[commandchars=!@|,fontsize=\small,frame=single,label=Example]
  !gapprompt@gap>| !gapinput@Dual(Hemi600Cell())=Hemi120Cell();|
  true
\end{Verbatim}
 

\subsection{\textcolor{Chapter }{BrucknerSphere}}
\logpage{[ 4, 1, 23 ]}\nobreak
\hyperdef{L}{X7E949BC7808403F4}{}
{\noindent\textcolor{FuncColor}{$\triangleright$\enspace\texttt{BrucknerSphere({\mdseries\slshape })\index{BrucknerSphere@\texttt{BrucknerSphere}}
\label{BrucknerSphere}
}\hfill{\scriptsize (operation)}}\\
\textbf{\indent Returns:\ }
IsPoset 



 Returns Bruckner's sphere. }

 
\begin{Verbatim}[commandchars=!@|,fontsize=\small,frame=single,label=Example]
  !gapprompt@gap>| !gapinput@IsLattice(BrucknerSphere());|
  true
\end{Verbatim}
 

\subsection{\textcolor{Chapter }{InternallySelfDualPolyhedron1 (for IsInt)}}
\logpage{[ 4, 1, 24 ]}\nobreak
\hyperdef{L}{X84DFF70F81CD6FF8}{}
{\noindent\textcolor{FuncColor}{$\triangleright$\enspace\texttt{InternallySelfDualPolyhedron1({\mdseries\slshape p})\index{InternallySelfDualPolyhedron1@\texttt{InternallySelfDualPolyhedron1}!for IsInt}
\label{InternallySelfDualPolyhedron1:for IsInt}
}\hfill{\scriptsize (operation)}}\\
\textbf{\indent Returns:\ }
IsPolytope 



 Constructs the internally self-dual polyhedron of type \texttt{\texttt{\symbol{123}}p, p\texttt{\symbol{125}}} described in Theorem 5.3 of \cite{CunMix17}. \#(\href{ https://doi.org/10.11575/cdm.v12i2.62785} {\texttt{ https://doi.org/10.11575/cdm.v12i2.62785}}). p must be at least 7. }

 
\begin{Verbatim}[commandchars=!@|,fontsize=\small,frame=single,label=Example]
  !gapprompt@gap>| !gapinput@SchlafliSymbol(InternallySelfDualPolyhedron1(40));|
  [ 40, 40 ]
\end{Verbatim}
 

\subsection{\textcolor{Chapter }{InternallySelfDualPolyhedron2 (for IsInt, IsInt)}}
\logpage{[ 4, 1, 25 ]}\nobreak
\hyperdef{L}{X8724E56280B600E1}{}
{\noindent\textcolor{FuncColor}{$\triangleright$\enspace\texttt{InternallySelfDualPolyhedron2({\mdseries\slshape p, k})\index{InternallySelfDualPolyhedron2@\texttt{InternallySelfDualPolyhedron2}!for IsInt, IsInt}
\label{InternallySelfDualPolyhedron2:for IsInt, IsInt}
}\hfill{\scriptsize (operation)}}\\
\textbf{\indent Returns:\ }
IsPolytope 



 Constructs the internally self-dual polyhedron of type \texttt{\texttt{\symbol{123}}p, p\texttt{\symbol{125}}} described in Theorem 5.8 of \cite{CunMix17}.\# (\href{ https://doi.org/10.11575/cdm.v12i2.62785} {\texttt{ https://doi.org/10.11575/cdm.v12i2.62785}}). \texttt{p} must be even and at least 6, and \texttt{k} must be odd. }

 
\begin{Verbatim}[commandchars=!@|,fontsize=\small,frame=single,label=Example]
  !gapprompt@gap>| !gapinput@SchlafliSymbol(InternallySelfDualPolyhedron2(40,7));|
  [ 40, 40 ]
\end{Verbatim}
 }

 
\section{\textcolor{Chapter }{Flat and tight polytopes}}\label{Chapter_Families_of_Polytopes_Section_Flat_and_tight_polytopes}
\logpage{[ 4, 2, 0 ]}
\hyperdef{L}{X7808135A8376515A}{}
{
  

\subsection{\textcolor{Chapter }{FlatOrientablyRegularPolyhedron (for IsInt, IsInt, IsInt, IsInt)}}
\logpage{[ 4, 2, 1 ]}\nobreak
\hyperdef{L}{X843987937D815A4D}{}
{\noindent\textcolor{FuncColor}{$\triangleright$\enspace\texttt{FlatOrientablyRegularPolyhedron({\mdseries\slshape p, q, i, j})\index{FlatOrientablyRegularPolyhedron@\texttt{FlatOrientablyRegularPolyhedron}!for IsInt, IsInt, IsInt, IsInt}
\label{FlatOrientablyRegularPolyhedron:for IsInt, IsInt, IsInt, IsInt}
}\hfill{\scriptsize (operation)}}\\
\textbf{\indent Returns:\ }
polyhedron 



 \texttt{polyhedron} is the flat orientably regular polyhedron with automorphism group [p, q] / (r2
r1 r0 r1 = (r0 r1)\texttt{\symbol{94}}i (r1 r2)\texttt{\symbol{94}}j). This
function validates the inputs to make sure that the polyhedron is
well-defined. Use FlatOrientablyRegularPolyhedronNC if you do not want this
validation. }

 
\begin{Verbatim}[commandchars=!@|,fontsize=\small,frame=single,label=Example]
  !gapprompt@gap>| !gapinput@FlatOrientablyRegularPolyhedron(4,2,3,3);|
  FlatOrientablyRegularPolyhedron(4,2,-1,1)
\end{Verbatim}
 

\subsection{\textcolor{Chapter }{FlatOrientablyRegularPolyhedraOfType (for IsList)}}
\logpage{[ 4, 2, 2 ]}\nobreak
\hyperdef{L}{X84B0BB9A83E67A3A}{}
{\noindent\textcolor{FuncColor}{$\triangleright$\enspace\texttt{FlatOrientablyRegularPolyhedraOfType({\mdseries\slshape sym})\index{FlatOrientablyRegularPolyhedraOfType@\texttt{Flat}\-\texttt{Orientably}\-\texttt{Regular}\-\texttt{Polyhedra}\-\texttt{Of}\-\texttt{Type}!for IsList}
\label{FlatOrientablyRegularPolyhedraOfType:for IsList}
}\hfill{\scriptsize (operation)}}\\


 Returns a list of all flat, orientably regular polyhedra with Schlafli symbol \mbox{\texttt{\mdseries\slshape sym}}. }

 
\begin{Verbatim}[commandchars=!@|,fontsize=\small,frame=single,label=Example]
  ap> FlatOrientablyRegularPolyhedraOfType([6,6]);
  [ FlatOrientablyRegularPolyhedron(6,6,3,1), FlatOrientablyRegularPolyhedron(6,6,-1,1), 
    FlatOrientablyRegularPolyhedron(6,6,-1,3) ]
\end{Verbatim}
 

\subsection{\textcolor{Chapter }{TightOrientablyRegularPolytopesOfType (for IsList)}}
\logpage{[ 4, 2, 3 ]}\nobreak
\hyperdef{L}{X83F74BA1814E8DF4}{}
{\noindent\textcolor{FuncColor}{$\triangleright$\enspace\texttt{TightOrientablyRegularPolytopesOfType({\mdseries\slshape sym})\index{TightOrientablyRegularPolytopesOfType@\texttt{Tight}\-\texttt{Orientably}\-\texttt{Regular}\-\texttt{Polytopes}\-\texttt{Of}\-\texttt{Type}!for IsList}
\label{TightOrientablyRegularPolytopesOfType:for IsList}
}\hfill{\scriptsize (operation)}}\\


 Returns a list of all tight, orientably regular polytopes with Schlafli symbol \mbox{\texttt{\mdseries\slshape sym}}. When \mbox{\texttt{\mdseries\slshape sym}} has length 2, this just calls FlatOrientablyRegularPolyhedraOfType(\mbox{\texttt{\mdseries\slshape sym}}). }

 
\begin{Verbatim}[commandchars=!@|,fontsize=\small,frame=single,label=Example]
  !gapprompt@gap>| !gapinput@TightOrientablyRegularPolytopesOfType([6,6]);|
  [ FlatOrientablyRegularPolyhedron(6,6,3,1), FlatOrientablyRegularPolyhedron(6,6,-1,1), 
    FlatOrientablyRegularPolyhedron(6,6,-1,3) ]
\end{Verbatim}
 }

 
\section{\textcolor{Chapter }{The Tomotope}}\label{Chapter_Families_of_Polytopes_Section_The_Tomotope}
\logpage{[ 4, 3, 0 ]}
\hyperdef{L}{X866DA2037902BBAD}{}
{
  

\subsection{\textcolor{Chapter }{Tomotope}}
\logpage{[ 4, 3, 1 ]}\nobreak
\hyperdef{L}{X7F5A6DF77ADAF2FE}{}
{\noindent\textcolor{FuncColor}{$\triangleright$\enspace\texttt{Tomotope({\mdseries\slshape })\index{Tomotope@\texttt{Tomotope}}
\label{Tomotope}
}\hfill{\scriptsize (operation)}}\\
\textbf{\indent Returns:\ }
maniplex 



 Constructs the \emph{Tomotope} from \cite{MonPelWil12} }

 
\begin{Verbatim}[commandchars=!@|,fontsize=\small,frame=single,label=Example]
  !gapprompt@gap>| !gapinput@SchlafliSymbol(Tomotope());|
  [ 3, [ 3, 4 ], 4 ]
\end{Verbatim}
 }

 
\section{\textcolor{Chapter }{Toroids}}\label{Chapter_Families_of_Polytopes_Section_Toroids}
\logpage{[ 4, 4, 0 ]}
\hyperdef{L}{X825A89737F01BFF1}{}
{
  

\subsection{\textcolor{Chapter }{ToroidalMap44}}
\logpage{[ 4, 4, 1 ]}\nobreak
\hyperdef{L}{X7F4B135A81962DB3}{}
{\noindent\textcolor{FuncColor}{$\triangleright$\enspace\texttt{ToroidalMap44({\mdseries\slshape u[, v]})\index{ToroidalMap44@\texttt{ToroidalMap44}}
\label{ToroidalMap44}
}\hfill{\scriptsize (function)}}\\
\textbf{\indent Returns:\ }
IsManiplex 



 Returns the toroidal map $\{4,4\}_{\vec u, \vec v}$. If only \mbox{\texttt{\mdseries\slshape u}} is given, then \mbox{\texttt{\mdseries\slshape v}} is taken to be \mbox{\texttt{\mdseries\slshape u}} rotated 90 degrees, in which case the resulting map is either reflexible or
chiral. }

 
\begin{Verbatim}[commandchars=!@|,fontsize=\small,frame=single,label=Example]
  !gapprompt@gap>| !gapinput@ToroidalMap44([3,0]) = ARP([4,4], "(r0 r1 r2 r1)^3");|
  true
  !gapprompt@gap>| !gapinput@M := ToroidalMap44([1,2]);; IsChiral(M);|
  true
  !gapprompt@gap>| !gapinput@ToroidalMap44([5,0]) = SmallestReflexibleCover(M);|
  true
  !gapprompt@gap>| !gapinput@M := ToroidalMap44([2,0],[0,3]);; NumberOfFlagOrbits(M);|
  2
  !gapprompt@gap>| !gapinput@M = ARP([4,4]) / "(r0 r1 r2 r1)^2, (r1 r0 r1 r2)^3";|
  true
  !gapprompt@gap>| !gapinput@SmallestReflexibleCover(M) = ToroidalMap44([6,0]);|
  true
  !gapprompt@gap>| !gapinput@ToroidalMap44([2,3],[4,1]) = ToroidalMap44([-3,2],[-1,4]);|
  true
\end{Verbatim}
 

\subsection{\textcolor{Chapter }{ToroidalMap36}}
\logpage{[ 4, 4, 2 ]}\nobreak
\hyperdef{L}{X855FEC2480DA687F}{}
{\noindent\textcolor{FuncColor}{$\triangleright$\enspace\texttt{ToroidalMap36({\mdseries\slshape u[, v]})\index{ToroidalMap36@\texttt{ToroidalMap36}}
\label{ToroidalMap36}
}\hfill{\scriptsize (function)}}\\
\textbf{\indent Returns:\ }
IsManiplex 



 Returns the toroidal map $\{3,6\}_{\vec u, \vec v}$. If only \mbox{\texttt{\mdseries\slshape u}} is given, then we return the corresponding reflexible map (if \mbox{\texttt{\mdseries\slshape u}} is [a, 0] or [a, a]) or chiral map. }

 
\begin{Verbatim}[commandchars=!@|,fontsize=\small,frame=single,label=Example]
  !gapprompt@gap>| !gapinput@Size(ToroidalMap36([3,0])) = 108;|
  true
  !gapprompt@gap>| !gapinput@SmallestReflexibleCover(ToroidalMap36([2,3])) = ToroidalMap36([19,0]);|
  true
  !gapprompt@gap>| !gapinput@ToroidalMap36([3,0]) = ToroidalMap36([3,0],[0,3]);|
  true
  !gapprompt@gap>| !gapinput@ToroidalMap36([2,3]) = ToroidalMap36([2,3],[-3,5]);|
  true
  !gapprompt@gap>| !gapinput@NumberOfFlagOrbits(ToroidalMap36([3,0],[-2,4]));|
  3
  !gapprompt@gap>| !gapinput@NumberOfFlagOrbits(ToroidalMap36([4,3],[5,0]));|
  6
\end{Verbatim}
 

\subsection{\textcolor{Chapter }{ToroidalMap63}}
\logpage{[ 4, 4, 3 ]}\nobreak
\hyperdef{L}{X858E3C287882A43F}{}
{\noindent\textcolor{FuncColor}{$\triangleright$\enspace\texttt{ToroidalMap63({\mdseries\slshape u[, v]})\index{ToroidalMap63@\texttt{ToroidalMap63}}
\label{ToroidalMap63}
}\hfill{\scriptsize (function)}}\\
\textbf{\indent Returns:\ }
IsManiplex 



 Returns the toroidal map $\{6,3\}_{\vec u, \vec v}$. If only \mbox{\texttt{\mdseries\slshape u}} is given, then we return the corresponding reflexible map (if \mbox{\texttt{\mdseries\slshape u}} is [a, 0] or [a, a]) or chiral map. }

 
\begin{Verbatim}[commandchars=!@|,fontsize=\small,frame=single,label=Example]
  !gapprompt@gap>| !gapinput@Size(ToroidalMap63([3,0])) = 108;|
  true
  !gapprompt@gap>| !gapinput@SmallestReflexibleCover(ToroidalMap63([2,3])) = ToroidalMap63([19,0]);|
  true
  !gapprompt@gap>| !gapinput@ToroidalMap63([3,0]) = ToroidalMap63([3,0],[0,3]);|
  true
  !gapprompt@gap>| !gapinput@ToroidalMap63([2,3]) = ToroidalMap63([2,3],[-3,5]);|
  true
  !gapprompt@gap>| !gapinput@NumberOfFlagOrbits(ToroidalMap63([3,0],[-2,4]));|
  3
  !gapprompt@gap>| !gapinput@NumberOfFlagOrbits(ToroidalMap63([4,3],[5,0]));|
  6
\end{Verbatim}
 

\subsection{\textcolor{Chapter }{CubicToroid (for IsInt,IsInt,IsInt)}}
\logpage{[ 4, 4, 4 ]}\nobreak
\hyperdef{L}{X80340D308628CE61}{}
{\noindent\textcolor{FuncColor}{$\triangleright$\enspace\texttt{CubicToroid({\mdseries\slshape s, k, n})\index{CubicToroid@\texttt{CubicToroid}!for IsInt,IsInt,IsInt}
\label{CubicToroid:for IsInt,IsInt,IsInt}
}\hfill{\scriptsize (operation)}}\\
\textbf{\indent Returns:\ }
IsManiplex 



 Given IsInt triple \mbox{\texttt{\mdseries\slshape s, k, n}}, will return the regular toroid $\{4, 3^{n-2},4\}_{\vec s}$ where $\vec s=(s^k, 0^{n-k})$. }

 
\begin{Verbatim}[commandchars=!@|,fontsize=\small,frame=single,label=Example]
  !gapprompt@gap>| !gapinput@m44:=CubicToroid(3,2,2);;|
  !gapprompt@gap>| !gapinput@m44=ToroidalMap44([3,3]);|
  true
\end{Verbatim}
 

\subsection{\textcolor{Chapter }{CubicToroid (for IsInt,IsList)}}
\logpage{[ 4, 4, 5 ]}\nobreak
\hyperdef{L}{X84CE798B7F485313}{}
{\noindent\textcolor{FuncColor}{$\triangleright$\enspace\texttt{CubicToroid({\mdseries\slshape n, vecs})\index{CubicToroid@\texttt{CubicToroid}!for IsInt,IsList}
\label{CubicToroid:for IsInt,IsList}
}\hfill{\scriptsize (operation)}}\\
\textbf{\indent Returns:\ }
IsManiplex 



 Given an integer n and a list of vectors \mbox{\texttt{\mdseries\slshape vecs}}, returns the cubic toroid that is a quotient of CubicTiling(n) by the
translation subgroup generated by the given vectors. The results may be
nonsensical if \mbox{\texttt{\mdseries\slshape vecs}} does not generate an n-dimensional translation group. }

 
\begin{Verbatim}[commandchars=!@|,fontsize=\small,frame=single,label=Example]
  !gapprompt@gap>| !gapinput@CubicToroid(2,[[2,0],[0,2]]);|
  3-maniplex
  !gapprompt@gap>| !gapinput@last=ToroidalMap44([2,0]);|
  true
\end{Verbatim}
 

\subsection{\textcolor{Chapter }{3343Toroid (for IsInt,IsInt)}}
\logpage{[ 4, 4, 6 ]}\nobreak
\hyperdef{L}{X870D878B843C9D5F}{}
{\noindent\textcolor{FuncColor}{$\triangleright$\enspace\texttt{3343Toroid({\mdseries\slshape s, k})\index{3343Toroid@\texttt{3343Toroid}!for IsInt,IsInt}
\label{3343Toroid:for IsInt,IsInt}
}\hfill{\scriptsize (operation)}}\\
\textbf{\indent Returns:\ }
IsManiplex 



 Given IsInt pair \mbox{\texttt{\mdseries\slshape s, k}}, will return the regular toroid $\{3,3,4,3\}_{\vec s}$ where $\vec s=(s^k, 0^{n-k})$. Note that $k$ must be 1 or 2. }

 
\begin{Verbatim}[commandchars=!@|,fontsize=\small,frame=single,label=Example]
  !gapprompt@gap>| !gapinput@M := 3343Toroid(3,1);|
  ReflexibleManiplex([ 3, 3, 4, 3 ], "(r0 r1 r2 r3 r2 r1 r4 r3 r2 r3 r4 r1 r2 r3 r2 r1)^3")
  !gapprompt@gap>| !gapinput@IsPolytopal(M);|
  true
  !gapprompt@gap>| !gapinput@IsPolytopal(3343Toroid(1,1));|
  false
\end{Verbatim}
 

\subsection{\textcolor{Chapter }{24CellToroid (for IsInt,IsInt)}}
\logpage{[ 4, 4, 7 ]}\nobreak
\hyperdef{L}{X830F2B047B0C3C13}{}
{\noindent\textcolor{FuncColor}{$\triangleright$\enspace\texttt{24CellToroid({\mdseries\slshape s, k})\index{24CellToroid@\texttt{24CellToroid}!for IsInt,IsInt}
\label{24CellToroid:for IsInt,IsInt}
}\hfill{\scriptsize (operation)}}\\
\textbf{\indent Returns:\ }
IsManiplex 



 Given IsInt pair \mbox{\texttt{\mdseries\slshape s, k}}, will return the regular toroid $\{3,4,3,3\}_{\vec s}$ where $\vec s=(s^k, 0^{n-k})$. Note that $k$ must be 1 or 2. }

 
\begin{Verbatim}[commandchars=!@|,fontsize=\small,frame=single,label=Example]
  !gapprompt@gap>| !gapinput@M := 24CellToroid(3,1);;|
  !gapprompt@gap>| !gapinput@Dual(M) = 3343Toroid(3,1);|
  true
\end{Verbatim}
 }

 
\section{\textcolor{Chapter }{Uniform and Archimedean polyhedra}}\label{Chapter_Families_of_Polytopes_Section_Uniform_and_Archimedean_polyhedra}
\logpage{[ 4, 5, 0 ]}
\hyperdef{L}{X858942527933AF92}{}
{
  Representations of the uniform and Archimedean polyhedra here are from \cite{HarWil10}. 

\subsection{\textcolor{Chapter }{Cuboctahedron}}
\logpage{[ 4, 5, 1 ]}\nobreak
\hyperdef{L}{X7D1C3FE3780376C5}{}
{\noindent\textcolor{FuncColor}{$\triangleright$\enspace\texttt{Cuboctahedron({\mdseries\slshape })\index{Cuboctahedron@\texttt{Cuboctahedron}}
\label{Cuboctahedron}
}\hfill{\scriptsize (operation)}}\\
\textbf{\indent Returns:\ }
maniplex 



 Constructs the cuboctahedron. }

 
\begin{Verbatim}[commandchars=!@|,fontsize=\small,frame=single,label=Example]
  !gapprompt@gap>| !gapinput@SchlafliSymbol(Cuboctahedron());|
  [ [ 3, 4 ], 4 ]
\end{Verbatim}
 

\subsection{\textcolor{Chapter }{TruncatedTetrahedron}}
\logpage{[ 4, 5, 2 ]}\nobreak
\hyperdef{L}{X7914BBCD85B92572}{}
{\noindent\textcolor{FuncColor}{$\triangleright$\enspace\texttt{TruncatedTetrahedron({\mdseries\slshape })\index{TruncatedTetrahedron@\texttt{TruncatedTetrahedron}}
\label{TruncatedTetrahedron}
}\hfill{\scriptsize (operation)}}\\
\textbf{\indent Returns:\ }
maniplex 



 Constructs the truncated tetrahedron. }

 
\begin{Verbatim}[commandchars=!@|,fontsize=\small,frame=single,label=Example]
  !gapprompt@gap>| !gapinput@SchlafliSymbol(TruncatedTetrahedron());|
  [ [ 3, 6 ], 3 ]
\end{Verbatim}
 

\subsection{\textcolor{Chapter }{TruncatedOctahedron}}
\logpage{[ 4, 5, 3 ]}\nobreak
\hyperdef{L}{X82251B0B7FF0867C}{}
{\noindent\textcolor{FuncColor}{$\triangleright$\enspace\texttt{TruncatedOctahedron({\mdseries\slshape })\index{TruncatedOctahedron@\texttt{TruncatedOctahedron}}
\label{TruncatedOctahedron}
}\hfill{\scriptsize (operation)}}\\
\textbf{\indent Returns:\ }
maniplex 



 Constructs the truncated octahedron. }

 
\begin{Verbatim}[commandchars=!@|,fontsize=\small,frame=single,label=Example]
  !gapprompt@gap>| !gapinput@Fvector(TruncatedOctahedron());|
  [ 24, 36, 14 ]
\end{Verbatim}
 

\subsection{\textcolor{Chapter }{TruncatedCube}}
\logpage{[ 4, 5, 4 ]}\nobreak
\hyperdef{L}{X83E8732F7B5723CA}{}
{\noindent\textcolor{FuncColor}{$\triangleright$\enspace\texttt{TruncatedCube({\mdseries\slshape })\index{TruncatedCube@\texttt{TruncatedCube}}
\label{TruncatedCube}
}\hfill{\scriptsize (operation)}}\\
\textbf{\indent Returns:\ }
maniplex 



 Constructs the truncated octahedron. }

 
\begin{Verbatim}[commandchars=!@|,fontsize=\small,frame=single,label=Example]
  !gapprompt@gap>| !gapinput@Fvector(TruncatedCube());|
  [ 24, 36, 14 ]
  !gapprompt@gap>| !gapinput@SchlafliSymbol(TruncatedCube());|
  [ [ 3, 8 ], 3 ]
\end{Verbatim}
 

\subsection{\textcolor{Chapter }{Icosadodecahedron}}
\logpage{[ 4, 5, 5 ]}\nobreak
\hyperdef{L}{X7C262C2B7B88190E}{}
{\noindent\textcolor{FuncColor}{$\triangleright$\enspace\texttt{Icosadodecahedron({\mdseries\slshape })\index{Icosadodecahedron@\texttt{Icosadodecahedron}}
\label{Icosadodecahedron}
}\hfill{\scriptsize (operation)}}\\
\textbf{\indent Returns:\ }
maniplex 



 Constructs the icosadodecahedron. }

 
\begin{Verbatim}[commandchars=!@|,fontsize=\small,frame=single,label=Example]
  !gapprompt@gap>| !gapinput@VertexFigure(Icosadodecahedron());|
  Pgon(4)
  !gapprompt@gap>| !gapinput@Facets(Icosadodecahedron());|
  [ Pgon(5), Pgon(3) ]
\end{Verbatim}
 

\subsection{\textcolor{Chapter }{TruncatedIcosahedron}}
\logpage{[ 4, 5, 6 ]}\nobreak
\hyperdef{L}{X81B089967C16B074}{}
{\noindent\textcolor{FuncColor}{$\triangleright$\enspace\texttt{TruncatedIcosahedron({\mdseries\slshape })\index{TruncatedIcosahedron@\texttt{TruncatedIcosahedron}}
\label{TruncatedIcosahedron}
}\hfill{\scriptsize (operation)}}\\
\textbf{\indent Returns:\ }
maniplex 



 Constructs the truncated icosahedron. }

 
\begin{Verbatim}[commandchars=!@|,fontsize=\small,frame=single,label=Example]
  !gapprompt@gap>| !gapinput@Facets(TruncatedIcosahedron());|
  [ Pgon(6), Pgon(5) ]
\end{Verbatim}
 

\subsection{\textcolor{Chapter }{SmallRhombicuboctahedron}}
\logpage{[ 4, 5, 7 ]}\nobreak
\hyperdef{L}{X838FD696802FEFD8}{}
{\noindent\textcolor{FuncColor}{$\triangleright$\enspace\texttt{SmallRhombicuboctahedron({\mdseries\slshape })\index{SmallRhombicuboctahedron@\texttt{SmallRhombicuboctahedron}}
\label{SmallRhombicuboctahedron}
}\hfill{\scriptsize (operation)}}\\
\textbf{\indent Returns:\ }
maniplex 



 Constructs the small rhombicuboctahedron. }

 
\begin{Verbatim}[commandchars=!@|,fontsize=\small,frame=single,label=Example]
  !gapprompt@gap>| !gapinput@ZigzagVector(SmallRhombicuboctahedron());|
  [ 12, 8 ]
\end{Verbatim}
 

\subsection{\textcolor{Chapter }{Pseudorhombicuboctahedron}}
\logpage{[ 4, 5, 8 ]}\nobreak
\hyperdef{L}{X84853A907A3EF1C5}{}
{\noindent\textcolor{FuncColor}{$\triangleright$\enspace\texttt{Pseudorhombicuboctahedron({\mdseries\slshape })\index{Pseudorhombicuboctahedron@\texttt{Pseudorhombicuboctahedron}}
\label{Pseudorhombicuboctahedron}
}\hfill{\scriptsize (operation)}}\\
\textbf{\indent Returns:\ }
maniplex 



 Constructs the pseudorhombicuboctahedron. }

 
\begin{Verbatim}[commandchars=!@|,fontsize=\small,frame=single,label=Example]
  !gapprompt@gap>| !gapinput@Size(ConnectionGroup(Pseudorhombicuboctahedron()));|
  16072626615091200
\end{Verbatim}
 

\subsection{\textcolor{Chapter }{SnubCube}}
\logpage{[ 4, 5, 9 ]}\nobreak
\hyperdef{L}{X7EA8AA0C84A43469}{}
{\noindent\textcolor{FuncColor}{$\triangleright$\enspace\texttt{SnubCube({\mdseries\slshape })\index{SnubCube@\texttt{SnubCube}}
\label{SnubCube}
}\hfill{\scriptsize (operation)}}\\
\textbf{\indent Returns:\ }
maniplex 



 Constructs the snub cube. }

 
\begin{Verbatim}[commandchars=!@|,fontsize=\small,frame=single,label=Example]
  !gapprompt@gap>| !gapinput@IsEquivelar(PetrieDual(SnubCube()));|
  true
  !gapprompt@gap>| !gapinput@SchlafliSymbol(PetrieDual(SnubCube()));|
  [ 30, 5 ]
  !gapprompt@gap>| !gapinput@Size(ConnectionGroup(PetrieDual(SnubCube())));|
  3804202857922560
  !gapprompt@gap>| !gapinput@Size(AutomorphismGroup(PetrieDual(SnubCube())));|
  24
\end{Verbatim}
 

\subsection{\textcolor{Chapter }{SmallRhombicosidodecahedron}}
\logpage{[ 4, 5, 10 ]}\nobreak
\hyperdef{L}{X7A2FC231845204F5}{}
{\noindent\textcolor{FuncColor}{$\triangleright$\enspace\texttt{SmallRhombicosidodecahedron({\mdseries\slshape })\index{SmallRhombicosidodecahedron@\texttt{SmallRhombicosidodecahedron}}
\label{SmallRhombicosidodecahedron}
}\hfill{\scriptsize (operation)}}\\
\textbf{\indent Returns:\ }
maniplex 



 Constructs the small rhombicosidodecahedron. }

 
\begin{Verbatim}[commandchars=!@|,fontsize=\small,frame=single,label=Example]
  !gapprompt@gap>| !gapinput@Facets(SmallRhombicosidodecahedron());|
  [ Pgon(5), Pgon(4), Pgon(3) ]
\end{Verbatim}
 

\subsection{\textcolor{Chapter }{GreatRhombicosidodecahedron}}
\logpage{[ 4, 5, 11 ]}\nobreak
\hyperdef{L}{X87A29A6B8375339F}{}
{\noindent\textcolor{FuncColor}{$\triangleright$\enspace\texttt{GreatRhombicosidodecahedron({\mdseries\slshape })\index{GreatRhombicosidodecahedron@\texttt{GreatRhombicosidodecahedron}}
\label{GreatRhombicosidodecahedron}
}\hfill{\scriptsize (operation)}}\\
\textbf{\indent Returns:\ }
maniplex 



 Constructs the great rhombicosidodecahedron. }

 
\begin{Verbatim}[commandchars=!@|,fontsize=\small,frame=single,label=Example]
  !gapprompt@gap>| !gapinput@Facets(GreatRhombicosidodecahedron());|
  [ Pgon(10), Pgon(4), Pgon(6) ]
\end{Verbatim}
 

\subsection{\textcolor{Chapter }{SnubDodecahedron}}
\logpage{[ 4, 5, 12 ]}\nobreak
\hyperdef{L}{X874B90BD82EA1192}{}
{\noindent\textcolor{FuncColor}{$\triangleright$\enspace\texttt{SnubDodecahedron({\mdseries\slshape })\index{SnubDodecahedron@\texttt{SnubDodecahedron}}
\label{SnubDodecahedron}
}\hfill{\scriptsize (operation)}}\\
\textbf{\indent Returns:\ }
maniplex 



 Constructs the small snub dodecahedron. }

 
\begin{Verbatim}[commandchars=!@|,fontsize=\small,frame=single,label=Example]
  !gapprompt@gap>| !gapinput@Facets(SnubDodecahedron());|
  [ Pgon(5), Pgon(3) ]
  !gapprompt@gap>| !gapinput@IsEquivelar(PetrieDual(SnubDodecahedron()));|
  true
\end{Verbatim}
 

\subsection{\textcolor{Chapter }{TruncatedDodecahedron}}
\logpage{[ 4, 5, 13 ]}\nobreak
\hyperdef{L}{X79A2DEE48126C748}{}
{\noindent\textcolor{FuncColor}{$\triangleright$\enspace\texttt{TruncatedDodecahedron({\mdseries\slshape })\index{TruncatedDodecahedron@\texttt{TruncatedDodecahedron}}
\label{TruncatedDodecahedron}
}\hfill{\scriptsize (operation)}}\\
\textbf{\indent Returns:\ }
maniplex 



 Constructs the truncated dodecahedron. }

 
\begin{Verbatim}[commandchars=!@|,fontsize=\small,frame=single,label=Example]
  !gapprompt@gap>| !gapinput@Facets(TruncatedDodecahedron());|
  [ Pgon(10), Pgon(3) ]
\end{Verbatim}
 

\subsection{\textcolor{Chapter }{GreatRhombicuboctahedron}}
\logpage{[ 4, 5, 14 ]}\nobreak
\hyperdef{L}{X7D08FB508708D8B2}{}
{\noindent\textcolor{FuncColor}{$\triangleright$\enspace\texttt{GreatRhombicuboctahedron({\mdseries\slshape })\index{GreatRhombicuboctahedron@\texttt{GreatRhombicuboctahedron}}
\label{GreatRhombicuboctahedron}
}\hfill{\scriptsize (operation)}}\\
\textbf{\indent Returns:\ }
maniplex 



 Constructs the great rhombicuboctahedron. }

 
\begin{Verbatim}[commandchars=!@|,fontsize=\small,frame=single,label=Example]
  !gapprompt@gap>| !gapinput@Size(ConnectionGroup(GreatRhombicuboctahedron()));|
  5308416
\end{Verbatim}
 }

 }

   
\chapter{\textcolor{Chapter }{Maniplexes}}\label{Chapter_Maniplexes}
\logpage{[ 5, 0, 0 ]}
\hyperdef{L}{X871229A5875C708A}{}
{
  
\section{\textcolor{Chapter }{Constructors}}\label{Chapter_Maniplexes_Section_Constructors}
\logpage{[ 5, 1, 0 ]}
\hyperdef{L}{X86EC0F0A78ECBC10}{}
{
  
\subsection{\textcolor{Chapter }{ReflexibleManiplex}}\label{ReflexibleManiplex}
\logpage{[ 5, 1, 1 ]}
\hyperdef{L}{X7F88B7097E9421AB}{}
{
\noindent\textcolor{FuncColor}{$\triangleright$\enspace\texttt{ReflexibleManiplex({\mdseries\slshape g})\index{ReflexibleManiplex@\texttt{ReflexibleManiplex}!for IsGroup}
\label{ReflexibleManiplex:for IsGroup}
}\hfill{\scriptsize (operation)}}\\
\noindent\textcolor{FuncColor}{$\triangleright$\enspace\texttt{ReflexibleManiplex({\mdseries\slshape sym[, relations]})\index{ReflexibleManiplex@\texttt{ReflexibleManiplex}!for IsList}
\label{ReflexibleManiplex:for IsList}
}\hfill{\scriptsize (operation)}}\\
\noindent\textcolor{FuncColor}{$\triangleright$\enspace\texttt{ReflexibleManiplex({\mdseries\slshape sym, words, orders})\index{ReflexibleManiplex@\texttt{ReflexibleManiplex}!for IsList, IsList, IsList}
\label{ReflexibleManiplex:for IsList, IsList, IsList}
}\hfill{\scriptsize (operation)}}\\
\noindent\textcolor{FuncColor}{$\triangleright$\enspace\texttt{ReflexibleManiplex({\mdseries\slshape name})\index{ReflexibleManiplex@\texttt{ReflexibleManiplex}!for IsString}
\label{ReflexibleManiplex:for IsString}
}\hfill{\scriptsize (operation)}}\\
\textbf{\indent Returns:\ }
\texttt{IsReflexibleManiplex} 



 In the first form, we are given an Sggi \mbox{\texttt{\mdseries\slshape g}} and we return the reflexible maniplex with that automorphism group, where the
privileged generators are those returned by GeneratorsOfGroup(g). 
\begin{Verbatim}[commandchars=!@|,fontsize=\small,frame=single,label=Example]
  !gapprompt@gap>| !gapinput@g := Group([(1,2), (2,3), (3,4)]);|
  !gapprompt@gap>| !gapinput@M := ReflexibleManiplex(g);|
  !gapprompt@gap>| !gapinput@M = Simplex(3);|
  true
\end{Verbatim}
 This function first checks whether g is an Sggi. Use \texttt{ReflexibleManiplexNC} to bypass that check. 

 The second form returns the universal reflexible maniplex with Schlafli symbol \mbox{\texttt{\mdseries\slshape sym}}. If the optional argument \mbox{\texttt{\mdseries\slshape relations}} is given, then we return the reflexible maniplex with the given defining
relations. The relations can be given by a list of Tietze words or as a string
of relators or relations that involve r0 etc. This method automatically calls \texttt{InterpolatedString} on the relations, so you may use \$variable in the relations, and it will be
replaced with the value of \texttt{variable} (but for global variables only). 
\begin{Verbatim}[commandchars=!@|,fontsize=\small,frame=single,label=Example]
  !gapprompt@gap>| !gapinput@q := ReflexibleManiplex([4,3,4], "(r0 r1 r2)^3, (r1 r2 r3)^3");;|
  !gapprompt@gap>| !gapinput@q = ReflexibleManiplex([4,3,4], "(r0 r1 r2)^3 = (r1 r2 r3)^3 = 1");|
  true
  !gapprompt@gap>| !gapinput@p := ReflexibleManiplex([infinity], "r0 r1 r0 = r1 r0 r1");;|
  !gapprompt@gap>| !gapinput@n := 3;;|
  !gapprompt@gap>| !gapinput@Size(ReflexibleManiplex([4,4], "(r0 r1 r2 r1)^$n"));|
  72
\end{Verbatim}
 The third form takes the Schlafli Symbol \mbox{\texttt{\mdseries\slshape sym}}, a list of \mbox{\texttt{\mdseries\slshape words}} in the generators r0 etc, and a list of \mbox{\texttt{\mdseries\slshape orders}}. It returns the reflexible maniplex that is the quotient of the universal
maniplex with that Schlalfi Symbol by the relations obtained by setting each \mbox{\texttt{\mdseries\slshape word[i]}} to have order \mbox{\texttt{\mdseries\slshape order[i]}}. This is primarily useful for quickly constructing a family of related
maniplexes. 
\begin{Verbatim}[commandchars=!@|,fontsize=\small,frame=single,label=Example]
  !gapprompt@gap>| !gapinput@L := List([1..5], k -> ReflexibleManiplex([4,4], ["r0 r1 r2 r1"], [k]));;|
  !gapprompt@gap>| !gapinput@List(L, IsPolytopal);|
  [ false, true, true, true, true ]
\end{Verbatim}
 

 The fourth form accepts common names for reflexible 3-maniplexes that specify
the lengths of holes and zigzags. That is, "\texttt{\symbol{123}}p, q | h2,
\texttt{\symbol{92}}ldots,
hk\texttt{\symbol{92}}\texttt{\symbol{125}}{\textunderscore}z1, ..., zL" is
the quotient of $\{p,q\}$ by the relations that make the 2-holes have length h2, ..., and the 1-zigzags
have length z1, etc. 
\begin{Verbatim}[commandchars=!@A,fontsize=\small,frame=single,label=Example]
  !gapprompt@gap>A !gapinput@ReflexibleManiplex("{4,4 | 6}") = ToroidalMap44([6,0]);A
  true
  !gapprompt@gap>A !gapinput@ReflexibleManiplex("{4,4}_4") = ToroidalMap44([2,2]);A
  true
  !gapprompt@gap>A !gapinput@M := ReflexibleManiplex("{6,6 | 6,2}_4");;A
  !gapprompt@gap>A !gapinput@HoleLength(M,2);A
  6
  !gapprompt@gap>A !gapinput@HoleLength(M,3);A
  2
  !gapprompt@gap>A !gapinput@ZigzagLength(M,1);A
  4
\end{Verbatim}
 }

 In the second and third forms, if the option set{\textunderscore}schlafli is
set, then we set the Schlafli symbol to the one given. This may not be the
correct Schlafli symbol, since the relations may cause a collapse, so this
should only be used if you know that the Schlafli symbol is correct. 

 The abbreviations \texttt{RefMan} and \texttt{RefManNC} are also available for all of these usages. 

\subsection{\textcolor{Chapter }{Maniplex (for IsPermGroup)}}
\logpage{[ 5, 1, 2 ]}\nobreak
\hyperdef{L}{X7F0E44EA7980CF74}{}
{\noindent\textcolor{FuncColor}{$\triangleright$\enspace\texttt{Maniplex({\mdseries\slshape G})\index{Maniplex@\texttt{Maniplex}!for IsPermGroup}
\label{Maniplex:for IsPermGroup}
}\hfill{\scriptsize (operation)}}\\
\textbf{\indent Returns:\ }
\texttt{IsManiplex} 



 Given a permutation group \mbox{\texttt{\mdseries\slshape G}} on the set [1..N], returns a maniplex with N flags with connection group \mbox{\texttt{\mdseries\slshape G}}. This function first checks whether g is an Sggi. Use \texttt{ManiplexNC} to bypass that check. 
\begin{Verbatim}[commandchars=!@|,fontsize=\small,frame=single,label=Example]
  !gapprompt@gap>| !gapinput@G := Group([(1,2)(3,4)(5,6), (2,3)(4,5)(1,6)]);;|
  !gapprompt@gap>| !gapinput@M := Maniplex(G);|
  Pgon(3)
  !gapprompt@gap>| !gapinput@c := ConnectionGroup(Cube(3));|
  <permutation group with 3 generators>
  !gapprompt@gap>| !gapinput@Maniplex(c) = Cube(3);|
  true
\end{Verbatim}
 }

 

\subsection{\textcolor{Chapter }{Maniplex (for IsReflexibleManiplex, IsGroup)}}
\logpage{[ 5, 1, 3 ]}\nobreak
\hyperdef{L}{X80B9127679FEBEAE}{}
{\noindent\textcolor{FuncColor}{$\triangleright$\enspace\texttt{Maniplex({\mdseries\slshape M, H})\index{Maniplex@\texttt{Maniplex}!for IsReflexibleManiplex, IsGroup}
\label{Maniplex:for IsReflexibleManiplex, IsGroup}
}\hfill{\scriptsize (operation)}}\\
\textbf{\indent Returns:\ }
\texttt{IsManiplex} 



 Let \mbox{\texttt{\mdseries\slshape M}} be a reflexible maniplex and let \mbox{\texttt{\mdseries\slshape H}} be a subgroup of AutomorphismGroup(\mbox{\texttt{\mdseries\slshape M}}). This returns the maniplex \mbox{\texttt{\mdseries\slshape M/H}}. This will be reflexible if and only if \mbox{\texttt{\mdseries\slshape H}} is normal. For most purposes, it is probably easier to use QuotientManiplex,
which takes a string of relations as input instead of a subgroup. The example
below builds the map $\{4, 4\}_{(1,0), (0,2)}$. 
\begin{Verbatim}[commandchars=!@|,fontsize=\small,frame=single,label=Example]
  !gapprompt@gap>| !gapinput@M := ReflexibleManiplex([4,4]);|
  CubicTiling(2)
  !gapprompt@gap>| !gapinput@G := AutomorphismGroup(M);|
  <fp group of size infinity on the generators [ r0, r1, r2 ]>
  !gapprompt@gap>| !gapinput@H := Subgroup(G, [G.1*G.2*G.3*G.2, (G.2*G.1*G.2*G.3)^2]);|
  Group([ r0*r1*r2*r1, (r1*r0*r1*r2)^2 ])
  !gapprompt@gap>| !gapinput@M2 := Maniplex(M, H);|
  3-maniplex
  !gapprompt@gap>| !gapinput@Size(M2);|
  16
\end{Verbatim}
 }

 

\subsection{\textcolor{Chapter }{Maniplex (for IsFunction, IsList)}}
\logpage{[ 5, 1, 4 ]}\nobreak
\hyperdef{L}{X81DA2F5185E38D67}{}
{\noindent\textcolor{FuncColor}{$\triangleright$\enspace\texttt{Maniplex({\mdseries\slshape F, inputs})\index{Maniplex@\texttt{Maniplex}!for IsFunction, IsList}
\label{Maniplex:for IsFunction, IsList}
}\hfill{\scriptsize (operation)}}\\
\textbf{\indent Returns:\ }
\texttt{IsManiplex} 



 Constructs a formal maniplex, represented by an operation \mbox{\texttt{\mdseries\slshape F}} and a list of arguments \mbox{\texttt{\mdseries\slshape inputs}}. By itself, this does not really {\textunderscore}do{\textunderscore}
anything -- it creates a maniplex object that only knows the operation \mbox{\texttt{\mdseries\slshape F}} and the \mbox{\texttt{\mdseries\slshape inputs}}. However, many polytope operations (such as Pyramid(M), Medial(M), etc) use
this construction as a base, and then add "attribute computers" that tell the
formal maniplex how to compute certain things in terms of properties of the
base. See AddAttrComputer for more information. }

 

\subsection{\textcolor{Chapter }{Maniplex (for IsPoset)}}
\logpage{[ 5, 1, 5 ]}\nobreak
\hyperdef{L}{X864C043380FB58D4}{}
{\noindent\textcolor{FuncColor}{$\triangleright$\enspace\texttt{Maniplex({\mdseries\slshape P})\index{Maniplex@\texttt{Maniplex}!for IsPoset}
\label{Maniplex:for IsPoset}
}\hfill{\scriptsize (operation)}}\\
\textbf{\indent Returns:\ }
\texttt{IsManiplex} 



 Constructs the maniplex from the given poset \mbox{\texttt{\mdseries\slshape P}}. This assumes that P actually defines a maniplex. }

 

\subsection{\textcolor{Chapter }{Maniplex (for IsEdgeLabeledGraph)}}
\logpage{[ 5, 1, 6 ]}\nobreak
\hyperdef{L}{X866AC7457DA8E2AC}{}
{\noindent\textcolor{FuncColor}{$\triangleright$\enspace\texttt{Maniplex({\mdseries\slshape P})\index{Maniplex@\texttt{Maniplex}!for IsEdgeLabeledGraph}
\label{Maniplex:for IsEdgeLabeledGraph}
}\hfill{\scriptsize (operation)}}\\
\textbf{\indent Returns:\ }
\texttt{IsManiplex} 



 Constructs the maniplex from its flag graph \mbox{\texttt{\mdseries\slshape F}}. This assumes that F actually defines a maniplex. }

 

\subsection{\textcolor{Chapter }{IsPolytopal (for IsManiplex)}}
\logpage{[ 5, 1, 7 ]}\nobreak
\hyperdef{L}{X79DD94037E0FACD4}{}
{\noindent\textcolor{FuncColor}{$\triangleright$\enspace\texttt{IsPolytopal({\mdseries\slshape M})\index{IsPolytopal@\texttt{IsPolytopal}!for IsManiplex}
\label{IsPolytopal:for IsManiplex}
}\hfill{\scriptsize (property)}}\\
\textbf{\indent Returns:\ }
\texttt{true} or \texttt{false} 



 Returns whether the maniplex \mbox{\texttt{\mdseries\slshape M}} is polytopal; i.e., the flag graph of a polytope. }

 }

 
\section{\textcolor{Chapter }{Mixing of Maniplexes functions}}\label{Chapter_Maniplexes_Section_Mixing_of_Maniplexes_functions}
\logpage{[ 5, 2, 0 ]}
\hyperdef{L}{X7EE6BE9E7BB852DC}{}
{
  
\subsection{\textcolor{Chapter }{Mix of groups}}\label{Mix}
\logpage{[ 5, 2, 1 ]}
\hyperdef{L}{X8648D2398775125D}{}
{
\noindent\textcolor{FuncColor}{$\triangleright$\enspace\texttt{Mix({\mdseries\slshape g, h})\index{Mix@\texttt{Mix}!for IsPermGroup, IsPermGroup}
\label{Mix:for IsPermGroup, IsPermGroup}
}\hfill{\scriptsize (operation)}}\\
\textbf{\indent Returns:\ }
\texttt{IsGroup }. 



 Given two groups (either both permutation groups or both FpGroups), returns
the mix of those groups. If g and h are permutation groups of degree m and n,
respectively, then the mix is a permutation group of degree m+n. }

 Here we build the mix of the connection groups of a 3-cube and an edge. 
\begin{Verbatim}[commandchars=!@|,fontsize=\small,frame=single,label=Example]
  !gapprompt@gap>| !gapinput@g1:=ConnectionGroup(Cube(3));|
  <permutation group with 3 generators>
  !gapprompt@gap>| !gapinput@g2:=ConnectionGroup(Edge());|
  Group([ (1,2) ])
  !gapprompt@gap>| !gapinput@Mix(g1,g2);|
  <permutation group with 3 generators>
\end{Verbatim}
 

\subsection{\textcolor{Chapter }{Mix (for IsPremaniplex, IsPremaniplex)}}
\logpage{[ 5, 2, 2 ]}\nobreak
\hyperdef{L}{X848CD6837BE77B6E}{}
{\noindent\textcolor{FuncColor}{$\triangleright$\enspace\texttt{Mix({\mdseries\slshape maniplex, maniplex})\index{Mix@\texttt{Mix}!for IsPremaniplex, IsPremaniplex}
\label{Mix:for IsPremaniplex, IsPremaniplex}
}\hfill{\scriptsize (operation)}}\\
\textbf{\indent Returns:\ }
\texttt{IsReflexibleManiplex }. 



 Given two (pre-)maniplexes, returns their mix. For two reflexible maniplexes
returns the IsReflexibleManiplex from the mix of their connection groups. In
general, it returns the "flag mix" of the two maniplexes (see \texttt{FlagMix}). }

 

\subsection{\textcolor{Chapter }{FlagMix (for IsPremaniplex, IsPremaniplex)}}
\logpage{[ 5, 2, 3 ]}\nobreak
\hyperdef{L}{X80CE0BA87B6CBDC2}{}
{\noindent\textcolor{FuncColor}{$\triangleright$\enspace\texttt{FlagMix({\mdseries\slshape maniplex, maniplex})\index{FlagMix@\texttt{FlagMix}!for IsPremaniplex, IsPremaniplex}
\label{FlagMix:for IsPremaniplex, IsPremaniplex}
}\hfill{\scriptsize (operation)}}\\
\textbf{\indent Returns:\ }
\texttt{IsManiplex }. 



 Given two (pre-)maniplexes p, q this returns the (pre-)maniplex of their
"flag" mix. That is, it constructs the mix of their connection groups, keeps
the connected component with the base flags of p and q, and then builds a
maniplex from this. }

 
\begin{Verbatim}[commandchars=!@|,fontsize=\small,frame=single,label=Example]
  !gapprompt@gap>| !gapinput@M := ToroidalMap44([1,2]);;|
  !gapprompt@gap>| !gapinput@FlagMix(M,M) = M;|
  true
  !gapprompt@gap>| !gapinput@R := FlagMix(M, EnantiomorphicForm(M));|
  3-maniplex with 200 flags
  !gapprompt@gap>| !gapinput@IsReflexible(R);|
  true
  !gapprompt@gap>| !gapinput@R = ToroidalMap44([5,0]);|
  true
\end{Verbatim}
 
\subsection{\textcolor{Chapter }{Comix}}\label{Comix}
\logpage{[ 5, 2, 4 ]}
\hyperdef{L}{X86B6B5DB7850FCA9}{}
{
\noindent\textcolor{FuncColor}{$\triangleright$\enspace\texttt{Comix({\mdseries\slshape fpgroup, fpgroup})\index{Comix@\texttt{Comix}!for IsFpGroup, IsFpGroup}
\label{Comix:for IsFpGroup, IsFpGroup}
}\hfill{\scriptsize (operation)}}\\
\noindent\textcolor{FuncColor}{$\triangleright$\enspace\texttt{Comix({\mdseries\slshape maniplex, maniplex})\index{Comix@\texttt{Comix}!for IsReflexibleManiplex, IsReflexibleManiplex}
\label{Comix:for IsReflexibleManiplex, IsReflexibleManiplex}
}\hfill{\scriptsize (operation)}}\\
\textbf{\indent Returns:\ }
\texttt{IsReflexibleManiplex }. 



 Returns the comix of two Finitely Presented groups gp and gq. Given maniplexes
returns the IsReflexibleManiplex from the comix of their connection groups }

 
\subsection{\textcolor{Chapter }{Indexed array tools}}\label{Tools}
\logpage{[ 5, 2, 5 ]}
\hyperdef{L}{X82C3A3447A93CDD6}{}
{
\noindent\textcolor{FuncColor}{$\triangleright$\enspace\texttt{CtoL({\mdseries\slshape int, int, int, int})\index{CtoL@\texttt{CtoL}!for IsInt,IsInt,IsInt,IsInt}
\label{CtoL:for IsInt,IsInt,IsInt,IsInt}
}\hfill{\scriptsize (operation)}}\\
\textbf{\indent Returns:\ }
\texttt{IsInteger }. 



 CtoL Returns an integer between 1 and N*M associated with the pair [a,b]. LtoC
Returns an ordered pair [a,b] associated with the integer between 1 and N*M. a
should range between 1 and N, and b should range between 1 and M N is how many
columns (x coordinates), M is how many rows (y coordinates) in a matrix
Functions are inverses. }

 
\begin{Verbatim}[commandchars=!@|,fontsize=\small,frame=single,label=Example]
  !gapprompt@gap>| !gapinput@LtoC(5,4,14);|
  [ 1, 2 ]
  !gapprompt@gap>| !gapinput@CtoL(3,2,5,4);|
  8
  !gapprompt@gap>| !gapinput@LtoC(8,5,4);|
  [ 3, 2 ]
\end{Verbatim}
 }

 
\section{\textcolor{Chapter }{Rotary maniplexes and rotation groups}}\label{Chapter_Maniplexes_Section_Rotary_maniplexes_and_rotation_groups}
\logpage{[ 5, 3, 0 ]}
\hyperdef{L}{X8064E656804172CF}{}
{
  

\subsection{\textcolor{Chapter }{UniversalRotationGroup (for IsInt)}}
\logpage{[ 5, 3, 1 ]}\nobreak
\hyperdef{L}{X8056E3557E06EFF6}{}
{\noindent\textcolor{FuncColor}{$\triangleright$\enspace\texttt{UniversalRotationGroup({\mdseries\slshape n})\index{UniversalRotationGroup@\texttt{UniversalRotationGroup}!for IsInt}
\label{UniversalRotationGroup:for IsInt}
}\hfill{\scriptsize (operation)}}\\


 Returns the rotation subgroup of the universal Coxeter Group of rank n. }

 
\begin{Verbatim}[commandchars=!@|,fontsize=\small,frame=single,label=Example]
  !gapprompt@gap>| !gapinput@UniversalRotationGroup(3);|
  <fp group of size infinity on the generators [ s1, s2 ]>
\end{Verbatim}
 

\subsection{\textcolor{Chapter }{UniversalRotationGroup (for IsList)}}
\logpage{[ 5, 3, 2 ]}\nobreak
\hyperdef{L}{X7D6042317A76FDC3}{}
{\noindent\textcolor{FuncColor}{$\triangleright$\enspace\texttt{UniversalRotationGroup({\mdseries\slshape sym})\index{UniversalRotationGroup@\texttt{UniversalRotationGroup}!for IsList}
\label{UniversalRotationGroup:for IsList}
}\hfill{\scriptsize (operation)}}\\


 Returns the rotation subgroup of the Coxeter Group with Schlafli symbol sym. }

 
\begin{Verbatim}[commandchars=!@|,fontsize=\small,frame=single,label=Example]
  !gapprompt@gap>| !gapinput@UniversalRotationGroup([4,4]);|
  <fp group of size infinity on the generators [ s1, s2 ]>
  !gapprompt@gap>| !gapinput@UniversalRotationGroup([3,3,3]);|
  <fp group of size 60 on the generators [ s1, s2, s3 ]>
\end{Verbatim}
 
\subsection{\textcolor{Chapter }{RotaryManiplex}}\label{RotaryManiplex}
\logpage{[ 5, 3, 3 ]}
\hyperdef{L}{X80791E367E43CA25}{}
{
\noindent\textcolor{FuncColor}{$\triangleright$\enspace\texttt{RotaryManiplex({\mdseries\slshape g})\index{RotaryManiplex@\texttt{RotaryManiplex}!for IsGroup}
\label{RotaryManiplex:for IsGroup}
}\hfill{\scriptsize (operation)}}\\
\noindent\textcolor{FuncColor}{$\triangleright$\enspace\texttt{RotaryManiplex({\mdseries\slshape sym})\index{RotaryManiplex@\texttt{RotaryManiplex}!for IsList}
\label{RotaryManiplex:for IsList}
}\hfill{\scriptsize (operation)}}\\
\noindent\textcolor{FuncColor}{$\triangleright$\enspace\texttt{RotaryManiplex({\mdseries\slshape sym, relations})\index{RotaryManiplex@\texttt{RotaryManiplex}!for IsList, IsList}
\label{RotaryManiplex:for IsList, IsList}
}\hfill{\scriptsize (operation)}}\\
\noindent\textcolor{FuncColor}{$\triangleright$\enspace\texttt{RotaryManiplex({\mdseries\slshape sym, words, orders})\index{RotaryManiplex@\texttt{RotaryManiplex}!for IsList, IsList, IsList}
\label{RotaryManiplex:for IsList, IsList, IsList}
}\hfill{\scriptsize (operation)}}\\


 In the first form, given a group g (which should be a string rotation group),
returns the rotary maniplex with that rotation group, where the privileged
generators are those returned by GeneratorsOfGroup(g). This function first
checks whether g is a StringRotationGroup. Use \texttt{RotaryManiplexNC} to bypass that check. 
\begin{Verbatim}[commandchars=!@|,fontsize=\small,frame=single,label=Example]
  !gapprompt@gap>| !gapinput@M := RotaryManiplex(UniversalRotationGroup([3,3]));;|
  !gapprompt@gap>| !gapinput@M = Simplex(3);|
  true
\end{Verbatim}
 

 The second form returns the universal rotary maniplex (in fact, regular
polytope) with Schlafli symbol \mbox{\texttt{\mdseries\slshape sym}}. 
\begin{Verbatim}[commandchars=!@|,fontsize=\small,frame=single,label=Example]
  !gapprompt@gap>| !gapinput@M := RotaryManiplex([4,3]);;|
  !gapprompt@gap>| !gapinput@M = Cube(3);|
  true
\end{Verbatim}
 

 The third form returns the rotary maniplex with the given Schlafli symbol and
with the given relations. The relations are given by a string that refers to
the generators s1, s2, etc. This method automatically calls \texttt{InterpolatedString} on the relations, so you may use \$variable in the relations, and it will be
replaced with the value of \texttt{variable} (but for global variables only). 
\begin{Verbatim}[commandchars=!@|,fontsize=\small,frame=single,label=Example]
  !gapprompt@gap>| !gapinput@M := RotaryManiplex([4,4], "(s2^-1 s1)^6");;|
  !gapprompt@gap>| !gapinput@M = ToroidalMap44([6,0]);|
  true
\end{Verbatim}
 

 The fourth form takes the Schlafli Symbol \mbox{\texttt{\mdseries\slshape sym}}, a list of \mbox{\texttt{\mdseries\slshape words}} in the generators r0 etc, and a list of \mbox{\texttt{\mdseries\slshape orders}}. It returns the rotary maniplex that is the quotient of the universal
maniplex with that Schlalfi Symbol by the relations obtained by setting each \mbox{\texttt{\mdseries\slshape word[i]}} to have order \mbox{\texttt{\mdseries\slshape order[i]}}. This is primarily useful for quickly constructing a family of related
maniplexes. }

 
\begin{Verbatim}[commandchars=!@|,fontsize=\small,frame=single,label=Example]
  !gapprompt@gap>| !gapinput@L := List([1..5], k -> RotaryManiplex([4,4], ["s1 s2^-1"], [k]));;|
  !gapprompt@gap>| !gapinput@List(L, IsPolytopal);|
  [ false, true, true, true, true ]
\end{Verbatim}
 

 In the last two forms, if the option set{\textunderscore}schlafli is set, then
we set the Schlafli symbol to the one given. This may not be the correct
Schlafli symbol, since the relations may cause a collapse, so this should only
be used if you know that the Schlafli symbol is correct. 
\begin{Verbatim}[commandchars=!@|,fontsize=\small,frame=single,label=Example]
  !gapprompt@gap>| !gapinput@M := RotaryManiplex([6,6], "(s1^2 s2^2)^8");;|
  !gapprompt@gap>| !gapinput@SchlafliSymbol(M);|
  #I  Coset table calculation failed -- trying with bigger table limit
  ... eventually give up with CTRL-C
  !gapprompt@gap>| !gapinput@M := RotaryManiplex([6,6], "(s1^2 s2^2)^8" : set_schlafli);;|
  !gapprompt@gap>| !gapinput@SchlafliSymbol(M);|
  [6, 6]
\end{Verbatim}
 

\subsection{\textcolor{Chapter }{EnantiomorphicForm (for IsManiplex)}}
\logpage{[ 5, 3, 4 ]}\nobreak
\hyperdef{L}{X8745361B80626A39}{}
{\noindent\textcolor{FuncColor}{$\triangleright$\enspace\texttt{EnantiomorphicForm({\mdseries\slshape M})\index{EnantiomorphicForm@\texttt{EnantiomorphicForm}!for IsManiplex}
\label{EnantiomorphicForm:for IsManiplex}
}\hfill{\scriptsize (operation)}}\\


 The \emph{enantiomorphic form} of a rotary maniplex is the same maniplex, but where we choose the new base
flag to be one of the flags that is adjacent to the original base flag. If M
is reflexible, then this choice has no effect. Otherwise, if M is chiral, then
the enantiomorphic form gives us a different presentation for the rotation
group. }

 
\begin{Verbatim}[commandchars=!@|,fontsize=\small,frame=single,label=Example]
  !gapprompt@gap>| !gapinput@M := ToroidalMap44([1,2]);;|
  !gapprompt@gap>| !gapinput@g := AutomorphismGroup(M);|
  <fp group of size 20 on the generators [ s1, s2 ]>
  !gapprompt@gap>| !gapinput@RelatorsOfFpGroup(g);|
  [ (s1*s2)^2, s1^4, s2^4, s2^-1*s1*(s2*s1^-1)^2 ]
  !gapprompt@gap>| !gapinput@h := AutomorphismGroup(EnantiomorphicForm(M));|
  <fp group of size 20 on the generators [ s1, s2 ]>
  !gapprompt@gap>| !gapinput@RelatorsOfFpGroup(h);|
  [ (s1*s2)^2, s1^4, s2^4, s2^-1*s1^-1*s2*s1^3*s2*s1 ]
\end{Verbatim}
 }

 }

   
\chapter{\textcolor{Chapter }{Maniplex Properties}}\label{Chapter_Maniplex_Properties}
\logpage{[ 6, 0, 0 ]}
\hyperdef{L}{X831BE6F07FED02F3}{}
{
  
\section{\textcolor{Chapter }{Automorphism group acting on faces and chains}}\label{Chapter_Maniplex_Properties_Section_Automorphism_group_acting_on_faces_and_chains}
\logpage{[ 6, 1, 0 ]}
\hyperdef{L}{X84F0885E8664F785}{}
{
  

\subsection{\textcolor{Chapter }{AutomorphismGroupOnChains (for IsManiplex, IsCollection)}}
\logpage{[ 6, 1, 1 ]}\nobreak
\hyperdef{L}{X84B5DAED82D403DE}{}
{\noindent\textcolor{FuncColor}{$\triangleright$\enspace\texttt{AutomorphismGroupOnChains({\mdseries\slshape M, I})\index{AutomorphismGroupOnChains@\texttt{AutomorphismGroupOnChains}!for IsManiplex, IsCollection}
\label{AutomorphismGroupOnChains:for IsManiplex, IsCollection}
}\hfill{\scriptsize (operation)}}\\
\textbf{\indent Returns:\ }
IsPermGroup 



 Returns a permutation group, representing the action of AutomorphismGroup(\mbox{\texttt{\mdseries\slshape M}}) on the chains of \mbox{\texttt{\mdseries\slshape M}} of type \mbox{\texttt{\mdseries\slshape I}}. }

 
\begin{Verbatim}[commandchars=!@|,fontsize=\small,frame=single,label=Example]
  !gapprompt@gap>| !gapinput@AutomorphismGroupOnChains(HemiCube(3),[0,2]);|
  Group([ (1,2)(3,4)(5,10)(6,9)(7,8)(11,12), (2,6)(3,5)(4,7)(8,11)(10,12), (1,3)(2,4)(6,11)(7,8)
    (9,12) ])
\end{Verbatim}
 

\subsection{\textcolor{Chapter }{AutomorphismGroupOnIFaces (for IsManiplex, IsInt)}}
\logpage{[ 6, 1, 2 ]}\nobreak
\hyperdef{L}{X8549DB2585A989E4}{}
{\noindent\textcolor{FuncColor}{$\triangleright$\enspace\texttt{AutomorphismGroupOnIFaces({\mdseries\slshape M, i})\index{AutomorphismGroupOnIFaces@\texttt{AutomorphismGroupOnIFaces}!for IsManiplex, IsInt}
\label{AutomorphismGroupOnIFaces:for IsManiplex, IsInt}
}\hfill{\scriptsize (operation)}}\\
\textbf{\indent Returns:\ }
IsPermGroup 



 Returns a permutation group, representing the action of AutomorphismGroup(\mbox{\texttt{\mdseries\slshape M}}) on the \mbox{\texttt{\mdseries\slshape i}}-faces of \mbox{\texttt{\mdseries\slshape M}}. }

 
\begin{Verbatim}[commandchars=!@|,fontsize=\small,frame=single,label=Example]
  !gapprompt@gap>| !gapinput@AutomorphismGroupOnIFaces(HemiCube(3),2);|
  Group([ (), (2,3), (1,2) ])
\end{Verbatim}
 

\subsection{\textcolor{Chapter }{AutomorphismGroupOnVertices (for IsManiplex)}}
\logpage{[ 6, 1, 3 ]}\nobreak
\hyperdef{L}{X81972F5187B9C2AF}{}
{\noindent\textcolor{FuncColor}{$\triangleright$\enspace\texttt{AutomorphismGroupOnVertices({\mdseries\slshape M})\index{AutomorphismGroupOnVertices@\texttt{AutomorphismGroupOnVertices}!for IsManiplex}
\label{AutomorphismGroupOnVertices:for IsManiplex}
}\hfill{\scriptsize (attribute)}}\\
\textbf{\indent Returns:\ }
IsPermGroup 



 Returns a permutation group, representing the action of AutomorphismGroup(\mbox{\texttt{\mdseries\slshape M}}) on the vertices of \mbox{\texttt{\mdseries\slshape M}}. }

 
\begin{Verbatim}[commandchars=!@|,fontsize=\small,frame=single,label=Example]
  !gapprompt@gap>| !gapinput@AutomorphismGroupOnVertices(HemiCube(4));|
  Group([ (1,2)(3,4)(5,6)(7,8), (2,3)(6,8), (3,5)(4,6), (5,7)(6,8) ])
\end{Verbatim}
 

\subsection{\textcolor{Chapter }{AutomorphismGroupOnEdges (for IsManiplex)}}
\logpage{[ 6, 1, 4 ]}\nobreak
\hyperdef{L}{X86CBC88A8718D7E4}{}
{\noindent\textcolor{FuncColor}{$\triangleright$\enspace\texttt{AutomorphismGroupOnEdges({\mdseries\slshape M})\index{AutomorphismGroupOnEdges@\texttt{AutomorphismGroupOnEdges}!for IsManiplex}
\label{AutomorphismGroupOnEdges:for IsManiplex}
}\hfill{\scriptsize (attribute)}}\\
\textbf{\indent Returns:\ }
IsPermGroup 



 Returns a permutation group, representing the action of AutomorphismGroup(\mbox{\texttt{\mdseries\slshape M}}) on the edges of \mbox{\texttt{\mdseries\slshape M}}. }

 
\begin{Verbatim}[commandchars=!@|,fontsize=\small,frame=single,label=Example]
  !gapprompt@gap>| !gapinput@AutomorphismGroupOnEdges(Simplex(4));|
  Group([ (2,5)(3,6)(4,7), (1,2)(6,8)(7,9), (2,3)(5,6)(9,10), (3,4)(6,7)(8,9) ])
\end{Verbatim}
 

\subsection{\textcolor{Chapter }{AutomorphismGroupOnFacets (for IsManiplex)}}
\logpage{[ 6, 1, 5 ]}\nobreak
\hyperdef{L}{X82726AF27C1E00D6}{}
{\noindent\textcolor{FuncColor}{$\triangleright$\enspace\texttt{AutomorphismGroupOnFacets({\mdseries\slshape M})\index{AutomorphismGroupOnFacets@\texttt{AutomorphismGroupOnFacets}!for IsManiplex}
\label{AutomorphismGroupOnFacets:for IsManiplex}
}\hfill{\scriptsize (attribute)}}\\
\textbf{\indent Returns:\ }
IsPermGroup 



 Returns a permutation group, representing the action of AutomorphismGroup(\mbox{\texttt{\mdseries\slshape M}}) on the facets of \mbox{\texttt{\mdseries\slshape M}}. }

 
\begin{Verbatim}[commandchars=!@|,fontsize=\small,frame=single,label=Example]
  !gapprompt@gap>| !gapinput@AutomorphismGroupOnFacets(HemiCube(5));|
  Group([ (), (4,5), (3,4), (2,3), (1,2) ])
\end{Verbatim}
 }

 
\section{\textcolor{Chapter }{Number of orbits and transitivity}}\label{Chapter_Maniplex_Properties_Section_Number_of_orbits_and_transitivity}
\logpage{[ 6, 2, 0 ]}
\hyperdef{L}{X7A07E55F7D617911}{}
{
  

\subsection{\textcolor{Chapter }{NumberOfChainOrbits (for IsManiplex, IsCollection)}}
\logpage{[ 6, 2, 1 ]}\nobreak
\hyperdef{L}{X8423C2647CFF8C2E}{}
{\noindent\textcolor{FuncColor}{$\triangleright$\enspace\texttt{NumberOfChainOrbits({\mdseries\slshape M, I})\index{NumberOfChainOrbits@\texttt{NumberOfChainOrbits}!for IsManiplex, IsCollection}
\label{NumberOfChainOrbits:for IsManiplex, IsCollection}
}\hfill{\scriptsize (operation)}}\\
\textbf{\indent Returns:\ }
IsInt 



 Returns the number of orbits of chains of type \mbox{\texttt{\mdseries\slshape I}} under the action of AutomorphismGroup(\mbox{\texttt{\mdseries\slshape M}}). }

 
\begin{Verbatim}[commandchars=!@|,fontsize=\small,frame=single,label=Example]
  !gapprompt@gap>| !gapinput@NumberOfChainOrbits(Cuboctahedron(),[0,2]);|
  2
\end{Verbatim}
 

\subsection{\textcolor{Chapter }{NumberOfIFaceOrbits (for IsManiplex, IsInt)}}
\logpage{[ 6, 2, 2 ]}\nobreak
\hyperdef{L}{X82D959517BC24461}{}
{\noindent\textcolor{FuncColor}{$\triangleright$\enspace\texttt{NumberOfIFaceOrbits({\mdseries\slshape M, i})\index{NumberOfIFaceOrbits@\texttt{NumberOfIFaceOrbits}!for IsManiplex, IsInt}
\label{NumberOfIFaceOrbits:for IsManiplex, IsInt}
}\hfill{\scriptsize (operation)}}\\
\textbf{\indent Returns:\ }
IsInt 



 Returns the number of orbits of \mbox{\texttt{\mdseries\slshape i}}-faces under the action of AutomorphismGroup(\mbox{\texttt{\mdseries\slshape M}}). }

 
\begin{Verbatim}[commandchars=!@|,fontsize=\small,frame=single,label=Example]
  !gapprompt@gap>| !gapinput@NumberOfIFaceOrbits(SnubDodecahedron(),2);|
  3
\end{Verbatim}
 

\subsection{\textcolor{Chapter }{NumberOfVertexOrbits (for IsManiplex)}}
\logpage{[ 6, 2, 3 ]}\nobreak
\hyperdef{L}{X86A6A073804A5780}{}
{\noindent\textcolor{FuncColor}{$\triangleright$\enspace\texttt{NumberOfVertexOrbits({\mdseries\slshape M})\index{NumberOfVertexOrbits@\texttt{NumberOfVertexOrbits}!for IsManiplex}
\label{NumberOfVertexOrbits:for IsManiplex}
}\hfill{\scriptsize (attribute)}}\\
\textbf{\indent Returns:\ }
IsInt 



 Returns the number of orbits of vertices under the action of
AutomorphismGroup(\mbox{\texttt{\mdseries\slshape M}}). }

 
\begin{Verbatim}[commandchars=!@|,fontsize=\small,frame=single,label=Example]
  !gapprompt@gap>| !gapinput@NumberOfVertexOrbits(Dual(SnubDodecahedron()));|
  3
\end{Verbatim}
 

\subsection{\textcolor{Chapter }{NumberOfEdgeOrbits (for IsManiplex)}}
\logpage{[ 6, 2, 4 ]}\nobreak
\hyperdef{L}{X7A41AD2B7A606FB2}{}
{\noindent\textcolor{FuncColor}{$\triangleright$\enspace\texttt{NumberOfEdgeOrbits({\mdseries\slshape M})\index{NumberOfEdgeOrbits@\texttt{NumberOfEdgeOrbits}!for IsManiplex}
\label{NumberOfEdgeOrbits:for IsManiplex}
}\hfill{\scriptsize (attribute)}}\\
\textbf{\indent Returns:\ }
IsInt 



 Returns the number of orbits of edges under the action of AutomorphismGroup(\mbox{\texttt{\mdseries\slshape M}}). }

 
\begin{Verbatim}[commandchars=!@|,fontsize=\small,frame=single,label=Example]
  !gapprompt@gap>| !gapinput@NumberOfEdgeOrbits(SnubDodecahedron());|
  3
\end{Verbatim}
 

\subsection{\textcolor{Chapter }{NumberOfFacetOrbits (for IsManiplex)}}
\logpage{[ 6, 2, 5 ]}\nobreak
\hyperdef{L}{X81C6FD5078C637E6}{}
{\noindent\textcolor{FuncColor}{$\triangleright$\enspace\texttt{NumberOfFacetOrbits({\mdseries\slshape M})\index{NumberOfFacetOrbits@\texttt{NumberOfFacetOrbits}!for IsManiplex}
\label{NumberOfFacetOrbits:for IsManiplex}
}\hfill{\scriptsize (attribute)}}\\
\textbf{\indent Returns:\ }
IsInt 



 Returns the number of orbits of facets under the action of AutomorphismGroup(\mbox{\texttt{\mdseries\slshape M}}). }

 
\begin{Verbatim}[commandchars=!@|,fontsize=\small,frame=single,label=Example]
  !gapprompt@gap>| !gapinput@NumberOfFacetOrbits(SnubCube());|
  3
\end{Verbatim}
 

\subsection{\textcolor{Chapter }{IsChainTransitive (for IsManiplex, IsCollection)}}
\logpage{[ 6, 2, 6 ]}\nobreak
\hyperdef{L}{X7A2326997FB2CAA8}{}
{\noindent\textcolor{FuncColor}{$\triangleright$\enspace\texttt{IsChainTransitive({\mdseries\slshape M, I})\index{IsChainTransitive@\texttt{IsChainTransitive}!for IsManiplex, IsCollection}
\label{IsChainTransitive:for IsManiplex, IsCollection}
}\hfill{\scriptsize (operation)}}\\
\textbf{\indent Returns:\ }
IsBool 



 Determines whether the action of AutomorphismGroup(\mbox{\texttt{\mdseries\slshape M}}) on chains of type \mbox{\texttt{\mdseries\slshape I}} is transitive. }

 
\begin{Verbatim}[commandchars=!@|,fontsize=\small,frame=single,label=Example]
  !gapprompt@gap>| !gapinput@IsChainTransitive(SmallRhombicuboctahedron(),[0,2]);|
  false
  !gapprompt@gap>| !gapinput@IsChainTransitive(SmallRhombicuboctahedron(),[0,1]);|
  false
  !gapprompt@gap>| !gapinput@IsChainTransitive(Cuboctahedron(),[0,1]);|
  true
\end{Verbatim}
 

\subsection{\textcolor{Chapter }{IsIFaceTransitive (for IsManiplex, IsInt)}}
\logpage{[ 6, 2, 7 ]}\nobreak
\hyperdef{L}{X80F9303F7A272F20}{}
{\noindent\textcolor{FuncColor}{$\triangleright$\enspace\texttt{IsIFaceTransitive({\mdseries\slshape M, i})\index{IsIFaceTransitive@\texttt{IsIFaceTransitive}!for IsManiplex, IsInt}
\label{IsIFaceTransitive:for IsManiplex, IsInt}
}\hfill{\scriptsize (operation)}}\\
\textbf{\indent Returns:\ }
IsBool 



 Determines whether the action of AutomorphismGroup(\mbox{\texttt{\mdseries\slshape M}}) on \mbox{\texttt{\mdseries\slshape i}}-faces is transitive. }

 
\begin{Verbatim}[commandchars=!@|,fontsize=\small,frame=single,label=Example]
  !gapprompt@gap>| !gapinput@IsIFaceTransitive(Cuboctahedron(),1);|
  true
\end{Verbatim}
 

\subsection{\textcolor{Chapter }{IsVertexTransitive (for IsManiplex)}}
\logpage{[ 6, 2, 8 ]}\nobreak
\hyperdef{L}{X7E29A16E7B724836}{}
{\noindent\textcolor{FuncColor}{$\triangleright$\enspace\texttt{IsVertexTransitive({\mdseries\slshape M})\index{IsVertexTransitive@\texttt{IsVertexTransitive}!for IsManiplex}
\label{IsVertexTransitive:for IsManiplex}
}\hfill{\scriptsize (property)}}\\
\textbf{\indent Returns:\ }
IsBool 



 Determines whether the action of AutomorphismGroup(\mbox{\texttt{\mdseries\slshape M}}) on vertices is transitive. }

 
\begin{Verbatim}[commandchars=!@|,fontsize=\small,frame=single,label=Example]
  !gapprompt@gap>| !gapinput@IsVertexTransitive(Bk2l(4,5));|
  true
\end{Verbatim}
 

\subsection{\textcolor{Chapter }{IsEdgeTransitive (for IsManiplex)}}
\logpage{[ 6, 2, 9 ]}\nobreak
\hyperdef{L}{X7CA8A3A17CB62B51}{}
{\noindent\textcolor{FuncColor}{$\triangleright$\enspace\texttt{IsEdgeTransitive({\mdseries\slshape M})\index{IsEdgeTransitive@\texttt{IsEdgeTransitive}!for IsManiplex}
\label{IsEdgeTransitive:for IsManiplex}
}\hfill{\scriptsize (property)}}\\
\textbf{\indent Returns:\ }
IsBool 



 Determines whether the action of AutomorphismGroup(\mbox{\texttt{\mdseries\slshape M}}) on edges is transitive. }

 
\begin{Verbatim}[commandchars=!@|,fontsize=\small,frame=single,label=Example]
  !gapprompt@gap>| !gapinput@IsEdgeTransitive(Prism(Simplex(3)));|
  false
\end{Verbatim}
 

\subsection{\textcolor{Chapter }{IsFacetTransitive (for IsManiplex)}}
\logpage{[ 6, 2, 10 ]}\nobreak
\hyperdef{L}{X85436C5C84C4031E}{}
{\noindent\textcolor{FuncColor}{$\triangleright$\enspace\texttt{IsFacetTransitive({\mdseries\slshape M})\index{IsFacetTransitive@\texttt{IsFacetTransitive}!for IsManiplex}
\label{IsFacetTransitive:for IsManiplex}
}\hfill{\scriptsize (property)}}\\
\textbf{\indent Returns:\ }
IsBool 



 Determines whether the action of AutomorphismGroup(\mbox{\texttt{\mdseries\slshape M}}) on facets is transitive. }

 
\begin{Verbatim}[commandchars=!@|,fontsize=\small,frame=single,label=Example]
  !gapprompt@gap>| !gapinput@IsFacetTransitive(Prism(HemiCube(3)));|
  false
\end{Verbatim}
 

\subsection{\textcolor{Chapter }{IsFullyTransitive (for IsManiplex)}}
\logpage{[ 6, 2, 11 ]}\nobreak
\hyperdef{L}{X7B898E19857B356F}{}
{\noindent\textcolor{FuncColor}{$\triangleright$\enspace\texttt{IsFullyTransitive({\mdseries\slshape M})\index{IsFullyTransitive@\texttt{IsFullyTransitive}!for IsManiplex}
\label{IsFullyTransitive:for IsManiplex}
}\hfill{\scriptsize (property)}}\\
\textbf{\indent Returns:\ }
IsBool 



 Determines whether the action of AutomorphismGroup(\mbox{\texttt{\mdseries\slshape M}}) on i-faces is transitive for every i. }

 
\begin{Verbatim}[commandchars=!@|,fontsize=\small,frame=single,label=Example]
  !gapprompt@gap>| !gapinput@IsFullyTransitive(SmallRhombicuboctahedron());|
  false
  !gapprompt@gap>| !gapinput@IsFullyTransitive(Bk2l(4,5));|
  true
\end{Verbatim}
 }

 
\section{\textcolor{Chapter }{Flag orbits}}\label{Chapter_Maniplex_Properties_Section_Flag_orbits}
\logpage{[ 6, 3, 0 ]}
\hyperdef{L}{X7BE5F0217862DBEF}{}
{
  

\subsection{\textcolor{Chapter }{Flags (for IsPremaniplex)}}
\logpage{[ 6, 3, 1 ]}\nobreak
\hyperdef{L}{X8668368E81B4B122}{}
{\noindent\textcolor{FuncColor}{$\triangleright$\enspace\texttt{Flags({\mdseries\slshape M})\index{Flags@\texttt{Flags}!for IsPremaniplex}
\label{Flags:for IsPremaniplex}
}\hfill{\scriptsize (attribute)}}\\
\textbf{\indent Returns:\ }
IsList 



 The list of flags of the premaniplex \mbox{\texttt{\mdseries\slshape M}}. }

 
\begin{Verbatim}[commandchars=!@|,fontsize=\small,frame=single,label=Example]
  !gapprompt@gap>| !gapinput@Flags(Pgon(5));|
  [ 1, 2, 3, 4, 5, 6, 7, 8, 9, 10 ]
  !gapprompt@gap>| !gapinput@M := Maniplex(Group((3,4)(5,6)(7,8)(9,10), (3,6)(4,5)(7,10)(8,9), (3,7)(4,8)(5,9)(6,10)));;|
  !gapprompt@gap>| !gapinput@Flags(M);|
  [ 3, 4, 5, 6, 7, 8, 9, 10 ]
\end{Verbatim}
 

\subsection{\textcolor{Chapter }{BaseFlag (for IsPremaniplex)}}
\logpage{[ 6, 3, 2 ]}\nobreak
\hyperdef{L}{X795A4A60852AF9A5}{}
{\noindent\textcolor{FuncColor}{$\triangleright$\enspace\texttt{BaseFlag({\mdseries\slshape M})\index{BaseFlag@\texttt{BaseFlag}!for IsPremaniplex}
\label{BaseFlag:for IsPremaniplex}
}\hfill{\scriptsize (attribute)}}\\
\textbf{\indent Returns:\ }
IsObject 



 The base flag of the premaniplex \mbox{\texttt{\mdseries\slshape M}}. By default, when the set of flags is a set of positive integers, the base
flag is the minimum of the set of flags. }

 
\begin{Verbatim}[commandchars=!@|,fontsize=\small,frame=single,label=Example]
  !gapprompt@gap>| !gapinput@BaseFlag(Cube(3));|
  1
  !gapprompt@gap>| !gapinput@M := Maniplex(Group((3,4)(5,6)(7,8)(9,10), (3,6)(4,5)(7,10)(8,9), (3,7)(4,8)(5,9)(6,10)));;|
  !gapprompt@gap>| !gapinput@BaseFlag(M);|
  3
\end{Verbatim}
 

\subsection{\textcolor{Chapter }{SymmetryTypeGraph (for IsPremaniplex)}}
\logpage{[ 6, 3, 3 ]}\nobreak
\hyperdef{L}{X87D5AE5F845BD88C}{}
{\noindent\textcolor{FuncColor}{$\triangleright$\enspace\texttt{SymmetryTypeGraph({\mdseries\slshape M[, A]})\index{SymmetryTypeGraph@\texttt{SymmetryTypeGraph}!for IsPremaniplex}
\label{SymmetryTypeGraph:for IsPremaniplex}
}\hfill{\scriptsize (attribute)}}\\
\textbf{\indent Returns:\ }
IsPremaniplex 



 Returns the Symmetry Type Graph of the premaniplex \mbox{\texttt{\mdseries\slshape M}} with respect to the subgroup \mbox{\texttt{\mdseries\slshape A}} of the automorphism group; that is, the quotient of the flag graph of \mbox{\texttt{\mdseries\slshape M}} by \mbox{\texttt{\mdseries\slshape A}}. If \mbox{\texttt{\mdseries\slshape A}} is not included, then returns the Symmetry Type Graph relative to the whole
automorphism group of \mbox{\texttt{\mdseries\slshape M}}. }

 
\begin{Verbatim}[commandchars=!@|,fontsize=\small,frame=single,label=Example]
  !gapprompt@gap>| !gapinput@SymmetryTypeGraph(Prism(Simplex(3)));|
  Edge labeled graph with 4 vertices, and edge labels [ 0, 1, 2, 3 ]
  !gapprompt@gap>| !gapinput@M:=Cube(3);;|
  !gapprompt@gap>| !gapinput@A:=AutomorphismGroupOnFlags(M);;|
  !gapprompt@gap>| !gapinput@B:=Group(A.1,A.2*A.3);;|
  !gapprompt@gap>| !gapinput@SymmetryTypeGraph(M,B);|
  Edge labeled graph with 2 vertices, and edge labels [ 0, 1, 2 ]
\end{Verbatim}
 

\subsection{\textcolor{Chapter }{NumberOfFlagOrbits (for IsPremaniplex)}}
\logpage{[ 6, 3, 4 ]}\nobreak
\hyperdef{L}{X7C14C6417F42D94D}{}
{\noindent\textcolor{FuncColor}{$\triangleright$\enspace\texttt{NumberOfFlagOrbits({\mdseries\slshape M})\index{NumberOfFlagOrbits@\texttt{NumberOfFlagOrbits}!for IsPremaniplex}
\label{NumberOfFlagOrbits:for IsPremaniplex}
}\hfill{\scriptsize (attribute)}}\\


 Returns the number of orbits of the automorphism group of \mbox{\texttt{\mdseries\slshape M}} on its flags. }

 
\begin{Verbatim}[commandchars=!@|,fontsize=\small,frame=single,label=Example]
  !gapprompt@gap>| !gapinput@NumberOfFlagOrbits(Prism(Simplex(3)));|
  4
\end{Verbatim}
 

\subsection{\textcolor{Chapter }{FlagOrbitRepresentatives (for IsPremaniplex)}}
\logpage{[ 6, 3, 5 ]}\nobreak
\hyperdef{L}{X82AFA2FE809AC358}{}
{\noindent\textcolor{FuncColor}{$\triangleright$\enspace\texttt{FlagOrbitRepresentatives({\mdseries\slshape M})\index{FlagOrbitRepresentatives@\texttt{FlagOrbitRepresentatives}!for IsPremaniplex}
\label{FlagOrbitRepresentatives:for IsPremaniplex}
}\hfill{\scriptsize (attribute)}}\\


 Returns one flag from each orbit under the action of AutomorphismGroup(\mbox{\texttt{\mdseries\slshape M}}). }

 
\begin{Verbatim}[commandchars=!@|,fontsize=\small,frame=single,label=Example]
  !gapprompt@gap>| !gapinput@FlagOrbitRepresentatives(Prism(Simplex(3)));|
  [ 1, 49, 97, 145 ]
\end{Verbatim}
 

\subsection{\textcolor{Chapter }{FlagOrbitsStabilizer (for IsPremaniplex)}}
\logpage{[ 6, 3, 6 ]}\nobreak
\hyperdef{L}{X7FC9977380733F65}{}
{\noindent\textcolor{FuncColor}{$\triangleright$\enspace\texttt{FlagOrbitsStabilizer({\mdseries\slshape M})\index{FlagOrbitsStabilizer@\texttt{FlagOrbitsStabilizer}!for IsPremaniplex}
\label{FlagOrbitsStabilizer:for IsPremaniplex}
}\hfill{\scriptsize (attribute)}}\\
\textbf{\indent Returns:\ }
g 



 Returns the subgroup of the connection group that preserves the flag orbits
under the action of the automorphism group. }

 
\begin{Verbatim}[commandchars=!@|,fontsize=\small,frame=single,label=Example]
  !gapprompt@gap>| !gapinput@m:=Prism(Dodecahedron());|
  Prism(Dodecahedron())
  !gapprompt@gap>| !gapinput@s:=FlagOrbitsStabilizer(m);|
  <permutation group of size 207360000 with 12 generators>
  !gapprompt@gap>| !gapinput@IsSubgroup(ConnectionGroup(m),s);|
  true
  !gapprompt@gap>| !gapinput@AsSet(Orbit(AutomorphismGroupOnFlags(m),1))=AsSet(Orbit(s,1));|
  true
\end{Verbatim}
 

\subsection{\textcolor{Chapter }{IsReflexible (for IsPremaniplex)}}
\logpage{[ 6, 3, 7 ]}\nobreak
\hyperdef{L}{X839E8D4783F91690}{}
{\noindent\textcolor{FuncColor}{$\triangleright$\enspace\texttt{IsReflexible({\mdseries\slshape M})\index{IsReflexible@\texttt{IsReflexible}!for IsPremaniplex}
\label{IsReflexible:for IsPremaniplex}
}\hfill{\scriptsize (property)}}\\
\textbf{\indent Returns:\ }
Whether the premaniplex \mbox{\texttt{\mdseries\slshape M}} is reflexible (has one flag orbit). 



 

 }

 
\begin{Verbatim}[commandchars=!@|,fontsize=\small,frame=single,label=Example]
  !gapprompt@gap>| !gapinput@IsReflexible(Epsilonk(6));|
  true
\end{Verbatim}
 

\subsection{\textcolor{Chapter }{IsChiral (for IsPremaniplex)}}
\logpage{[ 6, 3, 8 ]}\nobreak
\hyperdef{L}{X7F0108D17B523A0C}{}
{\noindent\textcolor{FuncColor}{$\triangleright$\enspace\texttt{IsChiral({\mdseries\slshape M})\index{IsChiral@\texttt{IsChiral}!for IsPremaniplex}
\label{IsChiral:for IsPremaniplex}
}\hfill{\scriptsize (property)}}\\
\textbf{\indent Returns:\ }
Whether the premaniplex \mbox{\texttt{\mdseries\slshape M}} is chiral. 



 

 }

 
\begin{Verbatim}[commandchars=!@|,fontsize=\small,frame=single,label=Example]
  !gapprompt@gap>| !gapinput@IsChiral(ToroidalMap44([2,3]));|
  true
\end{Verbatim}
 

\subsection{\textcolor{Chapter }{IsRotary (for IsPremaniplex)}}
\logpage{[ 6, 3, 9 ]}\nobreak
\hyperdef{L}{X8718754784A4FCB2}{}
{\noindent\textcolor{FuncColor}{$\triangleright$\enspace\texttt{IsRotary({\mdseries\slshape M})\index{IsRotary@\texttt{IsRotary}!for IsPremaniplex}
\label{IsRotary:for IsPremaniplex}
}\hfill{\scriptsize (property)}}\\
\textbf{\indent Returns:\ }
Whether the maniplex \mbox{\texttt{\mdseries\slshape M}} is rotary; i.e., whether it is either reflexible or chiral. 



 

 }

 
\begin{Verbatim}[commandchars=!@|,fontsize=\small,frame=single,label=Example]
  !gapprompt@gap>| !gapinput@IsRotary(ToroidalMap44([3,5]));|
  true
\end{Verbatim}
 

\subsection{\textcolor{Chapter }{FlagOrbits (for IsPremaniplex)}}
\logpage{[ 6, 3, 10 ]}\nobreak
\hyperdef{L}{X84E84D0E84BC7F59}{}
{\noindent\textcolor{FuncColor}{$\triangleright$\enspace\texttt{FlagOrbits({\mdseries\slshape M})\index{FlagOrbits@\texttt{FlagOrbits}!for IsPremaniplex}
\label{FlagOrbits:for IsPremaniplex}
}\hfill{\scriptsize (attribute)}}\\


 Returns a list of lists of flags, representing the orbits of flags under the
action of AutomorphismGroup(\mbox{\texttt{\mdseries\slshape M}}). }

 
\begin{Verbatim}[commandchars=!@|,fontsize=\small,frame=single,label=Example]
  !gapprompt@gap>| !gapinput@FlagOrbits(ToroidalMap44([3,2]));|
  [ [ 1, 9, 7, 33, 15, 63, 5, 65, 39, 23, 13, 71, 61, 101, 3, 89, 47, 37, 95, 21, 11, 79, 69, 29, 59, 77, 99, 51, 49, 55, 45, 35, 103, 93, 19, 83, 87, 67, 85, 27, 57, 75, 91, 97, 43, 81, 53, 31, 17, 25, 73, 41 ], 
    [ 2, 10, 8, 34, 16, 64, 6, 66, 40, 24, 14, 72, 62, 102, 4, 90, 48, 38, 96, 22, 12, 80, 70, 30, 60, 78, 100, 52, 50, 56, 46, 36, 104, 94, 20, 84, 88, 68, 86, 28, 58, 76, 92, 98, 44, 82, 54, 32, 18, 26, 74, 42 ] ]
\end{Verbatim}
 }

 
\section{\textcolor{Chapter }{Orientability}}\label{Chapter_Maniplex_Properties_Section_Orientability}
\logpage{[ 6, 4, 0 ]}
\hyperdef{L}{X861E4BAD800B2785}{}
{
  

\subsection{\textcolor{Chapter }{IsOrientable (for IsManiplex)}}
\logpage{[ 6, 4, 1 ]}\nobreak
\hyperdef{L}{X7DCAF9D27F36EBD3}{}
{\noindent\textcolor{FuncColor}{$\triangleright$\enspace\texttt{IsOrientable({\mdseries\slshape M})\index{IsOrientable@\texttt{IsOrientable}!for IsManiplex}
\label{IsOrientable:for IsManiplex}
}\hfill{\scriptsize (property)}}\\
\textbf{\indent Returns:\ }
\texttt{true} or \texttt{false} 



 A maniplex is orientable if its flag graph is bipartite. }

 
\begin{Verbatim}[commandchars=!@|,fontsize=\small,frame=single,label=Example]
  !gapprompt@gap>| !gapinput@IsOrientable(HemiCube(3));|
  false
  !gapprompt@gap>| !gapinput@IsOrientable(Cube(3));|
  true
\end{Verbatim}
 

\subsection{\textcolor{Chapter }{IsIOrientable (for IsManiplex, IsList)}}
\logpage{[ 6, 4, 2 ]}\nobreak
\hyperdef{L}{X79F849DD7F778616}{}
{\noindent\textcolor{FuncColor}{$\triangleright$\enspace\texttt{IsIOrientable({\mdseries\slshape M, I})\index{IsIOrientable@\texttt{IsIOrientable}!for IsManiplex, IsList}
\label{IsIOrientable:for IsManiplex, IsList}
}\hfill{\scriptsize (operation)}}\\


 For a subset I of \texttt{\symbol{123}}0, ..., n-1\texttt{\symbol{125}}, a
maniplex is I-orientable if every closed path in its flag graph contains an
even number of edges with colors in I. }

 
\begin{Verbatim}[commandchars=!@|,fontsize=\small,frame=single,label=Example]
  !gapprompt@gap>| !gapinput@IsIOrientable(HemiCube(3),[1,2]);|
  true
\end{Verbatim}
 

\subsection{\textcolor{Chapter }{IsVertexBipartite (for IsManiplex)}}
\logpage{[ 6, 4, 3 ]}\nobreak
\hyperdef{L}{X877609AB7ABD24AC}{}
{\noindent\textcolor{FuncColor}{$\triangleright$\enspace\texttt{IsVertexBipartite({\mdseries\slshape M})\index{IsVertexBipartite@\texttt{IsVertexBipartite}!for IsManiplex}
\label{IsVertexBipartite:for IsManiplex}
}\hfill{\scriptsize (property)}}\\
\textbf{\indent Returns:\ }
\texttt{true} or \texttt{false} 



 A maniplex is vertex-bipartite if its 1-skeleton is bipartite. This is
equivalent to being I-orientable for I =
\texttt{\symbol{123}}0\texttt{\symbol{125}}. }

 
\begin{Verbatim}[commandchars=!@|,fontsize=\small,frame=single,label=Example]
  !gapprompt@gap>| !gapinput@IsVertexBipartite(HemiCube(4));|
  true
\end{Verbatim}
 

\subsection{\textcolor{Chapter }{IsFacetBipartite (for IsManiplex)}}
\logpage{[ 6, 4, 4 ]}\nobreak
\hyperdef{L}{X7BD9E8A87D6E8FAB}{}
{\noindent\textcolor{FuncColor}{$\triangleright$\enspace\texttt{IsFacetBipartite({\mdseries\slshape M})\index{IsFacetBipartite@\texttt{IsFacetBipartite}!for IsManiplex}
\label{IsFacetBipartite:for IsManiplex}
}\hfill{\scriptsize (property)}}\\
\textbf{\indent Returns:\ }
\texttt{true} or \texttt{false} 



 A maniplex is facet-bipartite if the 1-skeleton of its dual is bipartite. This
is equivalent to being I-orientable for I =
\texttt{\symbol{123}}n-1\texttt{\symbol{125}}. }

 
\begin{Verbatim}[commandchars=!@|,fontsize=\small,frame=single,label=Example]
  !gapprompt@gap>| !gapinput@IsFacetBipartite(HemiCube(4));|
  false
\end{Verbatim}
 

\subsection{\textcolor{Chapter }{OrientableCover (for IsManiplex)}}
\logpage{[ 6, 4, 5 ]}\nobreak
\hyperdef{L}{X7BAA22327D298902}{}
{\noindent\textcolor{FuncColor}{$\triangleright$\enspace\texttt{OrientableCover({\mdseries\slshape M})\index{OrientableCover@\texttt{OrientableCover}!for IsManiplex}
\label{OrientableCover:for IsManiplex}
}\hfill{\scriptsize (attribute)}}\\


 Returns the minimal \emph{orientable cover} of the maniplex \mbox{\texttt{\mdseries\slshape M}}. }

 
\begin{Verbatim}[commandchars=!@|,fontsize=\small,frame=single,label=Example]
  !gapprompt@gap>| !gapinput@OrientableCover(HemiCube(3))=Cube(3);|
  true
\end{Verbatim}
 

\subsection{\textcolor{Chapter }{IOrientableCover (for IsManiplex, IsList)}}
\logpage{[ 6, 4, 6 ]}\nobreak
\hyperdef{L}{X81EF6062840E85AF}{}
{\noindent\textcolor{FuncColor}{$\triangleright$\enspace\texttt{IOrientableCover({\mdseries\slshape M, I})\index{IOrientableCover@\texttt{IOrientableCover}!for IsManiplex, IsList}
\label{IOrientableCover:for IsManiplex, IsList}
}\hfill{\scriptsize (operation)}}\\


 Returns the minimal \emph{I-orientable cover} of the maniplex \mbox{\texttt{\mdseries\slshape M}}. }

 
\begin{Verbatim}[commandchars=!@|,fontsize=\small,frame=single,label=Example]
  !gapprompt@gap>| !gapinput@SchlafliSymbol(IOrientableCover(Cube(3), [2]));|
  [ 4, 6 ]
\end{Verbatim}
 }

 
\section{\textcolor{Chapter }{Faithfulness}}\label{Chapter_Maniplex_Properties_Section_Faithfulness}
\logpage{[ 6, 5, 0 ]}
\hyperdef{L}{X83355D207F38F997}{}
{
  

\subsection{\textcolor{Chapter }{IsVertexFaithful (for IsManiplex)}}
\logpage{[ 6, 5, 1 ]}\nobreak
\hyperdef{L}{X7FFD0ADD7A2F6044}{}
{\noindent\textcolor{FuncColor}{$\triangleright$\enspace\texttt{IsVertexFaithful({\mdseries\slshape M})\index{IsVertexFaithful@\texttt{IsVertexFaithful}!for IsManiplex}
\label{IsVertexFaithful:for IsManiplex}
}\hfill{\scriptsize (property)}}\\
\textbf{\indent Returns:\ }
\texttt{true} or \texttt{false} 



 Returns whether the reflexible maniplex \mbox{\texttt{\mdseries\slshape M}} is vertex-faithful; i.e., whether the action of the automorphism group on the
vertices is faithful. }

 
\begin{Verbatim}[commandchars=!@|,fontsize=\small,frame=single,label=Example]
  !gapprompt@gap>| !gapinput@IsVertexFaithful(HemiCube(3));|
  true
\end{Verbatim}
 

\subsection{\textcolor{Chapter }{IsFacetFaithful (for IsManiplex)}}
\logpage{[ 6, 5, 2 ]}\nobreak
\hyperdef{L}{X7BB02182844EA5EB}{}
{\noindent\textcolor{FuncColor}{$\triangleright$\enspace\texttt{IsFacetFaithful({\mdseries\slshape M})\index{IsFacetFaithful@\texttt{IsFacetFaithful}!for IsManiplex}
\label{IsFacetFaithful:for IsManiplex}
}\hfill{\scriptsize (property)}}\\
\textbf{\indent Returns:\ }
\texttt{true} or \texttt{false} 



 Returns whether the reflexible maniplex \mbox{\texttt{\mdseries\slshape M}} is facet-faithful; i.e., whether the action of the automorphism group on the
facets is faithful. }

 
\begin{Verbatim}[commandchars=!@|,fontsize=\small,frame=single,label=Example]
  !gapprompt@gap>| !gapinput@IsFacetFaithful(HemiCube(3));|
  false
  !gapprompt@gap>| !gapinput@IsFacetFaithful(Cube(3));|
  true
\end{Verbatim}
 

\subsection{\textcolor{Chapter }{MaxVertexFaithfulQuotient (for IsManiplex)}}
\logpage{[ 6, 5, 3 ]}\nobreak
\hyperdef{L}{X7F2F8EEE8663AB89}{}
{\noindent\textcolor{FuncColor}{$\triangleright$\enspace\texttt{MaxVertexFaithfulQuotient({\mdseries\slshape M})\index{MaxVertexFaithfulQuotient@\texttt{MaxVertexFaithfulQuotient}!for IsManiplex}
\label{MaxVertexFaithfulQuotient:for IsManiplex}
}\hfill{\scriptsize (operation)}}\\
\textbf{\indent Returns:\ }
Q 



 Returns the maximal vertex-faithful reflexible maniplex covered by \mbox{\texttt{\mdseries\slshape M}}. }

 
\begin{Verbatim}[commandchars=!@|,fontsize=\small,frame=single,label=Example]
  !gapprompt@gap>| !gapinput@MaxVertexFaithfulQuotient(HemiCrossPolytope(3));|
  reflexible 3-maniplex
  !gapprompt@gap>| !gapinput@SchlafliSymbol(last);|
  [ 3, 2 ]
\end{Verbatim}
 

\subsection{\textcolor{Chapter }{SatisfiesWeakPathIntersectionProperty (for IsManiplex)}}
\logpage{[ 6, 5, 4 ]}\nobreak
\hyperdef{L}{X7F788D2E7F41CD69}{}
{\noindent\textcolor{FuncColor}{$\triangleright$\enspace\texttt{SatisfiesWeakPathIntersectionProperty({\mdseries\slshape M})\index{SatisfiesWeakPathIntersectionProperty@\texttt{Satisfies}\-\texttt{Weak}\-\texttt{Path}\-\texttt{Intersection}\-\texttt{Property}!for IsManiplex}
\label{SatisfiesWeakPathIntersectionProperty:for IsManiplex}
}\hfill{\scriptsize (property)}}\\
\textbf{\indent Returns:\ }
IsBool 



 Tests for the weak path intersection property in a maniplex. Definitions and
description available in \cite{GleHub18}. }

 

\subsection{\textcolor{Chapter }{IsFaithful (for IsManiplex)}}
\logpage{[ 6, 5, 5 ]}\nobreak
\hyperdef{L}{X7DEE9BDE7EA6F995}{}
{\noindent\textcolor{FuncColor}{$\triangleright$\enspace\texttt{IsFaithful({\mdseries\slshape m})\index{IsFaithful@\texttt{IsFaithful}!for IsManiplex}
\label{IsFaithful:for IsManiplex}
}\hfill{\scriptsize (operation)}}\\


 Returns whether the maniplex \mbox{\texttt{\mdseries\slshape m}} is faithful, as defined in "Polytopality of Maniplexes"; i.e., whether for
each flag the intersection of all the i-faces containing that flag is just the
flag itself. }

 
\begin{Verbatim}[commandchars=!@|,fontsize=\small,frame=single,label=Example]
  !gapprompt@gap>| !gapinput@IsFaithful(Cube(3));|
  true
  !gapprompt@gap>| !gapinput@IsFaithful(ToroidalMap44([1,0]));|
  false
\end{Verbatim}
 }

 }

   
\chapter{\textcolor{Chapter }{Comparing maniplexes}}\label{Chapter_Comparing_maniplexes}
\logpage{[ 7, 0, 0 ]}
\hyperdef{L}{X7A3CC7F9873E02BF}{}
{
  
\section{\textcolor{Chapter }{Quotients and covers}}\label{Chapter_Comparing_maniplexes_Section_Quotients_and_covers}
\logpage{[ 7, 1, 0 ]}
\hyperdef{L}{X7CD5138B85A97590}{}
{
  Many of the quotient operations let you describe some relations in terms of a
string. We assume that Sggis have a generating set of $\{r0, r1, ..., r_{n-1}\}$, so these relation strings will look something like "(r0 r1
r2)\texttt{\symbol{94}}5, r2 = (r0 r1)\texttt{\symbol{94}}3". Notice that we
can mix relations like "r2 = (r0 r1)\texttt{\symbol{94}}3" with relators like
"(r0 r1 r2)\texttt{\symbol{94}}5"; the latter is treated as the relation "(r0
r1 r2)\texttt{\symbol{94}}5 = 1". For convenience, we also allow relations to
contain the following strings: s1, s2, s3, etc, where si is expanded to r(i-1)
ri. For example, s2 becomes r1 r2. z1, z2, z3, etc, where zi is expanded to r0
(r1 r2)\texttt{\symbol{94}}i (the "i-zigzag" word). h1, h2, h3, etc, where hi
is expanded to r0 (r1 r2)\texttt{\symbol{94}}(j-1) r1 (the "i-hole" word). We
note that these strings are all restricted to have $i \leq 9$, \emph{including ri}. This restriction might be changed eventually, but it will require a rewrite
of the method ParseStringCRels that underlies many quotient operations. 
\subsection{\textcolor{Chapter }{IsQuotient}}\label{AutoDoc_generated_group2}
\logpage{[ 7, 1, 1 ]}
\hyperdef{L}{X87EB22A184A49FB8}{}
{
\noindent\textcolor{FuncColor}{$\triangleright$\enspace\texttt{IsQuotient({\mdseries\slshape M1, M2})\index{IsQuotient@\texttt{IsQuotient}!for IsPremaniplex, IsPremaniplex}
\label{IsQuotient:for IsPremaniplex, IsPremaniplex}
}\hfill{\scriptsize (operation)}}\\
\noindent\textcolor{FuncColor}{$\triangleright$\enspace\texttt{IsQuotient({\mdseries\slshape g, h})\index{IsQuotient@\texttt{IsQuotient}!for IsSggi, IsSggi}
\label{IsQuotient:for IsSggi, IsSggi}
}\hfill{\scriptsize (operation)}}\\
\textbf{\indent Returns:\ }
\texttt{IsBool} 



 Returns whether \mbox{\texttt{\mdseries\slshape M2}} is a quotient of \mbox{\texttt{\mdseries\slshape M1}}. Returns whether \mbox{\texttt{\mdseries\slshape h}} is a quotient of \mbox{\texttt{\mdseries\slshape g}}. That is, whether there is a homomorphism sending each generator of g to the
corresponding generator of h. }

 
\begin{Verbatim}[commandchars=!@|,fontsize=\small,frame=single,label=Example]
  !gapprompt@gap>| !gapinput@IsQuotient(Cube(3),HemiCube(3));|
  true
  !gapprompt@gap>| !gapinput@IsQuotient(UniversalSggi([4,3]),AutomorphismGroup(HemiCube(3)));|
  true
\end{Verbatim}
 

\subsection{\textcolor{Chapter }{IsRootedQuotient (for IsManiplex, IsManiplex, IsInt, IsInt)}}
\logpage{[ 7, 1, 2 ]}\nobreak
\hyperdef{L}{X8407FBD78060191A}{}
{\noindent\textcolor{FuncColor}{$\triangleright$\enspace\texttt{IsRootedQuotient({\mdseries\slshape M1, M2, i, j})\index{IsRootedQuotient@\texttt{IsRootedQuotient}!for IsManiplex, IsManiplex, IsInt, IsInt}
\label{IsRootedQuotient:for IsManiplex, IsManiplex, IsInt, IsInt}
}\hfill{\scriptsize (operation)}}\\
\textbf{\indent Returns:\ }
\texttt{IsBool} 



 Returns whether there is a maniplex homomorphism from \mbox{\texttt{\mdseries\slshape M1}} to \mbox{\texttt{\mdseries\slshape M2}} that sends flag \mbox{\texttt{\mdseries\slshape i}} of \mbox{\texttt{\mdseries\slshape M1}} to flag \mbox{\texttt{\mdseries\slshape j}} of \mbox{\texttt{\mdseries\slshape M2}}. }

 
\begin{Verbatim}[commandchars=!@|,fontsize=\small,frame=single,label=Example]
  !gapprompt@gap>| !gapinput@IsRootedQuotient(Pyramid(8), Pyramid(4), 1, 1);|
  true
  !gapprompt@gap>| !gapinput@IsRootedQuotient(Pyramid(8), Pyramid(4), 1, 2);|
  false
\end{Verbatim}
 

\subsection{\textcolor{Chapter }{IsRootedQuotient (for IsManiplex, IsManiplex)}}
\logpage{[ 7, 1, 3 ]}\nobreak
\hyperdef{L}{X7F097A917AE87005}{}
{\noindent\textcolor{FuncColor}{$\triangleright$\enspace\texttt{IsRootedQuotient({\mdseries\slshape M1, M2})\index{IsRootedQuotient@\texttt{IsRootedQuotient}!for IsManiplex, IsManiplex}
\label{IsRootedQuotient:for IsManiplex, IsManiplex}
}\hfill{\scriptsize (operation)}}\\
\textbf{\indent Returns:\ }
\texttt{IsBool} 



 Returns whether there is a maniplex homomorphism from \mbox{\texttt{\mdseries\slshape M1}} to \mbox{\texttt{\mdseries\slshape M2}} that sends \texttt{BaseFlag(M1)} to \texttt{BaseFlag(M2)}. }

 
\begin{Verbatim}[commandchars=!@|,fontsize=\small,frame=single,label=Example]
  !gapprompt@gap>| !gapinput@IsRootedQuotient(ToroidalMap44([4,4]), ToroidalMap44([4,0]));|
  true
  !gapprompt@gap>| !gapinput@IsRootedQuotient(ToroidalMap44([1,2]), ToroidalMap44([2,1]));|
  false
\end{Verbatim}
 

\subsection{\textcolor{Chapter }{IsCover (for IsPremaniplex, IsPremaniplex)}}
\logpage{[ 7, 1, 4 ]}\nobreak
\hyperdef{L}{X7FB4904D7E99C0A0}{}
{\noindent\textcolor{FuncColor}{$\triangleright$\enspace\texttt{IsCover({\mdseries\slshape M1, M2})\index{IsCover@\texttt{IsCover}!for IsPremaniplex, IsPremaniplex}
\label{IsCover:for IsPremaniplex, IsPremaniplex}
}\hfill{\scriptsize (operation)}}\\
\textbf{\indent Returns:\ }
\texttt{IsBool} 



 Returns whether \mbox{\texttt{\mdseries\slshape M2}} is a cover of \mbox{\texttt{\mdseries\slshape M1}}. }

 
\begin{Verbatim}[commandchars=!@|,fontsize=\small,frame=single,label=Example]
  !gapprompt@gap>| !gapinput@IsCover(HemiDodecahedron(),Dodecahedron());|
  true
\end{Verbatim}
 

\subsection{\textcolor{Chapter }{IsRootedCover (for IsManiplex, IsManiplex, IsInt, IsInt)}}
\logpage{[ 7, 1, 5 ]}\nobreak
\hyperdef{L}{X83165F047B217751}{}
{\noindent\textcolor{FuncColor}{$\triangleright$\enspace\texttt{IsRootedCover({\mdseries\slshape M1, M2, i, j})\index{IsRootedCover@\texttt{IsRootedCover}!for IsManiplex, IsManiplex, IsInt, IsInt}
\label{IsRootedCover:for IsManiplex, IsManiplex, IsInt, IsInt}
}\hfill{\scriptsize (operation)}}\\
\textbf{\indent Returns:\ }
\texttt{IsBool} 



 Returns whether there is a maniplex homomorphism from \mbox{\texttt{\mdseries\slshape M2}} to \mbox{\texttt{\mdseries\slshape M1}} that sends flag \mbox{\texttt{\mdseries\slshape j}} of \mbox{\texttt{\mdseries\slshape M2}} to flag \mbox{\texttt{\mdseries\slshape i}} of \mbox{\texttt{\mdseries\slshape M1}}. }

 
\begin{Verbatim}[commandchars=!@|,fontsize=\small,frame=single,label=Example]
  !gapprompt@gap>| !gapinput@IsRootedCover(Pyramid(4), Pyramid(8), 1, 1);|
  true
  !gapprompt@gap>| !gapinput@IsRootedCover(Pyramid(4), Pyramid(8), 1, 2);|
  false
\end{Verbatim}
 

\subsection{\textcolor{Chapter }{IsRootedCover (for IsManiplex, IsManiplex)}}
\logpage{[ 7, 1, 6 ]}\nobreak
\hyperdef{L}{X7923C770859A958F}{}
{\noindent\textcolor{FuncColor}{$\triangleright$\enspace\texttt{IsRootedCover({\mdseries\slshape M1, M2})\index{IsRootedCover@\texttt{IsRootedCover}!for IsManiplex, IsManiplex}
\label{IsRootedCover:for IsManiplex, IsManiplex}
}\hfill{\scriptsize (operation)}}\\
\textbf{\indent Returns:\ }
\texttt{IsBool} 



 Returns whether there is a maniplex homomorphism from \mbox{\texttt{\mdseries\slshape M2}} to \mbox{\texttt{\mdseries\slshape M1}} that sends \texttt{BaseFlag(M2)} to \texttt{BaseFlag(M1)}. }

 
\begin{Verbatim}[commandchars=!@|,fontsize=\small,frame=single,label=Example]
  !gapprompt@gap>| !gapinput@IsRootedCover(ToroidalMap44([4,0]), ToroidalMap44([4,4]));|
  true
  !gapprompt@gap>| !gapinput@IsRootedCover(ToroidalMap44([1,2]), ToroidalMap44([2,1]));|
  false
\end{Verbatim}
 

\subsection{\textcolor{Chapter }{IsIsomorphicManiplex (for IsManiplex, IsManiplex)}}
\logpage{[ 7, 1, 7 ]}\nobreak
\hyperdef{L}{X7B1006697C002741}{}
{\noindent\textcolor{FuncColor}{$\triangleright$\enspace\texttt{IsIsomorphicManiplex({\mdseries\slshape M1, M2})\index{IsIsomorphicManiplex@\texttt{IsIsomorphicManiplex}!for IsManiplex, IsManiplex}
\label{IsIsomorphicManiplex:for IsManiplex, IsManiplex}
}\hfill{\scriptsize (operation)}}\\
\textbf{\indent Returns:\ }
\texttt{IsBool} 



 Returns whether \mbox{\texttt{\mdseries\slshape M1}} is isomorphic to \mbox{\texttt{\mdseries\slshape M2}}. }

 
\begin{Verbatim}[commandchars=!@|,fontsize=\small,frame=single,label=Example]
  !gapprompt@gap>| !gapinput@IsIsomorphicManiplex(HemiCube(3),Petrial(Simplex(3)));|
  true
\end{Verbatim}
 

\subsection{\textcolor{Chapter }{IsIsomorphicRootedManiplex (for IsManiplex, IsManiplex, IsInt, IsInt)}}
\logpage{[ 7, 1, 8 ]}\nobreak
\hyperdef{L}{X7FCE9DF47E54D12D}{}
{\noindent\textcolor{FuncColor}{$\triangleright$\enspace\texttt{IsIsomorphicRootedManiplex({\mdseries\slshape M1, M2, i, j})\index{IsIsomorphicRootedManiplex@\texttt{IsIsomorphicRootedManiplex}!for IsManiplex, IsManiplex, IsInt, IsInt}
\label{IsIsomorphicRootedManiplex:for IsManiplex, IsManiplex, IsInt, IsInt}
}\hfill{\scriptsize (operation)}}\\
\textbf{\indent Returns:\ }
\texttt{IsBool} 



 Returns whether there is an isomorphism from \mbox{\texttt{\mdseries\slshape M1}} to \mbox{\texttt{\mdseries\slshape M2}} that sends flag \mbox{\texttt{\mdseries\slshape j}} of \mbox{\texttt{\mdseries\slshape M2}} to flag \mbox{\texttt{\mdseries\slshape i}} of \mbox{\texttt{\mdseries\slshape M1}}. }

 
\begin{Verbatim}[commandchars=!@|,fontsize=\small,frame=single,label=Example]
  !gapprompt@gap>| !gapinput@IsIsomorphicManiplex(ToroidalMap44([1,2]), ToroidalMap44([2,1]));|
  true
  !gapprompt@gap>| !gapinput@FlagOrbitRepresentatives(ToroidalMap44([1,2]));|
  [1, 21]
  !gapprompt@gap>| !gapinput@IsIsomorphicRootedManiplex(ToroidalMap44([1,2]), ToroidalMap44([1,2]), 1, 1);|
  true
  !gapprompt@gap>| !gapinput@IsIsomorphicRootedManiplex(ToroidalMap44([1,2]), ToroidalMap44([1,2]), 1, 21);|
  false
  !gapprompt@gap>| !gapinput@IsIsomorphicRootedManiplex(ToroidalMap44([1,2]), ToroidalMap44([2,1]), 1, 1);|
  false
\end{Verbatim}
 

\subsection{\textcolor{Chapter }{IsIsomorphicRootedManiplex (for IsManiplex, IsManiplex)}}
\logpage{[ 7, 1, 9 ]}\nobreak
\hyperdef{L}{X85A79420835EAF2A}{}
{\noindent\textcolor{FuncColor}{$\triangleright$\enspace\texttt{IsIsomorphicRootedManiplex({\mdseries\slshape M1, M2})\index{IsIsomorphicRootedManiplex@\texttt{IsIsomorphicRootedManiplex}!for IsManiplex, IsManiplex}
\label{IsIsomorphicRootedManiplex:for IsManiplex, IsManiplex}
}\hfill{\scriptsize (operation)}}\\
\textbf{\indent Returns:\ }
\texttt{IsBool} 



 Returns whether there is an isomorphism from \mbox{\texttt{\mdseries\slshape M1}} to \mbox{\texttt{\mdseries\slshape M2}} that sends \texttt{BaseFlag(M2)} to \texttt{BaseFlag(M1)}. }

 
\begin{Verbatim}[commandchars=!@|,fontsize=\small,frame=single,label=Example]
  !gapprompt@gap>| !gapinput@IsIsomorphicManiplex(ToroidalMap44([1,2]), ToroidalMap44([2,1]));|
  true
  !gapprompt@gap>| !gapinput@IsIsomorphicRootedManiplex(ToroidalMap44([1,2]), ToroidalMap44([2,1]));|
  false
  !gapprompt@gap>| !gapinput@IsIsomorphicRootedManiplex(ToroidalMap44([1,2]), EnantiomorphicForm(ToroidalMap44([2,1])));|
  true
\end{Verbatim}
 

\subsection{\textcolor{Chapter }{SmallestReflexibleCover (for IsManiplex)}}
\logpage{[ 7, 1, 10 ]}\nobreak
\hyperdef{L}{X8017B6F880D162E7}{}
{\noindent\textcolor{FuncColor}{$\triangleright$\enspace\texttt{SmallestReflexibleCover({\mdseries\slshape M})\index{SmallestReflexibleCover@\texttt{SmallestReflexibleCover}!for IsManiplex}
\label{SmallestReflexibleCover:for IsManiplex}
}\hfill{\scriptsize (attribute)}}\\


 Returns the smallest regular cover of \mbox{\texttt{\mdseries\slshape M}}, which is the maniplex whose automorphism group is isomorphic to the
connection group of \mbox{\texttt{\mdseries\slshape M}}. }

 
\begin{Verbatim}[commandchars=!@|,fontsize=\small,frame=single,label=Example]
  !gapprompt@gap>| !gapinput@SmallestReflexibleCover(ToroidalMap44([2,3],[3,2]));|
  reflexible 3-maniplex
  !gapprompt@gap>| !gapinput@last=ToroidalMap44([5,0]);|
  true
\end{Verbatim}
 

\subsection{\textcolor{Chapter }{QuotientManiplex (for IsReflexibleManiplex, IsString)}}
\logpage{[ 7, 1, 11 ]}\nobreak
\hyperdef{L}{X85BE14877927DB85}{}
{\noindent\textcolor{FuncColor}{$\triangleright$\enspace\texttt{QuotientManiplex({\mdseries\slshape M, relStr})\index{QuotientManiplex@\texttt{QuotientManiplex}!for IsReflexibleManiplex, IsString}
\label{QuotientManiplex:for IsReflexibleManiplex, IsString}
}\hfill{\scriptsize (operation)}}\\


 Given a reflexible maniplex \mbox{\texttt{\mdseries\slshape M}}, generates the subgroup S of AutomorphismGroup(\mbox{\texttt{\mdseries\slshape M}}) given by relStr, and returns the quotient maniplex M / S. For example,
QuotientManiplex(CubicTiling(2), "(r0 r1 r2 r1)\texttt{\symbol{94}}5, (r1 r0
r1 r2)\texttt{\symbol{94}}2") returns the toroidal map
\texttt{\symbol{123}}4,4\texttt{\symbol{125}}{\textunderscore}\texttt{\symbol{123}}(5,0),(0,2)\texttt{\symbol{125}}.
You can also input this as CubicTiling(2) / "(r0 r1 r2
r1)\texttt{\symbol{94}}5, (r1 r0 r1 r2)\texttt{\symbol{94}}2". }

 
\begin{Verbatim}[commandchars=!@|,fontsize=\small,frame=single,label=Example]
  !gapprompt@gap>| !gapinput@q:=QuotientManiplex(CubicTiling(2),"(r0 r1 r2 r1)^5, (r1 r0 r1 r2)^2");|
  3-maniplex
  !gapprompt@gap>| !gapinput@SchlafliSymbol(q);|
  [ 4, 4 ]
\end{Verbatim}
 

\subsection{\textcolor{Chapter }{ReflexibleQuotientManiplex (for IsManiplex, IsList)}}
\logpage{[ 7, 1, 12 ]}\nobreak
\hyperdef{L}{X87DCBB9E78E383BB}{}
{\noindent\textcolor{FuncColor}{$\triangleright$\enspace\texttt{ReflexibleQuotientManiplex({\mdseries\slshape M, rels})\index{ReflexibleQuotientManiplex@\texttt{ReflexibleQuotientManiplex}!for IsManiplex, IsList}
\label{ReflexibleQuotientManiplex:for IsManiplex, IsList}
}\hfill{\scriptsize (operation)}}\\


 Given a reflexible maniplex \mbox{\texttt{\mdseries\slshape M}}, generates the normal closure N of the subgroup S of AutomorphismGroup(\mbox{\texttt{\mdseries\slshape M}}) given by relStr, and returns the quotient maniplex M / N, which will be
reflexible. For example, QuotientManiplex(CubicTiling(2), "(r0 r1 r2
r1)\texttt{\symbol{94}}5, (r1 r0 r1 r2)\texttt{\symbol{94}}2") returns the
toroidal map
\texttt{\symbol{123}}4,4\texttt{\symbol{125}}{\textunderscore}\texttt{\symbol{123}}(1,0)\texttt{\symbol{125}},
because the normal closure of the group generated by (r0 r1 r2
r1)\texttt{\symbol{94}}5 and (r1 r0 r1 r2)\texttt{\symbol{94}}2 is the group
generated by r0 r1 r2 r1 and r1 r0 r1 r2. }

 
\begin{Verbatim}[commandchars=!@|,fontsize=\small,frame=single,label=Example]
  !gapprompt@gap>| !gapinput@q:=ReflexibleQuotientManiplex(CubicTiling(2),"(r0 r1 r2 r1)^5, (r1 r0 r1 r2)^2");|
  reflexible 3-maniplex with 8 flags
  !gapprompt@gap>| !gapinput@last=ToroidalMap44([1,0]);|
  true
\end{Verbatim}
 

\subsection{\textcolor{Chapter }{QuotientSggi (for IsGroup, IsList)}}
\logpage{[ 7, 1, 13 ]}\nobreak
\hyperdef{L}{X83D969847E68299D}{}
{\noindent\textcolor{FuncColor}{$\triangleright$\enspace\texttt{QuotientSggi({\mdseries\slshape g, rels})\index{QuotientSggi@\texttt{QuotientSggi}!for IsGroup, IsList}
\label{QuotientSggi:for IsGroup, IsList}
}\hfill{\scriptsize (operation)}}\\
\textbf{\indent Returns:\ }
the quotient of \mbox{\texttt{\mdseries\slshape g}} by \mbox{\texttt{\mdseries\slshape rels}}, which is either a list of Tietze words or a string of relations that is
parsed by ParseStringCRels. 



 
\begin{Verbatim}[commandchars=!@|,fontsize=\small,frame=single,label=Example]
  !gapprompt@gap>| !gapinput@g := UniversalSggi(3);|
  <fp group of size infinity on the generators [ r0, r1, r2 ]>
  !gapprompt@gap>| !gapinput@h := QuotientSggi(g, "(r0 r1)^5, (r1 r2)^3, (r0 r1 r2)^5");|
  <fp group on the generators [ r0, r1, r2 ]>
  !gapprompt@gap>| !gapinput@Size(h);|
  60
\end{Verbatim}
 }

 

\subsection{\textcolor{Chapter }{QuotientSggiByNormalSubgroup (for IsGroup,IsGroup)}}
\logpage{[ 7, 1, 14 ]}\nobreak
\hyperdef{L}{X819D85E57B505196}{}
{\noindent\textcolor{FuncColor}{$\triangleright$\enspace\texttt{QuotientSggiByNormalSubgroup({\mdseries\slshape g, n})\index{QuotientSggiByNormalSubgroup@\texttt{QuotientSggiByNormalSubgroup}!for IsGroup,IsGroup}
\label{QuotientSggiByNormalSubgroup:for IsGroup,IsGroup}
}\hfill{\scriptsize (operation)}}\\
\textbf{\indent Returns:\ }
g/n 



 Given an sggi \mbox{\texttt{\mdseries\slshape g}} and a normal subgroup \mbox{\texttt{\mdseries\slshape n}} in \mbox{\texttt{\mdseries\slshape g}}, this function will give you the quotient in a way that respects the
generators (i.e., the generators of the quotient will be the images of the
generators of the original group). }

 
\begin{Verbatim}[commandchars=!@|,fontsize=\small,frame=single,label=Example]
  !gapprompt@gap>| !gapinput@g:=AutomorphismGroup(Cube(3));|
  <fp group of size 48 on the generators [ r0, r1, r2 ]>
  !gapprompt@gap>| !gapinput@q:=QuotientSggiByNormalSubgroup(g,Group([(g.1*g.2*g.3)^3]));|
  Group([ (1,2)(3,7)(4,6)(5,10)(8,14)(9,16)(11,18)(12,20)(13,17)(15,23)(19,22)(21,24), (1,3)(2,5)(4,9)(6,12)(7,13)(8,15)(10,17)(11,19)(14,22)(16,24)(18,23)(20,21), (1,4)(2,6)(3,8)(5,11)(7,14)(9,15)(10,18)(12,19)(13,21)(16,23)(17,24)(20,22) ])
  !gapprompt@gap>| !gapinput@Maniplex(q)=HemiCube(3);|
  true
\end{Verbatim}
 

\subsection{\textcolor{Chapter }{QuotientManiplexByAutomorphismSubgroup (for IsManiplex,IsPermGroup)}}
\logpage{[ 7, 1, 15 ]}\nobreak
\hyperdef{L}{X86725DD8781BF181}{}
{\noindent\textcolor{FuncColor}{$\triangleright$\enspace\texttt{QuotientManiplexByAutomorphismSubgroup({\mdseries\slshape m, h})\index{QuotientManiplexByAutomorphismSubgroup@\texttt{Quotient}\-\texttt{Maniplex}\-\texttt{By}\-\texttt{Automorphism}\-\texttt{Subgroup}!for IsManiplex,IsPermGroup}
\label{QuotientManiplexByAutomorphismSubgroup:for IsManiplex,IsPermGroup}
}\hfill{\scriptsize (operation)}}\\
\textbf{\indent Returns:\ }
m/h 



 Given a maniplex \mbox{\texttt{\mdseries\slshape m}}, and a subgroup \mbox{\texttt{\mdseries\slshape h}} of the automorphism group on the flags, this function will give you the
maniplex in which the orbits of flags under the action of \mbox{\texttt{\mdseries\slshape h}} are identified. Note that this function doesn't do any prechecks, and may
break easily when \texttt{m/h} {\textunderscore}isn't{\textunderscore} a maniplex or when \texttt{m/h} is of lower rank (sorry!). }

 
\begin{Verbatim}[commandchars=!@|,fontsize=\small,frame=single,label=Example]
  !gapprompt@gap>| !gapinput@m:=Cube(3);|
  Cube(3)
  !gapprompt@gap>| !gapinput@a:=AutomorphismGroupOnFlags(m);|
  <permutation group with 3 generators>
  !gapprompt@gap>| !gapinput@h:=Group((a.3*a.1*a.2)^3);|
  Group([ (1,7)(2,3)(4,18)(5,19)(6,20)(8,11)(9,12)(10,13)(14,32)(15,33)(16,34)(17,35)(21,25)(22,26)  (23,27)(24,28)(29,43)(30,44)(31,45)(36,39)(37,40)(38,41)(42,48)(46,47) ])
  !gapprompt@gap>| !gapinput@q:=QuotientManiplexByAutomorphismSubgroup(m,h);|
  3-maniplex with 24 flags
  !gapprompt@gap>| !gapinput@last=HemiCube(3);|
  true
\end{Verbatim}
 }

 }

   
\chapter{\textcolor{Chapter }{Posets}}\label{Chapter_Posets}
\logpage{[ 8, 0, 0 ]}
\hyperdef{L}{X79540DAB85902432}{}
{
  
\section{\textcolor{Chapter }{Poset constructors}}\label{Chapter_Posets_Section_Poset_constructors}
\logpage{[ 8, 1, 0 ]}
\hyperdef{L}{X87ED175B85A4C5B7}{}
{
  I'm in the process of reconciling all of this, but there are going to be a
number of ways to \texttt{define} a poset: 
\begin{itemize}
\item  As an \texttt{IsPosetOfFlags}, where the underlying description is an ordered list of length $n+2$. Each of the $n+2$ list elements is a list of faces, and the assumption is that these are the
faces of rank $i-2$, where $i$ is the index in the master list (e.g., \texttt{l[1][1]} would usually correspond to the unique $-1$ face of a polytope -- and there won't be an \texttt{l[1][2]}). Each face is then a list of the flags incident with that face. 
\item  As an \texttt{IsPosetOfIndices}, where the underlying description is a binary relation on a set of indices,
which correspond to labels for the elements of the poset. 
\item  If the poset is known to be atomic, then by a description of the faces in
terms of the atoms... usually we'll just need the list of the elements of
maximal rank, from which all other elements may be obtained. 
\item  As an \texttt{IsPosetOfElements}, where the elements could be anything, and we have a known function
determining the partial order on the elements. 
\end{itemize}
 Usually, we assume that the poset will have a natural rank function on it.
More information on the poset attributes that are important in the study of
abstract polytopes and maniplexes is available in \cite{McMSch02}, \cite{MonPelWil14}, and \cite{Wil12}. 

\subsection{\textcolor{Chapter }{PosetFromFaceListOfFlags (for IsList)}}
\logpage{[ 8, 1, 1 ]}\nobreak
\hyperdef{L}{X7D495B407CDFAEA2}{}
{\noindent\textcolor{FuncColor}{$\triangleright$\enspace\texttt{PosetFromFaceListOfFlags({\mdseries\slshape list})\index{PosetFromFaceListOfFlags@\texttt{PosetFromFaceListOfFlags}!for IsList}
\label{PosetFromFaceListOfFlags:for IsList}
}\hfill{\scriptsize (operation)}}\\
\textbf{\indent Returns:\ }
\texttt{IsPosetOfFlags}. 



 Given a \mbox{\texttt{\mdseries\slshape list}} of lists of faces in increasing rank, where each face is described by the
incident flags, gives you a IsPosetOfFlags object back. Posets constructed
this way are assumed to be IsP1 and IsP2. }

 Here we have a poset using the \texttt{IsPosetOfFlags} description for the triangle. 
\begin{Verbatim}[commandchars=!@|,fontsize=\small,frame=single,label=Example]
  !gapprompt@gap>| !gapinput@poset:=PosetFromFaceListOfFlags([[[1,2,3,4,5,6]],[[1,2],[3,6],[4,5]],[[1,4],[2,3],[5,6]],[[1,2,3,4,5,6]]]);|
  A poset using the IsPosetOfFlags representation with 8 faces.
  !gapprompt@gap>| !gapinput@FaceListOfPoset(poset);|
  [ [ [ 1, 2, 3, 4, 5, 6 ] ], [ [ 1, 2 ], [ 3, 6 ], [ 4, 5 ] ], [ [ 1, 4 ], [ 2, 3 ], [ 5, 6 ] ], [ [ 1, 2, 3, 4, 5, 6 ] ] ]
\end{Verbatim}
 

\subsection{\textcolor{Chapter }{PosetFromConnectionGroup (for IsPermGroup)}}
\logpage{[ 8, 1, 2 ]}\nobreak
\hyperdef{L}{X8720C4757EE12082}{}
{\noindent\textcolor{FuncColor}{$\triangleright$\enspace\texttt{PosetFromConnectionGroup({\mdseries\slshape g})\index{PosetFromConnectionGroup@\texttt{PosetFromConnectionGroup}!for IsPermGroup}
\label{PosetFromConnectionGroup:for IsPermGroup}
}\hfill{\scriptsize (operation)}}\\
\textbf{\indent Returns:\ }
\texttt{IsPosetOfFlags} with \texttt{IsP1}=true. 



 Given a group, returns a poset with an internal representation as a list of
faces ordered by rank, where each face is represented as a list of the flags
it contains. Note that this function includes the minimal (empty) face and the
maximal face of the maniplex. Note that the $i$-faces correspond to the $i+1$ item in the list because of how GAP indexes lists. }

 
\begin{Verbatim}[commandchars=!@|,fontsize=\small,frame=single,label=Example]
  !gapprompt@gap>| !gapinput@g:=Group([(1,4)(2,3)(5,6),(1,2)(3,6)(4,5)]);|
  Group([ (1,4)(2,3)(5,6), (1,2)(3,6)(4,5) ])
  !gapprompt@gap>| !gapinput@PosetFromConnectionGroup(g);|
  A poset using the IsPosetOfFlags representation with 8 faces.
\end{Verbatim}
 

\subsection{\textcolor{Chapter }{PosetFromManiplex (for IsManiplex)}}
\logpage{[ 8, 1, 3 ]}\nobreak
\hyperdef{L}{X7C4F3D307C5A7E7A}{}
{\noindent\textcolor{FuncColor}{$\triangleright$\enspace\texttt{PosetFromManiplex({\mdseries\slshape mani})\index{PosetFromManiplex@\texttt{PosetFromManiplex}!for IsManiplex}
\label{PosetFromManiplex:for IsManiplex}
}\hfill{\scriptsize (operation)}}\\
\textbf{\indent Returns:\ }
\texttt{IsPosetOfFlags} 



 Given a maniplex, returns a poset of the maniplex with an internal
representation as a list of faces ordered by rank, where each face is
represented as a list of the flags it contains. Note that this function does
include the minimal (empty) face and the maximal face of the maniplex. Note
that the $i$-faces correspond to the $i+1$ item in the list because of how GAP indexes lists. }

 
\begin{Verbatim}[commandchars=!@|,fontsize=\small,frame=single,label=Example]
  !gapprompt@gap>| !gapinput@p:=HemiCube(3);|
  Regular 3-polytope of type [ 4, 3 ] with 24 flags
  !gapprompt@gap>| !gapinput@PosetFromManiplex(p);|
  A poset using the IsPosetOfFlags representation with 15 faces.
\end{Verbatim}
 

\subsection{\textcolor{Chapter }{PosetFromPartialOrder (for IsBinaryRelation)}}
\logpage{[ 8, 1, 4 ]}\nobreak
\hyperdef{L}{X7ECC2FBC7AF3A960}{}
{\noindent\textcolor{FuncColor}{$\triangleright$\enspace\texttt{PosetFromPartialOrder({\mdseries\slshape partialOrder})\index{PosetFromPartialOrder@\texttt{PosetFromPartialOrder}!for IsBinaryRelation}
\label{PosetFromPartialOrder:for IsBinaryRelation}
}\hfill{\scriptsize (operation)}}\\
\textbf{\indent Returns:\ }
\texttt{IsPosetOfIndices} 



 Given a partial order on a finite set of size $n$, this function will create a partial order on \texttt{[1..n]}. }

 
\begin{Verbatim}[commandchars=!@|,fontsize=\small,frame=single,label=Example]
  !gapprompt@gap>| !gapinput@l:=List([[1,1],[1,2],[1,3],[1,4],[2,4],[2,2],[3,3],[4,4]],x->Tuple(x));|
  !gapprompt@gap>| !gapinput@r:=BinaryRelationByElements(Domain([1..4]), l);|
  <general mapping: Domain([ 1 .. 4 ]) -> Domain([ 1 .. 4 ]) >
  !gapprompt@gap>| !gapinput@poset:=PosetFromPartialOrder(r);|
  A poset using the IsPosetOfIndices representation 
  !gapprompt@gap>| !gapinput@h:=HasseDiagramBinaryRelation(PartialOrder(poset));|
  <general mapping: Domain([ 1 .. 4 ]) -> Domain([ 1 .. 4 ]) >
  !gapprompt@gap>| !gapinput@Successors(h);|
  [ [ 2, 3 ], [ 4 ], [  ], [  ] ]
\end{Verbatim}
 Note that what we've accomplished here is the poset containing the elements 1,
2, 3, 4 with partial order determined by whether the first element divides the
second. The essential information about the poset can be obtained from the
Hasse diagram. 

\subsection{\textcolor{Chapter }{PosetFromAtomicList (for IsList)}}
\logpage{[ 8, 1, 5 ]}\nobreak
\hyperdef{L}{X80052F8E835F45A4}{}
{\noindent\textcolor{FuncColor}{$\triangleright$\enspace\texttt{PosetFromAtomicList({\mdseries\slshape list})\index{PosetFromAtomicList@\texttt{PosetFromAtomicList}!for IsList}
\label{PosetFromAtomicList:for IsList}
}\hfill{\scriptsize (operation)}}\\
\textbf{\indent Returns:\ }
\texttt{IsPosetOfAtoms} 



 Given a list of elements, where each element is given as a list of atoms, this
function will construct the corresponding poset. Note that this will construct
any implied faces as well (i.e., all possible intersections of the listed
faces). }

 
\begin{Verbatim}[commandchars=!@|,fontsize=\small,frame=single,label=Example]
  !gapprompt@gap>| !gapinput@list:=[[1,2,3],[1,2,4],[1,3,4],[2,3,4]];|
  [ [ 1, 2, 3 ], [ 1, 2, 4 ], [ 1, 3, 4 ], [ 2, 3, 4 ] ]
  !gapprompt@gap>| !gapinput@poset:=PosetFromAtomicList(list);;|
  !gapprompt@gap>| !gapinput@List(Faces(poset),AtomList);|
  [ [  ], [ 1 ], [ 1, 2 ], [ 1, 2, 3 ], [ 1, 2, 4 ], [ 1, 3 ], [ 1, 3, 4 ], [ 1, 4 ], [ 2 ], [ 2, 3 ], 
    [ 2, 3, 4 ], [ 2, 4 ], [ 3 ], [ 3, 4 ], [ 4 ], [ 1 .. 4 ] ]
  !gapprompt@gap>| !gapinput@ml:=["abc","abd","acd","bcd"];;|
  !gapprompt@gap>| !gapinput@p:=PosetFromAtomicList(ml);;|
  !gapprompt@gap>| !gapinput@List(Flags(p),x->List(x,AtomList));|
  [ [ [  ], "a", "ab", "abc", "abcd" ], [ [  ], "a", "ab", "abd", "abcd" ], 
    [ [  ], "a", "ac", "abc", "abcd" ], [ [  ], "a", "ac", "acd", "abcd" ], 
    [ [  ], "a", "ad", "abd", "abcd" ], [ [  ], "a", "ad", "acd", "abcd" ], 
    [ [  ], "b", "ab", "abc", "abcd" ], [ [  ], "b", "ab", "abd", "abcd" ], 
    [ [  ], "b", "bc", "abc", "abcd" ], [ [  ], "b", "bc", "bcd", "abcd" ], 
    [ [  ], "b", "bd", "abd", "abcd" ], [ [  ], "b", "bd", "bcd", "abcd" ], 
    [ [  ], "c", "ac", "abc", "abcd" ], [ [  ], "c", "ac", "acd", "abcd" ], 
    [ [  ], "c", "bc", "abc", "abcd" ], [ [  ], "c", "bc", "bcd", "abcd" ], 
    [ [  ], "c", "cd", "acd", "abcd" ], [ [  ], "c", "cd", "bcd", "abcd" ], 
    [ [  ], "d", "ad", "abd", "abcd" ], [ [  ], "d", "ad", "acd", "abcd" ], 
    [ [  ], "d", "bd", "abd", "abcd" ], [ [  ], "d", "bd", "bcd", "abcd" ], 
    [ [  ], "d", "cd", "acd", "abcd" ], [ [  ], "d", "cd", "bcd", "abcd" ] ]
\end{Verbatim}
 

\subsection{\textcolor{Chapter }{PosetFromElements (for IsList,IsFunction)}}
\logpage{[ 8, 1, 6 ]}\nobreak
\hyperdef{L}{X79AEF6FF8699020E}{}
{\noindent\textcolor{FuncColor}{$\triangleright$\enspace\texttt{PosetFromElements({\mdseries\slshape list{\textunderscore}of{\textunderscore}faces, func})\index{PosetFromElements@\texttt{PosetFromElements}!for IsList,IsFunction}
\label{PosetFromElements:for IsList,IsFunction}
}\hfill{\scriptsize (operation)}}\\
\textbf{\indent Returns:\ }
\texttt{IsPosetOfElements} 



 This is for gathering elements with a known ordering \mbox{\texttt{\mdseries\slshape func}} on two variables into a poset. Also note, the expectation is that \mbox{\texttt{\mdseries\slshape func}} behaves similarly to IsSubset, i.e., \mbox{\texttt{\mdseries\slshape func}} (x,y)=true means $y$ is less than $x$ in the order. }

 
\begin{Verbatim}[commandchars=!@|,fontsize=\small,frame=single,label=Example]
  !gapprompt@gap>| !gapinput@ g:=SymmetricGroup(3);|
  Sym( [ 1 .. 3 ] )
  !gapprompt@gap>| !gapinput@asg:=AllSubgroups(g);|
  [ Group(()), Group([ (2,3) ]), Group([ (1,2) ]), Group([ (1,3) ]), Group([ (1,2,3) ]),   Group([ (1,2,3), (2,3) ]) ]
  !gapprompt@gap>| !gapinput@poset:=PosetFromElements(asg,IsSubgroup);|
  A poset on 6 elements using the IsPosetOfIndices representation.
  !gapprompt@gap>| !gapinput@HasseDiagramBinaryRelation(PartialOrder(poset));|
  <general mapping: Domain([ 1 .. 6 ]) -> Domain([ 1 .. 6 ]) >
  !gapprompt@gap>| !gapinput@Successors(last);|
  [ [ 2, 3, 4, 5 ], [ 6 ], [ 6 ], [ 6 ], [ 6 ], [  ] ]
  !gapprompt@gap>| !gapinput@List( ElementsList(poset){[2,6]}, ElementObject);|
  [ Group([ (2,3) ]), Group([ (1,2,3), (2,3) ]) ]
\end{Verbatim}
 

\subsection{\textcolor{Chapter }{PosetFromSuccessorList (for IsList)}}
\logpage{[ 8, 1, 7 ]}\nobreak
\hyperdef{L}{X856446F583AAF703}{}
{\noindent\textcolor{FuncColor}{$\triangleright$\enspace\texttt{PosetFromSuccessorList({\mdseries\slshape successorsList})\index{PosetFromSuccessorList@\texttt{PosetFromSuccessorList}!for IsList}
\label{PosetFromSuccessorList:for IsList}
}\hfill{\scriptsize (operation)}}\\
\textbf{\indent Returns:\ }
poset 



 Given a list of immediate successors, will construct the poset. A valid list
of successors is of the form \texttt{[[2,3],[3],[]]} where the $i$-th entry is a list of elements that are greater than the $i$-th element in the partial order that determines the poset. If the given list
isn't reflexive and transitive, this function will induce those properties
from the given list of successors. }

 
\begin{Verbatim}[commandchars=!@|,fontsize=\small,frame=single,label=Example]
  !gapprompt@gap>| !gapinput@p:=PosetFromManiplex(HemiCube(3));;|
  !gapprompt@gap>| !gapinput@Print(p);|
  PosetFromSuccessorList([ [ 2, 3, 4, 5 ], [ 6, 7, 9 ], [ 6, 8, 11 ], [ 7, 10, 11 ], 
  [ 8, 9, 10 ], [ 1, 2, 13 ], [ 12, 14 ], [ 12, 14 ], [ 13, 14 ], [ 12, 13 ], [ 13, 14 ], 
  [ 15 ], [ 15 ], [ 15 ], [ ] ]);
\end{Verbatim}
 
\subsection{\textcolor{Chapter }{Helper functions for special partial orders}}\label{Helper_functions}
\logpage{[ 8, 1, 8 ]}
\hyperdef{L}{X82DB1F987C68B392}{}
{
\noindent\textcolor{FuncColor}{$\triangleright$\enspace\texttt{PairCompareFlagsList({\mdseries\slshape list1, list2})\index{PairCompareFlagsList@\texttt{PairCompareFlagsList}!for IsList,IsList}
\label{PairCompareFlagsList:for IsList,IsList}
}\hfill{\scriptsize (operation)}}\\
\noindent\textcolor{FuncColor}{$\triangleright$\enspace\texttt{PairCompareAtomsList({\mdseries\slshape list1, list2})\index{PairCompareAtomsList@\texttt{PairCompareAtomsList}!for IsList,IsList}
\label{PairCompareAtomsList:for IsList,IsList}
}\hfill{\scriptsize (operation)}}\\
\textbf{\indent Returns:\ }
\texttt{true} or \texttt{false} 



 The functions PairCompareFlagsList and PairCompareAtomsList are used in poset
construction. Function assumes \mbox{\texttt{\mdseries\slshape list1}} and \mbox{\texttt{\mdseries\slshape list2}} are of the form [\texttt{listOfFlags},\texttt{i}] where \texttt{listOfFlags} is a list of flags in the face and \texttt{i} is the rank of the face. Allows comparison of HasFlagList elements. Function
assumes \mbox{\texttt{\mdseries\slshape list1}} and \mbox{\texttt{\mdseries\slshape list2}} are of the form \texttt{[listOfAtoms,int]} where \texttt{listOfAtoms} is a list of flags in the face and \texttt{int} is the rank of the face. Allows comparison of HasAtomList elements. }

 

\subsection{\textcolor{Chapter }{DualPoset (for IsPoset)}}
\logpage{[ 8, 1, 9 ]}\nobreak
\hyperdef{L}{X878F761878BCB6B9}{}
{\noindent\textcolor{FuncColor}{$\triangleright$\enspace\texttt{DualPoset({\mdseries\slshape poset})\index{DualPoset@\texttt{DualPoset}!for IsPoset}
\label{DualPoset:for IsPoset}
}\hfill{\scriptsize (operation)}}\\
\textbf{\indent Returns:\ }
dual 



 Given a \mbox{\texttt{\mdseries\slshape poset}}, will construct a poset isomorphic to the dual of \mbox{\texttt{\mdseries\slshape poset}}. }

 
\begin{Verbatim}[commandchars=!@|,fontsize=\small,frame=single,label=Example]
  !gapprompt@gap>| !gapinput@p:=PosetFromManiplex(Cube(3));; c:=PosetFromManiplex(CrossPolytope(3));;|
  !gapprompt@gap>| !gapinput@IsIsomorphicPoset(DualPoset(DualPoset(p)),p);|
  true
  !gapprompt@gap>| !gapinput@IsIsomorphicPoset(DualPoset(p),c);|
  true
  !gapprompt@gap>| !gapinput@IsIsomorphicPoset(DualPoset(p),p);|
  false
\end{Verbatim}
 

\subsection{\textcolor{Chapter }{Section (for IsFace, IsFace, IsPoset)}}
\logpage{[ 8, 1, 10 ]}\nobreak
\hyperdef{L}{X7923E0E47CAFBD9D}{}
{\noindent\textcolor{FuncColor}{$\triangleright$\enspace\texttt{Section({\mdseries\slshape face1, face2, poset})\index{Section@\texttt{Section}!for IsFace, IsFace, IsPoset}
\label{Section:for IsFace, IsFace, IsPoset}
}\hfill{\scriptsize (operation)}}\\
\textbf{\indent Returns:\ }
section 



 Constructs the section of the \mbox{\texttt{\mdseries\slshape poset}} \mbox{\texttt{\mdseries\slshape face1}}$/$\mbox{\texttt{\mdseries\slshape face2}}. }

 
\begin{Verbatim}[commandchars=!@|,fontsize=\small,frame=single,label=Example]
  !gapprompt@gap>| !gapinput@poset:=PosetFromManiplex(PyramidOver(Cube(2)));;|
  !gapprompt@gap>| !gapinput@faces:=Faces(poset);;List(faces,x->RankInPoset(x,poset));|
  [ -1, 0, 0, 0, 0, 0, 1, 1, 1, 1, 1, 1, 1, 1, 2, 2, 2, 2, 2, 3 ]
  !gapprompt@gap>| !gapinput@IsIsomorphicPoset(Section(faces[15],faces[1],poset),PosetFromManiplex(Simplex(2)));|
  true
  !gapprompt@gap>| !gapinput@IsIsomorphicPoset(Section(faces[16],faces[1],poset),PosetFromManiplex(Cube(2)));|
  true
  !gapprompt@gap>| !gapinput@IsIsomorphicPoset(Section(faces[20],faces[2],poset),PosetFromManiplex(Cube(2)));|
  true
\end{Verbatim}
 
\subsection{\textcolor{Chapter }{Cleaving polytopes}}\label{Cleaving}
\logpage{[ 8, 1, 11 ]}
\hyperdef{L}{X7B693D8884B77C7C}{}
{
\noindent\textcolor{FuncColor}{$\triangleright$\enspace\texttt{Cleave({\mdseries\slshape p, k})\index{Cleave@\texttt{Cleave}!for IsPoset,IsInt}
\label{Cleave:for IsPoset,IsInt}
}\hfill{\scriptsize (operation)}}\\
\noindent\textcolor{FuncColor}{$\triangleright$\enspace\texttt{PartiallyCleave({\mdseries\slshape p, k})\index{PartiallyCleave@\texttt{PartiallyCleave}!for IsPoset,IsInt}
\label{PartiallyCleave:for IsPoset,IsInt}
}\hfill{\scriptsize (operation)}}\\
\textbf{\indent Returns:\ }
IsPolytope 



 Given an IsPolytope \mbox{\texttt{\mdseries\slshape p}}, and an IsInt \mbox{\texttt{\mdseries\slshape k}}, \texttt{Cleave(polytope,k)} will construct the $k^{th}$-cleaved polytope of \mbox{\texttt{\mdseries\slshape p}}. Cleaved polytopes were introduced by Daniel Pellicer \cite{Pel18}. \texttt{PartiallyCleave(p,k)} will construct the $k^{\textrm{th}}$-partially cleaved polytope of \mbox{\texttt{\mdseries\slshape p}}. }

 
\begin{Verbatim}[commandchars=!@|,fontsize=\small,frame=single,label=Example]
  !gapprompt@gap>| !gapinput@Cleave(PosetFromManiplex(Cube(4)),3);|
  A poset on 290 elements using the IsPosetOfIndices representation.
\end{Verbatim}
 }

 
\section{\textcolor{Chapter }{Poset attributes}}\label{Chapter_Posets_Section_Poset_attributes}
\logpage{[ 8, 2, 0 ]}
\hyperdef{L}{X859F4CD47E3311AD}{}
{
  Posets have many properties we might be interested in. Here's a few. All
abstract polytope definitions in use here are from Schulte and McMullen's \emph{Abstract Regular Polytopes} \cite{McMSch02}. 

\subsection{\textcolor{Chapter }{MaximalChains (for IsPoset)}}
\logpage{[ 8, 2, 1 ]}\nobreak
\hyperdef{L}{X81E0832F7B92DB7A}{}
{\noindent\textcolor{FuncColor}{$\triangleright$\enspace\texttt{MaximalChains({\mdseries\slshape poset})\index{MaximalChains@\texttt{MaximalChains}!for IsPoset}
\label{MaximalChains:for IsPoset}
}\hfill{\scriptsize (attribute)}}\\


 Gives the list of maximal chains in a poset in terms of the elements of the
poset. Synonyms are \texttt{FlagsList} and \texttt{Flags}. Tends to work faster (sometimes significantly) if the poset \texttt{HasPartialOrder}. }

 Synonym is \texttt{FlagsList}. 
\begin{Verbatim}[commandchars=!@|,fontsize=\small,frame=single,label=Example]
  !gapprompt@gap>| !gapinput@poset:=PosetFromManiplex(HemiCube(3));|
  A poset using the IsPosetOfFlags representation.
  !gapprompt@gap>| !gapinput@MaximalChains(poset)[1];|
  [ An element of a poset made of flags, An element of a poset made of flags, 
    An element of a poset made of flags, An element of a poset made of flags, 
    An element of a poset made of flags ]
  !gapprompt@gap>| !gapinput@List(last,x->RankInPoset(x,poset));|
  [ -1, 0, 1, 2, 3 ]
\end{Verbatim}
 

\subsection{\textcolor{Chapter }{RankPoset (for IsPoset)}}
\logpage{[ 8, 2, 2 ]}\nobreak
\hyperdef{L}{X7C9B4AA77C73A761}{}
{\noindent\textcolor{FuncColor}{$\triangleright$\enspace\texttt{RankPoset({\mdseries\slshape poset})\index{RankPoset@\texttt{RankPoset}!for IsPoset}
\label{RankPoset:for IsPoset}
}\hfill{\scriptsize (attribute)}}\\


 If the poset \texttt{IsP1}, ranks are assumed to run from $-1$ to $n$, and function will return $n$. If \texttt{IsP1(poset)=false}, ranks are assumed to run from 1 to $n$. In RAMP, at least currently, we are assuming that graded/ranked posets are
bounded. Note that in general what you \emph{actually} want to do is call \texttt{Rank(poset)}. The reason is that \texttt{Rank} will calculate the \texttt{RankPoset} if it isn't set, and then set and store the value in the poset. }

 

\subsection{\textcolor{Chapter }{ElementsList (for IsPoset)}}
\logpage{[ 8, 2, 3 ]}\nobreak
\label{elements}
\hyperdef{L}{X8448C7067B0EF49D}{}
{\noindent\textcolor{FuncColor}{$\triangleright$\enspace\texttt{ElementsList({\mdseries\slshape poset})\index{ElementsList@\texttt{ElementsList}!for IsPoset}
\label{ElementsList:for IsPoset}
}\hfill{\scriptsize (attribute)}}\\


 Will recover the list of faces of the poset, format may depend on \emph{type} of representation of \texttt{poset}. 
\begin{itemize}
\item  We also have \texttt{FacesList} and \texttt{Faces} as synonyms for this command. 
\end{itemize}
 }

 

\subsection{\textcolor{Chapter }{OrderingFunction (for IsPoset)}}
\logpage{[ 8, 2, 4 ]}\nobreak
\hyperdef{L}{X7C5580EC83A4955B}{}
{\noindent\textcolor{FuncColor}{$\triangleright$\enspace\texttt{OrderingFunction({\mdseries\slshape poset})\index{OrderingFunction@\texttt{OrderingFunction}!for IsPoset}
\label{OrderingFunction:for IsPoset}
}\hfill{\scriptsize (attribute)}}\\


 \texttt{OrderingFunction} is an attribute of a poset which stores a function for ordering elements. }

 
\begin{Verbatim}[commandchars=!@|,fontsize=\small,frame=single,label=Example]
  !gapprompt@gap>| !gapinput@p:=PosetFromManiplex(Cube(2));;|
  !gapprompt@gap>| !gapinput@p3:=PosetFromElements(RankedFaceListOfPoset(p),PairCompareFlagsList);;|
  !gapprompt@gap>| !gapinput@f3:=FacesList(p3);;|
  !gapprompt@gap>| !gapinput@OrderingFunction(p3)(ElementObject(f3[2]),ElementObject(f3[1]));|
  true
  !gapprompt@gap>| !gapinput@OrderingFunction(p3)(ElementObject(f3[1]),ElementObject(f3[2]));|
  false
\end{Verbatim}
 

\subsection{\textcolor{Chapter }{IsFlaggable (for IsPoset)}}
\logpage{[ 8, 2, 5 ]}\nobreak
\hyperdef{L}{X842B0F7E81670077}{}
{\noindent\textcolor{FuncColor}{$\triangleright$\enspace\texttt{IsFlaggable({\mdseries\slshape poset})\index{IsFlaggable@\texttt{IsFlaggable}!for IsPoset}
\label{IsFlaggable:for IsPoset}
}\hfill{\scriptsize (property)}}\\
\textbf{\indent Returns:\ }
\texttt{true} or \texttt{false} 



 Checks or creates the value of the attribute \texttt{IsFlaggable} for an \texttt{IsPoset}. Point here is to see if the structure of the poset is sufficient to
determine the flag graph. For IsPosetOfFlags this is another way of saying
that the intersection of the faces (thought of as collections of flags)
containing a flag is that selfsame flag. (Might be equivalent to
prepolytopal... but Gabe was tired and Gordon hasn't bothered to think about
it yet.) Now also works with generic poset element types (not just \texttt{IsPosetOfFlags}). }

 

\subsection{\textcolor{Chapter }{IsAtomic (for IsPoset)}}
\logpage{[ 8, 2, 6 ]}\nobreak
\hyperdef{L}{X816460188169E650}{}
{\noindent\textcolor{FuncColor}{$\triangleright$\enspace\texttt{IsAtomic({\mdseries\slshape poset})\index{IsAtomic@\texttt{IsAtomic}!for IsPoset}
\label{IsAtomic:for IsPoset}
}\hfill{\scriptsize (property)}}\\
\textbf{\indent Returns:\ }
\texttt{true} or \texttt{false} 



 This checks whether or not the faces of an IsP1 poset may be described
uniquely in terms of the posets atoms. 

 The terminology as used here is approximately that of Ziegler's \emph{Lectures on Polytopes} where a lattice is atomic if every element is the join of atoms. }

 
\begin{Verbatim}[commandchars=!@|,fontsize=\small,frame=single,label=Example]
  !gapprompt@gap>| !gapinput@po:=BinaryRelationOnPoints([[2,3],[4,5],[4,5],[6],[6],[]]);;|
  !gapprompt@gap>| !gapinput@po:=ReflexiveClosureBinaryRelation(TransitiveClosureBinaryRelation(po));;|
  !gapprompt@gap>| !gapinput@p:=PosetFromPartialOrder(po);; IsAtomic(p);|
  false
  !gapprompt@gap>| !gapinput@p2:=PosetFromManiplex(Cube(3));; IsAtomic(p2);|
  true
\end{Verbatim}
 

\subsection{\textcolor{Chapter }{PartialOrder (for IsPoset)}}
\logpage{[ 8, 2, 7 ]}\nobreak
\hyperdef{L}{X85A9CAA682FE56EB}{}
{\noindent\textcolor{FuncColor}{$\triangleright$\enspace\texttt{PartialOrder({\mdseries\slshape poset})\index{PartialOrder@\texttt{PartialOrder}!for IsPoset}
\label{PartialOrder:for IsPoset}
}\hfill{\scriptsize (attribute)}}\\
\textbf{\indent Returns:\ }
\texttt{partial order} 



 HasPartialOrder Checks if \mbox{\texttt{\mdseries\slshape poset}} has a declared partial order (binary relation). SetPartialOrder assigns a
partial order to the \mbox{\texttt{\mdseries\slshape poset}}. In many cases, PartialOrder is able to compute one from structural
information. }

 
\subsection{\textcolor{Chapter }{Lattices}}\label{Lattices}
\logpage{[ 8, 2, 8 ]}
\hyperdef{L}{X851265E17D42FCAD}{}
{
\noindent\textcolor{FuncColor}{$\triangleright$\enspace\texttt{IsLattice({\mdseries\slshape poset})\index{IsLattice@\texttt{IsLattice}!for IsPoset}
\label{IsLattice:for IsPoset}
}\hfill{\scriptsize (property)}}\\
\noindent\textcolor{FuncColor}{$\triangleright$\enspace\texttt{IsAllMeets({\mdseries\slshape arg})\index{IsAllMeets@\texttt{IsAllMeets}!for IsPoset}
\label{IsAllMeets:for IsPoset}
}\hfill{\scriptsize (property)}}\\
\noindent\textcolor{FuncColor}{$\triangleright$\enspace\texttt{IsAllJoins({\mdseries\slshape arg})\index{IsAllJoins@\texttt{IsAllJoins}!for IsPoset}
\label{IsAllJoins:for IsPoset}
}\hfill{\scriptsize (property)}}\\
\textbf{\indent Returns:\ }
IsBool 



 IsLattice determines whether a poset is a lattice or not. IsAllMeets
determines whether all meets in a poset are unique. IsAllJoins determines
whether all joins in a poset are unique. }

 
\begin{Verbatim}[commandchars=!@|,fontsize=\small,frame=single,label=Example]
  !gapprompt@gap>| !gapinput@poset:=PosetFromManiplex(Cube(3));;|
  !gapprompt@gap>| !gapinput@IsLattice(poset);|
  true
  !gapprompt@gap>| !gapinput@bad:=PosetFromManiplex(HemiCube(3));;|
  !gapprompt@gap>| !gapinput@IsLattice(bad);|
  fail
\end{Verbatim}
 Here's a simple example of when a lattice isn't atomic. 
\begin{Verbatim}[commandchars=!@|,fontsize=\small,frame=single,label=Example]
  !gapprompt@gap>| !gapinput@l:=[[2,3,4],[5,7],[5,6],[6,7],[8],[8],[8,9],[10],[10],[]];;|
  !gapprompt@gap>| !gapinput@b:=BinaryRelationOnPoints(l);; |
  po:=ReflexiveClosureBinaryRelation(TransitiveClosureBinaryRelation(b));;
  !gapprompt@gap>| !gapinput@poset:=PosetFromPartialOrder(po);;|
  !gapprompt@gap>| !gapinput@IsLattice(poset);|
  true
  !gapprompt@gap>| !gapinput@IsAtomic(poset);|
  false
\end{Verbatim}
 

\subsection{\textcolor{Chapter }{ListIsP1Poset (for IsList)}}
\logpage{[ 8, 2, 9 ]}\nobreak
\hyperdef{L}{X8455987F7E9A88FF}{}
{\noindent\textcolor{FuncColor}{$\triangleright$\enspace\texttt{ListIsP1Poset({\mdseries\slshape list})\index{ListIsP1Poset@\texttt{ListIsP1Poset}!for IsList}
\label{ListIsP1Poset:for IsList}
}\hfill{\scriptsize (operation)}}\\
\textbf{\indent Returns:\ }
\texttt{true} or \texttt{false} 



 Given \mbox{\texttt{\mdseries\slshape list}}, comprised of sublists of faces ordered by rank, each face listing the flags
on the face, this function will tell you if the list corresponds to a P1 poset
or not. }

 

\subsection{\textcolor{Chapter }{IsP1 (for IsPoset)}}
\logpage{[ 8, 2, 10 ]}\nobreak
\hyperdef{L}{X81269ACB7EF202C4}{}
{\noindent\textcolor{FuncColor}{$\triangleright$\enspace\texttt{IsP1({\mdseries\slshape poset})\index{IsP1@\texttt{IsP1}!for IsPoset}
\label{IsP1:for IsPoset}
}\hfill{\scriptsize (property)}}\\
\textbf{\indent Returns:\ }
\texttt{true} or \texttt{false} 



 Determines whether a poset has property P1 from ARP. Recall that a poset is P1
if it has a unique least, and a unique maximal element/face. }

 
\begin{Verbatim}[commandchars=!@|,fontsize=\small,frame=single,label=Example]
  !gapprompt@gap>| !gapinput@p:=PosetFromElements(AllSubgroups(AlternatingGroup(4)),IsSubgroup);|
  A poset using the IsPosetOfIndices representation 
  !gapprompt@gap>| !gapinput@IsP1(p);|
  true
  !gapprompt@gap>| !gapinput@p2:=PosetFromFaceListOfFlags([[[1],[2]],[[1,2]]]);|
  A poset using the IsPosetOfFlags representation with 3 faces.
  !gapprompt@gap>| !gapinput@IsP1(p2);|
  false
\end{Verbatim}
 

\subsection{\textcolor{Chapter }{IsP2 (for IsPoset)}}
\logpage{[ 8, 2, 11 ]}\nobreak
\hyperdef{L}{X7AFA3C077FD9583A}{}
{\noindent\textcolor{FuncColor}{$\triangleright$\enspace\texttt{IsP2({\mdseries\slshape poset})\index{IsP2@\texttt{IsP2}!for IsPoset}
\label{IsP2:for IsPoset}
}\hfill{\scriptsize (property)}}\\
\textbf{\indent Returns:\ }
\texttt{true} or \texttt{false} 



 Determines whether a poset has property P2 from ARP. Recall that a poset is P2
if each maximal chain in the poset has the same length (for $n$-polytopes, this means each flag containes $n+2$ faces). }

 
\begin{Verbatim}[commandchars=!@|,fontsize=\small,frame=single,label=Example]
  !gapprompt@gap>| !gapinput@poset:=PosetFromManiplex(HemiCube(3)); |
  !gapprompt@gap>| !gapinput@IsP2(poset);|
  true
\end{Verbatim}
 Another nice example 
\begin{Verbatim}[commandchars=!@|,fontsize=\small,frame=single,label=Example]
  !gapprompt@gap>| !gapinput@g:=AlternatingGroup(4);; a:=AllSubgroups(g);; poset:=PosetFromElements(a,IsSubgroup);|
  A poset using the IsPosetOfIndices representation 
  !gapprompt@gap>| !gapinput@IsP2(poset);|
  false
\end{Verbatim}
 

\subsection{\textcolor{Chapter }{IsP3 (for IsPoset)}}
\logpage{[ 8, 2, 12 ]}\nobreak
\hyperdef{L}{X8723519F84529E4C}{}
{\noindent\textcolor{FuncColor}{$\triangleright$\enspace\texttt{IsP3({\mdseries\slshape poset})\index{IsP3@\texttt{IsP3}!for IsPoset}
\label{IsP3:for IsPoset}
}\hfill{\scriptsize (property)}}\\
\textbf{\indent Returns:\ }
\texttt{true} or \texttt{false} 



 Determines whether a poset is strongly flag connected (property P3' from ARP).
May also be called with command \texttt{IsStronglyFlagConnected}. If you are not working with a pre-polytope, expect this to take a LONG time.
This means that given flags $\Phi$ and $\Psi$, not only is there a sequence of flags $\Psi=\Phi_0=\Phi_1=\cdots=\Phi_k=\Psi$ such that each $\Phi_i$ shares all but once face with $\Phi_{i+1}$, but that each $\Phi_i\supseteq \Phi\cap\Psi$. }

 Helper for IsP3 

\subsection{\textcolor{Chapter }{IsFlagConnected (for IsPoset)}}
\logpage{[ 8, 2, 13 ]}\nobreak
\hyperdef{L}{X85F554347C5AEF30}{}
{\noindent\textcolor{FuncColor}{$\triangleright$\enspace\texttt{IsFlagConnected({\mdseries\slshape poset})\index{IsFlagConnected@\texttt{IsFlagConnected}!for IsPoset}
\label{IsFlagConnected:for IsPoset}
}\hfill{\scriptsize (property)}}\\
\textbf{\indent Returns:\ }
\texttt{true} or \texttt{false} 



 Determines whether a poset is flag connected. }

 

\subsection{\textcolor{Chapter }{IsP4 (for IsPoset)}}
\logpage{[ 8, 2, 14 ]}\nobreak
\hyperdef{L}{X80F461FA7D8FEDC7}{}
{\noindent\textcolor{FuncColor}{$\triangleright$\enspace\texttt{IsP4({\mdseries\slshape poset})\index{IsP4@\texttt{IsP4}!for IsPoset}
\label{IsP4:for IsPoset}
}\hfill{\scriptsize (property)}}\\
\textbf{\indent Returns:\ }
\texttt{true} or \texttt{false} 



 Determines whether a poset satisfies the diamond condition. May also be
invoked using \texttt{IsDiamondCondition}. Recall that this means that if $F,G$ elements of the poset of ranks $i-1$ and $i+1$, respectively, where $F$ less than $G$, then there are precisely two $i$-faces $H$ such that $F$ is less than $H$ and $H$ is less than $G$. }

 

\subsection{\textcolor{Chapter }{IsPolytope (for IsPoset)}}
\logpage{[ 8, 2, 15 ]}\nobreak
\hyperdef{L}{X83D63725789E745D}{}
{\noindent\textcolor{FuncColor}{$\triangleright$\enspace\texttt{IsPolytope({\mdseries\slshape poset})\index{IsPolytope@\texttt{IsPolytope}!for IsPoset}
\label{IsPolytope:for IsPoset}
}\hfill{\scriptsize (property)}}\\
\textbf{\indent Returns:\ }
\texttt{true} or \texttt{false} 



 Determines whether a poset is an abstract polytope. }

 
\begin{Verbatim}[commandchars=!@|,fontsize=\small,frame=single,label=Example]
  !gapprompt@gap>| !gapinput@poset:=PosetFromManiplex(Cube(3));|
  A poset using the IsPosetOfFlags representation with 28 faces.
  !gapprompt@gap>| !gapinput@IsPolytope(poset);|
  true
  !gapprompt@gap>| !gapinput@KnownPropertiesOfObject(poset);|
  [ "IsP1", "IsP2", "IsP3", "IsP4", "IsPolytope" ]
  !gapprompt@gap>| !gapinput@poset2:=PosetFromElements(AllSubgroups(AlternatingGroup(4)),IsSubgroup);|
  A poset using the IsPosetOfIndices representation 
  !gapprompt@gap>| !gapinput@IsPolytope(poset2);|
  false
  !gapprompt@gap>| !gapinput@KnownPropertiesOfObject(poset2);|
  [ "IsP1", "IsP2", "IsPolytope" ]
\end{Verbatim}
 

\subsection{\textcolor{Chapter }{IsPrePolytope (for IsPoset)}}
\logpage{[ 8, 2, 16 ]}\nobreak
\hyperdef{L}{X8085287683CCF81B}{}
{\noindent\textcolor{FuncColor}{$\triangleright$\enspace\texttt{IsPrePolytope({\mdseries\slshape poset})\index{IsPrePolytope@\texttt{IsPrePolytope}!for IsPoset}
\label{IsPrePolytope:for IsPoset}
}\hfill{\scriptsize (property)}}\\
\textbf{\indent Returns:\ }
\texttt{true} or \texttt{false} 



 Determines whether a poset is an abstract pre-polytope. }

 

\subsection{\textcolor{Chapter }{IsSelfDual (for IsPoset)}}
\logpage{[ 8, 2, 17 ]}\nobreak
\hyperdef{L}{X7876E8577DB97677}{}
{\noindent\textcolor{FuncColor}{$\triangleright$\enspace\texttt{IsSelfDual({\mdseries\slshape poset})\index{IsSelfDual@\texttt{IsSelfDual}!for IsPoset}
\label{IsSelfDual:for IsPoset}
}\hfill{\scriptsize (property)}}\\
\textbf{\indent Returns:\ }
IsBool 



 Determines whether a poset is self dual. }

 
\begin{Verbatim}[commandchars=!@|,fontsize=\small,frame=single,label=Example]
  !gapprompt@gap>| !gapinput@poset:=PosetFromManiplex(Simplex(5));;|
  A poset using the IsPosetOfFlags representation.
  !gapprompt@gap>| !gapinput@IsSelfDual(poset);|
  true
  !gapprompt@gap>| !gapinput@poset2:=PosetFromManiplex(PyramidOver(Cube(3)));;|
  !gapprompt@gap>| !gapinput@IsSelfDual(poset2);|
  false
\end{Verbatim}
 }

 
\section{\textcolor{Chapter }{Working with posets}}\label{Chapter_Posets_Section_Working_with_posets}
\logpage{[ 8, 3, 0 ]}
\hyperdef{L}{X8163385F7B822934}{}
{
  

\subsection{\textcolor{Chapter }{IsIsomorphicPoset (for IsPoset,IsPoset)}}
\logpage{[ 8, 3, 1 ]}\nobreak
\hyperdef{L}{X80A2785B7D079FD1}{}
{\noindent\textcolor{FuncColor}{$\triangleright$\enspace\texttt{IsIsomorphicPoset({\mdseries\slshape poset1, poset2})\index{IsIsomorphicPoset@\texttt{IsIsomorphicPoset}!for IsPoset,IsPoset}
\label{IsIsomorphicPoset:for IsPoset,IsPoset}
}\hfill{\scriptsize (operation)}}\\
\textbf{\indent Returns:\ }
\texttt{true} or \texttt{false} 



 Determines whether \mbox{\texttt{\mdseries\slshape poset1}} and \mbox{\texttt{\mdseries\slshape poset2}} are isomorphic by checking to see if their Hasse diagrams are isomorphic. }

 
\begin{Verbatim}[commandchars=!@|,fontsize=\small,frame=single,label=Example]
  !gapprompt@gap>| !gapinput@ IsIsomorphicPoset( PosetFromManiplex( PyramidOver( Cube(3) ) ),  PosetFromManiplex( PrismOver (PyramidOver( Cube(2) ) ) ) );|
  false
  !gapprompt@gap>| !gapinput@ IsIsomorphicPoset( PosetFromManiplex( PyramidOver( Cube(3) ) ), PosetFromManiplex( PyramidOver( PrismOver( Cube(2) ) ) ) );|
  true
\end{Verbatim}
 

\subsection{\textcolor{Chapter }{PosetIsomorphism (for IsPoset,IsPoset)}}
\logpage{[ 8, 3, 2 ]}\nobreak
\hyperdef{L}{X8340F2597BD837A5}{}
{\noindent\textcolor{FuncColor}{$\triangleright$\enspace\texttt{PosetIsomorphism({\mdseries\slshape poset1, poset2})\index{PosetIsomorphism@\texttt{PosetIsomorphism}!for IsPoset,IsPoset}
\label{PosetIsomorphism:for IsPoset,IsPoset}
}\hfill{\scriptsize (operation)}}\\
\textbf{\indent Returns:\ }
map on face indices 



 When \mbox{\texttt{\mdseries\slshape poset1}} and \mbox{\texttt{\mdseries\slshape poset2}} are isomorphic, will give you a map from the faces of \mbox{\texttt{\mdseries\slshape poset1}} to the faces of \mbox{\texttt{\mdseries\slshape poset2}}. }

 

\subsection{\textcolor{Chapter }{FlagsAsFlagListFaces (for IsPoset)}}
\logpage{[ 8, 3, 3 ]}\nobreak
\hyperdef{L}{X828C2C297FAAB627}{}
{\noindent\textcolor{FuncColor}{$\triangleright$\enspace\texttt{FlagsAsFlagListFaces({\mdseries\slshape poset})\index{FlagsAsFlagListFaces@\texttt{FlagsAsFlagListFaces}!for IsPoset}
\label{FlagsAsFlagListFaces:for IsPoset}
}\hfill{\scriptsize (operation)}}\\
\textbf{\indent Returns:\ }
\texttt{IsList} 



 Given a \mbox{\texttt{\mdseries\slshape poset}}, this will give you a version of the list of flags in terms of the proper
faces described in the \mbox{\texttt{\mdseries\slshape poset}}; i.e., this gives a list of flags where each face is described in terms of
its (enumerated) list of incident flags. Note that the flag list does not
include the minimal face or the maximal face if the poset IsP2. }

 

\subsection{\textcolor{Chapter }{RankedFaceListOfPoset (for IsPoset)}}
\logpage{[ 8, 3, 4 ]}\nobreak
\hyperdef{L}{X7A80223E83EF52E8}{}
{\noindent\textcolor{FuncColor}{$\triangleright$\enspace\texttt{RankedFaceListOfPoset({\mdseries\slshape IsPosetOfFlags})\index{RankedFaceListOfPoset@\texttt{RankedFaceListOfPoset}!for IsPoset}
\label{RankedFaceListOfPoset:for IsPoset}
}\hfill{\scriptsize (operation)}}\\
\textbf{\indent Returns:\ }
\texttt{list} 



 Gives a list of [\mbox{\texttt{\mdseries\slshape face}},\mbox{\texttt{\mdseries\slshape rank}}] pairs for all the faces of \mbox{\texttt{\mdseries\slshape poset}}. Assumptions here are that faces are lists of incident flags. }

 

\subsection{\textcolor{Chapter }{AdjacentFlag (for IsPosetOfFlags,IsList,IsInt)}}
\logpage{[ 8, 3, 5 ]}\nobreak
\hyperdef{L}{X7ACE88EC803EDE07}{}
{\noindent\textcolor{FuncColor}{$\triangleright$\enspace\texttt{AdjacentFlag({\mdseries\slshape poset, flag, i})\index{AdjacentFlag@\texttt{AdjacentFlag}!for IsPosetOfFlags,IsList,IsInt}
\label{AdjacentFlag:for IsPosetOfFlags,IsList,IsInt}
}\hfill{\scriptsize (operation)}}\\
\textbf{\indent Returns:\ }
\texttt{flag(s)} 



 Given a poset, a flag, and a rank, this function will give you the \mbox{\texttt{\mdseries\slshape i}}-adjacent flag. Note that adjacencies are listed from ranks 0 to one less than
the dimension. You can replace \mbox{\texttt{\mdseries\slshape flag}} with the integer corresponding to that flag. Appending \texttt{true} to the arguments will give the position of the flag instead of its description
from \texttt{FlagsAsFlagListFaces}. }

 

\subsection{\textcolor{Chapter }{AdjacentFlags (for IsPoset,IsList,IsInt)}}
\logpage{[ 8, 3, 6 ]}\nobreak
\hyperdef{L}{X7C94E895794BA41F}{}
{\noindent\textcolor{FuncColor}{$\triangleright$\enspace\texttt{AdjacentFlags({\mdseries\slshape poset, flagaslistoffaces, adjacencyrank})\index{AdjacentFlags@\texttt{AdjacentFlags}!for IsPoset,IsList,IsInt}
\label{AdjacentFlags:for IsPoset,IsList,IsInt}
}\hfill{\scriptsize (operation)}}\\


 If your poset isn't P4, there may be multiple adjacent maximal chains at a
given rank. This function handles that case. May substitute \texttt{IsInt} for \texttt{flagaslistoffaces} corresponding to position of \texttt{flag} in list of maximal chains. }

 

\subsection{\textcolor{Chapter }{EqualChains (for IsList,IsList)}}
\logpage{[ 8, 3, 7 ]}\nobreak
\hyperdef{L}{X7E1C59DF835343FE}{}
{\noindent\textcolor{FuncColor}{$\triangleright$\enspace\texttt{EqualChains({\mdseries\slshape flag1, flag2})\index{EqualChains@\texttt{EqualChains}!for IsList,IsList}
\label{EqualChains:for IsList,IsList}
}\hfill{\scriptsize (operation)}}\\


 Determines whether two chains are equal. }

 

\subsection{\textcolor{Chapter }{ConnectionGeneratorOfPoset (for IsPoset,IsInt)}}
\logpage{[ 8, 3, 8 ]}\nobreak
\hyperdef{L}{X84E3B63C8639771B}{}
{\noindent\textcolor{FuncColor}{$\triangleright$\enspace\texttt{ConnectionGeneratorOfPoset({\mdseries\slshape poset, i})\index{ConnectionGeneratorOfPoset@\texttt{ConnectionGeneratorOfPoset}!for IsPoset,IsInt}
\label{ConnectionGeneratorOfPoset:for IsPoset,IsInt}
}\hfill{\scriptsize (operation)}}\\
\textbf{\indent Returns:\ }
A permutation on the flags. 



 Given a \mbox{\texttt{\mdseries\slshape poset}} and an integer $i$, this function will give you the associated permutation for the rank $i$-connection. }

 

\subsection{\textcolor{Chapter }{ConnectionGroup (for IsPoset)}}
\logpage{[ 8, 3, 9 ]}\nobreak
\hyperdef{L}{X859C651184ED9424}{}
{\noindent\textcolor{FuncColor}{$\triangleright$\enspace\texttt{ConnectionGroup({\mdseries\slshape poset})\index{ConnectionGroup@\texttt{ConnectionGroup}!for IsPoset}
\label{ConnectionGroup:for IsPoset}
}\hfill{\scriptsize (attribute)}}\\
\textbf{\indent Returns:\ }
\texttt{IsPermGroup} 



 Given a \mbox{\texttt{\mdseries\slshape poset}} that is \texttt{IsPrePolytope}, this function will give you the connection group. }

 

\subsection{\textcolor{Chapter }{AutomorphismGroup (for IsPoset)}}
\logpage{[ 8, 3, 10 ]}\nobreak
\hyperdef{L}{X7C6813D27F0BD0F8}{}
{\noindent\textcolor{FuncColor}{$\triangleright$\enspace\texttt{AutomorphismGroup({\mdseries\slshape poset})\index{AutomorphismGroup@\texttt{AutomorphismGroup}!for IsPoset}
\label{AutomorphismGroup:for IsPoset}
}\hfill{\scriptsize (attribute)}}\\


 Given a \mbox{\texttt{\mdseries\slshape poset}}, gives the automorphism group of the poset as an action on the maximal
chains. }

 

\subsection{\textcolor{Chapter }{AutomorphismGroupOnElements (for IsPoset)}}
\logpage{[ 8, 3, 11 ]}\nobreak
\hyperdef{L}{X80C2783D7FD60E56}{}
{\noindent\textcolor{FuncColor}{$\triangleright$\enspace\texttt{AutomorphismGroupOnElements({\mdseries\slshape poset})\index{AutomorphismGroupOnElements@\texttt{AutomorphismGroupOnElements}!for IsPoset}
\label{AutomorphismGroupOnElements:for IsPoset}
}\hfill{\scriptsize (attribute)}}\\


 Given a \mbox{\texttt{\mdseries\slshape poset}}, gives the automorphism group of the poset as an action on the elements. }

 

\subsection{\textcolor{Chapter }{AutomorphismGroupOnChains (for IsPoset, IsCollection)}}
\logpage{[ 8, 3, 12 ]}\nobreak
\hyperdef{L}{X8359754487DF73D9}{}
{\noindent\textcolor{FuncColor}{$\triangleright$\enspace\texttt{AutomorphismGroupOnChains({\mdseries\slshape poset, I})\index{AutomorphismGroupOnChains@\texttt{AutomorphismGroupOnChains}!for IsPoset, IsCollection}
\label{AutomorphismGroupOnChains:for IsPoset, IsCollection}
}\hfill{\scriptsize (operation)}}\\
\textbf{\indent Returns:\ }
group 



 Returns the permutation group, representing the action of the automorphism
group of \mbox{\texttt{\mdseries\slshape poset}} on the chains of \mbox{\texttt{\mdseries\slshape poset}} of type \mbox{\texttt{\mdseries\slshape I}}. }

 
\begin{Verbatim}[commandchars=!@|,fontsize=\small,frame=single,label=Example]
  gap>
\end{Verbatim}
 

\subsection{\textcolor{Chapter }{AutomorphismGroupOnIFaces (for IsPoset, IsInt)}}
\logpage{[ 8, 3, 13 ]}\nobreak
\hyperdef{L}{X80355CD07D65A962}{}
{\noindent\textcolor{FuncColor}{$\triangleright$\enspace\texttt{AutomorphismGroupOnIFaces({\mdseries\slshape poset, i})\index{AutomorphismGroupOnIFaces@\texttt{AutomorphismGroupOnIFaces}!for IsPoset, IsInt}
\label{AutomorphismGroupOnIFaces:for IsPoset, IsInt}
}\hfill{\scriptsize (operation)}}\\
\textbf{\indent Returns:\ }
group 



 Returns the permutation group, representing the action of the automorphism
group of \mbox{\texttt{\mdseries\slshape poset}} on the faces of \mbox{\texttt{\mdseries\slshape poset}} of rank \mbox{\texttt{\mdseries\slshape I}}. }

 

\subsection{\textcolor{Chapter }{AutomorphismGroupOnFacets (for IsPoset)}}
\logpage{[ 8, 3, 14 ]}\nobreak
\hyperdef{L}{X7E0F912A7D9C4743}{}
{\noindent\textcolor{FuncColor}{$\triangleright$\enspace\texttt{AutomorphismGroupOnFacets({\mdseries\slshape poset})\index{AutomorphismGroupOnFacets@\texttt{AutomorphismGroupOnFacets}!for IsPoset}
\label{AutomorphismGroupOnFacets:for IsPoset}
}\hfill{\scriptsize (attribute)}}\\
\textbf{\indent Returns:\ }
group 



 Returns the permutation group, representing the action of the automorphism
group of \mbox{\texttt{\mdseries\slshape poset}} on the faces of \mbox{\texttt{\mdseries\slshape poset}} of rank $d-1$. }

 

\subsection{\textcolor{Chapter }{AutomorphismGroupOnEdges (for IsPoset)}}
\logpage{[ 8, 3, 15 ]}\nobreak
\hyperdef{L}{X86EA69DA7E2B35B9}{}
{\noindent\textcolor{FuncColor}{$\triangleright$\enspace\texttt{AutomorphismGroupOnEdges({\mdseries\slshape poset})\index{AutomorphismGroupOnEdges@\texttt{AutomorphismGroupOnEdges}!for IsPoset}
\label{AutomorphismGroupOnEdges:for IsPoset}
}\hfill{\scriptsize (attribute)}}\\
\textbf{\indent Returns:\ }
group 



 Returns the permutation group, representing the action of the automorphism
group of \mbox{\texttt{\mdseries\slshape poset}} on the faces of \mbox{\texttt{\mdseries\slshape poset}} of rank 1. }

 

\subsection{\textcolor{Chapter }{AutomorphismGroupOnVertices (for IsPoset)}}
\logpage{[ 8, 3, 16 ]}\nobreak
\hyperdef{L}{X784F0CFE7DAC8C49}{}
{\noindent\textcolor{FuncColor}{$\triangleright$\enspace\texttt{AutomorphismGroupOnVertices({\mdseries\slshape poset})\index{AutomorphismGroupOnVertices@\texttt{AutomorphismGroupOnVertices}!for IsPoset}
\label{AutomorphismGroupOnVertices:for IsPoset}
}\hfill{\scriptsize (attribute)}}\\
\textbf{\indent Returns:\ }
group 



 Returns the permutation group, representing the action of the automorphism
group of \mbox{\texttt{\mdseries\slshape poset}} on the faces of \mbox{\texttt{\mdseries\slshape poset}} of rank 0. }

 

\subsection{\textcolor{Chapter }{FaceListOfPoset (for IsPoset)}}
\logpage{[ 8, 3, 17 ]}\nobreak
\hyperdef{L}{X79F1837C835A5EFF}{}
{\noindent\textcolor{FuncColor}{$\triangleright$\enspace\texttt{FaceListOfPoset({\mdseries\slshape poset})\index{FaceListOfPoset@\texttt{FaceListOfPoset}!for IsPoset}
\label{FaceListOfPoset:for IsPoset}
}\hfill{\scriptsize (operation)}}\\
\textbf{\indent Returns:\ }
\texttt{list} 



 Gives a list of faces collected into lists ordered by increasing rank.
Suitable as input for \texttt{PosetFromFaceListOfFlags}. Argument must be IsPosetOfFlags. }

 

\subsection{\textcolor{Chapter }{RankPosetElements (for IsPoset)}}
\logpage{[ 8, 3, 18 ]}\nobreak
\hyperdef{L}{X7980D74F7D8C1B95}{}
{\noindent\textcolor{FuncColor}{$\triangleright$\enspace\texttt{RankPosetElements({\mdseries\slshape poset})\index{RankPosetElements@\texttt{RankPosetElements}!for IsPoset}
\label{RankPosetElements:for IsPoset}
}\hfill{\scriptsize (operation)}}\\


 Assigns to each face of a poset (when possible) the rank of the element in the
poset. }

 

\subsection{\textcolor{Chapter }{FacesByRankOfPoset (for IsPoset)}}
\logpage{[ 8, 3, 19 ]}\nobreak
\hyperdef{L}{X82CBC68480F222C7}{}
{\noindent\textcolor{FuncColor}{$\triangleright$\enspace\texttt{FacesByRankOfPoset({\mdseries\slshape poset})\index{FacesByRankOfPoset@\texttt{FacesByRankOfPoset}!for IsPoset}
\label{FacesByRankOfPoset:for IsPoset}
}\hfill{\scriptsize (operation)}}\\
\textbf{\indent Returns:\ }
\texttt{list} 



 Gives lists of faces ordered by rank. Also sets the rank for each of the
faces. }

 

\subsection{\textcolor{Chapter }{HasseDiagramOfPoset (for IsPoset)}}
\logpage{[ 8, 3, 20 ]}\nobreak
\hyperdef{L}{X7AE5050E7F6182C8}{}
{\noindent\textcolor{FuncColor}{$\triangleright$\enspace\texttt{HasseDiagramOfPoset({\mdseries\slshape poset})\index{HasseDiagramOfPoset@\texttt{HasseDiagramOfPoset}!for IsPoset}
\label{HasseDiagramOfPoset:for IsPoset}
}\hfill{\scriptsize (operation)}}\\
\textbf{\indent Returns:\ }
directed graph 



 

 }

 

\subsection{\textcolor{Chapter }{AsPosetOfAtoms (for IsPoset)}}
\logpage{[ 8, 3, 21 ]}\nobreak
\hyperdef{L}{X83ACE1CB7DE65C44}{}
{\noindent\textcolor{FuncColor}{$\triangleright$\enspace\texttt{AsPosetOfAtoms({\mdseries\slshape poset})\index{AsPosetOfAtoms@\texttt{AsPosetOfAtoms}!for IsPoset}
\label{AsPosetOfAtoms:for IsPoset}
}\hfill{\scriptsize (operation)}}\\
\textbf{\indent Returns:\ }
posetFromAtoms 



 If \mbox{\texttt{\mdseries\slshape poset}} is an IsP1 poset admits a description of its elements in terms of its atoms,
this function will construct an isomorphic poset whose faces are described
using PosetFromAtomList. }

 
\begin{Verbatim}[commandchars=!@|,fontsize=\small,frame=single,label=Example]
  !gapprompt@gap>| !gapinput@poset:=PosetFromManiplex(Cube(2));;|
  !gapprompt@gap>| !gapinput@p2:=AsPosetOfAtoms(poset);|
  A poset on 10 elements using the IsPosetOfIndices representation.
  !gapprompt@gap>| !gapinput@IsIsomorphicPoset(poset,p2);|
  true
\end{Verbatim}
 
\subsection{\textcolor{Chapter }{Max/min faces}}\label{Special_Faces}
\logpage{[ 8, 3, 22 ]}
\hyperdef{L}{X838947F98660C2C4}{}
{
\noindent\textcolor{FuncColor}{$\triangleright$\enspace\texttt{MinFace({\mdseries\slshape poset})\index{MinFace@\texttt{MinFace}!for IsPoset}
\label{MinFace:for IsPoset}
}\hfill{\scriptsize (operation)}}\\
\noindent\textcolor{FuncColor}{$\triangleright$\enspace\texttt{MaxFace({\mdseries\slshape arg})\index{MaxFace@\texttt{MaxFace}!for IsPoset}
\label{MaxFace:for IsPoset}
}\hfill{\scriptsize (operation)}}\\
\textbf{\indent Returns:\ }
face 



 Gives the minimal/maximal face of a \mbox{\texttt{\mdseries\slshape poset}} when it IsP1 and IsP2. 

 }

 }

 
\section{\textcolor{Chapter }{Element constructors}}\label{Chapter_Posets_Section_Element_constructors}
\logpage{[ 8, 4, 0 ]}
\hyperdef{L}{X85445BE07F161F88}{}
{
  

\subsection{\textcolor{Chapter }{PosetElementWithOrder (for IsObject,IsFunction)}}
\logpage{[ 8, 4, 1 ]}\nobreak
\hyperdef{L}{X81CE0D9C872BE42B}{}
{\noindent\textcolor{FuncColor}{$\triangleright$\enspace\texttt{PosetElementWithOrder({\mdseries\slshape obj, func})\index{PosetElementWithOrder@\texttt{PosetElementWithOrder}!for IsObject,IsFunction}
\label{PosetElementWithOrder:for IsObject,IsFunction}
}\hfill{\scriptsize (operation)}}\\
\textbf{\indent Returns:\ }
\texttt{IsFace} 



 Creates a \texttt{face} with \mbox{\texttt{\mdseries\slshape obj}} and ordering function \texttt{func}. Note that by convetiontion \texttt{func(a,b)} should return true when $b\le a$. }

 

\subsection{\textcolor{Chapter }{PosetElementFromListOfFlags (for IsList,IsPoset,IsInt)}}
\logpage{[ 8, 4, 2 ]}\nobreak
\hyperdef{L}{X79608DCB8749DEDC}{}
{\noindent\textcolor{FuncColor}{$\triangleright$\enspace\texttt{PosetElementFromListOfFlags({\mdseries\slshape list, poset, n})\index{PosetElementFromListOfFlags@\texttt{PosetElementFromListOfFlags}!for IsList,IsPoset,IsInt}
\label{PosetElementFromListOfFlags:for IsList,IsPoset,IsInt}
}\hfill{\scriptsize (operation)}}\\
\textbf{\indent Returns:\ }
\texttt{IsPosetElement} 



 This is used to create a face of rank \mbox{\texttt{\mdseries\slshape n}} from a \mbox{\texttt{\mdseries\slshape list}} of flags of \mbox{\texttt{\mdseries\slshape poset}}. }

 

\subsection{\textcolor{Chapter }{PosetElementFromAtomList (for IsList)}}
\logpage{[ 8, 4, 3 ]}\nobreak
\hyperdef{L}{X794F736784C735B0}{}
{\noindent\textcolor{FuncColor}{$\triangleright$\enspace\texttt{PosetElementFromAtomList({\mdseries\slshape list})\index{PosetElementFromAtomList@\texttt{PosetElementFromAtomList}!for IsList}
\label{PosetElementFromAtomList:for IsList}
}\hfill{\scriptsize (operation)}}\\
\textbf{\indent Returns:\ }
\texttt{IsFace} 



 Creates a \texttt{face} with \mbox{\texttt{\mdseries\slshape list}} of atoms. If you wish to assign ranks or membership in a poset, you must do
this separately. }

 

\subsection{\textcolor{Chapter }{PosetElementFromIndex (for IsObject)}}
\logpage{[ 8, 4, 4 ]}\nobreak
\hyperdef{L}{X7D7FDF76792A6C04}{}
{\noindent\textcolor{FuncColor}{$\triangleright$\enspace\texttt{PosetElementFromIndex({\mdseries\slshape obj})\index{PosetElementFromIndex@\texttt{PosetElementFromIndex}!for IsObject}
\label{PosetElementFromIndex:for IsObject}
}\hfill{\scriptsize (operation)}}\\
\textbf{\indent Returns:\ }
\texttt{IsFace} 



 Creates a \texttt{face} with index \mbox{\texttt{\mdseries\slshape obj}} at rank \mbox{\texttt{\mdseries\slshape n}}. }

 

\subsection{\textcolor{Chapter }{PosetElementWithPartialOrder (for IsObject, IsBinaryRelation)}}
\logpage{[ 8, 4, 5 ]}\nobreak
\hyperdef{L}{X8141EB127AA1FFE1}{}
{\noindent\textcolor{FuncColor}{$\triangleright$\enspace\texttt{PosetElementWithPartialOrder({\mdseries\slshape obj, order})\index{PosetElementWithPartialOrder@\texttt{PosetElementWithPartialOrder}!for IsObject, IsBinaryRelation}
\label{PosetElementWithPartialOrder:for IsObject, IsBinaryRelation}
}\hfill{\scriptsize (operation)}}\\
\textbf{\indent Returns:\ }
\texttt{IsFace} 



 Creates a \texttt{face} with index \mbox{\texttt{\mdseries\slshape obj}} and BinaryRelation \mbox{\texttt{\mdseries\slshape order}} on \mbox{\texttt{\mdseries\slshape obj}}. Function does not check to make sure \mbox{\texttt{\mdseries\slshape order}} has \mbox{\texttt{\mdseries\slshape obj}} in its domain. }

 

\subsection{\textcolor{Chapter }{RanksInPosets (for IsPosetElement)}}
\logpage{[ 8, 4, 6 ]}\nobreak
\hyperdef{L}{X7D43D66E82CC96DF}{}
{\noindent\textcolor{FuncColor}{$\triangleright$\enspace\texttt{RanksInPosets({\mdseries\slshape posetelement})\index{RanksInPosets@\texttt{RanksInPosets}!for IsPosetElement}
\label{RanksInPosets:for IsPosetElement}
}\hfill{\scriptsize (attribute)}}\\
\textbf{\indent Returns:\ }
list 



 Gives the \texttt{list} of posets \mbox{\texttt{\mdseries\slshape posetelement}} is in, and the corresponding rank (if available) as a list of ordered pairs of
the form \texttt{[poset,rank]}. \#! Note that this attribute is mutable, so if you modify it you may break
things. }

 

\subsection{\textcolor{Chapter }{AddRanksInPosets (for IsPosetElement,IsPoset,IsInt)}}
\logpage{[ 8, 4, 7 ]}\nobreak
\hyperdef{L}{X8399E8CD8022DF5F}{}
{\noindent\textcolor{FuncColor}{$\triangleright$\enspace\texttt{AddRanksInPosets({\mdseries\slshape posetelement, poset, int})\index{AddRanksInPosets@\texttt{AddRanksInPosets}!for IsPosetElement,IsPoset,IsInt}
\label{AddRanksInPosets:for IsPosetElement,IsPoset,IsInt}
}\hfill{\scriptsize (operation)}}\\
\textbf{\indent Returns:\ }
null 



 Adds an entry in the list of RanksInPosets for \mbox{\texttt{\mdseries\slshape posetelement}} corresponding to \mbox{\texttt{\mdseries\slshape poset}} with assigned rank \mbox{\texttt{\mdseries\slshape int}}. }

 

\subsection{\textcolor{Chapter }{FlagList (for IsPosetElement)}}
\logpage{[ 8, 4, 8 ]}\nobreak
\hyperdef{L}{X819B1AEF84E08B4D}{}
{\noindent\textcolor{FuncColor}{$\triangleright$\enspace\texttt{FlagList({\mdseries\slshape posetelement, \texttt{\symbol{123}}face\texttt{\symbol{125}}})\index{FlagList@\texttt{FlagList}!for IsPosetElement}
\label{FlagList:for IsPosetElement}
}\hfill{\scriptsize (attribute)}}\\
\textbf{\indent Returns:\ }
\texttt{list} 



 Description of \mbox{\texttt{\mdseries\slshape posetelement}} n as a list of incident flags (when present). }

 

\subsection{\textcolor{Chapter }{AtomList (for IsPosetElement)}}
\logpage{[ 8, 4, 9 ]}\nobreak
\hyperdef{L}{X878F991183754744}{}
{\noindent\textcolor{FuncColor}{$\triangleright$\enspace\texttt{AtomList({\mdseries\slshape posetelement, \texttt{\symbol{123}}face\texttt{\symbol{125}}})\index{AtomList@\texttt{AtomList}!for IsPosetElement}
\label{AtomList:for IsPosetElement}
}\hfill{\scriptsize (attribute)}}\\
\textbf{\indent Returns:\ }
\texttt{list} 



 Description of \mbox{\texttt{\mdseries\slshape posetelement}} n as a list of atoms (when present). }

 }

 
\section{\textcolor{Chapter }{Element operations}}\label{Chapter_Posets_Section_Element_operations}
\logpage{[ 8, 5, 0 ]}
\hyperdef{L}{X874169E2825654E9}{}
{
  

\subsection{\textcolor{Chapter }{RankInPoset (for IsPosetElement,IsPoset)}}
\logpage{[ 8, 5, 1 ]}\nobreak
\hyperdef{L}{X78A47D5C8513F4E2}{}
{\noindent\textcolor{FuncColor}{$\triangleright$\enspace\texttt{RankInPoset({\mdseries\slshape face, poset})\index{RankInPoset@\texttt{RankInPoset}!for IsPosetElement,IsPoset}
\label{RankInPoset:for IsPosetElement,IsPoset}
}\hfill{\scriptsize (operation)}}\\
\textbf{\indent Returns:\ }
\texttt{IsInt} 



 Given an element \mbox{\texttt{\mdseries\slshape face}} and a poset \mbox{\texttt{\mdseries\slshape poset}} to which it belongs, will give you the rank of \mbox{\texttt{\mdseries\slshape face}} in \mbox{\texttt{\mdseries\slshape poset}}. }

 

\subsection{\textcolor{Chapter }{IsSubface (for IsFace,IsFace,IsPoset)}}
\logpage{[ 8, 5, 2 ]}\nobreak
\hyperdef{L}{X798D28CC83E97F05}{}
{\noindent\textcolor{FuncColor}{$\triangleright$\enspace\texttt{IsSubface({\mdseries\slshape face1, face2, poset})\index{IsSubface@\texttt{IsSubface}!for IsFace,IsFace,IsPoset}
\label{IsSubface:for IsFace,IsFace,IsPoset}
}\hfill{\scriptsize (operation)}}\\
\textbf{\indent Returns:\ }
\texttt{true} or \texttt{false} 



 \mbox{\texttt{\mdseries\slshape face1}} and \mbox{\texttt{\mdseries\slshape face2}} are IsFace or IsPosetElement. IsSubface will check to see if \mbox{\texttt{\mdseries\slshape face2}} is a subface of \mbox{\texttt{\mdseries\slshape face1}} in \mbox{\texttt{\mdseries\slshape poset}}. You may drop the argument \mbox{\texttt{\mdseries\slshape poset}} if the faces only belong to one poset in common. Warning: if the elements are
made up of atoms, then IsSubface doesn't need to know what poset you are
working with. }

 

\subsection{\textcolor{Chapter }{IsEqualFaces (for IsFace, IsFace, IsPoset)}}
\logpage{[ 8, 5, 3 ]}\nobreak
\hyperdef{L}{X855B0DBE82C67770}{}
{\noindent\textcolor{FuncColor}{$\triangleright$\enspace\texttt{IsEqualFaces({\mdseries\slshape arg1, arg2, arg3})\index{IsEqualFaces@\texttt{IsEqualFaces}!for IsFace, IsFace, IsPoset}
\label{IsEqualFaces:for IsFace, IsFace, IsPoset}
}\hfill{\scriptsize (operation)}}\\


 Determines whether two faces are equal in a poset. Note that \texttt{\texttt{\symbol{92}}=} tests whether they are the identical object or not. }

 

\subsection{\textcolor{Chapter }{AreIncidentElements (for IsObject,IsObject)}}
\logpage{[ 8, 5, 4 ]}\nobreak
\hyperdef{L}{X84B2B67A805CD31D}{}
{\noindent\textcolor{FuncColor}{$\triangleright$\enspace\texttt{AreIncidentElements({\mdseries\slshape object1, object2})\index{AreIncidentElements@\texttt{AreIncidentElements}!for IsObject,IsObject}
\label{AreIncidentElements:for IsObject,IsObject}
}\hfill{\scriptsize (operation)}}\\
\textbf{\indent Returns:\ }
\texttt{true} or \texttt{false} 



 Given two poset elements, will tell you if they are incident. 
\begin{itemize}
\item  Synonym function: \texttt{AreIncidentFaces}. 
\end{itemize}
 }

 

\subsection{\textcolor{Chapter }{Meet (for IsFace, IsFace, IsPoset)}}
\logpage{[ 8, 5, 5 ]}\nobreak
\hyperdef{L}{X83066BBD80D8C755}{}
{\noindent\textcolor{FuncColor}{$\triangleright$\enspace\texttt{Meet({\mdseries\slshape face1, face2, poset})\index{Meet@\texttt{Meet}!for IsFace, IsFace, IsPoset}
\label{Meet:for IsFace, IsFace, IsPoset}
}\hfill{\scriptsize (operation)}}\\
\textbf{\indent Returns:\ }
meet 



 Finds (when possible) the meet of two elements in a poset. }

 

\subsection{\textcolor{Chapter }{Join (for IsFace, IsFace, IsPoset)}}
\logpage{[ 8, 5, 6 ]}\nobreak
\hyperdef{L}{X7A70132C85FB946E}{}
{\noindent\textcolor{FuncColor}{$\triangleright$\enspace\texttt{Join({\mdseries\slshape face1, face2, poset})\index{Join@\texttt{Join}!for IsFace, IsFace, IsPoset}
\label{Join:for IsFace, IsFace, IsPoset}
}\hfill{\scriptsize (operation)}}\\
\textbf{\indent Returns:\ }
meet 



 Finds (when possible) the join of two elements in a poset. }

 }

 This uses the work of Gleason and Hubard. 
\section{\textcolor{Chapter }{Product operations}}\label{Chapter_Posets_Section_Product_operations}
\logpage{[ 8, 6, 0 ]}
\hyperdef{L}{X84B6533683018DBA}{}
{
  The products documented in this section were defined by Gleason and Hubard in \cite{GleHub18} (\href{https://doi.org/10.1016/j.jcta.2018.02.002} {\texttt{https://doi.org/10.1016/j.jcta.2018.02.002}}). 

\subsection{\textcolor{Chapter }{JoinProduct (for IsPoset,IsPoset)}}
\logpage{[ 8, 6, 1 ]}\nobreak
\hyperdef{L}{X87B984FB860BDF3F}{}
{\noindent\textcolor{FuncColor}{$\triangleright$\enspace\texttt{JoinProduct({\mdseries\slshape poset1, poset2})\index{JoinProduct@\texttt{JoinProduct}!for IsPoset,IsPoset}
\label{JoinProduct:for IsPoset,IsPoset}
}\hfill{\scriptsize (operation)}}\\
\textbf{\indent Returns:\ }
poset 



 Given two posets, this forms the join product. If given two partial orders,
returns the join product of the partial orders. If given two maniplexes,
returns the join product of the maniplexes. }

 
\begin{Verbatim}[commandchars=!@|,fontsize=\small,frame=single,label=Example]
  !gapprompt@gap>| !gapinput@p:=PosetFromManiplex(Cube(2));|
  A poset
  !gapprompt@gap>| !gapinput@rel:=BinaryRelationOnPoints([[1,2],[2]]);|
  Binary Relation on 2 points
  !gapprompt@gap>| !gapinput@p1:=PosetFromPartialOrder(rel);|
  A poset using the IsPosetOfIndices representation
  !gapprompt@gap>| !gapinput@j:=JoinProduct(p,p1);|
  A poset using the IsPosetOfIndices representation
  !gapprompt@gap>| !gapinput@IsIsomorphicPoset(j,PosetFromManiplex(PyramidOver(Cube(2))));|
  true
\end{Verbatim}
 

\subsection{\textcolor{Chapter }{CartesianProduct (for IsPoset,IsPoset)}}
\logpage{[ 8, 6, 2 ]}\nobreak
\hyperdef{L}{X80F7AC5385F572D2}{}
{\noindent\textcolor{FuncColor}{$\triangleright$\enspace\texttt{CartesianProduct({\mdseries\slshape polytope1, polytope2})\index{CartesianProduct@\texttt{CartesianProduct}!for IsPoset,IsPoset}
\label{CartesianProduct:for IsPoset,IsPoset}
}\hfill{\scriptsize (operation)}}\\
\textbf{\indent Returns:\ }
polytope 



 Given two polytopes, forms the cartesian product of the polytopes. Should also
work if you give it any two posets. If given two maniplexes, returns the join
product of the maniplexes. }

 
\begin{Verbatim}[commandchars=!@|,fontsize=\small,frame=single,label=Example]
  !gapprompt@gap>| !gapinput@p1:=PosetFromManiplex(Edge());|
  A poset
  !gapprompt@gap>| !gapinput@p2:=PosetFromManiplex(Simplex(2));|
  A poset
  !gapprompt@gap>| !gapinput@c:=CartesianProduct(p1,p2);|
  A poset using the IsPosetOfIndices representation
  !gapprompt@gap>| !gapinput@IsIsomorphicPoset(c,PosetFromManiplex(PrismOver(Simplex(2))));|
  true
\end{Verbatim}
 

\subsection{\textcolor{Chapter }{DirectSumOfPosets (for IsPoset,IsPoset)}}
\logpage{[ 8, 6, 3 ]}\nobreak
\hyperdef{L}{X8124A8217B62E4D9}{}
{\noindent\textcolor{FuncColor}{$\triangleright$\enspace\texttt{DirectSumOfPosets({\mdseries\slshape polytope1, polytope2})\index{DirectSumOfPosets@\texttt{DirectSumOfPosets}!for IsPoset,IsPoset}
\label{DirectSumOfPosets:for IsPoset,IsPoset}
}\hfill{\scriptsize (operation)}}\\
\textbf{\indent Returns:\ }
polytope 



 Given two polytopes, forms the direct sum of the polytopes. }

 
\begin{Verbatim}[commandchars=!@|,fontsize=\small,frame=single,label=Example]
  !gapprompt@gap>| !gapinput@p1:=PosetFromManiplex(Cube(2));;p2:=PosetFromManiplex(Edge());;|
  !gapprompt@gap>| !gapinput@ds:=DirectSumOfPosets(p1,p2);|
  A poset using the IsPosetOfIndices representation.
  !gapprompt@gap>| !gapinput@IsIsomorphicPoset(ds,PosetFromManiplex(CrossPolytope(3)));|
  true
\end{Verbatim}
 

\subsection{\textcolor{Chapter }{TopologicalProduct (for IsPoset,IsPoset)}}
\logpage{[ 8, 6, 4 ]}\nobreak
\hyperdef{L}{X7D5144DE79A8E5BF}{}
{\noindent\textcolor{FuncColor}{$\triangleright$\enspace\texttt{TopologicalProduct({\mdseries\slshape polytope1, polytope2})\index{TopologicalProduct@\texttt{TopologicalProduct}!for IsPoset,IsPoset}
\label{TopologicalProduct:for IsPoset,IsPoset}
}\hfill{\scriptsize (operation)}}\\
\textbf{\indent Returns:\ }
polytope 



 Given two polytopes, forms the topological product of the polytopes. If given
two maniplexes, returns the join product of the maniplexes. }

 Here we demonstrate that the topological product (as expected) when taking the
product of a triangle with itself gives us the torus $\{4,4\}_{(3,0)}$ with 72 flags. 
\begin{Verbatim}[commandchars=!@|,fontsize=\small,frame=single,label=Example]
  !gapprompt@gap>| !gapinput@p:=PosetFromManiplex(Pgon(3));|
  A poset using the IsPosetOfFlags representation.
  !gapprompt@gap>| !gapinput@tp:=TopologicalProduct(p,p);|
  A poset using the IsPosetOfIndices representation.
  !gapprompt@gap>| !gapinput@s0 := (5,6);;|
  !gapprompt@gap>| !gapinput@s1 := (1,2)(3,5)(4,6);;|
  !gapprompt@gap>| !gapinput@s2 := (2,3);;|
  !gapprompt@gap>| !gapinput@poly := Group([s0,s1,s2]);;|
  !gapprompt@gap>| !gapinput@torus:=PosetFromManiplex(ReflexibleManiplex(poly));|
  A poset using the IsPosetOfFlags representation.
  !gapprompt@gap>| !gapinput@IsIsomorphicPoset(p,tp);|
  false
  !gapprompt@gap>| !gapinput@IsIsomorphicPoset(torus,tp);|
  true
\end{Verbatim}
 

\subsection{\textcolor{Chapter }{Antiprism (for IsPoset)}}
\logpage{[ 8, 6, 5 ]}\nobreak
\hyperdef{L}{X78FB044D7B9F5DB6}{}
{\noindent\textcolor{FuncColor}{$\triangleright$\enspace\texttt{Antiprism({\mdseries\slshape polytope})\index{Antiprism@\texttt{Antiprism}!for IsPoset}
\label{Antiprism:for IsPoset}
}\hfill{\scriptsize (operation)}}\\
\textbf{\indent Returns:\ }
poset 



 Given a \mbox{\texttt{\mdseries\slshape polytope}} (actually, should work for any poset), will return the antiprism of the \mbox{\texttt{\mdseries\slshape polytope}} (poset). If given two maniplexes, returns the join product of the maniplexes. }

 
\begin{Verbatim}[commandchars=!@|,fontsize=\small,frame=single,label=Example]
  !gapprompt@gap>| !gapinput@p:=PosetFromManiplex(Pgon(3));;|
  !gapprompt@gap>| !gapinput@a:=Antiprism(p);;|
  !gapprompt@gap>| !gapinput@IsIsomorphicPoset(a,PosetFromManiplex(CrossPolytope(3)));|
  true
  !gapprompt@gap>| !gapinput@p:=PosetFromManiplex(Pgon(4));;a:=Antiprism(p);;|
  !gapprompt@gap>| !gapinput@d:=DualPoset(p);;ad:=Antiprism(d);;|
  !gapprompt@gap>| !gapinput@IsIsomorphicPoset(a,ad);|
  true
\end{Verbatim}
 }

 }

   
\chapter{\textcolor{Chapter }{Polytope Constructions and Operations}}\label{Chapter_Polytope_Constructions_and_Operations}
\logpage{[ 9, 0, 0 ]}
\hyperdef{L}{X823655C77BD84AB5}{}
{
  
\section{\textcolor{Chapter }{Extensions, amalgamations, and quotients}}\label{Chapter_Polytope_Constructions_and_Operations_Section_Extensions_amalgamations_and_quotients}
\logpage{[ 9, 1, 0 ]}
\hyperdef{L}{X871C2D73829F1FC1}{}
{
  

\subsection{\textcolor{Chapter }{UniversalPolytope (for IsInt)}}
\logpage{[ 9, 1, 1 ]}\nobreak
\hyperdef{L}{X808AEA5281D52911}{}
{\noindent\textcolor{FuncColor}{$\triangleright$\enspace\texttt{UniversalPolytope({\mdseries\slshape n})\index{UniversalPolytope@\texttt{UniversalPolytope}!for IsInt}
\label{UniversalPolytope:for IsInt}
}\hfill{\scriptsize (operation)}}\\
\textbf{\indent Returns:\ }
IsManiplex 



 Constructs the universal polytope of rank \mbox{\texttt{\mdseries\slshape n}}. }

 
\begin{Verbatim}[commandchars=!@|,fontsize=\small,frame=single,label=Example]
  !gapprompt@gap>| !gapinput@UniversalPolytope(3);|
  UniversalPolytope(3)
  !gapprompt@gap>| !gapinput@Rank(last);|
  3
\end{Verbatim}
 

\subsection{\textcolor{Chapter }{UniversalExtension (for IsManiplex)}}
\logpage{[ 9, 1, 2 ]}\nobreak
\hyperdef{L}{X8582D0CF7EA1CB34}{}
{\noindent\textcolor{FuncColor}{$\triangleright$\enspace\texttt{UniversalExtension({\mdseries\slshape M})\index{UniversalExtension@\texttt{UniversalExtension}!for IsManiplex}
\label{UniversalExtension:for IsManiplex}
}\hfill{\scriptsize (operation)}}\\
\textbf{\indent Returns:\ }
IsManiplex 



 Constructs the universal extension of \mbox{\texttt{\mdseries\slshape M}}, i.e. the maniplex with facets isomorphic to \mbox{\texttt{\mdseries\slshape M}} that covers all other maniplexes with facets isomorphic to \mbox{\texttt{\mdseries\slshape M}}. Currently only defined for reflexible maniplexes. }

 
\begin{Verbatim}[commandchars=!@|,fontsize=\small,frame=single,label=Example]
  !gapprompt@gap>| !gapinput@UniversalExtension(Cube(3));|
  regular 4-polytope of type [ 4, 3, infinity ] with infinity flags
\end{Verbatim}
 

\subsection{\textcolor{Chapter }{UniversalExtension (for IsManiplex, IsInt)}}
\logpage{[ 9, 1, 3 ]}\nobreak
\hyperdef{L}{X78B4610E7DE67E32}{}
{\noindent\textcolor{FuncColor}{$\triangleright$\enspace\texttt{UniversalExtension({\mdseries\slshape M, k})\index{UniversalExtension@\texttt{UniversalExtension}!for IsManiplex, IsInt}
\label{UniversalExtension:for IsManiplex, IsInt}
}\hfill{\scriptsize (operation)}}\\
\textbf{\indent Returns:\ }
IsManiplex 



 Constructs the universal extension of \mbox{\texttt{\mdseries\slshape M}} with last entry of Schlafli symbol \mbox{\texttt{\mdseries\slshape k}}. Currently only defined for reflexible maniplexes. }

 
\begin{Verbatim}[commandchars=!@|,fontsize=\small,frame=single,label=Example]
  !gapprompt@gap>| !gapinput@UniversalExtension(Cube(3),2);|
  regular 4-polytope of type [ 4, 3, 2 ] with 96 flags
\end{Verbatim}
 

\subsection{\textcolor{Chapter }{TrivialExtension (for IsManiplex)}}
\logpage{[ 9, 1, 4 ]}\nobreak
\hyperdef{L}{X84BB6B21853173FD}{}
{\noindent\textcolor{FuncColor}{$\triangleright$\enspace\texttt{TrivialExtension({\mdseries\slshape M})\index{TrivialExtension@\texttt{TrivialExtension}!for IsManiplex}
\label{TrivialExtension:for IsManiplex}
}\hfill{\scriptsize (operation)}}\\
\textbf{\indent Returns:\ }
IsManiplex 



 Constructs the trivial extension of \mbox{\texttt{\mdseries\slshape M}}, also known as \texttt{\symbol{123}}\mbox{\texttt{\mdseries\slshape M}}, 2\texttt{\symbol{125}}. }

 
\begin{Verbatim}[commandchars=!@|,fontsize=\small,frame=single,label=Example]
  !gapprompt@gap>| !gapinput@TrivialExtension(Dodecahedron());|
  regular 4-polytope of type [ 5, 3, 2 ] with 240 flags
\end{Verbatim}
 

\subsection{\textcolor{Chapter }{FlatExtension (for IsManiplex, IsInt)}}
\logpage{[ 9, 1, 5 ]}\nobreak
\hyperdef{L}{X7A5853E278A83D74}{}
{\noindent\textcolor{FuncColor}{$\triangleright$\enspace\texttt{FlatExtension({\mdseries\slshape M, k})\index{FlatExtension@\texttt{FlatExtension}!for IsManiplex, IsInt}
\label{FlatExtension:for IsManiplex, IsInt}
}\hfill{\scriptsize (operation)}}\\
\textbf{\indent Returns:\ }
IsManiplex\#! @Description Constructs the flat extension of \mbox{\texttt{\mdseries\slshape M}} with last entry of Schlafli symbol \mbox{\texttt{\mdseries\slshape k}}. (As defined in \emph{Flat Extensions of Abstract Polytopes} \cite{Cun21}.) 



 Currently only defined for reflexible maniplexes. }

 
\begin{Verbatim}[commandchars=!@|,fontsize=\small,frame=single,label=Example]
  !gapprompt@gap>| !gapinput@FlatExtension(Simplex(3),4);|
  reflexible 4-maniplex of type [ 3, 3, 4 ] with 48 flags
\end{Verbatim}
 

\subsection{\textcolor{Chapter }{Amalgamate (for IsManiplex, IsManiplex)}}
\logpage{[ 9, 1, 6 ]}\nobreak
\hyperdef{L}{X790F22D47CD222EB}{}
{\noindent\textcolor{FuncColor}{$\triangleright$\enspace\texttt{Amalgamate({\mdseries\slshape M1, M2})\index{Amalgamate@\texttt{Amalgamate}!for IsManiplex, IsManiplex}
\label{Amalgamate:for IsManiplex, IsManiplex}
}\hfill{\scriptsize (operation)}}\\
\textbf{\indent Returns:\ }
IsManiplex 



 Constructs the amalgamation of \mbox{\texttt{\mdseries\slshape M1}} and \mbox{\texttt{\mdseries\slshape M2}}. Implicitly assumes that \mbox{\texttt{\mdseries\slshape M1}} and \mbox{\texttt{\mdseries\slshape M2}} are compatible. Currently only defined for reflexible maniplexes. }

 
\begin{Verbatim}[commandchars=!@|,fontsize=\small,frame=single,label=Example]
  !gapprompt@gap>| !gapinput@Amalgamate(Cube(3),CrossPolytope(3));|
  reflexible 4-maniplex of type [ 4, 3, 4 ]
\end{Verbatim}
 

\subsection{\textcolor{Chapter }{Medial (for IsManiplex)}}
\logpage{[ 9, 1, 7 ]}\nobreak
\hyperdef{L}{X840BC19484E0E9CC}{}
{\noindent\textcolor{FuncColor}{$\triangleright$\enspace\texttt{Medial({\mdseries\slshape M})\index{Medial@\texttt{Medial}!for IsManiplex}
\label{Medial:for IsManiplex}
}\hfill{\scriptsize (operation)}}\\
\textbf{\indent Returns:\ }
IsManiplex 



 Given a 3-maniplex \mbox{\texttt{\mdseries\slshape M}}, returns its medial. }

 
\begin{Verbatim}[commandchars=!@|,fontsize=\small,frame=single,label=Example]
  !gapprompt@gap>| !gapinput@SchlafliSymbol(Medial(Dodecahedron()));|
  [ [ 3, 5 ], 4 ]
\end{Verbatim}
 }

 
\section{\textcolor{Chapter }{Duality}}\label{Chapter_Polytope_Constructions_and_Operations_Section_Duality}
\logpage{[ 9, 2, 0 ]}
\hyperdef{L}{X87FD993F7F6F2FAB}{}
{
  

\subsection{\textcolor{Chapter }{Dual (for IsManiplex)}}
\logpage{[ 9, 2, 1 ]}\nobreak
\hyperdef{L}{X7D62BD3E7F941F5B}{}
{\noindent\textcolor{FuncColor}{$\triangleright$\enspace\texttt{Dual({\mdseries\slshape M})\index{Dual@\texttt{Dual}!for IsManiplex}
\label{Dual:for IsManiplex}
}\hfill{\scriptsize (operation)}}\\
\textbf{\indent Returns:\ }
The maniplex that is dual to \mbox{\texttt{\mdseries\slshape M}}. 



 

 }

 
\begin{Verbatim}[commandchars=!@|,fontsize=\small,frame=single,label=Example]
  !gapprompt@gap>| !gapinput@Dual(CrossPolytope(3));|
  Cube(3)
\end{Verbatim}
 

\subsection{\textcolor{Chapter }{IsSelfDual (for IsManiplex)}}
\logpage{[ 9, 2, 2 ]}\nobreak
\hyperdef{L}{X7FF8B96C83DA60CB}{}
{\noindent\textcolor{FuncColor}{$\triangleright$\enspace\texttt{IsSelfDual({\mdseries\slshape M})\index{IsSelfDual@\texttt{IsSelfDual}!for IsManiplex}
\label{IsSelfDual:for IsManiplex}
}\hfill{\scriptsize (property)}}\\
\textbf{\indent Returns:\ }
Whether this maniplex is isomorphic to its dual. 



 Also works for IsPoset objects. }

 
\begin{Verbatim}[commandchars=!@|,fontsize=\small,frame=single,label=Example]
  !gapprompt@gap>| !gapinput@IsSelfDual(Cube(3));|
  false
  !gapprompt@gap>| !gapinput@IsSelfDual(Simplex(5));|
  true
\end{Verbatim}
 

\subsection{\textcolor{Chapter }{IsInternallySelfDual (for IsManiplex)}}
\logpage{[ 9, 2, 3 ]}\nobreak
\hyperdef{L}{X79287DCF7835E0F4}{}
{\noindent\textcolor{FuncColor}{$\triangleright$\enspace\texttt{IsInternallySelfDual({\mdseries\slshape M[, x]})\index{IsInternallySelfDual@\texttt{IsInternallySelfDual}!for IsManiplex}
\label{IsInternallySelfDual:for IsManiplex}
}\hfill{\scriptsize (property)}}\\
\textbf{\indent Returns:\ }
\texttt{true} or \texttt{false} 



 Returns whether this maniplex is "internally self-dual", as defined by
Cunningham and Mixer in \cite{CunMix17} (\href{ https://doi.org/10.11575/cdm.v12i2.62785} {\texttt{ https://doi.org/10.11575/cdm.v12i2.62785}}). That is, if \mbox{\texttt{\mdseries\slshape M}} is self-dual, and the automorphism of AutomorphismGroup(M) that induces the
isomorphism between \mbox{\texttt{\mdseries\slshape M}} and its dual is an inner automorphism. If the optional group element \mbox{\texttt{\mdseries\slshape x}} is given, then we first check whether \mbox{\texttt{\mdseries\slshape x}} is a dualizing automorphism of \mbox{\texttt{\mdseries\slshape M}}, and if not, then we search the whole automorphism group of \mbox{\texttt{\mdseries\slshape M}}. You can also input a string for \mbox{\texttt{\mdseries\slshape x}}, in which case it is converted to \texttt{SggiElement(M, x)}. This property is only defined for rotary (chiral or reflexible) maniplexes. }

 
\begin{Verbatim}[commandchars=!@|,fontsize=\small,frame=single,label=Example]
  !gapprompt@gap>| !gapinput@IsInternallySelfDual(Simplex(4));|
  true
  !gapprompt@gap>| !gapinput@IsInternallySelfDual(Simplex(3), "r0")|
  #I  The given automorphism is not dualizing; searching for another...
  true
  !gapprompt@gap>| !gapinput@IsInternallySelfDual(Simplex(3), "r0 r1 r2 r0 r1 r0");|
  true
  !gapprompt@gap>| !gapinput@IsInternallySelfDual(ToroidalMap44([6,0]));|
  false
  !gapprompt@gap>| !gapinput@IsInternallySelfDual(ToroidalMap44([5,0]));|
  true
  !gapprompt@gap>| !gapinput@IsInternallySelfDual(ToroidalMap44([1,2]));|
  false
  !gapprompt@gap>| !gapinput@IsInternallySelfDual(Cube(3));|
  false
  !gapprompt@gap>| !gapinput@M := InternallySelfDualPolyhedron2(10,1);;|
  !gapprompt@gap>| !gapinput@g := AutomorphismGroup(M);;|
  !gapprompt@gap>| !gapinput@IsInternallySelfDual(M, (g.1*g.3*g.2)^6);|
  true
\end{Verbatim}
 

\subsection{\textcolor{Chapter }{IsExternallySelfDual (for IsManiplex)}}
\logpage{[ 9, 2, 4 ]}\nobreak
\hyperdef{L}{X871BE6737FF36FE6}{}
{\noindent\textcolor{FuncColor}{$\triangleright$\enspace\texttt{IsExternallySelfDual({\mdseries\slshape M})\index{IsExternallySelfDual@\texttt{IsExternallySelfDual}!for IsManiplex}
\label{IsExternallySelfDual:for IsManiplex}
}\hfill{\scriptsize (property)}}\\
\textbf{\indent Returns:\ }
\texttt{true} or \texttt{false} 



 Returns whether this maniplex is "externally self-dual", as defined by
Cunningham and Mixer in \cite{CunMix17} (\href{ https://doi.org/10.11575/cdm.v12i2.62785} {\texttt{ https://doi.org/10.11575/cdm.v12i2.62785}}). That is, if \mbox{\texttt{\mdseries\slshape M}} is self-dual, and the automorphism of AutomorphismGroup(M) that induces the
isomorphism between \mbox{\texttt{\mdseries\slshape M}} and its dual is an outer automorphism. }

 
\begin{Verbatim}[commandchars=!@|,fontsize=\small,frame=single,label=Example]
  !gapprompt@gap>| !gapinput@IsExternallySelfDual(Simplex(4));|
  false
  !gapprompt@gap>| !gapinput@IsExternallySelfDual(ToroidalMap44([6,0]));|
  true
  !gapprompt@gap>| !gapinput@IsExternallySelfDual(ToroidalMap44([5,0]));|
  false
  !gapprompt@gap>| !gapinput@IsExternallySelfDual(Cube(3));|
  false
\end{Verbatim}
 

\subsection{\textcolor{Chapter }{IsProperlySelfDual (for IsManiplex)}}
\logpage{[ 9, 2, 5 ]}\nobreak
\hyperdef{L}{X879D377778BF2175}{}
{\noindent\textcolor{FuncColor}{$\triangleright$\enspace\texttt{IsProperlySelfDual({\mdseries\slshape M})\index{IsProperlySelfDual@\texttt{IsProperlySelfDual}!for IsManiplex}
\label{IsProperlySelfDual:for IsManiplex}
}\hfill{\scriptsize (property)}}\\
\textbf{\indent Returns:\ }
\texttt{true} or \texttt{false} 



 Returns whether this rooted maniplex is "properly self-dual", meaning that
there is an isomorphism of \mbox{\texttt{\mdseries\slshape M}} to its dual that preserves the flag-orbits. Note that all reflexible self-dual
maniplexes are properly self-dual. }

 
\begin{Verbatim}[commandchars=!@|,fontsize=\small,frame=single,label=Example]
  !gapprompt@gap>| !gapinput@IsProperlySelfDual(Cube(4));|
  false
  !gapprompt@gap>| !gapinput@IsProperlySelfDual(Simplex(4));|
  true
  !gapprompt@gap>| !gapinput@IsProperlySelfDual(ARP([4,5,4]));|
  true
  !gapprompt@gap>| !gapinput@IsProperlySelfDual(ToroidalMap44([1,2]));|
  false
  !gapprompt@gap>| !gapinput@IsProperlySelfDual(RotaryManiplex([4,4,4],"(s2^-1 s1) (s2 s1^-1)^3, (s2 s3^-1) (s2^-1 s3)^3"));|
  true
  !gapprompt@gap>| !gapinput@IsProperlySelfDual(RotaryManiplex([4,4,4],"(s2^-1 s1)^3 (s2 s1^-1), (s2 s3^-1) (s2^-1 s3)^3"));|
  false
\end{Verbatim}
 
\subsection{\textcolor{Chapter }{Petrie Dual}}\label{AutoDoc_generated_group1}
\logpage{[ 9, 2, 6 ]}
\hyperdef{L}{X86E347E67EE31555}{}
{
\noindent\textcolor{FuncColor}{$\triangleright$\enspace\texttt{Petrial({\mdseries\slshape M})\index{Petrial@\texttt{Petrial}!for IsManiplex}
\label{Petrial:for IsManiplex}
}\hfill{\scriptsize (attribute)}}\\
\textbf{\indent Returns:\ }
The Petrial (Petrie dual) of \mbox{\texttt{\mdseries\slshape M}}. 



 Note that this is not necessarily a polytope, even if \mbox{\texttt{\mdseries\slshape M}} is. When Rank(M) {\textgreater} 3, this is the "generalized Petrial" which
essentially replaces $r_{n-3}$ with $r_{n-3} r_{n-1}$ in the set of generators. 

 Synonym for the command is \texttt{PetrieDual}. }

 
\begin{Verbatim}[commandchars=!@|,fontsize=\small,frame=single,label=Example]
  !gapprompt@gap>| !gapinput@Petrial(HemiCube(3));|
  ReflexibleManiplex([ 3, 3 ], "((r0 r2)*r1*r2)^3,z1^4")
\end{Verbatim}
 

\subsection{\textcolor{Chapter }{IsSelfPetrial (for IsManiplex)}}
\logpage{[ 9, 2, 7 ]}\nobreak
\hyperdef{L}{X78FDB0047E3388A5}{}
{\noindent\textcolor{FuncColor}{$\triangleright$\enspace\texttt{IsSelfPetrial({\mdseries\slshape M})\index{IsSelfPetrial@\texttt{IsSelfPetrial}!for IsManiplex}
\label{IsSelfPetrial:for IsManiplex}
}\hfill{\scriptsize (property)}}\\
\textbf{\indent Returns:\ }
Whether this maniplex is isomorphic to its Petrial. 



 

 }

 
\begin{Verbatim}[commandchars=!@|,fontsize=\small,frame=single,label=Example]
  !gapprompt@gap>| !gapinput@s0 := ( 2, 3)( 4, 6)( 7,10)( 9,12)(11,14)(13,15);;|
  !gapprompt@gap>| !gapinput@s1 := ( 1, 2)( 3, 5)( 4, 7)( 6, 9)( 8,11)(10,13)(12,15)(14,16);;|
  !gapprompt@gap>| !gapinput@s2 := ( 2, 4)( 3, 6)( 5, 8)( 9,12)(11,15)(13,14);;|
  !gapprompt@gap>| !gapinput@poly := Group([s0,s1,s2]);;|
  !gapprompt@gap>| !gapinput@p:=ARP(poly);|
  regular 3-polytope
  !gapprompt@gap>| !gapinput@IsSelfPetrial(p);|
  true
\end{Verbatim}
 

\subsection{\textcolor{Chapter }{DirectDerivates (for IsManiplex)}}
\logpage{[ 9, 2, 8 ]}\nobreak
\hyperdef{L}{X7F8BED777CD6B80F}{}
{\noindent\textcolor{FuncColor}{$\triangleright$\enspace\texttt{DirectDerivates({\mdseries\slshape M})\index{DirectDerivates@\texttt{DirectDerivates}!for IsManiplex}
\label{DirectDerivates:for IsManiplex}
}\hfill{\scriptsize (operation)}}\\


 Returns a list of the \emph{direct derivates} of \mbox{\texttt{\mdseries\slshape M}}, which are the images of M under duality and Petriality. A 3-maniplex has up
to 6 direct derivates, and an n-maniplex with $n \geq 4$ has up to 8. If the option 'polytopal' is set, then only returns those direct
derivates that are polytopal. }

 
\begin{Verbatim}[commandchars=!@|,fontsize=\small,frame=single,label=Example]
  !gapprompt@gap>| !gapinput@DirectDerivates(Cube(3));|
  [ Cube(3), CrossPolytope(3), ReflexibleManiplex([ 6, 3 ], "z1^4"), 
    ReflexibleManiplex([ 6, 4 ], "z1^3"), ReflexibleManiplex([ 3, 6 ], "(r2*r1*r0)^4"), 
    ReflexibleManiplex([ 4, 6 ], "(r2*r1*r0)^3") ]
\end{Verbatim}
 

\subsection{\textcolor{Chapter }{IsKaleidoscopic (for IsMapOnSurface)}}
\logpage{[ 9, 2, 9 ]}\nobreak
\hyperdef{L}{X814A6AB1824DBB9B}{}
{\noindent\textcolor{FuncColor}{$\triangleright$\enspace\texttt{IsKaleidoscopic({\mdseries\slshape M})\index{IsKaleidoscopic@\texttt{IsKaleidoscopic}!for IsMapOnSurface}
\label{IsKaleidoscopic:for IsMapOnSurface}
}\hfill{\scriptsize (property)}}\\
\textbf{\indent Returns:\ }
\texttt{true} or \texttt{false} 



 Returns whether the map \mbox{\texttt{\mdseries\slshape M}} with $q$-valent vertices is \emph{kaleidoscopic}, meaning that for each integer e in [2..q-1] that is coprime to q, the map
'Hole(M, e)' is isomorphic to M as a rooted map. }

 
\begin{Verbatim}[commandchars=!@|,fontsize=\small,frame=single,label=Example]
  !gapprompt@gap>| !gapinput@M := AbstractRegularPolytope([4,5], "(r0 r1 r2)^5");;|
  !gapprompt@gap>| !gapinput@ForAll([2,3,4], e -> IsIsomorphicRootedManiplex(Hole(M,e), M));|
  true
  !gapprompt@gap>| !gapinput@IsKaleidoscopic(M);|
  true
  !gapprompt@gap>| !gapinput@Filtered(SmallChiralPolyhedra(200), IsKaleidoscopic);|
  [ ]
\end{Verbatim}
 }

 
\section{\textcolor{Chapter }{Products}}\label{Chapter_Polytope_Constructions_and_Operations_Section_Products}
\logpage{[ 9, 3, 0 ]}
\hyperdef{L}{X86CE352C7851221F}{}
{
  
\subsection{\textcolor{Chapter }{Pyramids}}\label{Pyramid}
\logpage{[ 9, 3, 1 ]}
\hyperdef{L}{X7CBB470F7D3135BC}{}
{
\noindent\textcolor{FuncColor}{$\triangleright$\enspace\texttt{Pyramid({\mdseries\slshape M})\index{Pyramid@\texttt{Pyramid}!for IsManiplex}
\label{Pyramid:for IsManiplex}
}\hfill{\scriptsize (operation)}}\\
\noindent\textcolor{FuncColor}{$\triangleright$\enspace\texttt{Pyramid({\mdseries\slshape k})\index{Pyramid@\texttt{Pyramid}!for IsInt}
\label{Pyramid:for IsInt}
}\hfill{\scriptsize (operation)}}\\


 In the first form, returns the pyramid over \mbox{\texttt{\mdseries\slshape M}}. In the second form, returns the pyramid over a \mbox{\texttt{\mdseries\slshape k}}-gon. }

 
\begin{Verbatim}[commandchars=!@|,fontsize=\small,frame=single,label=Example]
  !gapprompt@gap>| !gapinput@SchlafliSymbol(Pyramid(Cube(3)));|
  [ [ 3, 4 ], [ 3, 4 ], 3 ]
  !gapprompt@gap>| !gapinput@SchlafliSymbol(Pyramid(4));|
  [ [ 3, 4 ], [ 3, 4 ] ]
\end{Verbatim}
 
\subsection{\textcolor{Chapter }{Prisms}}\label{Prisms}
\logpage{[ 9, 3, 2 ]}
\hyperdef{L}{X7B13F48A7AF5DF42}{}
{
\noindent\textcolor{FuncColor}{$\triangleright$\enspace\texttt{Prism({\mdseries\slshape M})\index{Prism@\texttt{Prism}!for IsManiplex}
\label{Prism:for IsManiplex}
}\hfill{\scriptsize (operation)}}\\
\noindent\textcolor{FuncColor}{$\triangleright$\enspace\texttt{Prism({\mdseries\slshape k})\index{Prism@\texttt{Prism}!for IsInt}
\label{Prism:for IsInt}
}\hfill{\scriptsize (operation)}}\\


 In the first form, returns the prism over \mbox{\texttt{\mdseries\slshape M}}. In the second form, returns the prism over a \mbox{\texttt{\mdseries\slshape k}}-gon. }

 
\begin{Verbatim}[commandchars=!@|,fontsize=\small,frame=single,label=Example]
  !gapprompt@gap>| !gapinput@Cube(3)=Prism(Cube(2));|
  true
  !gapprompt@gap>| !gapinput@Prism(4)=Cube(3);|
  true
\end{Verbatim}
 
\subsection{\textcolor{Chapter }{Antiprisms}}\label{Antiprisms}
\logpage{[ 9, 3, 3 ]}
\hyperdef{L}{X808566B57F296C93}{}
{
\noindent\textcolor{FuncColor}{$\triangleright$\enspace\texttt{Antiprism({\mdseries\slshape M})\index{Antiprism@\texttt{Antiprism}!for IsManiplex}
\label{Antiprism:for IsManiplex}
}\hfill{\scriptsize (operation)}}\\
\noindent\textcolor{FuncColor}{$\triangleright$\enspace\texttt{Antiprism({\mdseries\slshape k})\index{Antiprism@\texttt{Antiprism}!for IsInt}
\label{Antiprism:for IsInt}
}\hfill{\scriptsize (operation)}}\\


 In the first form, returns the antiprism over \mbox{\texttt{\mdseries\slshape M}}. In the second form, returns the antiprism over a \mbox{\texttt{\mdseries\slshape k}}-gon. }

 
\begin{Verbatim}[commandchars=!@|,fontsize=\small,frame=single,label=Example]
  !gapprompt@gap>| !gapinput@SchlafliSymbol(Antiprism(Dodecahedron()));|
  [ [ 3, 5 ], [ 3, 5 ], 4 ]
  !gapprompt@gap>| !gapinput@SchlafliSymbol(Antiprism(5));|
  [ [ 3, 5 ], 4 ]
\end{Verbatim}
 

\subsection{\textcolor{Chapter }{JoinProduct (for IsManiplex, IsManiplex)}}
\logpage{[ 9, 3, 4 ]}\nobreak
\hyperdef{L}{X7FA28B5C7AC1C32E}{}
{\noindent\textcolor{FuncColor}{$\triangleright$\enspace\texttt{JoinProduct({\mdseries\slshape M1, M2})\index{JoinProduct@\texttt{JoinProduct}!for IsManiplex, IsManiplex}
\label{JoinProduct:for IsManiplex, IsManiplex}
}\hfill{\scriptsize (operation)}}\\
\textbf{\indent Returns:\ }
Maniplex 



 Given two maniplexes, this forms the join product. May give weird results if
the maniplexes aren't faithfully represented by their posets. }

 
\begin{Verbatim}[commandchars=!@|,fontsize=\small,frame=single,label=Example]
  !gapprompt@gap>| !gapinput@SchlafliSymbol(last);|
  [ [ 3, 4 ], [ 3, 4 ], [ 3, 4 ], [ 3, 4 ], 3 ]
\end{Verbatim}
 

\subsection{\textcolor{Chapter }{CartesianProduct (for IsManiplex, IsManiplex)}}
\logpage{[ 9, 3, 5 ]}\nobreak
\hyperdef{L}{X7C116F2C83963ADC}{}
{\noindent\textcolor{FuncColor}{$\triangleright$\enspace\texttt{CartesianProduct({\mdseries\slshape M1, M2})\index{CartesianProduct@\texttt{CartesianProduct}!for IsManiplex, IsManiplex}
\label{CartesianProduct:for IsManiplex, IsManiplex}
}\hfill{\scriptsize (operation)}}\\
\textbf{\indent Returns:\ }
Maniplex 



 Given two maniplexes, this forms the cartesian product. May give weird results
if the maniplexes aren't faithfully represented by their posets. }

 
\begin{Verbatim}[commandchars=!@|,fontsize=\small,frame=single,label=Example]
  !gapprompt@gap>| !gapinput@SchlafliSymbol(CartesianProduct(HemiCube(3),Simplex(2)));|
  [ [ 3, 4 ], 3, 3, 3 ]
\end{Verbatim}
 

\subsection{\textcolor{Chapter }{DirectSumOfManiplexes (for IsManiplex, IsManiplex)}}
\logpage{[ 9, 3, 6 ]}\nobreak
\hyperdef{L}{X8022B9AC85D38D7D}{}
{\noindent\textcolor{FuncColor}{$\triangleright$\enspace\texttt{DirectSumOfManiplexes({\mdseries\slshape M1, M2})\index{DirectSumOfManiplexes@\texttt{DirectSumOfManiplexes}!for IsManiplex, IsManiplex}
\label{DirectSumOfManiplexes:for IsManiplex, IsManiplex}
}\hfill{\scriptsize (operation)}}\\
\textbf{\indent Returns:\ }
Maniplex 



 Given two maniplexes, this forms the direct sum. May give weird results if the
maniplexes aren't faithfully represented by their posets. }

 
\begin{Verbatim}[commandchars=!@|,fontsize=\small,frame=single,label=Example]
  !gapprompt@gap>| !gapinput@SchlafliSymbol(DirectSumOfManiplexes(HemiCube(3),Simplex(2)));|
  [ [ 3, 4 ], [ 3, 4 ], [ 3, 4 ], [ 3, 4 ] ]
\end{Verbatim}
 

\subsection{\textcolor{Chapter }{TopologicalProduct (for IsManiplex, IsManiplex)}}
\logpage{[ 9, 3, 7 ]}\nobreak
\hyperdef{L}{X833534418543FB94}{}
{\noindent\textcolor{FuncColor}{$\triangleright$\enspace\texttt{TopologicalProduct({\mdseries\slshape M1, M2})\index{TopologicalProduct@\texttt{TopologicalProduct}!for IsManiplex, IsManiplex}
\label{TopologicalProduct:for IsManiplex, IsManiplex}
}\hfill{\scriptsize (operation)}}\\
\textbf{\indent Returns:\ }
Maniplex 



 Given two maniplexes, this forms the direct sum. May give weird results if the
maniplexes aren't faithfully represented by their posets. }

 
\begin{Verbatim}[commandchars=!@|,fontsize=\small,frame=single,label=Example]
  !gapprompt@gap>| !gapinput@SchlafliSymbol(TopologicalProduct(HemiCube(3),Simplex(2)));|
  [ 4, 3, [ 3, 4 ] ]
\end{Verbatim}
 }

 }

   
\chapter{\textcolor{Chapter }{Combinatorics and Structure}}\label{Chapter_Combinatorics_and_Structure}
\logpage{[ 10, 0, 0 ]}
\hyperdef{L}{X8153E7658330D710}{}
{
  
\section{\textcolor{Chapter }{Faces}}\label{Chapter_Combinatorics_and_Structure_Section_Faces}
\logpage{[ 10, 1, 0 ]}
\hyperdef{L}{X872AD1E785C7EB03}{}
{
  

\subsection{\textcolor{Chapter }{NumberOfIFaces (for IsManiplex, IsInt)}}
\logpage{[ 10, 1, 1 ]}\nobreak
\hyperdef{L}{X86085A967E396035}{}
{\noindent\textcolor{FuncColor}{$\triangleright$\enspace\texttt{NumberOfIFaces({\mdseries\slshape M, i})\index{NumberOfIFaces@\texttt{NumberOfIFaces}!for IsManiplex, IsInt}
\label{NumberOfIFaces:for IsManiplex, IsInt}
}\hfill{\scriptsize (operation)}}\\


 Returns The number of \mbox{\texttt{\mdseries\slshape i}}-faces of \mbox{\texttt{\mdseries\slshape M}}. }

 
\begin{Verbatim}[commandchars=!@|,fontsize=\small,frame=single,label=Example]
  !gapprompt@gap>| !gapinput@NumberOfIFaces(Dodecahedron(),1);|
  30
\end{Verbatim}
 

\subsection{\textcolor{Chapter }{NumberOfVertices (for IsManiplex)}}
\logpage{[ 10, 1, 2 ]}\nobreak
\hyperdef{L}{X7C126944873D2461}{}
{\noindent\textcolor{FuncColor}{$\triangleright$\enspace\texttt{NumberOfVertices({\mdseries\slshape M})\index{NumberOfVertices@\texttt{NumberOfVertices}!for IsManiplex}
\label{NumberOfVertices:for IsManiplex}
}\hfill{\scriptsize (attribute)}}\\


 Returns the number of vertices of \mbox{\texttt{\mdseries\slshape M}}. }

 
\begin{Verbatim}[commandchars=!@|,fontsize=\small,frame=single,label=Example]
  !gapprompt@gap>| !gapinput@NumberOfVertices(HemiDodecahedron());|
  10
\end{Verbatim}
 

\subsection{\textcolor{Chapter }{NumberOfEdges (for IsManiplex)}}
\logpage{[ 10, 1, 3 ]}\nobreak
\hyperdef{L}{X86CDC9B38792B038}{}
{\noindent\textcolor{FuncColor}{$\triangleright$\enspace\texttt{NumberOfEdges({\mdseries\slshape M})\index{NumberOfEdges@\texttt{NumberOfEdges}!for IsManiplex}
\label{NumberOfEdges:for IsManiplex}
}\hfill{\scriptsize (attribute)}}\\


 Returns the number of edges of \mbox{\texttt{\mdseries\slshape M}}. }

 
\begin{Verbatim}[commandchars=!@|,fontsize=\small,frame=single,label=Example]
  !gapprompt@gap>| !gapinput@NumberOfEdges(HemiIcosahedron());|
  15
\end{Verbatim}
 

\subsection{\textcolor{Chapter }{NumberOfFacets (for IsManiplex)}}
\logpage{[ 10, 1, 4 ]}\nobreak
\hyperdef{L}{X7DE9F1057B309554}{}
{\noindent\textcolor{FuncColor}{$\triangleright$\enspace\texttt{NumberOfFacets({\mdseries\slshape M})\index{NumberOfFacets@\texttt{NumberOfFacets}!for IsManiplex}
\label{NumberOfFacets:for IsManiplex}
}\hfill{\scriptsize (attribute)}}\\


 Returns the number of facets of \mbox{\texttt{\mdseries\slshape M}}. }

 
\begin{Verbatim}[commandchars=!@|,fontsize=\small,frame=single,label=Example]
  !gapprompt@gap>| !gapinput@NumberOfFacets(Bk2l(4,6));|
  4
\end{Verbatim}
 

\subsection{\textcolor{Chapter }{NumberOfRidges (for IsManiplex)}}
\logpage{[ 10, 1, 5 ]}\nobreak
\hyperdef{L}{X7C5446247FD63F28}{}
{\noindent\textcolor{FuncColor}{$\triangleright$\enspace\texttt{NumberOfRidges({\mdseries\slshape M})\index{NumberOfRidges@\texttt{NumberOfRidges}!for IsManiplex}
\label{NumberOfRidges:for IsManiplex}
}\hfill{\scriptsize (attribute)}}\\


 Returns the number of ridges ((n-2)-faces) of \mbox{\texttt{\mdseries\slshape M}}. }

 
\begin{Verbatim}[commandchars=!@|,fontsize=\small,frame=single,label=Example]
  !gapprompt@gap>| !gapinput@NumberOfRidges(CrossPolytope(5));|
  80
\end{Verbatim}
 

\subsection{\textcolor{Chapter }{Fvector (for IsManiplex)}}
\logpage{[ 10, 1, 6 ]}\nobreak
\hyperdef{L}{X816A47B879737629}{}
{\noindent\textcolor{FuncColor}{$\triangleright$\enspace\texttt{Fvector({\mdseries\slshape M})\index{Fvector@\texttt{Fvector}!for IsManiplex}
\label{Fvector:for IsManiplex}
}\hfill{\scriptsize (attribute)}}\\


 Returns the f-vector of \mbox{\texttt{\mdseries\slshape M}}. }

 
\begin{Verbatim}[commandchars=!@|,fontsize=\small,frame=single,label=Example]
  !gapprompt@gap>| !gapinput@Fvector(HemiIcosahedron());|
  [ 6, 15, 10 ]
\end{Verbatim}
 
\subsection{\textcolor{Chapter }{Section(s)}}\label{Sections}
\logpage{[ 10, 1, 7 ]}
\hyperdef{L}{X8586EE158331B46C}{}
{
\noindent\textcolor{FuncColor}{$\triangleright$\enspace\texttt{Section({\mdseries\slshape M, j, i})\index{Section@\texttt{Section}!for IsManiplex, IsInt, IsInt}
\label{Section:for IsManiplex, IsInt, IsInt}
}\hfill{\scriptsize (operation)}}\\
\noindent\textcolor{FuncColor}{$\triangleright$\enspace\texttt{Section({\mdseries\slshape M, j, i, k})\index{Section@\texttt{Section}!for IsManiplex, IsInt, IsInt, IsInt}
\label{Section:for IsManiplex, IsInt, IsInt, IsInt}
}\hfill{\scriptsize (operation)}}\\
\noindent\textcolor{FuncColor}{$\triangleright$\enspace\texttt{Sections({\mdseries\slshape M, j, i})\index{Sections@\texttt{Sections}!for IsManiplex, IsInt, IsInt}
\label{Sections:for IsManiplex, IsInt, IsInt}
}\hfill{\scriptsize (operation)}}\\


 \texttt{Section(M,j,i)} returns the section \texttt{F{\textunderscore}j / F{\textunderscore}i}, where \texttt{F{\textunderscore}j} is the $j$-face of the base flag of \mbox{\texttt{\mdseries\slshape M}} and \texttt{F{\textunderscore}i} is the $i$-face of the base flag. \texttt{Section(M,j,i,k)} returns the section \texttt{F{\textunderscore}j / F{\textunderscore}i}, where \texttt{F{\textunderscore}j} is the $j$-face of flag number \mbox{\texttt{\mdseries\slshape k}} of \mbox{\texttt{\mdseries\slshape M}} and \texttt{F{\textunderscore}i} is the $i$-face of the same flag. \texttt{Sections(M,j,i)} returns all sections of type \texttt{F{\textunderscore}j / F{\textunderscore}i}, where \texttt{F{\textunderscore}j} is a $j$-face and \texttt{F{\textunderscore}i} is an incident $i$-face. }

 
\begin{Verbatim}[commandchars=!@|,fontsize=\small,frame=single,label=Example]
  !gapprompt@gap>| !gapinput@Section(ToroidalMap44([2,2]),3,0);|
  Pgon(4)
  !gapprompt@gap>| !gapinput@Section(Cuboctahedron(),2,-1,1);|
  Pgon(4)
  !gapprompt@gap>| !gapinput@Section(Cuboctahedron(),2,-1,4);|
  Pgon(3)
  !gapprompt@gap>| !gapinput@Sections(Cuboctahedron(),2,-1);|
  [ Pgon(4), Pgon(3) ]
\end{Verbatim}
 
\subsection{\textcolor{Chapter }{Facet(s)}}\label{Facet}
\logpage{[ 10, 1, 8 ]}
\hyperdef{L}{X85D359FD7E3C191B}{}
{
\noindent\textcolor{FuncColor}{$\triangleright$\enspace\texttt{Facets({\mdseries\slshape M})\index{Facets@\texttt{Facets}!for IsManiplex}
\label{Facets:for IsManiplex}
}\hfill{\scriptsize (attribute)}}\\
\noindent\textcolor{FuncColor}{$\triangleright$\enspace\texttt{Facet({\mdseries\slshape M, k})\index{Facet@\texttt{Facet}!for IsManiplex, IsInt}
\label{Facet:for IsManiplex, IsInt}
}\hfill{\scriptsize (operation)}}\\
\noindent\textcolor{FuncColor}{$\triangleright$\enspace\texttt{Facet({\mdseries\slshape M})\index{Facet@\texttt{Facet}!for IsManiplex}
\label{Facet:for IsManiplex}
}\hfill{\scriptsize (attribute)}}\\


 Returns the facet-types of \mbox{\texttt{\mdseries\slshape M}} (i.e. the maniplexes corresponding to the facets). Returns the facet of \mbox{\texttt{\mdseries\slshape M}} that contains the flag number \mbox{\texttt{\mdseries\slshape k}} (that is, the maniplex corresponding to the facet). Returns the facet of \mbox{\texttt{\mdseries\slshape M}} that contains flag number 1 (that is, the maniplex corresponding to the
facet). }

 
\begin{Verbatim}[commandchars=!@|,fontsize=\small,frame=single,label=Example]
  !gapprompt@gap>| !gapinput@Facets(Cuboctahedron());|
  [ Pgon(4), Pgon(3) ]
  !gapprompt@gap>| !gapinput@Facet(Cuboctahedron(),4);|
  Pgon(3)
  !gapprompt@gap>| !gapinput@Facet(Cuboctahedron());|
  Pgon(4)
\end{Verbatim}
 
\subsection{\textcolor{Chapter }{Vertex Figure(s)}}\label{VertexFigure}
\logpage{[ 10, 1, 9 ]}
\hyperdef{L}{X859E4169801C72D9}{}
{
\noindent\textcolor{FuncColor}{$\triangleright$\enspace\texttt{VertexFigures({\mdseries\slshape M})\index{VertexFigures@\texttt{VertexFigures}!for IsManiplex}
\label{VertexFigures:for IsManiplex}
}\hfill{\scriptsize (attribute)}}\\
\noindent\textcolor{FuncColor}{$\triangleright$\enspace\texttt{VertexFigure({\mdseries\slshape M, k})\index{VertexFigure@\texttt{VertexFigure}!for IsManiplex, IsInt}
\label{VertexFigure:for IsManiplex, IsInt}
}\hfill{\scriptsize (operation)}}\\
\noindent\textcolor{FuncColor}{$\triangleright$\enspace\texttt{VertexFigure({\mdseries\slshape M})\index{VertexFigure@\texttt{VertexFigure}!for IsManiplex}
\label{VertexFigure:for IsManiplex}
}\hfill{\scriptsize (attribute)}}\\


 Returns the types of vertex-figures of \mbox{\texttt{\mdseries\slshape M}} (i.e. the maniplexes corresponding to the vertex-figures). Returns the
vertex-figure of \mbox{\texttt{\mdseries\slshape M}} that contains flag number \mbox{\texttt{\mdseries\slshape k}}. Returns the vertex-figure of \mbox{\texttt{\mdseries\slshape M}} that contains the base flag. }

 
\begin{Verbatim}[commandchars=!@|,fontsize=\small,frame=single,label=Example]
  !gapprompt@gap>| !gapinput@p:=Dual(SmallRhombicosidodecahedron());|
  Dual(3-maniplex)
  !gapprompt@gap>| !gapinput@VertexFigures(p);|
  [ Pgon(5), Pgon(4), Pgon(3) ]
  !gapprompt@gap>| !gapinput@VertexFigure(p,4);|
  Pgon(4)
  !gapprompt@gap>| !gapinput@VertexFigure(p);|
  Pgon(5)
\end{Verbatim}
 }

 
\section{\textcolor{Chapter }{Flatness}}\label{Chapter_Combinatorics_and_Structure_Section_Flatness}
\logpage{[ 10, 2, 0 ]}
\hyperdef{L}{X877E841E7FBCEAAE}{}
{
  
\subsection{\textcolor{Chapter }{Flatness}}\label{IsFlat}
\logpage{[ 10, 2, 1 ]}
\hyperdef{L}{X877E841E7FBCEAAE}{}
{
\noindent\textcolor{FuncColor}{$\triangleright$\enspace\texttt{IsFlat({\mdseries\slshape M})\index{IsFlat@\texttt{IsFlat}!for IsManiplex}
\label{IsFlat:for IsManiplex}
}\hfill{\scriptsize (property)}}\\
\noindent\textcolor{FuncColor}{$\triangleright$\enspace\texttt{IsFlat({\mdseries\slshape M, i, j})\index{IsFlat@\texttt{IsFlat}!for IsManiplex, IsInt, IsInt}
\label{IsFlat:for IsManiplex, IsInt, IsInt}
}\hfill{\scriptsize (operation)}}\\
\textbf{\indent Returns:\ }
\texttt{true} or \texttt{false} 



 In the first form, returns true if every vertex of the maniplex \mbox{\texttt{\mdseries\slshape M}} is incident to every facet. In the second form, returns true if every i-face
of the maniplex \mbox{\texttt{\mdseries\slshape M}} is incident to every j-face. }

 
\begin{Verbatim}[commandchars=!@|,fontsize=\small,frame=single,label=Example]
  !gapprompt@gap>| !gapinput@IsFlat(HemiCube(3));|
  true
  !gapprompt@gap>| !gapinput@IsFlat(Simplex(3),0,2);|
  false
\end{Verbatim}
 }

 
\section{\textcolor{Chapter }{Schlafli symbol}}\label{Chapter_Combinatorics_and_Structure_Section_Schlafli_symbol}
\logpage{[ 10, 3, 0 ]}
\hyperdef{L}{X8052D0A68676A82C}{}
{
  

\subsection{\textcolor{Chapter }{SchlafliSymbol (for IsManiplex)}}
\logpage{[ 10, 3, 1 ]}\nobreak
\hyperdef{L}{X7FA0B161838A442A}{}
{\noindent\textcolor{FuncColor}{$\triangleright$\enspace\texttt{SchlafliSymbol({\mdseries\slshape M})\index{SchlafliSymbol@\texttt{SchlafliSymbol}!for IsManiplex}
\label{SchlafliSymbol:for IsManiplex}
}\hfill{\scriptsize (attribute)}}\\


 Returns the Schlafli symbol of the maniplex \mbox{\texttt{\mdseries\slshape M}}. Each entry is either an integer or a set of integers, where entry number i
shows the polygons that we obtain as sections of (i+1)-faces over (i-2)-faces. }

 
\begin{Verbatim}[commandchars=!@|,fontsize=\small,frame=single,label=Example]
  !gapprompt@gap>| !gapinput@SchlafliSymbol(SmallRhombicosidodecahedron());|
  [ [ 3, 4, 5 ], 4 ]
\end{Verbatim}
 

\subsection{\textcolor{Chapter }{PseudoSchlafliSymbol (for IsManiplex)}}
\logpage{[ 10, 3, 2 ]}\nobreak
\hyperdef{L}{X78447BE8877668EA}{}
{\noindent\textcolor{FuncColor}{$\triangleright$\enspace\texttt{PseudoSchlafliSymbol({\mdseries\slshape M})\index{PseudoSchlafliSymbol@\texttt{PseudoSchlafliSymbol}!for IsManiplex}
\label{PseudoSchlafliSymbol:for IsManiplex}
}\hfill{\scriptsize (attribute)}}\\


 Sometimes when we make a maniplex, we know that the Schlafli symbol must be a
quotient of some symbol. This most frequently happens because we start with a
maniplex with a given Schlafli symbol and then take a quotient of it. In this
case, we store the given Schlafli symbol and call it a \emph{pseudo-Schlafli symbol} of \mbox{\texttt{\mdseries\slshape M}}. Note that whenever we compute the actual Schlafli symbol of \mbox{\texttt{\mdseries\slshape M}}, we update the pseudo-Schlafli symbol to match. }

 
\begin{Verbatim}[commandchars=!@|,fontsize=\small,frame=single,label=Example]
  !gapprompt@gap>| !gapinput@M := ReflexibleManiplex([4,4], "(r0 r1)^2");;|
  !gapprompt@gap>| !gapinput@PseudoSchlafliSymbol(M);|
  [4, 4]
  !gapprompt@gap>| !gapinput@SchlafliSymbol(M);|
  [2, 4]
  !gapprompt@gap>| !gapinput@PseudoSchlafliSymbol(M);|
  [2, 4]
\end{Verbatim}
 

\subsection{\textcolor{Chapter }{IsEquivelar (for IsManiplex)}}
\logpage{[ 10, 3, 3 ]}\nobreak
\hyperdef{L}{X7E9CCE9D794DC285}{}
{\noindent\textcolor{FuncColor}{$\triangleright$\enspace\texttt{IsEquivelar({\mdseries\slshape M})\index{IsEquivelar@\texttt{IsEquivelar}!for IsManiplex}
\label{IsEquivelar:for IsManiplex}
}\hfill{\scriptsize (property)}}\\
\textbf{\indent Returns:\ }
the the maniplex \mbox{\texttt{\mdseries\slshape M}} is equivelar; i.e., whether its Schlafli Symbol consists of integers at each
position (no lists). 



 

 }

 
\begin{Verbatim}[commandchars=!@|,fontsize=\small,frame=single,label=Example]
  !gapprompt@gap>| !gapinput@IsEquivelar(Bk2l(6,18));|
  true
\end{Verbatim}
 

\subsection{\textcolor{Chapter }{IsDegenerate (for IsManiplex)}}
\logpage{[ 10, 3, 4 ]}\nobreak
\hyperdef{L}{X8483BE798579B521}{}
{\noindent\textcolor{FuncColor}{$\triangleright$\enspace\texttt{IsDegenerate({\mdseries\slshape M})\index{IsDegenerate@\texttt{IsDegenerate}!for IsManiplex}
\label{IsDegenerate:for IsManiplex}
}\hfill{\scriptsize (property)}}\\
\textbf{\indent Returns:\ }
\texttt{true} or \texttt{false} 



 Returns whether the maniplex \mbox{\texttt{\mdseries\slshape M}} has any sections that are digons. We may eventually want to include maniplexes
with even smaller sections. }

 
\begin{Verbatim}[commandchars=!@|,fontsize=\small,frame=single,label=Example]
  !gapprompt@gap>| !gapinput@F := FreeGroup("s0","s1","s2","s3");;|
  !gapprompt@gap>| !gapinput@s0 := F.1;;  s1 := F.2;;  s2 := F.3;;  s3 := F.4;;  |
  !gapprompt@gap>| !gapinput@rels := [ s0*s0, s1*s1, s2*s2, s3*s3, s0*s2*s0*s2, |
  !gapprompt@>| !gapinput@s1*s2*s1*s2, s0*s3*s0*s3, s1*s3*s1*s3, |
  !gapprompt@>| !gapinput@s2*s3*s2*s3*s2*s3*s2*s3, s0*s1*s0*s1*s0*s1*s0*s1*s0*s1 ];;|
  !gapprompt@gap>| !gapinput@poly := F / rels;;|
  !gapprompt@gap>| !gapinput@IsDegenerate(ARP(poly));|
  true
\end{Verbatim}
 

\subsection{\textcolor{Chapter }{IsTight (for IsManiplex)}}
\logpage{[ 10, 3, 5 ]}\nobreak
\hyperdef{L}{X8088FFB2851334AB}{}
{\noindent\textcolor{FuncColor}{$\triangleright$\enspace\texttt{IsTight({\mdseries\slshape P})\index{IsTight@\texttt{IsTight}!for IsManiplex}
\label{IsTight:for IsManiplex}
}\hfill{\scriptsize (property)}}\\
\textbf{\indent Returns:\ }
\texttt{true} or \texttt{false} 



 Returns whether the polytope \mbox{\texttt{\mdseries\slshape P}} is tight, meaning that it has a Schlafli symbol
\texttt{\symbol{123}}k{\textunderscore}1, ...,
k{\textunderscore}\texttt{\symbol{123}}n-1\texttt{\symbol{125}}\texttt{\symbol{125}}
and has 2 k{\textunderscore}1 ...
k{\textunderscore}\texttt{\symbol{123}}n-1\texttt{\symbol{125}} flags, which
is the minimum possible. This property doesn't make any sense for
non-polytopal maniplexes, which aren't constrained by this lower bound. }

 
\begin{Verbatim}[commandchars=!@|,fontsize=\small,frame=single,label=Example]
  !gapprompt@gap>| !gapinput@IsTight(ToroidalMap44([2,0]));|
  true
\end{Verbatim}
 

\subsection{\textcolor{Chapter }{EulerCharacteristic (for IsManiplex)}}
\logpage{[ 10, 3, 6 ]}\nobreak
\hyperdef{L}{X7CEC34DF795B8DDF}{}
{\noindent\textcolor{FuncColor}{$\triangleright$\enspace\texttt{EulerCharacteristic({\mdseries\slshape M})\index{EulerCharacteristic@\texttt{EulerCharacteristic}!for IsManiplex}
\label{EulerCharacteristic:for IsManiplex}
}\hfill{\scriptsize (attribute)}}\\
\textbf{\indent Returns:\ }
The Euler characteristic of the maniplex, given by $f_0 - f_1 + f_2 - \cdots + (-1)^{n-1} f_{n-1}$. 



 

 }

 
\begin{Verbatim}[commandchars=!@|,fontsize=\small,frame=single,label=Example]
  !gapprompt@gap>| !gapinput@EulerCharacteristic(Bk2lStar(3,5));|
  -10
\end{Verbatim}
 

\subsection{\textcolor{Chapter }{Genus (for IsManiplex)}}
\logpage{[ 10, 3, 7 ]}\nobreak
\hyperdef{L}{X7FA861D87C58E1EF}{}
{\noindent\textcolor{FuncColor}{$\triangleright$\enspace\texttt{Genus({\mdseries\slshape M})\index{Genus@\texttt{Genus}!for IsManiplex}
\label{Genus:for IsManiplex}
}\hfill{\scriptsize (attribute)}}\\
\textbf{\indent Returns:\ }
The genus of the given 3-maniplex. 



 

 }

 
\begin{Verbatim}[commandchars=!@|,fontsize=\small,frame=single,label=Example]
  !gapprompt@gap>| !gapinput@Genus(Bk2lStar(3,5));|
  6
\end{Verbatim}
 

\subsection{\textcolor{Chapter }{IsSpherical (for IsManiplex)}}
\logpage{[ 10, 3, 8 ]}\nobreak
\hyperdef{L}{X84F212F485BA012F}{}
{\noindent\textcolor{FuncColor}{$\triangleright$\enspace\texttt{IsSpherical({\mdseries\slshape M})\index{IsSpherical@\texttt{IsSpherical}!for IsManiplex}
\label{IsSpherical:for IsManiplex}
}\hfill{\scriptsize (property)}}\\
\textbf{\indent Returns:\ }
Whether the 3-maniplex \mbox{\texttt{\mdseries\slshape M}} is spherical, which is to say, whether the Euler characteristic is equal to 2. 



 
\begin{Verbatim}[commandchars=!@|,fontsize=\small,frame=single,label=Example]
  !gapprompt@gap>| !gapinput@IsSpherical(Simplex(3));|
  true
  !gapprompt@gap>| !gapinput@IsSpherical(AbstractRegularPolytope([4,4],"h2^3"));|
  false
  !gapprompt@gap>| !gapinput@IsSpherical(Pyramid(5));|
  true
  !gapprompt@gap>| !gapinput@IsSpherical(CubicTiling(2));|
  false
\end{Verbatim}
 }

 

\subsection{\textcolor{Chapter }{IsLocallySpherical (for IsManiplex)}}
\logpage{[ 10, 3, 9 ]}\nobreak
\hyperdef{L}{X85B48E0678A65F40}{}
{\noindent\textcolor{FuncColor}{$\triangleright$\enspace\texttt{IsLocallySpherical({\mdseries\slshape M})\index{IsLocallySpherical@\texttt{IsLocallySpherical}!for IsManiplex}
\label{IsLocallySpherical:for IsManiplex}
}\hfill{\scriptsize (property)}}\\
\textbf{\indent Returns:\ }
Whether the 4-maniplex \mbox{\texttt{\mdseries\slshape M}} is locally spherical, which is to say, whether its facets and vertex-figures
are both spherical. 



 
\begin{Verbatim}[commandchars=!@|,fontsize=\small,frame=single,label=Example]
  !gapprompt@gap>| !gapinput@IsLocallySpherical(Simplex(4));|
  true
  !gapprompt@gap>| !gapinput@IsLocallySpherical(AbstractRegularPolytope([4,4,4]));|
  false
  !gapprompt@gap>| !gapinput@IsLocallySpherical(CubicTiling(3));|
  true
  !gapprompt@gap>| !gapinput@IsLocallySpherical(Pyramid(Cube(3)));|
  true
\end{Verbatim}
 }

 

\subsection{\textcolor{Chapter }{IsToroidal (for IsManiplex)}}
\logpage{[ 10, 3, 10 ]}\nobreak
\hyperdef{L}{X7886C68887E4DD20}{}
{\noindent\textcolor{FuncColor}{$\triangleright$\enspace\texttt{IsToroidal({\mdseries\slshape M})\index{IsToroidal@\texttt{IsToroidal}!for IsManiplex}
\label{IsToroidal:for IsManiplex}
}\hfill{\scriptsize (property)}}\\
\textbf{\indent Returns:\ }
Whether the 3-maniplex \mbox{\texttt{\mdseries\slshape M}} is toroidal, which is to say, whether the Euler characteristic is equal to 0. 



 
\begin{Verbatim}[commandchars=!@|,fontsize=\small,frame=single,label=Example]
  !gapprompt@gap>| !gapinput@IsToroidal(Simplex(3));|
  false
  !gapprompt@gap>| !gapinput@IsToroidal(AbstractRegularPolytope([4,4],"h2^3"));|
  true
  !gapprompt@gap>| !gapinput@IsToroidal(Pyramid(5));|
  false
\end{Verbatim}
 }

 

\subsection{\textcolor{Chapter }{IsLocallyToroidal (for IsManiplex)}}
\logpage{[ 10, 3, 11 ]}\nobreak
\hyperdef{L}{X7E0FF0C982FC49D2}{}
{\noindent\textcolor{FuncColor}{$\triangleright$\enspace\texttt{IsLocallyToroidal({\mdseries\slshape M})\index{IsLocallyToroidal@\texttt{IsLocallyToroidal}!for IsManiplex}
\label{IsLocallyToroidal:for IsManiplex}
}\hfill{\scriptsize (property)}}\\
\textbf{\indent Returns:\ }
Whether the 4-maniplex \mbox{\texttt{\mdseries\slshape M}} is locally toroidal, which is to say, whether it has at least one toroidal
facet or vertex-figure, and all of its facets and vertex-figures are either
spherical or toroidal. 



 
\begin{Verbatim}[commandchars=!@|,fontsize=\small,frame=single,label=Example]
  !gapprompt@gap>| !gapinput@IsLocallyToroidal(Simplex(4));|
  false
  !gapprompt@gap>| !gapinput@IsLocallyToroidal(AbstractRegularPolytope([4,4,3],"(r0 r1 r2 r1)^2"));|
  true
  !gapprompt@gap>| !gapinput@IsLocallyToroidal(AbstractRegularPolytope([4,4,4],"(r0 r1 r2 r1)^2, (r1 r2 r3 r2)^2"));|
  true
\end{Verbatim}
 }

 }

 
\section{\textcolor{Chapter }{Basics}}\label{Chapter_Combinatorics_and_Structure_Section_Basics}
\logpage{[ 10, 4, 0 ]}
\hyperdef{L}{X868F7BAB7AC2EEBC}{}
{
  

\subsection{\textcolor{Chapter }{Size (for IsManiplex)}}
\logpage{[ 10, 4, 1 ]}\nobreak
\hyperdef{L}{X7C2ADC507F19E2E3}{}
{\noindent\textcolor{FuncColor}{$\triangleright$\enspace\texttt{Size({\mdseries\slshape M})\index{Size@\texttt{Size}!for IsManiplex}
\label{Size:for IsManiplex}
}\hfill{\scriptsize (attribute)}}\\
\textbf{\indent Returns:\ }
The number of flags of the maniplex \mbox{\texttt{\mdseries\slshape M}}. 



 Synonym: \texttt{NumberOfFlags}. }

 

\subsection{\textcolor{Chapter }{RankManiplex (for IsPremaniplex)}}
\logpage{[ 10, 4, 2 ]}\nobreak
\hyperdef{L}{X7D63F98D7B6EBAC9}{}
{\noindent\textcolor{FuncColor}{$\triangleright$\enspace\texttt{RankManiplex({\mdseries\slshape M})\index{RankManiplex@\texttt{RankManiplex}!for IsPremaniplex}
\label{RankManiplex:for IsPremaniplex}
}\hfill{\scriptsize (attribute)}}\\
\textbf{\indent Returns:\ }
The rank of the premaniplex \mbox{\texttt{\mdseries\slshape M}}. 



 

 }

 }

 
\section{\textcolor{Chapter }{Zigzags and holes}}\label{Chapter_Combinatorics_and_Structure_Section_Zigzags_and_holes}
\logpage{[ 10, 5, 0 ]}
\hyperdef{L}{X780FC00678185849}{}
{
  

\subsection{\textcolor{Chapter }{ZigzagLength (for IsManiplex, IsInt)}}
\logpage{[ 10, 5, 1 ]}\nobreak
\hyperdef{L}{X7D0AAD597CDE5697}{}
{\noindent\textcolor{FuncColor}{$\triangleright$\enspace\texttt{ZigzagLength({\mdseries\slshape M, j})\index{ZigzagLength@\texttt{ZigzagLength}!for IsManiplex, IsInt}
\label{ZigzagLength:for IsManiplex, IsInt}
}\hfill{\scriptsize (operation)}}\\
\textbf{\indent Returns:\ }
The lengths of \mbox{\texttt{\mdseries\slshape j}}-zigzags of the 3-maniplex \mbox{\texttt{\mdseries\slshape M}}. 



 This corresponds to the lengths of orbits under r0 (r1
r2)\texttt{\symbol{94}}j. }

 
\begin{Verbatim}[commandchars=!@|,fontsize=\small,frame=single,label=Example]
  !gapprompt@gap>| !gapinput@ZigzagLength(Cube(3),1);|
  6
  !gapprompt@gap>| !gapinput@ZigzagLength(Cube(3),2);|
  6
  !gapprompt@gap>| !gapinput@ZigzagLength(Cube(3),3);|
  2
\end{Verbatim}
 

\subsection{\textcolor{Chapter }{ZigzagVector (for IsManiplex)}}
\logpage{[ 10, 5, 2 ]}\nobreak
\hyperdef{L}{X7FC09F34870E6CDE}{}
{\noindent\textcolor{FuncColor}{$\triangleright$\enspace\texttt{ZigzagVector({\mdseries\slshape M})\index{ZigzagVector@\texttt{ZigzagVector}!for IsManiplex}
\label{ZigzagVector:for IsManiplex}
}\hfill{\scriptsize (attribute)}}\\
\textbf{\indent Returns:\ }
The lengths of all zigzags of the 3-maniplex \mbox{\texttt{\mdseries\slshape M}}. 



 A rank 3 maniplex of type \texttt{\symbol{123}}p, q\texttt{\symbol{125}} has
Floor(q/2) distinct zigzag lengths because the j-zigzags are the same as the
(q-j)-zigzags. }

 
\begin{Verbatim}[commandchars=!@|,fontsize=\small,frame=single,label=Example]
  !gapprompt@gap>| !gapinput@ZigzagVector(Pseudorhombicuboctahedron());|
  [ [ 40, 56 ], [ 8, 32 ] ]
\end{Verbatim}
 

\subsection{\textcolor{Chapter }{PetrieLength (for IsManiplex)}}
\logpage{[ 10, 5, 3 ]}\nobreak
\hyperdef{L}{X7C4A38D780F6AECD}{}
{\noindent\textcolor{FuncColor}{$\triangleright$\enspace\texttt{PetrieLength({\mdseries\slshape M})\index{PetrieLength@\texttt{PetrieLength}!for IsManiplex}
\label{PetrieLength:for IsManiplex}
}\hfill{\scriptsize (attribute)}}\\
\textbf{\indent Returns:\ }
The length of the petrie polygons of the maniplex \mbox{\texttt{\mdseries\slshape M}}. 



 

 }

 
\begin{Verbatim}[commandchars=!@|,fontsize=\small,frame=single,label=Example]
  !gapprompt@gap>| !gapinput@PetrieLength(Cube(3));|
  6
\end{Verbatim}
 

\subsection{\textcolor{Chapter }{PetrieRelation (for IsInt, IsInt)}}
\logpage{[ 10, 5, 4 ]}\nobreak
\hyperdef{L}{X7C93C4708678EB87}{}
{\noindent\textcolor{FuncColor}{$\triangleright$\enspace\texttt{PetrieRelation({\mdseries\slshape i, j})\index{PetrieRelation@\texttt{PetrieRelation}!for IsInt, IsInt}
\label{PetrieRelation:for IsInt, IsInt}
}\hfill{\scriptsize (operation)}}\\
\textbf{\indent Returns:\ }
relation 



 Returns the Petrie relation for a rank \mbox{\texttt{\mdseries\slshape i}} maniplex of length \mbox{\texttt{\mdseries\slshape j}}. }

 
\begin{Verbatim}[commandchars=!@|,fontsize=\small,frame=single,label=Example]
  !gapprompt@gap>| !gapinput@p:=PetrieRelation(3,3);|
  "(r0r1r2)^3"
  !gapprompt@gap>| !gapinput@q:=Cube(3)/p;|
  3-maniplex
  !gapprompt@gap>| !gapinput@Size(q);|
  24
\end{Verbatim}
 

\subsection{\textcolor{Chapter }{HoleLength (for IsManiplex, IsInt)}}
\logpage{[ 10, 5, 5 ]}\nobreak
\hyperdef{L}{X80FF41567B259CC1}{}
{\noindent\textcolor{FuncColor}{$\triangleright$\enspace\texttt{HoleLength({\mdseries\slshape M, j})\index{HoleLength@\texttt{HoleLength}!for IsManiplex, IsInt}
\label{HoleLength:for IsManiplex, IsInt}
}\hfill{\scriptsize (operation)}}\\
\textbf{\indent Returns:\ }
The lengths of \mbox{\texttt{\mdseries\slshape j}}-holes of the 3-maniplex \mbox{\texttt{\mdseries\slshape M}}. 



 This corresponds to the lengths of orbits under r0 (r1
r2)\texttt{\symbol{94}}(j-1) r2. }

 
\begin{Verbatim}[commandchars=!@|,fontsize=\small,frame=single,label=Example]
  !gapprompt@gap>| !gapinput@HoleLength(ToroidalMap44([3,0]),2);|
  3
\end{Verbatim}
 

\subsection{\textcolor{Chapter }{HoleVector (for IsManiplex)}}
\logpage{[ 10, 5, 6 ]}\nobreak
\hyperdef{L}{X7EE84DCF7EFC02C6}{}
{\noindent\textcolor{FuncColor}{$\triangleright$\enspace\texttt{HoleVector({\mdseries\slshape M})\index{HoleVector@\texttt{HoleVector}!for IsManiplex}
\label{HoleVector:for IsManiplex}
}\hfill{\scriptsize (attribute)}}\\
\textbf{\indent Returns:\ }
The lengths of all zigzags of the 3-maniplex \mbox{\texttt{\mdseries\slshape M}}. 



 A rank 3 maniplex of type \texttt{\symbol{123}}p, q\texttt{\symbol{125}} has
Floor(q/2) distinct zigzag lengths because the j-zigzags are the same as the
(q-j)-zigzags. }

 
\begin{Verbatim}[commandchars=!@|,fontsize=\small,frame=single,label=Example]
  !gapprompt@gap>| !gapinput@HoleVector(ToroidalMap44([3,0],[0,5]));|
  [ [ 3, 5 ] ]
\end{Verbatim}
 }

 }

   
\chapter{\textcolor{Chapter }{Graphs for Maniplexes}}\label{Chapter_Graphs_for_Maniplexes}
\logpage{[ 11, 0, 0 ]}
\hyperdef{L}{X84DE212D847339C0}{}
{
  
\section{\textcolor{Chapter }{Graph families}}\label{Chapter_Graphs_for_Maniplexes_Section_Graph_families}
\logpage{[ 11, 1, 0 ]}
\hyperdef{L}{X8533D8D97CDE9DF1}{}
{
  

\subsection{\textcolor{Chapter }{HeawoodGraph}}
\logpage{[ 11, 1, 1 ]}\nobreak
\hyperdef{L}{X7DC6E7897C266B73}{}
{\noindent\textcolor{FuncColor}{$\triangleright$\enspace\texttt{HeawoodGraph({\mdseries\slshape })\index{HeawoodGraph@\texttt{HeawoodGraph}}
\label{HeawoodGraph}
}\hfill{\scriptsize (operation)}}\\
\textbf{\indent Returns:\ }
\texttt{IsGraph} 



 Heawood Graph as described at
https://www.distanceregular.org/graphs/heawood.html }

 

\subsection{\textcolor{Chapter }{PetersenGraph}}
\logpage{[ 11, 1, 2 ]}\nobreak
\hyperdef{L}{X823F43217A6C375D}{}
{\noindent\textcolor{FuncColor}{$\triangleright$\enspace\texttt{PetersenGraph({\mdseries\slshape })\index{PetersenGraph@\texttt{PetersenGraph}}
\label{PetersenGraph}
}\hfill{\scriptsize (operation)}}\\
\textbf{\indent Returns:\ }
\texttt{IsGraph} 



 Petersen Graph as described at
https://www.gap-system.org/Manuals/pkg/grape/htm/CHAP002.htm }

 

\subsection{\textcolor{Chapter }{CirculantGraph (for IsInt,IsList)}}
\logpage{[ 11, 1, 3 ]}\nobreak
\hyperdef{L}{X7E247B6783FBF9F9}{}
{\noindent\textcolor{FuncColor}{$\triangleright$\enspace\texttt{CirculantGraph({\mdseries\slshape int, list})\index{CirculantGraph@\texttt{CirculantGraph}!for IsInt,IsList}
\label{CirculantGraph:for IsInt,IsList}
}\hfill{\scriptsize (operation)}}\\
\textbf{\indent Returns:\ }
\texttt{IsGraph} 



 Given an integer n and a list L, this returns the Circulant Graph with n
vertices For each i in the list L and each vertex v, there is an edge from v
to v+i and v-i (mod n) }

 

\subsection{\textcolor{Chapter }{CompleteBipartiteGraph (for IsInt,IsInt)}}
\logpage{[ 11, 1, 4 ]}\nobreak
\hyperdef{L}{X7D9F8CF285289177}{}
{\noindent\textcolor{FuncColor}{$\triangleright$\enspace\texttt{CompleteBipartiteGraph({\mdseries\slshape int, list})\index{CompleteBipartiteGraph@\texttt{CompleteBipartiteGraph}!for IsInt,IsInt}
\label{CompleteBipartiteGraph:for IsInt,IsInt}
}\hfill{\scriptsize (operation)}}\\
\textbf{\indent Returns:\ }
\texttt{IsGraph} 



 Given two integers n, m, this returns the Complete Bipartite Graph
K{\textunderscore}\texttt{\symbol{123}}n,m\texttt{\symbol{125}} }

 }

 
\section{\textcolor{Chapter }{Graph constructors for maniplexes}}\label{Chapter_Graphs_for_Maniplexes_Section_Graph_constructors_for_maniplexes}
\logpage{[ 11, 2, 0 ]}
\hyperdef{L}{X8142EE7180458E04}{}
{
  

\subsection{\textcolor{Chapter }{DirectedGraphFromListOfEdges (for IsList,IsList)}}
\logpage{[ 11, 2, 1 ]}\nobreak
\hyperdef{L}{X7FB3733D81739F2C}{}
{\noindent\textcolor{FuncColor}{$\triangleright$\enspace\texttt{DirectedGraphFromListOfEdges({\mdseries\slshape list, list})\index{DirectedGraphFromListOfEdges@\texttt{DirectedGraphFromListOfEdges}!for IsList,IsList}
\label{DirectedGraphFromListOfEdges:for IsList,IsList}
}\hfill{\scriptsize (operation)}}\\
\textbf{\indent Returns:\ }
\texttt{IsGraph}. Note this returns a directed graph. 



 Given a list of vertices and a list of directed-edges (represented as ordered
pairs), this outputs the directed graph with the appropriate vertex and
directed-edge set. }

 Here we have a directed cycle on 3 vertices. 
\begin{Verbatim}[commandchars=!@|,fontsize=\small,frame=single,label=Example]
  !gapprompt@gap>| !gapinput@g:= DirectedGraphFromListOfEdges([1,2,3],[[1,2],[2,3],[3,1]]);|
  rec( adjacencies := [ [ 2 ], [ 3 ], [ 1 ] ], group := Group(()), 
   isGraph := true, names := [ 1, 2, 3 ], order := 3, 
   representatives := [ 1, 2, 3 ], schreierVector := [ -1, -2, -3 ] )
\end{Verbatim}
 

\subsection{\textcolor{Chapter }{GraphFromListOfEdges (for IsList,IsList)}}
\logpage{[ 11, 2, 2 ]}\nobreak
\hyperdef{L}{X82810A0A7CF30BB4}{}
{\noindent\textcolor{FuncColor}{$\triangleright$\enspace\texttt{GraphFromListOfEdges({\mdseries\slshape list, list})\index{GraphFromListOfEdges@\texttt{GraphFromListOfEdges}!for IsList,IsList}
\label{GraphFromListOfEdges:for IsList,IsList}
}\hfill{\scriptsize (operation)}}\\
\textbf{\indent Returns:\ }
\texttt{IsGraph}. Note this returns an undirected graph. 



 Given a list of vertices and a list of (directed) edges (represented as
ordered pairs), this outputs the simple underlying graph with the appropriate
vertex and directed-edge set. }

 Here we have a simple complete graph on 4 vertices. 
\begin{Verbatim}[commandchars=!@|,fontsize=\small,frame=single,label=Example]
  !gapprompt@gap>| !gapinput@g:= GraphFromListOfEdges([1,2,3,4],[[1,2],[2,3],[3,1], [1,4], [2,4], [3,4]]);|
  rec( 
   adjacencies := [ [ 2, 3, 4 ], [ 1, 3, 4 ], [ 1, 2, 4 ], [ 1, 2, 3 ] ],
   group := Group(()), isGraph := true, isSimple := true, 
   names := [ 1, 2, 3, 4 ], order := 4, representatives := [ 1, 2, 3, 4 ]
     , schreierVector := [ -1, -2, -3, -4 ] )
\end{Verbatim}
 

\subsection{\textcolor{Chapter }{UnlabeledFlagGraph (for IsGroup)}}
\logpage{[ 11, 2, 3 ]}\nobreak
\hyperdef{L}{X86BE61C88420C6F6}{}
{\noindent\textcolor{FuncColor}{$\triangleright$\enspace\texttt{UnlabeledFlagGraph({\mdseries\slshape group})\index{UnlabeledFlagGraph@\texttt{UnlabeledFlagGraph}!for IsGroup}
\label{UnlabeledFlagGraph:for IsGroup}
}\hfill{\scriptsize (operation)}}\\
\textbf{\indent Returns:\ }
\texttt{IsGraph}. Note this returns an undirected graph. 



 Given a group (assumed to be the connection group of a maniplex), this outputs
the simple underlying flag graph. }

 Here we build the flag graph for the cube from its connection group. 
\begin{Verbatim}[commandchars=!@|,fontsize=\small,frame=single,label=Example]
  !gapprompt@gap>| !gapinput@g:= UnlabeledFlagGraph(ConnectionGroup(Cube(3)));|
  rec( 
   adjacencies := [ [ 3, 11, 20 ], [ 7, 13, 18 ], [ 1, 4, 10 ], 
       [ 3, 25, 34 ], [ 26, 28, 35 ], [ 7, 13, 41 ], [ 2, 6, 8 ], 
       [ 7, 27, 32 ], [ 28, 33, 35 ], [ 3, 20, 45 ], [ 1, 14, 23 ], 
       [ 15, 17, 24 ], [ 2, 6, 31 ], [ 11, 25, 44 ], [ 12, 45, 47 ], 
       [ 18, 28, 40 ], [ 12, 19, 27 ], [ 2, 16, 21 ], [ 17, 22, 24 ], 
       [ 1, 10, 38 ], [ 18, 32, 40 ], [ 19, 41, 48 ], [ 11, 35, 44 ], 
       [ 12, 19, 34 ], [ 4, 14, 37 ], [ 5, 38, 42 ], [ 8, 17, 30 ], 
       [ 5, 9, 16 ], [ 39, 41, 48 ], [ 27, 32, 47 ], [ 13, 33, 39 ], 
       [ 8, 21, 30 ], [ 9, 31, 46 ], [ 4, 24, 37 ], [ 5, 9, 23 ], 
       [ 43, 45, 47 ], [ 25, 34, 48 ], [ 20, 26, 43 ], [ 29, 31, 46 ], 
       [ 16, 21, 42 ], [ 6, 22, 29 ], [ 26, 40, 43 ], [ 36, 38, 42 ], 
       [ 14, 23, 46 ], [ 10, 15, 36 ], [ 33, 39, 44 ], [ 15, 30, 36 ], 
       [ 22, 29, 37 ] ], group := Group(()), isGraph := true, 
   isSimple := true, names := [ 1 .. 48 ], order := 48, 
   representatives := [ 1, 2, 3, 4, 5, 6, 7, 8, 9, 10, 11, 12, 13, 14, 
       15, 16, 17, 18, 19, 20, 21, 22, 23, 24, 25, 26, 27, 28, 29, 30, 
       31, 32, 33, 34, 35, 36, 37, 38, 39, 40, 41, 42, 43, 44, 45, 46, 
       47, 48 ], 
   schreierVector := [ -1, -2, -3, -4, -5, -6, -7, -8, -9, -10, -11, 
       -12, -13, -14, -15, -16, -17, -18, -19, -20, -21, -22, -23, -24, 
       -25, -26, -27, -28, -29, -30, -31, -32, -33, -34, -35, -36, -37, 
       -38, -39, -40, -41, -42, -43, -44, -45, -46, -47, -48 ] )
\end{Verbatim}
 This also works with a maniplex input. Here we build the flag graph for the
cube. 
\begin{Verbatim}[commandchars=!@|,fontsize=\small,frame=single,label=Example]
  !gapprompt@gap>| !gapinput@g:= UnlabeledFlagGraph(Cube(3));|
\end{Verbatim}
 

\subsection{\textcolor{Chapter }{FlagGraphWithLabels (for IsGroup)}}
\logpage{[ 11, 2, 4 ]}\nobreak
\hyperdef{L}{X791AEF1C7AA7138C}{}
{\noindent\textcolor{FuncColor}{$\triangleright$\enspace\texttt{FlagGraphWithLabels({\mdseries\slshape group})\index{FlagGraphWithLabels@\texttt{FlagGraphWithLabels}!for IsGroup}
\label{FlagGraphWithLabels:for IsGroup}
}\hfill{\scriptsize (operation)}}\\
\textbf{\indent Returns:\ }
a triple [\texttt{IsGraph}, \texttt{IsList}, \texttt{IsList}]. 



 Given a group (assumed to be the connection group of a maniplex), this outputs
a triple [graph,list,list]. The graph is the unlabeled flag graph of the
connection group. The first list gives the undirected edges in the flag
graphs. The second list gives the labels for these edges. }

 Here we again build the flag graph for the cube from its connection group, but
this time keep track of labels of the edges. 
\begin{Verbatim}[commandchars=!@|,fontsize=\small,frame=single,label=Example]
  !gapprompt@gap>| !gapinput@g:= FlagGraphWithLabels(ConnectionGroup(Cube(3)));|
  [ rec( 
       adjacencies := [ [ 3, 11, 20 ], [ 7, 13, 18 ], [ 1, 4, 10 ], 
           [ 3, 25, 34 ], [ 26, 28, 35 ], [ 7, 13, 41 ], [ 2, 6, 8 ], 
           [ 7, 27, 32 ], [ 28, 33, 35 ], [ 3, 20, 45 ], [ 1, 14, 23 ], 
           [ 15, 17, 24 ], [ 2, 6, 31 ], [ 11, 25, 44 ], [ 12, 45, 47 ], 
           [ 18, 28, 40 ], [ 12, 19, 27 ], [ 2, 16, 21 ], 
           [ 17, 22, 24 ], [ 1, 10, 38 ], [ 18, 32, 40 ], 
           [ 19, 41, 48 ], [ 11, 35, 44 ], [ 12, 19, 34 ], 
           [ 4, 14, 37 ], [ 5, 38, 42 ], [ 8, 17, 30 ], [ 5, 9, 16 ], 
           [ 39, 41, 48 ], [ 27, 32, 47 ], [ 13, 33, 39 ], 
           [ 8, 21, 30 ], [ 9, 31, 46 ], [ 4, 24, 37 ], [ 5, 9, 23 ], 
           [ 43, 45, 47 ], [ 25, 34, 48 ], [ 20, 26, 43 ], 
           [ 29, 31, 46 ], [ 16, 21, 42 ], [ 6, 22, 29 ], 
           [ 26, 40, 43 ], [ 36, 38, 42 ], [ 14, 23, 46 ], 
           [ 10, 15, 36 ], [ 33, 39, 44 ], [ 15, 30, 36 ], 
           [ 22, 29, 37 ] ], group := Group(()), isGraph := true, 
       isSimple := true, names := [ 1 .. 48 ], order := 48, 
       representatives := [ 1, 2, 3, 4, 5, 6, 7, 8, 9, 10, 11, 12, 13, 
           14, 15, 16, 17, 18, 19, 20, 21, 22, 23, 24, 25, 26, 27, 28, 
           29, 30, 31, 32, 33, 34, 35, 36, 37, 38, 39, 40, 41, 42, 43, 
           44, 45, 46, 47, 48 ], 
       schreierVector := [ -1, -2, -3, -4, -5, -6, -7, -8, -9, -10, -11, 
           -12, -13, -14, -15, -16, -17, -18, -19, -20, -21, -22, -23, 
           -24, -25, -26, -27, -28, -29, -30, -31, -32, -33, -34, -35, 
           -36, -37, -38, -39, -40, -41, -42, -43, -44, -45, -46, -47, 
           -48 ] ), 
   [ [ 1, 3 ], [ 1, 11 ], [ 1, 20 ], [ 2, 7 ], [ 2, 13 ], [ 2, 18 ], 
       [ 3, 4 ], [ 3, 10 ], [ 4, 25 ], [ 4, 34 ], [ 5, 26 ], [ 5, 28 ], 
       [ 5, 35 ], [ 6, 7 ], [ 6, 13 ], [ 6, 41 ], [ 7, 8 ], [ 8, 27 ], 
       [ 8, 32 ], [ 9, 28 ], [ 9, 33 ], [ 9, 35 ], [ 10, 20 ], 
       [ 10, 45 ], [ 11, 14 ], [ 11, 23 ], [ 12, 15 ], [ 12, 17 ], 
       [ 12, 24 ], [ 13, 31 ], [ 14, 25 ], [ 14, 44 ], [ 15, 45 ], 
       [ 15, 47 ], [ 16, 18 ], [ 16, 28 ], [ 16, 40 ], [ 17, 19 ], 
       [ 17, 27 ], [ 18, 21 ], [ 19, 22 ], [ 19, 24 ], [ 20, 38 ], 
       [ 21, 32 ], [ 21, 40 ], [ 22, 41 ], [ 22, 48 ], [ 23, 35 ], 
       [ 23, 44 ], [ 24, 34 ], [ 25, 37 ], [ 26, 38 ], [ 26, 42 ], 
       [ 27, 30 ], [ 29, 39 ], [ 29, 41 ], [ 29, 48 ], [ 30, 32 ], 
       [ 30, 47 ], [ 31, 33 ], [ 31, 39 ], [ 33, 46 ], [ 34, 37 ], 
       [ 36, 43 ], [ 36, 45 ], [ 36, 47 ], [ 37, 48 ], [ 38, 43 ], 
       [ 39, 46 ], [ 40, 42 ], [ 42, 43 ], [ 44, 46 ] ], 
   [ 3, 2, 1, 3, 1, 2, 2, 1, 3, 1, 2, 3, 1, 1, 3, 2, 2, 1, 3, 1, 2, 3, 
       3, 2, 3, 1, 2, 3, 1, 2, 2, 1, 1, 3, 1, 2, 3, 1, 2, 3, 2, 3, 2, 2, 
       1, 1, 3, 2, 3, 2, 1, 1, 3, 3, 2, 3, 1, 1, 2, 1, 3, 3, 3, 2, 3, 1, 
       2, 3, 1, 2, 1, 2 ] ]
\end{Verbatim}
 This also works with a maniplex input. Here we build the flag graph for the
cube. 
\begin{Verbatim}[commandchars=!@|,fontsize=\small,frame=single,label=Example]
  !gapprompt@gap>| !gapinput@g:= FlagGraphWithLabels(Cube(3));|
\end{Verbatim}
 

\subsection{\textcolor{Chapter }{LayerGraph (for IsGroup, IsInt, IsInt)}}
\logpage{[ 11, 2, 5 ]}\nobreak
\hyperdef{L}{X809815D08250B57F}{}
{\noindent\textcolor{FuncColor}{$\triangleright$\enspace\texttt{LayerGraph({\mdseries\slshape [group, int, int]})\index{LayerGraph@\texttt{LayerGraph}!for IsGroup, IsInt, IsInt}
\label{LayerGraph:for IsGroup, IsInt, IsInt}
}\hfill{\scriptsize (operation)}}\\
\textbf{\indent Returns:\ }
\texttt{IsGraph}. Note this returns an undirected graph. 



 Given a group (assumed to be the connection group of a maniplex), and two
integers, this outputs the simple underlying graph given by incidences of
faces of those ranks. Note: There are no warnings yet to make sure that i,j
are bounded by the rank. }

 Here we build the graph given by the 6 faces and 12 edges of a cube from its
connection group. 
\begin{Verbatim}[commandchars=!@|,fontsize=\small,frame=single,label=Example]
  !gapprompt@gap>| !gapinput@g:= LayerGraph(ConnectionGroup(Cube(3)),2,1);|
  rec( 
   adjacencies := [ [ 7, 10, 12, 17 ], [ 8, 10, 15, 18 ], 
       [ 7, 9, 13, 14 ], [ 8, 11, 13, 16 ], [ 9, 12, 16, 18 ], 
       [ 11, 14, 15, 17 ], [ 1, 3 ], [ 2, 4 ], [ 3, 5 ], [ 1, 2 ], 
       [ 4, 6 ], [ 1, 5 ], [ 3, 4 ], [ 3, 6 ], [ 2, 6 ], [ 4, 5 ], 
       [ 1, 6 ], [ 2, 5 ] ], group := Group(()), isGraph := true, 
   isSimple := true, names := [ 1 .. 18 ], order := 18, 
   representatives := [ 1, 2, 3, 4, 5, 6, 7, 8, 9, 10, 11, 12, 13, 14, 
       15, 16, 17, 18 ], 
   schreierVector := [ -1, -2, -3, -4, -5, -6, -7, -8, -9, -10, -11, 
       -12, -13, -14, -15, -16, -17, -18 ] )
\end{Verbatim}
 This also works with a maniplex input. Here we build the graph given by the 6
faces and 12 edges of a cube. 
\begin{Verbatim}[commandchars=!@|,fontsize=\small,frame=single,label=Example]
  !gapprompt@gap>| !gapinput@g:= LayerGraph(Cube(3),2,1);;|
\end{Verbatim}
 

\subsection{\textcolor{Chapter }{Skeleton (for IsManiplex)}}
\logpage{[ 11, 2, 6 ]}\nobreak
\hyperdef{L}{X7BB443CD832DAC12}{}
{\noindent\textcolor{FuncColor}{$\triangleright$\enspace\texttt{Skeleton({\mdseries\slshape maniplex})\index{Skeleton@\texttt{Skeleton}!for IsManiplex}
\label{Skeleton:for IsManiplex}
}\hfill{\scriptsize (operation)}}\\
\textbf{\indent Returns:\ }
\texttt{IsGraph}. Note this returns an undirected graph. 



 Given a maniplex, this outputs the 0-1 skeleton. The vertices are the 0-faces,
and the edges are the 1-faces. }

 Here we build the skeleton of the dodecahedron. 
\begin{Verbatim}[commandchars=!@|,fontsize=\small,frame=single,label=Example]
  !gapprompt@gap>| !gapinput@g:= Skeleton(Dodecahedron());;|
\end{Verbatim}
 

\subsection{\textcolor{Chapter }{CoSkeleton (for IsManiplex)}}
\logpage{[ 11, 2, 7 ]}\nobreak
\hyperdef{L}{X85F3D43E7D1BA8E5}{}
{\noindent\textcolor{FuncColor}{$\triangleright$\enspace\texttt{CoSkeleton({\mdseries\slshape maniplex})\index{CoSkeleton@\texttt{CoSkeleton}!for IsManiplex}
\label{CoSkeleton:for IsManiplex}
}\hfill{\scriptsize (operation)}}\\
\textbf{\indent Returns:\ }
\texttt{IsGraph}. Note this returns an undirected graph. 



 Given a maniplex, this outputs the $(n-1)$-$(n-2)$ skeleton, i.e., the 0-1 skeleton of the dual. The vertices are the $(n-1)$-faces, and the edges are the $(n-2)$-faces. }

 Here we build the co-skeleton of the dodecahedron and verify that it is the
skeleton of the icosahedron. 
\begin{Verbatim}[commandchars=!@|,fontsize=\small,frame=single,label=Example]
  !gapprompt@gap>| !gapinput@g:=CoSkeleton(Dodecahedron());;|
  !gapprompt@gap>| !gapinput@h:=Skeleton(Icosahedron());;|
  !gapprompt@gap>| !gapinput@g=h;|
  true
\end{Verbatim}
 

\subsection{\textcolor{Chapter }{Hasse (for IsManiplex)}}
\logpage{[ 11, 2, 8 ]}\nobreak
\hyperdef{L}{X7A0FB13C7EF68CBB}{}
{\noindent\textcolor{FuncColor}{$\triangleright$\enspace\texttt{Hasse({\mdseries\slshape group})\index{Hasse@\texttt{Hasse}!for IsManiplex}
\label{Hasse:for IsManiplex}
}\hfill{\scriptsize (operation)}}\\
\textbf{\indent Returns:\ }
\texttt{IsGraph}. Note this returns a directed graph. 



 Given a group, assumed to be the connection group of a maniplex, this outputs
the Hasse Diagram as a directed graph. Note: The unique minimal and maximal
face are assumed. }

 Here we build the Hasse Diagram of a 3-simplex from its representation as a
maniplex. 
\begin{Verbatim}[commandchars=!@|,fontsize=\small,frame=single,label=Example]
  !gapprompt@gap>| !gapinput@Hasse(Simplex(3));|
  rec( 
   adjacencies := [ [  ], [ 1 ], [ 1 ], [ 1 ], [ 1 ], [ 2, 4 ], 
       [ 2, 3 ], [ 3, 5 ], [ 2, 5 ], [ 4, 5 ], [ 3, 4 ], [ 6, 9, 10 ], 
       [ 6, 7, 11 ], [ 8, 10, 11 ], [ 7, 8, 9 ], [ 12, 13, 14, 15 ] ], 
   group := Group(()), isGraph := true, names := [ 1 .. 16 ], 
   order := 16, 
   representatives := [ 1, 2, 3, 4, 5, 6, 7, 8, 9, 10, 11, 12, 13, 14, 
       15, 16 ], 
   schreierVector := [ -1, -2, -3, -4, -5, -6, -7, -8, -9, -10, -11, 
       -12, -13, -14, -15, -16 ] )
\end{Verbatim}
 

\subsection{\textcolor{Chapter }{QuotientByLabel (for IsObject,IsList, IsList, IsList)}}
\logpage{[ 11, 2, 9 ]}\nobreak
\hyperdef{L}{X860AACCF860A3F66}{}
{\noindent\textcolor{FuncColor}{$\triangleright$\enspace\texttt{QuotientByLabel({\mdseries\slshape object, list, list, list})\index{QuotientByLabel@\texttt{QuotientByLabel}!for IsObject,IsList, IsList, IsList}
\label{QuotientByLabel:for IsObject,IsList, IsList, IsList}
}\hfill{\scriptsize (operation)}}\\
\textbf{\indent Returns:\ }
\texttt{IsGraph}. Note this returns an undirected graph. 



 Given a graph, its edges, and its edge labels, and a sublist of labels, this
creates the underlying simple graph of the quotient identifying vertices
connected by labels not in the sublist. }

 Here we start with the flag graph of the 3-cube (with edge labels 1,2,3), and
identify any vertices not connected by edge by edges of label 1. We can then
check that this new graph is bipartite. 
\begin{Verbatim}[commandchars=!@|,fontsize=\small,frame=single,label=Example]
  !gapprompt@gap>| !gapinput@P:=Cube(3);;|
  !gapprompt@gap>| !gapinput@f:=FlagGraphWithLabels(P);;|
  !gapprompt@gap>| !gapinput@g:=f[1];;|
  !gapprompt@gap>| !gapinput@ed:=f[2];;|
  !gapprompt@gap>| !gapinput@lab:=f[3];  #Note This triple is to be replace by a single object.|
  [ 3, 2, 1, 3, 1, 2, 1, 2, 3, 2, 1, 3, 2, 1, 1, 3, 2, 2, 3, 1, 3, 1, 2, 3, 2, 1, 1, 2, 2, 3, 1, 3, 1, 2, 
    3, 1, 2, 1, 3, 2, 2, 1, 2, 2, 3, 1, 1, 3, 1, 3, 3, 2, 1, 2, 1, 3, 3, 1, 3, 2, 2, 2, 2, 3, 3, 1, 3, 1, 1, 3, 2, 3 ]
  !gapprompt@gap>| !gapinput@Q:=QuotientByLabel(g,ed,lab,[1]);|
  rec( adjacencies := [ [ 5, 6, 8 ], [ 3, 4, 7 ], [ 2, 6, 8 ], [ 2, 5, 8 ], [ 1, 4, 7 ], [ 1, 3, 7 ], [ 2, 5, 6 ], [ 1, 3, 4 ] ], group := Group(()), isGraph := true, 
   isSimple := true, names := [ 1 .. 8 ], order := 8, representatives := [ 1, 2, 3, 4, 5, 6, 7, 8 ], schreierVector := [ -1, -2, -3, -4, -5, -6, -7, -8 ] )
  !gapprompt@gap>| !gapinput@IsBipartite(Q);|
  true
\end{Verbatim}
 

\subsection{\textcolor{Chapter }{EdgeLabeledGraphFromEdges (for IsList, IsList,IsList)}}
\logpage{[ 11, 2, 10 ]}\nobreak
\hyperdef{L}{X84FE2C317CFA6AFD}{}
{\noindent\textcolor{FuncColor}{$\triangleright$\enspace\texttt{EdgeLabeledGraphFromEdges({\mdseries\slshape list, list, list})\index{EdgeLabeledGraphFromEdges@\texttt{EdgeLabeledGraphFromEdges}!for IsList, IsList,IsList}
\label{EdgeLabeledGraphFromEdges:for IsList, IsList,IsList}
}\hfill{\scriptsize (operation)}}\\
\textbf{\indent Returns:\ }
\texttt{IsEdgeLabeledGraph}. 



 Given a list of vertices, a list of edges, and a list of edge labels, this
represents the edge labeled (multi)-graph with those parameters. Semi-edges
are represented by a singleton in the edge list. Loops are represented by
edges [i,i] }

 Here we have an edge labeled cycle graph with 6 vertices and edges alternating
in labels 0,1. 
\begin{Verbatim}[commandchars=!@|,fontsize=\small,frame=single,label=Example]
  V:=[1..6];;
  Edges:=[[1,2],[2,3],[3,4],[4,5],[5,6],[6,1]];;
  L:=[0,1,0,1,0,1];;
  gamma:=EdgeLabeledGraphFromEdges(V,Edges,L);
\end{Verbatim}
 

\subsection{\textcolor{Chapter }{EdgeLabeledGraphFromLabeledEdges (for IsList)}}
\logpage{[ 11, 2, 11 ]}\nobreak
\hyperdef{L}{X83DD0B547D5DDE35}{}
{\noindent\textcolor{FuncColor}{$\triangleright$\enspace\texttt{EdgeLabeledGraphFromLabeledEdges({\mdseries\slshape list})\index{EdgeLabeledGraphFromLabeledEdges@\texttt{EdgeLabeledGraphFromLabeledEdges}!for IsList}
\label{EdgeLabeledGraphFromLabeledEdges:for IsList}
}\hfill{\scriptsize (operation)}}\\
\textbf{\indent Returns:\ }
\texttt{IsEdgeLabeledGraph}. 



 Given a list of labeled edges this represents the edge labeled (multi)-graph
with those parameters. Semi-edges are represented by a singleton in the edge
list. }

 
\begin{Verbatim}[commandchars=!@|,fontsize=\small,frame=single,label=Example]
  L:=[[[1],0],[[2],0],  [ [1,2],1]];;
  X2:=EdgeLabeledGraphFromLabeledEdges(L);
\end{Verbatim}
 

\subsection{\textcolor{Chapter }{FlagGraph (for IsGroup)}}
\logpage{[ 11, 2, 12 ]}\nobreak
\hyperdef{L}{X7A958E7C821DD8F8}{}
{\noindent\textcolor{FuncColor}{$\triangleright$\enspace\texttt{FlagGraph({\mdseries\slshape group})\index{FlagGraph@\texttt{FlagGraph}!for IsGroup}
\label{FlagGraph:for IsGroup}
}\hfill{\scriptsize (operation)}}\\
\textbf{\indent Returns:\ }
\texttt{IsEdgeLabeledGraph}. 



 Given group, assumed to be a connection group, output the labeled flag graph.
The input could also be a premaniplex, then the connection group is
calculated. }

 Here we have the flag graph of the 3-simplex from its connection group. 
\begin{Verbatim}[commandchars=!@|,fontsize=\small,frame=single,label=Example]
  !gapprompt@gap>| !gapinput@C:=ConnectionGroup(Simplex(3));;|
  !gapprompt@gap>| !gapinput@gamma:=FlagGraph(C);|
  Edge labeled graph with 24 vertices, and edge labels [ 0, 1, 2 ]
  !gapprompt@gap>| !gapinput@STG3(4,1);;|
  !gapprompt@gap>| !gapinput@FlagGraph(last);|
  Edge labeled graph with 3 vertices, and edge labels [ 0, 1, 2, 3 ]
\end{Verbatim}
 

\subsection{\textcolor{Chapter }{UnlabeledSimpleGraph (for IsEdgeLabeledGraph)}}
\logpage{[ 11, 2, 13 ]}\nobreak
\hyperdef{L}{X82A8C3AD871EC6D7}{}
{\noindent\textcolor{FuncColor}{$\triangleright$\enspace\texttt{UnlabeledSimpleGraph({\mdseries\slshape edge-labeled-graph})\index{UnlabeledSimpleGraph@\texttt{UnlabeledSimpleGraph}!for IsEdgeLabeledGraph}
\label{UnlabeledSimpleGraph:for IsEdgeLabeledGraph}
}\hfill{\scriptsize (operation)}}\\
\textbf{\indent Returns:\ }
\texttt{IsGraph}. 



 Given an edge labeled (multi) graph, it returns the underlying simple graph,
with semi-edges, loops, and muliple-edges removed. }

 Here we have underlying simple graph for the flag graph of the cube. 
\begin{Verbatim}[commandchars=!@|,fontsize=\small,frame=single,label=Example]
  gamma:=UnlabeledSimpleGraph(FlagGraph(Cube(3)));
\end{Verbatim}
 

\subsection{\textcolor{Chapter }{EdgeLabelPreservingAutomorphismGroup (for IsEdgeLabeledGraph)}}
\logpage{[ 11, 2, 14 ]}\nobreak
\hyperdef{L}{X84263C6C81AB458C}{}
{\noindent\textcolor{FuncColor}{$\triangleright$\enspace\texttt{EdgeLabelPreservingAutomorphismGroup({\mdseries\slshape edge-labeled-graph})\index{EdgeLabelPreservingAutomorphismGroup@\texttt{Edge}\-\texttt{Label}\-\texttt{Preserving}\-\texttt{Automorphism}\-\texttt{Group}!for IsEdgeLabeledGraph}
\label{EdgeLabelPreservingAutomorphismGroup:for IsEdgeLabeledGraph}
}\hfill{\scriptsize (operation)}}\\
\textbf{\indent Returns:\ }
\texttt{IsGroup}. 



 Given an edge labeled (multi) graph, it returns automorphism group (preserving
the labels). Note, for now the labels are assumed to be [1..n]. Note This
tends to be very slow. I would like to look for a way to go back and forth
between flag automorphisms and poset automorphisms, as the latter are much
faster to compute. }

 Here we have the automorphism group of the flag graph of the cube. 
\begin{Verbatim}[commandchars=!@|,fontsize=\small,frame=single,label=Example]
  g:=EdgeLabelPreservingAutomorphismGroup(FlagGraph(Cube(3)));;
  Size(g);
\end{Verbatim}
 

\subsection{\textcolor{Chapter }{Simple (for IsEdgeLabeledGraph)}}
\logpage{[ 11, 2, 15 ]}\nobreak
\hyperdef{L}{X7D7BF5608431FED8}{}
{\noindent\textcolor{FuncColor}{$\triangleright$\enspace\texttt{Simple({\mdseries\slshape edge-labeled-graph})\index{Simple@\texttt{Simple}!for IsEdgeLabeledGraph}
\label{Simple:for IsEdgeLabeledGraph}
}\hfill{\scriptsize (operation)}}\\
\textbf{\indent Returns:\ }
\texttt{IsEdgeLabeledGraph }. 



 Given an edge labeled (multi) graph, it returns another edge labeled graph
where semi-edges, loops, and multiple edges are removed. Note only the "first"
edge label is retained if there are multiple edges. }

 

\subsection{\textcolor{Chapter }{ConnectedComponents (for IsEdgeLabeledGraph, IsList)}}
\logpage{[ 11, 2, 16 ]}\nobreak
\hyperdef{L}{X8159C96B8757873F}{}
{\noindent\textcolor{FuncColor}{$\triangleright$\enspace\texttt{ConnectedComponents({\mdseries\slshape edge-labeled-graph})\index{ConnectedComponents@\texttt{ConnectedComponents}!for IsEdgeLabeledGraph, IsList}
\label{ConnectedComponents:for IsEdgeLabeledGraph, IsList}
}\hfill{\scriptsize (operation)}}\\
\textbf{\indent Returns:\ }
\texttt{IsGraph}. 



 Given an edge labeled (multi) graph and a list of labels, it returns connected
components of the graph not using edges in the list of labels. Note if the
second argument is not used, it is assumed to be an empty list, and the
connected components of the original graph are returned. }

 Here we see that each connected component of the flag graph of the cube (which
has labels 1,2,3) where edges of label 2 are removed, is a 4 cycle. 
\begin{Verbatim}[commandchars=!@|,fontsize=\small,frame=single,label=Example]
  gamma:=ConnectedComponents(FlagGraph(Cube(3)),[2]);
\end{Verbatim}
 

\subsection{\textcolor{Chapter }{PRGraph (for IsGroup)}}
\logpage{[ 11, 2, 17 ]}\nobreak
\hyperdef{L}{X7A53CE8B820FD6A1}{}
{\noindent\textcolor{FuncColor}{$\triangleright$\enspace\texttt{PRGraph({\mdseries\slshape group})\index{PRGraph@\texttt{PRGraph}!for IsGroup}
\label{PRGraph:for IsGroup}
}\hfill{\scriptsize (operation)}}\\
\textbf{\indent Returns:\ }
\texttt{IsEdgeLabeledGraph }. 



 Given a group, it returns the permutation representation graph for that group.
When the group is a string C-group this is also called a CPR graph. The labels
of the edges are [1...r] where r is the number of generators of the group. }

 Here we see the CPR graph of the automorphism group of a cube (acting on its 8
vertices). 
\begin{Verbatim}[commandchars=!@|,fontsize=\small,frame=single,label=Example]
  G:=AutomorphismGroup(Cube(3));
  H:=Group(G.2,G.3);
  phi:=FactorCosetAction(G,H);
  G2:=Range(phi);
  gamma:=PRGraph(G2);
\end{Verbatim}
 

\subsection{\textcolor{Chapter }{CPRGraphFromGroups (for IsGroup,IsGroup)}}
\logpage{[ 11, 2, 18 ]}\nobreak
\hyperdef{L}{X85BBB22F8196DC0B}{}
{\noindent\textcolor{FuncColor}{$\triangleright$\enspace\texttt{CPRGraphFromGroups({\mdseries\slshape group, subgroup})\index{CPRGraphFromGroups@\texttt{CPRGraphFromGroups}!for IsGroup,IsGroup}
\label{CPRGraphFromGroups:for IsGroup,IsGroup}
}\hfill{\scriptsize (operation)}}\\
\textbf{\indent Returns:\ }
\texttt{IsEdgeLabeledGraph}. 



 Given a group and a subgroup. Returns the graph of the action of the first
group on cosets of the subgroup. }

 

\subsection{\textcolor{Chapter }{AdjacentVertices (for IsEdgeLabeledGraph, IsObject)}}
\logpage{[ 11, 2, 19 ]}\nobreak
\hyperdef{L}{X7E52D3CB7A84139A}{}
{\noindent\textcolor{FuncColor}{$\triangleright$\enspace\texttt{AdjacentVertices({\mdseries\slshape EdgeLabeledGraph, vertex})\index{AdjacentVertices@\texttt{AdjacentVertices}!for IsEdgeLabeledGraph, IsObject}
\label{AdjacentVertices:for IsEdgeLabeledGraph, IsObject}
}\hfill{\scriptsize (operation)}}\\
\textbf{\indent Returns:\ }
\texttt{IsList}. 



 Takes in an edge labeled graph and a vertex, and outputs a list of the
adjacent vertices. }

 

\subsection{\textcolor{Chapter }{LabeledAdjacentVertices (for IsEdgeLabeledGraph, IsObject)}}
\logpage{[ 11, 2, 20 ]}\nobreak
\hyperdef{L}{X78BAE96984A79ACD}{}
{\noindent\textcolor{FuncColor}{$\triangleright$\enspace\texttt{LabeledAdjacentVertices({\mdseries\slshape EdgeLabeledGraph, vertex})\index{LabeledAdjacentVertices@\texttt{LabeledAdjacentVertices}!for IsEdgeLabeledGraph, IsObject}
\label{LabeledAdjacentVertices:for IsEdgeLabeledGraph, IsObject}
}\hfill{\scriptsize (operation)}}\\
\textbf{\indent Returns:\ }
\texttt{IsList, IsList}. 



 Takes in an edge labeled graph and a vertex, and outputs two lists: the list
of adjacent vertices, and the labels of the corresponding edges. }

 

\subsection{\textcolor{Chapter }{SemiEdges (for IsEdgeLabeledGraph)}}
\logpage{[ 11, 2, 21 ]}\nobreak
\hyperdef{L}{X791A59B97A44470F}{}
{\noindent\textcolor{FuncColor}{$\triangleright$\enspace\texttt{SemiEdges({\mdseries\slshape EdgeLabeledGraph})\index{SemiEdges@\texttt{SemiEdges}!for IsEdgeLabeledGraph}
\label{SemiEdges:for IsEdgeLabeledGraph}
}\hfill{\scriptsize (attribute)}}\\
\textbf{\indent Returns:\ }
\texttt{IsList}. 



 Takes in an edge labeled graph and a vertex, and outputs a list of semiedges }

 

\subsection{\textcolor{Chapter }{LabeledSemiEdges (for IsEdgeLabeledGraph)}}
\logpage{[ 11, 2, 22 ]}\nobreak
\hyperdef{L}{X82490C307F022606}{}
{\noindent\textcolor{FuncColor}{$\triangleright$\enspace\texttt{LabeledSemiEdges({\mdseries\slshape EdgeLabeledGraph})\index{LabeledSemiEdges@\texttt{LabeledSemiEdges}!for IsEdgeLabeledGraph}
\label{LabeledSemiEdges:for IsEdgeLabeledGraph}
}\hfill{\scriptsize (attribute)}}\\
\textbf{\indent Returns:\ }
\texttt{IsList, IsList}. 



 Takes in an edge labeled graph and a vertex, and outputs two lists: SemiEdges
and their labels }

 

\subsection{\textcolor{Chapter }{LabeledDarts (for IsEdgeLabeledGraph)}}
\logpage{[ 11, 2, 23 ]}\nobreak
\hyperdef{L}{X811C8FB3842D6D2E}{}
{\noindent\textcolor{FuncColor}{$\triangleright$\enspace\texttt{LabeledDarts({\mdseries\slshape EdgeLabeledGraph})\index{LabeledDarts@\texttt{LabeledDarts}!for IsEdgeLabeledGraph}
\label{LabeledDarts:for IsEdgeLabeledGraph}
}\hfill{\scriptsize (attribute)}}\\
\textbf{\indent Returns:\ }
\texttt{IsList}. 



 Takes in an edge labeled graph and outputs the labeled darts. }

 

\subsection{\textcolor{Chapter }{DerivedGraph (for IsList,IsList,IsList)}}
\logpage{[ 11, 2, 24 ]}\nobreak
\hyperdef{L}{X8529379678912775}{}
{\noindent\textcolor{FuncColor}{$\triangleright$\enspace\texttt{DerivedGraph({\mdseries\slshape list, list, list})\index{DerivedGraph@\texttt{DerivedGraph}!for IsList,IsList,IsList}
\label{DerivedGraph:for IsList,IsList,IsList}
}\hfill{\scriptsize (operation)}}\\
\textbf{\indent Returns:\ }
\texttt{IsEdgeLabeledGraph}. 



 Given a a pre-maniplex (entered as its vertices and labeled darts) and
voltages Return the connected derived graph from a pre-maniplex Careful, the
order of our automorphisms. Do we want them on left or right? Does it matter?
Can make another version with non-connected results, where the group is also
an input }

 Here we can build the flag graph of a 3-orbit polyhedron. 
\begin{Verbatim}[commandchars=!@|,fontsize=\small,frame=single,label=Example]
  !gapprompt@gap>| !gapinput@V:=[1,2,3];;|
  !gapprompt@gap>| !gapinput@Ed:=[[1],[1],[1,2],[2],[2,3],[3],[3]];;|
  !gapprompt@gap>| !gapinput@L:=[1,2,0,2,1,0,2];;|
  !gapprompt@gap>| !gapinput@g:=EdgeLabeledGraphFromEdges(V,Ed,L);;|
  !gapprompt@gap>| !gapinput@L:=LabeledDarts(g);;|
  !gapprompt@gap>| !gapinput@volt:=[ (1,2), (3,4), (), (), (3,4), (), (), (4,5), (2,3) ];;|
  !gapprompt@gap>| !gapinput@D:=DerivedGraph(V,L,volt);|
  Edge labeled graph with 360 vertices, and edge labels [ 0, 1, 2 ]
\end{Verbatim}
 

\subsection{\textcolor{Chapter }{ViewGraph (for IsObject, IsString)}}
\logpage{[ 11, 2, 25 ]}\nobreak
\hyperdef{L}{X83990BFD7D887A02}{}
{\noindent\textcolor{FuncColor}{$\triangleright$\enspace\texttt{ViewGraph({\mdseries\slshape G, software{\textunderscore}name})\index{ViewGraph@\texttt{ViewGraph}!for IsObject, IsString}
\label{ViewGraph:for IsObject, IsString}
}\hfill{\scriptsize (operation)}}\\
\textbf{\indent Returns:\ }
\texttt{IsString}. 



 Given a Graph or EdgeLabeledGraph \mbox{\texttt{\mdseries\slshape G}}, outputs code to view the graph in other software. Currently mathematica and
sage are supported. }

 

\subsection{\textcolor{Chapter }{ConnectionGroup (for IsEdgeLabeledGraph)}}
\logpage{[ 11, 2, 26 ]}\nobreak
\hyperdef{L}{X8147A743805AD49D}{}
{\noindent\textcolor{FuncColor}{$\triangleright$\enspace\texttt{ConnectionGroup({\mdseries\slshape F})\index{ConnectionGroup@\texttt{ConnectionGroup}!for IsEdgeLabeledGraph}
\label{ConnectionGroup:for IsEdgeLabeledGraph}
}\hfill{\scriptsize (attribute)}}\\
\textbf{\indent Returns:\ }
\texttt{IsPermGroup} 



 Constructs the connection group from an edge labeled graph. Loops, semi-edges,
and non-edges give fixed points. Graph is assumed to be coming from a
maniplex. Some weird things could happen if it is not }

 }

 }

   
\chapter{\textcolor{Chapter }{Databases}}\label{Chapter_Databases}
\logpage{[ 12, 0, 0 ]}
\hyperdef{L}{X7EB183C3780A475B}{}
{
  We are indebted to those who have made their data on polytopes and maps freely
available. Data on small regular polytopes is from Marston Conder: 

 https://www.math.auckland.ac.nz/\texttt{\symbol{126}}conder/RegularPolytopesWithUpTo4000Flags-ByOrder.txt 

 Data on small reflexible maniplexes was produced for RAMP by Mark Mixer. 

 Data on small chiral polytopes is from Marston Conder: 

 https://www.math.auckland.ac.nz/\texttt{\symbol{126}}conder/ChiralPolytopesWithUpTo4000Flags-ByOrder.txt 

 Data on small 2-orbit polyhedra in class 2{\textunderscore}0 (available in
Rank3AG{\textunderscore}2{\textunderscore}0.txt in the data folder) was
produced for RAMP by Mark Mixer. 
\section{\textcolor{Chapter }{Regular polyhedra}}\label{Chapter_Databases_Section_Regular_polyhedra}
\logpage{[ 12, 1, 0 ]}
\hyperdef{L}{X8062E376879531A7}{}
{
  

\subsection{\textcolor{Chapter }{WriteManiplexesToFile}}
\logpage{[ 12, 1, 1 ]}\nobreak
\hyperdef{L}{X87BC7EE2835BE22A}{}
{\noindent\textcolor{FuncColor}{$\triangleright$\enspace\texttt{WriteManiplexesToFile({\mdseries\slshape maniplexes, filename, attributeNames})\index{WriteManiplexesToFile@\texttt{WriteManiplexesToFile}}
\label{WriteManiplexesToFile}
}\hfill{\scriptsize (function)}}\\


 Writes the data in \mbox{\texttt{\mdseries\slshape maniplexes}} to the designated file, including the defining information and the values of
the attributes in \mbox{\texttt{\mdseries\slshape attributeNames}}. This calls \texttt{DatabaseString} on each maniplex in \mbox{\texttt{\mdseries\slshape maniplexes}} to get the file representation. }

 

\subsection{\textcolor{Chapter }{ManiplexesFromFile}}
\logpage{[ 12, 1, 2 ]}\nobreak
\hyperdef{L}{X81B0523079986448}{}
{\noindent\textcolor{FuncColor}{$\triangleright$\enspace\texttt{ManiplexesFromFile({\mdseries\slshape filename})\index{ManiplexesFromFile@\texttt{ManiplexesFromFile}}
\label{ManiplexesFromFile}
}\hfill{\scriptsize (function)}}\\
\textbf{\indent Returns:\ }
IsList 



 Reads the maniplexes from \mbox{\texttt{\mdseries\slshape filename}} in the data directory of RAMP and returns them as a list. Note that for
performance reasons, some safety checks are disabled for data read from a
file. For example, \texttt{AbstractRegularPolytope} usually checks its input to make sure that it defines a polytope, but \texttt{ManiplexesFromFile} just assumes that any maniplex defined using \texttt{AbstractRegularPolytope} really is a polytope. }

 

\subsection{\textcolor{Chapter }{DegeneratePolyhedra}}
\logpage{[ 12, 1, 3 ]}\nobreak
\hyperdef{L}{X7B2285CD79A7DF27}{}
{\noindent\textcolor{FuncColor}{$\triangleright$\enspace\texttt{DegeneratePolyhedra({\mdseries\slshape sizerange})\index{DegeneratePolyhedra@\texttt{DegeneratePolyhedra}}
\label{DegeneratePolyhedra}
}\hfill{\scriptsize (function)}}\\
\textbf{\indent Returns:\ }
IsList 



 Gives all degenerate polyhedra (of type $\{2, q\}$ and $\{p, 2\}$) with sizes in \mbox{\texttt{\mdseries\slshape sizerange}}. Also accepts a single integer \emph{maxsize} as input to indicate a sizerange of \texttt{[1..maxsize]}. }

 
\begin{Verbatim}[commandchars=!@|,fontsize=\small,frame=single,label=Example]
  !gapprompt@gap>| !gapinput@DegeneratePolyhedra(24);|
  [ AbstractRegularPolytope([ 2, 2 ]), AbstractRegularPolytope([ 2, 3 ]), 
    AbstractRegularPolytope([ 3, 2 ]), AbstractRegularPolytope([ 2, 4 ]), 
    AbstractRegularPolytope([ 4, 2 ]), AbstractRegularPolytope([ 2, 5 ]), 
    AbstractRegularPolytope([ 5, 2 ]), AbstractRegularPolytope([ 2, 6 ]), 
    AbstractRegularPolytope([ 6, 2 ]) ]
\end{Verbatim}
 

\subsection{\textcolor{Chapter }{FlatRegularPolyhedra}}
\logpage{[ 12, 1, 4 ]}\nobreak
\hyperdef{L}{X86869B8A80564ACF}{}
{\noindent\textcolor{FuncColor}{$\triangleright$\enspace\texttt{FlatRegularPolyhedra({\mdseries\slshape sizerange})\index{FlatRegularPolyhedra@\texttt{FlatRegularPolyhedra}}
\label{FlatRegularPolyhedra}
}\hfill{\scriptsize (function)}}\\
\textbf{\indent Returns:\ }
IsList 



 Gives all nondegenerate flat regular polyhedra with sizes in \mbox{\texttt{\mdseries\slshape sizerange}}. Also accepts a single integer \emph{maxsize} as input to indicate a sizerange of \texttt{[1..maxsize]}. Currently supports a maxsize of 4000 or less. }

 
\begin{Verbatim}[commandchars=!@|,fontsize=\small,frame=single,label=Example]
  !gapprompt@gap>| !gapinput@FlatRegularPolyhedra([10..24]);|
  [ AbstractRegularPolytope([ 2, 3 ]), AbstractRegularPolytope([ 3, 2 ]), 
    AbstractRegularPolytope([ 2, 4 ]), AbstractRegularPolytope([ 4, 2 ]), 
    AbstractRegularPolytope([ 2, 5 ]), AbstractRegularPolytope([ 5, 2 ]), 
    AbstractRegularPolytope([ 4, 3 ], "r2 r1 r0 r1 = (r0 r1)^2 r1 (r1 r2)^1, r2 r1 r2 r1 r0 r1 = (r0\
   r1)^3 (r1 r2)^2"), 
    ReflexibleManiplex([ 3, 4 ], "(r2*r1)^2*r1^2*r0*r1*r2*r1*r0,(r2*r1)^3*(r1*r0)^2*r1*r2*(r1*r0)^2"\
  ), AbstractRegularPolytope([ 2, 6 ]), AbstractRegularPolytope([ 6, 2 ]) ]
\end{Verbatim}
 

\subsection{\textcolor{Chapter }{RegularToroidalPolyhedra44}}
\logpage{[ 12, 1, 5 ]}\nobreak
\hyperdef{L}{X8229CA838473FF2B}{}
{\noindent\textcolor{FuncColor}{$\triangleright$\enspace\texttt{RegularToroidalPolyhedra44({\mdseries\slshape sizerange})\index{RegularToroidalPolyhedra44@\texttt{RegularToroidalPolyhedra44}}
\label{RegularToroidalPolyhedra44}
}\hfill{\scriptsize (function)}}\\
\textbf{\indent Returns:\ }
IsList 



 Gives all regular toroidal polyhedra of type $\{4,4\}$ with sizes in \mbox{\texttt{\mdseries\slshape sizerange}}. Also accepts a single integer \emph{maxsize} as input to indicate a sizerange of \texttt{[1..maxsize]}. }

 
\begin{Verbatim}[commandchars=!@|,fontsize=\small,frame=single,label=Example]
  !gapprompt@gap>| !gapinput@RegularToroidalPolyhedra44([60..100]);|
  [ AbstractRegularPolytope([ 4, 4 ], "(r0 r1 r2)^4"), 
    AbstractRegularPolytope([ 4, 4 ], "(r0 r1 r2 r1)^3") ]
\end{Verbatim}
 

\subsection{\textcolor{Chapter }{RegularToroidalPolyhedra36}}
\logpage{[ 12, 1, 6 ]}\nobreak
\hyperdef{L}{X783D35FD79C25457}{}
{\noindent\textcolor{FuncColor}{$\triangleright$\enspace\texttt{RegularToroidalPolyhedra36({\mdseries\slshape sizerange})\index{RegularToroidalPolyhedra36@\texttt{RegularToroidalPolyhedra36}}
\label{RegularToroidalPolyhedra36}
}\hfill{\scriptsize (function)}}\\
\textbf{\indent Returns:\ }
IsList 



 Gives all regular toroidal polyhedra of type $\{3,6\}$ with sizes in \mbox{\texttt{\mdseries\slshape sizerange}}. Also accepts a single integer \emph{maxsize} as input to indicate a sizerange of \texttt{[1..maxsize]}. }

 
\begin{Verbatim}[commandchars=!@|,fontsize=\small,frame=single,label=Example]
  !gapprompt@gap>| !gapinput@RegularToroidalPolyhedra36([100..150]);|
  [ AbstractRegularPolytope([ 3, 6 ], "(r0 r1 r2)^6"), 
    AbstractRegularPolytope([ 3, 6 ], "(r0 r1 r2 r1 r2)^4") ]
\end{Verbatim}
 

\subsection{\textcolor{Chapter }{SmallRegularPolyhedraFromFile}}
\logpage{[ 12, 1, 7 ]}\nobreak
\hyperdef{L}{X814F1BC07C480F5D}{}
{\noindent\textcolor{FuncColor}{$\triangleright$\enspace\texttt{SmallRegularPolyhedraFromFile({\mdseries\slshape sizerange})\index{SmallRegularPolyhedraFromFile@\texttt{SmallRegularPolyhedraFromFile}}
\label{SmallRegularPolyhedraFromFile}
}\hfill{\scriptsize (function)}}\\
\textbf{\indent Returns:\ }
IsList 



 Gives all regular polyhedra with sizes in \mbox{\texttt{\mdseries\slshape sizerange}} flags that are stored separately in a file. These are polyhedra that are not
part of one of several infinite families that are covered by the other
generators. The return value of this function is unstable and may change as
more infinite familes of polyhedra are identified and written as separate
generators. }

 
\begin{Verbatim}[commandchars=!@|,fontsize=\small,frame=single,label=Example]
  !gapprompt@gap>| !gapinput@SmallRegularPolyhedraFromFile(64);|
  [ Simplex(3), AbstractRegularPolytope([ 3, 6 ], "(r2*r0*r1)^2*(r0*r2*r1)^2 "), CrossPolytope(3), 
    AbstractRegularPolytope([ 6, 3 ], "(r0*r2*r1)^2*(r2*r0*r1)^2 "), Cube(3), 
    AbstractRegularPolytope([ 5, 5 ], "r1*r2*r0*(r1*r0*r2)^2 "), 
    AbstractRegularPolytope([ 3, 5 ], "(r2*r0*r1)^3*(r0*r2*r1)^2 "), 
    AbstractRegularPolytope([ 5, 3 ], "(r0*r2*r1)^3*(r2*r0*r1)^2 ") ]
\end{Verbatim}
 

\subsection{\textcolor{Chapter }{SmallRegularPolyhedra}}
\logpage{[ 12, 1, 8 ]}\nobreak
\hyperdef{L}{X7BD5DABD833B9700}{}
{\noindent\textcolor{FuncColor}{$\triangleright$\enspace\texttt{SmallRegularPolyhedra({\mdseries\slshape sizerange})\index{SmallRegularPolyhedra@\texttt{SmallRegularPolyhedra}}
\label{SmallRegularPolyhedra}
}\hfill{\scriptsize (function)}}\\
\textbf{\indent Returns:\ }
IsList 



 Gives all regular polyhedra with sizes in \mbox{\texttt{\mdseries\slshape sizerange}} flags. Currently supports a \texttt{maxsize} of 4000 or less. You can also set options \texttt{nondegenerate}, \texttt{nonflat}, and \texttt{nontoroidal}. }

 
\begin{Verbatim}[commandchars=!@|,fontsize=\small,frame=single,label=Example]
  !gapprompt@gap>| !gapinput@L1 := SmallRegularPolyhedra(500);;|
  !gapprompt@gap>| !gapinput@L2 := SmallRegularPolyhedra(1000 : nondegenerate);;|
  !gapprompt@gap>| !gapinput@L3 := SmallRegularPolyhedra(2000 : nondegenerate, nonflat);;|
  !gapprompt@gap>| !gapinput@Length(SmallRegularPolyhedra(64));|
  53
\end{Verbatim}
 

\subsection{\textcolor{Chapter }{SmallDegenerateRegular4Polytopes}}
\logpage{[ 12, 1, 9 ]}\nobreak
\hyperdef{L}{X8670A2B078096E46}{}
{\noindent\textcolor{FuncColor}{$\triangleright$\enspace\texttt{SmallDegenerateRegular4Polytopes({\mdseries\slshape sizerange})\index{SmallDegenerateRegular4Polytopes@\texttt{SmallDegenerateRegular4Polytopes}}
\label{SmallDegenerateRegular4Polytopes}
}\hfill{\scriptsize (function)}}\\
\textbf{\indent Returns:\ }
IsList 



 Gives all degenerate regular 4-polytopes with sizes in \mbox{\texttt{\mdseries\slshape sizerange}} flags. Currently supports a \texttt{maxsize} of 8000 or less. }

 
\begin{Verbatim}[commandchars=!@|,fontsize=\small,frame=single,label=Example]
  !gapprompt@gap>| !gapinput@ SmallDegenerateRegular4Polytopes([64]);|
  [ AbstractRegularPolytope([ 4, 2, 4 ]), AbstractRegularPolytope([ 2, 8, 2 ]), 
    regular 4-polytope of type [ 4, 4, 2 ] with 64 flags, 
    ReflexibleManiplex([ 2, 4, 4 ], "(r2*r1*r2*r3)^2,(r1*r2*r3*r2)^2") ]
\end{Verbatim}
 

\subsection{\textcolor{Chapter }{SmallRegular4Polytopes}}
\logpage{[ 12, 1, 10 ]}\nobreak
\hyperdef{L}{X83C76844861B7B5A}{}
{\noindent\textcolor{FuncColor}{$\triangleright$\enspace\texttt{SmallRegular4Polytopes({\mdseries\slshape sizerange})\index{SmallRegular4Polytopes@\texttt{SmallRegular4Polytopes}}
\label{SmallRegular4Polytopes}
}\hfill{\scriptsize (function)}}\\
\textbf{\indent Returns:\ }
IsList 



 Gives all regular 4-polytopes with sizes in \mbox{\texttt{\mdseries\slshape sizerange}} flags. Currently supports a \texttt{maxsize} of 4000 or less. }

 
\begin{Verbatim}[commandchars=!@|,fontsize=\small,frame=single,label=Example]
  !gapprompt@gap>| !gapinput@SmallRegular4Polytopes([100]);|
  [ AbstractRegularPolytope([ 5, 2, 5 ]) ]
\end{Verbatim}
 

\subsection{\textcolor{Chapter }{SmallChiralPolyhedra}}
\logpage{[ 12, 1, 11 ]}\nobreak
\hyperdef{L}{X7C5BEAD87A858EE7}{}
{\noindent\textcolor{FuncColor}{$\triangleright$\enspace\texttt{SmallChiralPolyhedra({\mdseries\slshape sizerange})\index{SmallChiralPolyhedra@\texttt{SmallChiralPolyhedra}}
\label{SmallChiralPolyhedra}
}\hfill{\scriptsize (function)}}\\
\textbf{\indent Returns:\ }
IsList 



 Gives all chiral polyhedra with sizes in \mbox{\texttt{\mdseries\slshape sizerange}} flags. Currently supports a \texttt{maxsize} of 4000 or less. }

 
\begin{Verbatim}[commandchars=!@|,fontsize=\small,frame=single,label=Example]
  !gapprompt@gap>| !gapinput@SmallChiralPolyhedra(100);|
  [ AbstractRotaryPolytope([ 4, 4 ], "s1*s2^-2*s1^2*s2^-1,(s1^-1*s2^-1)^2"), 
    AbstractRotaryPolytope([ 4, 4 ], "s2*s1^-1*s2*s1^2*s2^2*s1^-1,(s1^-1*s2^-1)^2"), 
    AbstractRotaryPolytope([ 3, 6 ], "s2^-1*s1*s2^-2*s1^-1*s2*s1^-1*s2^-2,(s1^-1*s2^-1)^2"), 
    AbstractRotaryPolytope([ 6, 3 ], "s1*s2^-1*s1^2*s2*s1^-1*s2*s1^2,(s2*s1)^2") ]
\end{Verbatim}
 

\subsection{\textcolor{Chapter }{SmallChiral4Polytopes}}
\logpage{[ 12, 1, 12 ]}\nobreak
\hyperdef{L}{X85ED37A97CE7F77C}{}
{\noindent\textcolor{FuncColor}{$\triangleright$\enspace\texttt{SmallChiral4Polytopes({\mdseries\slshape sizerange})\index{SmallChiral4Polytopes@\texttt{SmallChiral4Polytopes}}
\label{SmallChiral4Polytopes}
}\hfill{\scriptsize (function)}}\\
\textbf{\indent Returns:\ }
IsList 



 Gives all chiral 4-polytopes with sizes in \mbox{\texttt{\mdseries\slshape sizerange}} flags. Currently supports a \texttt{maxsize} of 4000 or less. }

 
\begin{Verbatim}[commandchars=!@|,fontsize=\small,frame=single,label=Example]
  !gapprompt@gap>| !gapinput@SmallChiral4Polytopes([200..250]);|
  [ AbstractRotaryPolytope([ 3, 4, 4 ], "s3^-1*s2^-2*s1^-1*s3*s1,s2^-1*s3^-2*s2^2*s3,(s2^-1*s3^-1)^2,(s1^-1*s2^-1)^2"), 
    AbstractRotaryPolytope([ 4, 4, 3 ], "s1*s2^2*s3*s1^-1*s3^-1,s2*s1^2*s2^-2*s1^-1,(s2*s1)^2,(s3*s2)^2"), 
    AbstractRotaryPolytope([ 4, 4, 4 ], "s2*s3^-2*s2^2*s3^-1,s3*s2*s1^-1*s3^2*s1,s3^-1*s2^-2*s1^-1*s3*s1,(s2^-1*s3^-1)^2,s1^-1*s2^-2*s1^2*s2,(s1^-1*s2^-1)^2") ]
\end{Verbatim}
 

\subsection{\textcolor{Chapter }{SmallReflexible3Maniplexes}}
\logpage{[ 12, 1, 13 ]}\nobreak
\hyperdef{L}{X7ED0E4347B0192A6}{}
{\noindent\textcolor{FuncColor}{$\triangleright$\enspace\texttt{SmallReflexible3Maniplexes({\mdseries\slshape sizerange})\index{SmallReflexible3Maniplexes@\texttt{SmallReflexible3Maniplexes}}
\label{SmallReflexible3Maniplexes}
}\hfill{\scriptsize (function)}}\\
\textbf{\indent Returns:\ }
IsList 



 Gives all regular 3-polytopes with sizes in \mbox{\texttt{\mdseries\slshape sizerange}} flags. Currently supports a \texttt{maxsize} of 2000 or less. If the option \texttt{nonpolytopal} is set, only returns maniplexes that are not polyhedra. }

 

\subsection{\textcolor{Chapter }{SmallReflexibleManiplexes}}
\logpage{[ 12, 1, 14 ]}\nobreak
\hyperdef{L}{X8082983086E31719}{}
{\noindent\textcolor{FuncColor}{$\triangleright$\enspace\texttt{SmallReflexibleManiplexes({\mdseries\slshape n, sizerange[, filt1, filt2, ...]})\index{SmallReflexibleManiplexes@\texttt{SmallReflexibleManiplexes}}
\label{SmallReflexibleManiplexes}
}\hfill{\scriptsize (function)}}\\
\textbf{\indent Returns:\ }
IsList 



 First finds a list of all reflexible maniplexes of rank \mbox{\texttt{\mdseries\slshape n}} where the number of flags is in \mbox{\texttt{\mdseries\slshape sizerange}}. Then applies the given filters and returns the result. Each filter is either
a function-value pair or a boolean function. In the first case, we keep only
those maniplexes such that applying the given function returns the given
value. In the second case, we keep only those maniplexes such that the given
boolean function returns \texttt{true}. }

 
\begin{Verbatim}[commandchars=!@|,fontsize=\small,frame=single,label=Example]
  !gapprompt@gap>| !gapinput@L := SmallReflexibleManiplexes(3, [100..200], IsPolytopal, [NumberOfVertices, 6]);;|
  !gapprompt@gap>| !gapinput@Size(L);|
  14
  !gapprompt@gap>| !gapinput@ForAll(L, IsPolytopal);|
  true
  !gapprompt@gap>| !gapinput@List(L, NumberOfVertices);|
  [ 6, 6, 6, 6, 6, 6, 6, 6, 6, 6, 6, 6, 6, 6 ]
\end{Verbatim}
 

\subsection{\textcolor{Chapter }{SmallTwoOrbit3Maniplexes}}
\logpage{[ 12, 1, 15 ]}\nobreak
\hyperdef{L}{X7D30D2C37BD1B341}{}
{\noindent\textcolor{FuncColor}{$\triangleright$\enspace\texttt{SmallTwoOrbit3Maniplexes({\mdseries\slshape I, sizerange})\index{SmallTwoOrbit3Maniplexes@\texttt{SmallTwoOrbit3Maniplexes}}
\label{SmallTwoOrbit3Maniplexes}
}\hfill{\scriptsize (function)}}\\
\textbf{\indent Returns:\ }
IsList 



 Gives all two-orbit 3-maniplexes in class $2_I$ with sizes in \mbox{\texttt{\mdseries\slshape sizerange}} flags. Currently supports a \texttt{maxsize} of 1000 or less. }

 
\begin{Verbatim}[commandchars=!@|,fontsize=\small,frame=single,label=Example]
  !gapprompt@gap>| !gapinput@L := SmallTwoOrbit3Maniplexes([0], 100);|
  [ TwoOrbit3ManiplexClass2_0([ 10, 4 ], "  r0*a21*a101*a21^-1, r0*a21^-1*a101*r0*a101*a21 "),
   TwoOrbit3ManiplexClass2_0([ 14, 3 ], "  r0*a21*a101*a21^-1, r0*a101*a21*(a101*r0)^2*a21^-1 ") ]
\end{Verbatim}
 }

 
\section{\textcolor{Chapter }{System internal representations}}\label{Chapter_Databases_Section_System_internal_representations}
\logpage{[ 12, 2, 0 ]}
\hyperdef{L}{X84020A6879FBFF00}{}
{
  

\subsection{\textcolor{Chapter }{DatabaseString (for IsManiplex)}}
\logpage{[ 12, 2, 1 ]}\nobreak
\hyperdef{L}{X7B89A32E7D94769F}{}
{\noindent\textcolor{FuncColor}{$\triangleright$\enspace\texttt{DatabaseString({\mdseries\slshape M})\index{DatabaseString@\texttt{DatabaseString}!for IsManiplex}
\label{DatabaseString:for IsManiplex}
}\hfill{\scriptsize (operation)}}\\
\textbf{\indent Returns:\ }
String 



 Given a maniplex \mbox{\texttt{\mdseries\slshape M}}, returns a string representation of \mbox{\texttt{\mdseries\slshape M}} suitable for saving in a database for later retrieval. This works for any
maniplex such that String(\mbox{\texttt{\mdseries\slshape M}}) contains defining information for \mbox{\texttt{\mdseries\slshape M}} - otherwise the output may not be so useful. }

 
\begin{Verbatim}[commandchars=!@|,fontsize=\small,frame=single,label=Example]
  !gapprompt@gap>| !gapinput@DatabaseString(Cube(3));|
  "Cube(3)#6#48"
  !gapprompt@gap>| !gapinput@M := ReflexibleManiplex(Group((1,2),(2,3),(3,4)));;|
  !gapprompt@gap>| !gapinput@DatabaseString(M);|
  "<object>#4#24"
\end{Verbatim}
 

\subsection{\textcolor{Chapter }{ManiplexFromDatabaseString (for IsString)}}
\logpage{[ 12, 2, 2 ]}\nobreak
\hyperdef{L}{X84FC221B80847F84}{}
{\noindent\textcolor{FuncColor}{$\triangleright$\enspace\texttt{ManiplexFromDatabaseString({\mdseries\slshape maniplexString})\index{ManiplexFromDatabaseString@\texttt{ManiplexFromDatabaseString}!for IsString}
\label{ManiplexFromDatabaseString:for IsString}
}\hfill{\scriptsize (operation)}}\\
\textbf{\indent Returns:\ }
IsManiplex 



 Given a string \mbox{\texttt{\mdseries\slshape maniplexString}}, representing a maniplex stored in a database, returns the maniplex that is
represented. In particular, ManiplexFromDatabaseString(DatabaseString(M)) is
isomorphic to M if DatabaseString(M) contains defining information for M. }

 
\begin{Verbatim}[commandchars=!@|,fontsize=\small,frame=single,label=Example]
  !gapprompt@gap>| !gapinput@ManiplexFromDatabaseString("Cube(3)#6#48") = Cube(3);|
  true
  !gapprompt@gap>| !gapinput@M := ReflexibleManiplex(Group((1,2),(2,3),(3,4)));;|
  !gapprompt@gap>| !gapinput@ManiplexFromDatabaseString(DatabaseString(M));|
  Syntax error: expression expected in stream:1
  _EVALSTRINGTMP:=<object>;
\end{Verbatim}
 

\subsection{\textcolor{Chapter }{InterpolatedString (for IsString)}}
\logpage{[ 12, 2, 3 ]}\nobreak
\hyperdef{L}{X7D879C01845D4BE9}{}
{\noindent\textcolor{FuncColor}{$\triangleright$\enspace\texttt{InterpolatedString({\mdseries\slshape str})\index{InterpolatedString@\texttt{InterpolatedString}!for IsString}
\label{InterpolatedString:for IsString}
}\hfill{\scriptsize (operation)}}\\
\textbf{\indent Returns:\ }
IsString 



 Given a string, replaces each instance of "\$variable" with
String(EvalString(variable)). Any character which cannot be used in a variable
name (such as spaces, commas, etc.) marks the end of the variable name. 

 Note that, due to limitations with EvalString, only global variables can be
interpolated this way. }

 
\begin{Verbatim}[commandchars=!@|,fontsize=\small,frame=single,label=Example]
  !gapprompt@gap>| !gapinput@n := 5;;|
  !gapprompt@gap>| !gapinput@InterpolatedString("2 + 3 = $n");|
  "2 + 3 = 5"
  !gapprompt@gap>| !gapinput@InterpolatedString("2 + 3 = $n, right?");|
  "2 + 3 = 5, right?"
  !gapprompt@gap>| !gapinput@nn := 17;;|
  !gapprompt@gap>| !gapinput@InterpolatedString("$n and $nn are different");|
  "5 and 17 are different"
\end{Verbatim}
 }

 }

   
\chapter{\textcolor{Chapter }{Stratified Operations}}\label{Chapter_Stratified_Operations}
\logpage{[ 13, 0, 0 ]}
\hyperdef{L}{X7B9A7024825B378C}{}
{
  
\section{\textcolor{Chapter }{Computational tools}}\label{Chapter_Stratified_Operations_Section_Computational_tools}
\logpage{[ 13, 1, 0 ]}
\hyperdef{L}{X871D56AE7EFE6B58}{}
{
  I should say something more here. 

\subsection{\textcolor{Chapter }{ChunkMultiply (for IsList,IsList)}}
\logpage{[ 13, 1, 1 ]}\nobreak
\hyperdef{L}{X794A197182124FB1}{}
{\noindent\textcolor{FuncColor}{$\triangleright$\enspace\texttt{ChunkMultiply({\mdseries\slshape element1, element2})\index{ChunkMultiply@\texttt{ChunkMultiply}!for IsList,IsList}
\label{ChunkMultiply:for IsList,IsList}
}\hfill{\scriptsize (operation)}}\\
\textbf{\indent Returns:\ }
element 



 Elements are ordered pairs of the form [perm, list], where the elements of
list are members of a group. Operation performed is consistent with that in
defined in \cite{PelWil18}. 

 }

 

\subsection{\textcolor{Chapter }{ChunkPower (for IsList,IsInt)}}
\logpage{[ 13, 1, 2 ]}\nobreak
\hyperdef{L}{X858FD5C2809828C3}{}
{\noindent\textcolor{FuncColor}{$\triangleright$\enspace\texttt{ChunkPower({\mdseries\slshape element, integer})\index{ChunkPower@\texttt{ChunkPower}!for IsList,IsInt}
\label{ChunkPower:for IsList,IsInt}
}\hfill{\scriptsize (operation)}}\\
\textbf{\indent Returns:\ }
element 



 Given an element compatible with the ChunkMultiply operation, this function
will compute the product of element with itself integer times. }

 

\subsection{\textcolor{Chapter }{ChunkGeneratedGroupElements (for IsList, IsGroup)}}
\logpage{[ 13, 1, 3 ]}\nobreak
\hyperdef{L}{X81AD40D884608C3B}{}
{\noindent\textcolor{FuncColor}{$\triangleright$\enspace\texttt{ChunkGeneratedGroupElements({\mdseries\slshape list, group})\index{ChunkGeneratedGroupElements@\texttt{ChunkGeneratedGroupElements}!for IsList, IsGroup}
\label{ChunkGeneratedGroupElements:for IsList, IsGroup}
}\hfill{\scriptsize (operation)}}\\
\textbf{\indent Returns:\ }
newList 



 Given a list of generators compatible with the ChunkMultiply operation, this
function will construct the associated list of group elements in a form
suitable for taking ChunkMultiply and ChunkPower. }

 
\begin{Verbatim}[commandchars=!@|,fontsize=\small,frame=single,label=Example]
  !gapprompt@gap>| !gapinput@g:=AutomorphismGroup(Simplex(2));|
  <fp group of size 6 on the generators [ r0, r1 ]>
  !gapprompt@gap>| !gapinput@AssignGeneratorVariables(g);|
  #I  Assigned the global variables [ r0, r1 ]
  !gapprompt@gap>| !gapinput@SetReducedMultiplication(r0);|
  !gapprompt@gap>| !gapinput@s0:=[(1,2),[r0,r0,r1]];;s1:=[(2,3),[r0,r1,r1]];;|
  !gapprompt@gap>| !gapinput@ChunkGeneratedGroupElements([s0,s1],g);|
  [ [ (1,2), [ r0, r0, r1 ] ], [ (2,3), [ r0, r1, r1 ] ], 
    [ (), [ <identity ...>, <identity ...>, <identity ...> ] ], 
    [ (1,3,2), [ <identity ...>, <identity ...>, r0*r1 ] ], 
    [ (1,2,3), [ r1*r0, <identity ...>, <identity ...> ] ], [ (1,3), [ r0, r0, r0 ] ], 
    [ (1,3), [ r1, r1, r1 ] ], [ (1,2,3), [ <identity ...>, r0*r1, r0*r1 ] ], 
    [ (1,3,2), [ r1*r0, r1*r0, <identity ...> ] ], [ (2,3), [ r0*r1*r0, r0, r0 ] ], 
    [ (1,2), [ r1, r1, r0*r1*r0 ] ], [ (), [ r0*r1, r0*r1, r0*r1 ] ], 
    [ (), [ r1*r0, r1*r0, r1*r0 ] ], [ (1,2), [ r0*r1*r0, r0*r1*r0, r0 ] ], 
    [ (2,3), [ r1, r0*r1*r0, r0*r1*r0 ] ], [ (1,3,2), [ r0*r1, r0*r1, r1*r0 ] ], 
    [ (1,2,3), [ r0*r1, r1*r0, r1*r0 ] ], [ (1,3), [ r0*r1*r0, r0*r1*r0, r0*r1*r0 ] ] ]
\end{Verbatim}
 

\subsection{\textcolor{Chapter }{ChunkGeneratedGroup (for IsList, IsPermGroup)}}
\logpage{[ 13, 1, 4 ]}\nobreak
\hyperdef{L}{X7E759EE8836E55E0}{}
{\noindent\textcolor{FuncColor}{$\triangleright$\enspace\texttt{ChunkGeneratedGroup({\mdseries\slshape list, group})\index{ChunkGeneratedGroup@\texttt{ChunkGeneratedGroup}!for IsList, IsPermGroup}
\label{ChunkGeneratedGroup:for IsList, IsPermGroup}
}\hfill{\scriptsize (operation)}}\\
\textbf{\indent Returns:\ }
permGroup 



 Given a list of generators compatible with the ChunkMultiply operation, this
function will construct a representation of the group as a permutation group.
Note that generators are of the form [perm, list], and each list is a list of
elements from group. }

 
\begin{Verbatim}[commandchars=!@|,fontsize=\small,frame=single,label=Example]
  !gapprompt@gap>| !gapinput@p:=Simplex(2); a:=AutomorphismGroup(p);|
  Pgon(3)
  <fp group of size 6 on the generators [ r0, r1 ]>
  !gapprompt@gap>| !gapinput@e:=One(a);; AssignGeneratorVariables(a);|
  !gapprompt@gap>| !gapinput@s0:=[(3,4),[r0,r0,e,e,r0,r0]];|
  [ (3,4), [ r0, r0, <identity ...>, <identity ...>, r0, r0 ] ]
  !gapprompt@gap>| !gapinput@s1:=[(2,3)(4,5),[r1,e,e,e,e,r1]];|
  [ (2,3)(4,5), [ r1, <identity ...>, <identity ...>, <identity ...>, <identity ...>, r1 ] ]
  !gapprompt@gap>| !gapinput@s2:=[(1,2)(5,6),[e,e,r1,r1,e,e]];|
  [ (1,2)(5,6), [ <identity ...>, <identity ...>, r1, r1, <identity ...>, <identity ...> ] ]
  !gapprompt@gap>| !gapinput@ gens:=[s0,s1,s2];;|
  !gapprompt@gap>| !gapinput@ChunkMultiply(s0,s1);|
  [ (2,3,5,4), [ r0*r1, <identity ...>, r0, r0, <identity ...>, r0*r1 ] ]
  !gapprompt@gap>| !gapinput@ChunkMultiply(s0,s0);|
  [ (), [ r0^2, r0^2, <identity ...>, <identity ...>, r0^2, r0^2 ] ]
  !gapprompt@gap>| !gapinput@SetReducedMultiplication(r1);|
  !gapprompt@gap>| !gapinput@ChunkMultiply(s0,s0);|
  [ (), [ <identity ...>, <identity ...>, <identity ...>, <identity ...>, <identity ...>,  <identity ...> ] ]
  !gapprompt@gap>| !gapinput@ChunkGeneratedGroup(gens,a);|
  <permutation group with 3 generators>
  !gapprompt@gap>| !gapinput@Size(last);|
  1296
\end{Verbatim}
 

\subsection{\textcolor{Chapter }{FullyStratifiedGroup (for IsList, IsGroup)}}
\logpage{[ 13, 1, 5 ]}\nobreak
\hyperdef{L}{X7C67C6B77C6C2114}{}
{\noindent\textcolor{FuncColor}{$\triangleright$\enspace\texttt{FullyStratifiedGroup({\mdseries\slshape list, group})\index{FullyStratifiedGroup@\texttt{FullyStratifiedGroup}!for IsList, IsGroup}
\label{FullyStratifiedGroup:for IsList, IsGroup}
}\hfill{\scriptsize (operation)}}\\
\textbf{\indent Returns:\ }
IsPermGroup 



 Implements fully stratified operations on maniplexes from \cite{CunPelWil22}. Given \mbox{\texttt{\mdseries\slshape list}} of generators compatible with the \texttt{ChunkMultiply} operation, \mbox{\texttt{\mdseries\slshape group}} is the underlying group in the representation (usually the connection group of
the base), this will calculate the connection group of the resulting maniplex
acting on the implicit flags of the construction. Function assumes that \mbox{\texttt{\mdseries\slshape list}} are the generators of the connection group of the resulting maniplex in the
order $\langle r_0, r_1, \ldots, r_{d-1}\rangle$. It is possible that for some groups this function will behave poorly because
GAP won't recognize equivalent representations of a group element. If so, try
again with a permutation representation and let us know so we can modify the
code to handle this problem better (didn't show up in our testing, but is a
theoretical possibility). }

 
\begin{Verbatim}[commandchars=!@|,fontsize=\small,frame=single,label=Example]
  !gapprompt@gap>| !gapinput@p:=Simplex(2);; a:=AutomorphismGroup(p);|
  <fp group of size 6 on the generators [ r0, r1 ]>
  !gapprompt@gap>| !gapinput@e:=One(a);|
  <identity ...>
  !gapprompt@gap>| !gapinput@AssignGeneratorVariables(a);|
  #I  Assigned the global variables [ r0, r1 ]
  !gapprompt@gap>| !gapinput@s0:=[(3,4),[r0,r0,e,e,r0,r0]];|
  [ (3,4), [ r0, r0, <identity ...>, <identity ...>, r0, r0 ] ]
  !gapprompt@gap>| !gapinput@s1:=[(2,3)(4,5),[r1,e,e,e,e,r1]];|
  [ (2,3)(4,5), [ r1, <identity ...>, <identity ...>, <identity ...>, <identity ...>, r1 ] ]
  !gapprompt@gap>| !gapinput@s2:=[(1,2)(5,6),[e,e,r1,r1,e,e]];|
  [ (1,2)(5,6), [ <identity ...>, <identity ...>, r1, r1, <identity ...>, <identity ...> ] ]
  !gapprompt@gap>| !gapinput@gens:=[s0,s1,s2];|
  [ [ (3,4), [ r0, r0, <identity ...>, <identity ...>, r0, r0 ] ], 
    [ (2,3)(4,5), [ r1, <identity ...>, <identity ...>, <identity ...>, <identity ...>, r1 ] ], 
    [ (1,2)(5,6), [ <identity ...>, <identity ...>, r1, r1, <identity ...>, <identity ...> ] ] ]
  !gapprompt@gap>| !gapinput@Maniplex(FullyStratifiedGroup(gens,a))=Prism(Simplex(2));|
  true
\end{Verbatim}
 }

 }

   
\chapter{\textcolor{Chapter }{Maps On Surfaces}}\label{Chapter_Maps_On_Surfaces}
\logpage{[ 14, 0, 0 ]}
\hyperdef{L}{X802F024B80294E55}{}
{
  
\section{\textcolor{Chapter }{Bicontactual regular maps}}\label{Chapter_Maps_On_Surfaces_Section_Bicontactual_regular_maps}
\logpage{[ 14, 1, 0 ]}
\hyperdef{L}{X81AFCB8E84DB41DE}{}
{
  The names for the maps in this section are from S.E. Wilson's \cite{Wil85}. 

\subsection{\textcolor{Chapter }{Epsilonk (for IsInt)}}
\logpage{[ 14, 1, 1 ]}\nobreak
\hyperdef{L}{X871814FE7CC7B178}{}
{\noindent\textcolor{FuncColor}{$\triangleright$\enspace\texttt{Epsilonk({\mdseries\slshape k})\index{Epsilonk@\texttt{Epsilonk}!for IsInt}
\label{Epsilonk:for IsInt}
}\hfill{\scriptsize (operation)}}\\
\textbf{\indent Returns:\ }
IsManiplex 



 Given an integer \mbox{\texttt{\mdseries\slshape k}}, gives the map $\epsilon_k$, which is $\{k,2\}_k$ when \mbox{\texttt{\mdseries\slshape k}} is even, and $\{k,2\}_{2k}$ when \mbox{\texttt{\mdseries\slshape k}} is odd. }

 
\begin{Verbatim}[commandchars=!@|,fontsize=\small,frame=single,label=Example]
  !gapprompt@gap>| !gapinput@Epsilonk(5);|
  AbstractRegularPolytope([ 5, 2 ])
  !gapprompt@gap>| !gapinput@Epsilonk(6);|
  AbstractRegularPolytope([ 6, 2 ])
\end{Verbatim}
 

\subsection{\textcolor{Chapter }{Deltak (for IsInt)}}
\logpage{[ 14, 1, 2 ]}\nobreak
\hyperdef{L}{X7F59553F84D04CAF}{}
{\noindent\textcolor{FuncColor}{$\triangleright$\enspace\texttt{Deltak({\mdseries\slshape k})\index{Deltak@\texttt{Deltak}!for IsInt}
\label{Deltak:for IsInt}
}\hfill{\scriptsize (operation)}}\\
\textbf{\indent Returns:\ }
IsManiplex 



 Given an integer \mbox{\texttt{\mdseries\slshape k}}, gives the map $\delta_k$, which is $\{2k,2\}/2$ when \mbox{\texttt{\mdseries\slshape k}} is even, and $\{2k,2\}_{k}$ when \mbox{\texttt{\mdseries\slshape k}} is odd. }

 
\begin{Verbatim}[commandchars=!@|,fontsize=\small,frame=single,label=Example]
  !gapprompt@gap>| !gapinput@Deltak(5);|
  ReflexibleManiplex([ 10, 2 ], "(r0 r1)^5 r2")
  !gapprompt@gap>| !gapinput@Deltak(6);|
  ReflexibleManiplex([ 12, 2 ], "(r0 r1)^6 r2")
\end{Verbatim}
 

\subsection{\textcolor{Chapter }{Mk (for IsInt)}}
\logpage{[ 14, 1, 3 ]}\nobreak
\hyperdef{L}{X7876740785ECAF02}{}
{\noindent\textcolor{FuncColor}{$\triangleright$\enspace\texttt{Mk({\mdseries\slshape k})\index{Mk@\texttt{Mk}!for IsInt}
\label{Mk:for IsInt}
}\hfill{\scriptsize (operation)}}\\
\textbf{\indent Returns:\ }
IsManiplex 



 Given an integer \mbox{\texttt{\mdseries\slshape k}}, gives the map $M_k$, which is $\{2k,2k\}_{1,0}$ when \mbox{\texttt{\mdseries\slshape k}} is even, and $\{2k,k\}_{2}$ when \mbox{\texttt{\mdseries\slshape k}} is odd. }

 
\begin{Verbatim}[commandchars=!@|,fontsize=\small,frame=single,label=Example]
  !gapprompt@gap>| !gapinput@Mk(5);Mk(6);|
  ReflexibleManiplex([ 10, 5 ], "(r0 r1)^5 r0 = r2")
  ReflexibleManiplex([ 12, 12 ], "(r0 r1)^6 r0 = r2")
\end{Verbatim}
 

\subsection{\textcolor{Chapter }{MkPrime (for IsInt)}}
\logpage{[ 14, 1, 4 ]}\nobreak
\hyperdef{L}{X7DAE3F0686CA7AA5}{}
{\noindent\textcolor{FuncColor}{$\triangleright$\enspace\texttt{MkPrime({\mdseries\slshape k})\index{MkPrime@\texttt{MkPrime}!for IsInt}
\label{MkPrime:for IsInt}
}\hfill{\scriptsize (operation)}}\\
\textbf{\indent Returns:\ }
IsManiplex 



 Given an integer \mbox{\texttt{\mdseries\slshape k}}, gives the map $M'_k$, which is $\{k,k\}_2$ when \mbox{\texttt{\mdseries\slshape k}} is even, and $\{k,2k\}_{2}$ when \mbox{\texttt{\mdseries\slshape k}} is odd. \texttt{MkPrime(k,i)} gives the map $M'_{k,i}$. }

 
\begin{Verbatim}[commandchars=!@|,fontsize=\small,frame=single,label=Example]
  !gapprompt@gap>| !gapinput@MkPrime(5);MkPrime(6);|
  ReflexibleManiplex([ 5, 10 ], "(r2*r1*(r0 r2))^5,z1^2")
  ReflexibleManiplex([ 6, 6 ], "(r2*r1*(r0 r2))^6,z1^2")
\end{Verbatim}
 

\subsection{\textcolor{Chapter }{Bk2l (for IsInt,IsInt)}}
\logpage{[ 14, 1, 5 ]}\nobreak
\hyperdef{L}{X7EB8C646795CB285}{}
{\noindent\textcolor{FuncColor}{$\triangleright$\enspace\texttt{Bk2l({\mdseries\slshape k, l})\index{Bk2l@\texttt{Bk2l}!for IsInt,IsInt}
\label{Bk2l:for IsInt,IsInt}
}\hfill{\scriptsize (operation)}}\\
\textbf{\indent Returns:\ }
IsManiplex 



 Given integers \mbox{\texttt{\mdseries\slshape k,l}}, gives the map $B(k,2l)$. }

 
\begin{Verbatim}[commandchars=!@|,fontsize=\small,frame=single,label=Example]
  !gapprompt@gap>| !gapinput@Bk2l(4,5);|
  3-maniplex with 80 flags
\end{Verbatim}
 

\subsection{\textcolor{Chapter }{Bk2lStar (for IsInt,IsInt)}}
\logpage{[ 14, 1, 6 ]}\nobreak
\hyperdef{L}{X8592D83D87DFCF3F}{}
{\noindent\textcolor{FuncColor}{$\triangleright$\enspace\texttt{Bk2lStar({\mdseries\slshape k, l})\index{Bk2lStar@\texttt{Bk2lStar}!for IsInt,IsInt}
\label{Bk2lStar:for IsInt,IsInt}
}\hfill{\scriptsize (operation)}}\\
\textbf{\indent Returns:\ }
IsManiplex 



 Given integers \mbox{\texttt{\mdseries\slshape k,l}}, gives the map $B^*(k,2l)$. }

 
\begin{Verbatim}[commandchars=!@|,fontsize=\small,frame=single,label=Example]
  !gapprompt@gap>| !gapinput@Bk2lStar(5,7);|
  3-maniplex with 140 flags
\end{Verbatim}
 }

 
\section{\textcolor{Chapter }{Operations on reflexible maps}}\label{Chapter_Maps_On_Surfaces_Section_Operations_on_reflexible_maps}
\logpage{[ 14, 2, 0 ]}
\hyperdef{L}{X80E866D6862E1FD0}{}
{
  

\subsection{\textcolor{Chapter }{Opp (for IsMapOnSurface)}}
\logpage{[ 14, 2, 1 ]}\nobreak
\hyperdef{L}{X788A29017AE71302}{}
{\noindent\textcolor{FuncColor}{$\triangleright$\enspace\texttt{Opp({\mdseries\slshape map})\index{Opp@\texttt{Opp}!for IsMapOnSurface}
\label{Opp:for IsMapOnSurface}
}\hfill{\scriptsize (operation)}}\\
\textbf{\indent Returns:\ }
IsManiplex 



 Forms the opposite map of the maniplex \mbox{\texttt{\mdseries\slshape map}}. }

 
\begin{Verbatim}[commandchars=!@|,fontsize=\small,frame=single,label=Example]
  !gapprompt@gap>| !gapinput@Opp(Bk2lStar(5,7));|
  Petrial(Dual(Petrial(3-maniplex with 140 flags)))
\end{Verbatim}
 

\subsection{\textcolor{Chapter }{Hole (for IsMapOnSurface,IsInt)}}
\logpage{[ 14, 2, 2 ]}\nobreak
\hyperdef{L}{X835627037B606BA9}{}
{\noindent\textcolor{FuncColor}{$\triangleright$\enspace\texttt{Hole({\mdseries\slshape map, j})\index{Hole@\texttt{Hole}!for IsMapOnSurface,IsInt}
\label{Hole:for IsMapOnSurface,IsInt}
}\hfill{\scriptsize (operation)}}\\
\textbf{\indent Returns:\ }
IsManiplex 



 Given \mbox{\texttt{\mdseries\slshape map}} and integer $j$, will form the map $H_j(map)$. Note that if the action of $[r_0,(r_1 r_2)^{j-1} r_1, r_2]$ on the flags forms multiple orbits, then the resulting map will be on just one
of those orbits. }

 
\begin{Verbatim}[commandchars=!@|,fontsize=\small,frame=single,label=Example]
  !gapprompt@gap>| !gapinput@Hole(Bk2lStar(5,7),2);|
  3-maniplex with 140 flags
\end{Verbatim}
 }

 
\section{\textcolor{Chapter }{Map properties}}\label{Chapter_Maps_On_Surfaces_Section_Map_properties}
\logpage{[ 14, 3, 0 ]}
\hyperdef{L}{X7BF319E97DEC64D2}{}
{
  \texttt{IsMapOnSurface} will test to see if you have rank 3 maniplex. 
\begin{Verbatim}[commandchars=!@|,fontsize=\small,frame=single,label=Example]
  !gapprompt@gap>| !gapinput@Filtered([HemiCube(3),Cube(4),Icosahedron()],IsMapOnSurface);|
  [ HemiCube(3), Icosahedron() ]
\end{Verbatim}
 }

 
\section{\textcolor{Chapter }{Operations on maps}}\label{Chapter_Maps_On_Surfaces_Section_Operations_on_maps}
\logpage{[ 14, 4, 0 ]}
\hyperdef{L}{X8688080A7A67C820}{}
{
  

\subsection{\textcolor{Chapter }{Truncation (for IsMapOnSurface)}}
\logpage{[ 14, 4, 1 ]}\nobreak
\hyperdef{L}{X836355A984E78063}{}
{\noindent\textcolor{FuncColor}{$\triangleright$\enspace\texttt{Truncation({\mdseries\slshape map})\index{Truncation@\texttt{Truncation}!for IsMapOnSurface}
\label{Truncation:for IsMapOnSurface}
}\hfill{\scriptsize (operation)}}\\
\textbf{\indent Returns:\ }
trunc{\textunderscore}map 



 Given a \mbox{\texttt{\mdseries\slshape map}} on a surface, this function will return the truncation of \mbox{\texttt{\mdseries\slshape map}}. }

 
\begin{Verbatim}[commandchars=!@|,fontsize=\small,frame=single,label=Example]
  !gapprompt@gap>| !gapinput@SchlafliSymbol(Truncation(Simplex(3)));|
  [ [ 3, 6 ], 3 ]
  !gapprompt@gap>| !gapinput@TruncatedTetrahedron()=Truncation(Simplex(3));|
  true
  !gapprompt@gap>| !gapinput@Truncation(CrossPolytope(3))=TruncatedOctahedron();|
  true
  !gapprompt@gap>| !gapinput@Truncation(Cube(3))=TruncatedCube();|
  true
\end{Verbatim}
 

\subsection{\textcolor{Chapter }{Snub (for IsMapOnSurface)}}
\logpage{[ 14, 4, 2 ]}\nobreak
\hyperdef{L}{X87314A378600B2D5}{}
{\noindent\textcolor{FuncColor}{$\triangleright$\enspace\texttt{Snub({\mdseries\slshape M})\index{Snub@\texttt{Snub}!for IsMapOnSurface}
\label{Snub:for IsMapOnSurface}
}\hfill{\scriptsize (operation)}}\\
\textbf{\indent Returns:\ }
snub{\textunderscore}map 



 Returns the snub of a given map; we require that the map have triangles as
vertex figures. }

 
\begin{Verbatim}[commandchars=!@|,fontsize=\small,frame=single,label=Example]
  !gapprompt@gap>| !gapinput@Snub(Dodecahedron())=SnubDodecahedron();|
  true
  !gapprompt@gap>| !gapinput@Snub(Cube(3))=SnubCube();|
  true
  !gapprompt@gap>| !gapinput@Snub(Simplex(3))=Icosahedron();|
  true
  !gapprompt@gap>| !gapinput@Snub(CrossPolytope(3))=SnubCube();|
  true
  !gapprompt@gap>| !gapinput@Snub(Dual(Cube(3)))=Reflection(Snub(Reflection(Cube(3))));|
  true
\end{Verbatim}
 

\subsection{\textcolor{Chapter }{Chamfer (for IsMapOnSurface)}}
\logpage{[ 14, 4, 3 ]}\nobreak
\hyperdef{L}{X781B0A0880BFD546}{}
{\noindent\textcolor{FuncColor}{$\triangleright$\enspace\texttt{Chamfer({\mdseries\slshape M})\index{Chamfer@\texttt{Chamfer}!for IsMapOnSurface}
\label{Chamfer:for IsMapOnSurface}
}\hfill{\scriptsize (operation)}}\\
\textbf{\indent Returns:\ }
chamfered{\textunderscore}map 



 Returns the map obtained from the chamfering operation described in \cite{Rio14} }

 
\begin{Verbatim}[commandchars=!@|,fontsize=\small,frame=single,label=Example]
  !gapprompt@gap>| !gapinput@s0 := (4,5)(6,7)(8,9);;|
  !gapprompt@gap>| !gapinput@s1 := (2,6)(3,4)(5,7);;|
  !gapprompt@gap>| !gapinput@s2 := (1,2)(4,8)(5,9);;|
  !gapprompt@gap>| !gapinput@poly := Group([s0,s1,s2]);;|
  !gapprompt@gap>| !gapinput@p:=ARP(poly);;|
  !gapprompt@gap>| !gapinput@SchlafliSymbol(p);|
  [ 6, 3 ]
  !gapprompt@gap>| !gapinput@ch:=Chamfer(p);|
  3-maniplex with 432 flags
  !gapprompt@gap>| !gapinput@SchlafliSymbol(ch);|
  [ 6, 3 ]
\end{Verbatim}
 

\subsection{\textcolor{Chapter }{Subdivision1 (for IsMapOnSurface)}}
\logpage{[ 14, 4, 4 ]}\nobreak
\hyperdef{L}{X7E0E11CE7CF33FBB}{}
{\noindent\textcolor{FuncColor}{$\triangleright$\enspace\texttt{Subdivision1({\mdseries\slshape M})\index{Subdivision1@\texttt{Subdivision1}!for IsMapOnSurface}
\label{Subdivision1:for IsMapOnSurface}
}\hfill{\scriptsize (operation)}}\\
\textbf{\indent Returns:\ }
Su1 



 Returns the One-dimensional subdivision of a map, which replaces each edge
with two edges. For more information on the oriented version of this, see \cite{BerPisWil17}. }

 
\begin{Verbatim}[commandchars=!@|,fontsize=\small,frame=single,label=Example]
  !gapprompt@gap>| !gapinput@m:=Subdivision1(Simplex(3));|
  3-maniplex with 48 flags
  !gapprompt@gap>| !gapinput@SchlafliSymbol(m);|
  [ 6, [ 2, 3 ] ]
\end{Verbatim}
 

\subsection{\textcolor{Chapter }{Subdivision2 (for IsMapOnSurface)}}
\logpage{[ 14, 4, 5 ]}\nobreak
\hyperdef{L}{X84F485127B8AD294}{}
{\noindent\textcolor{FuncColor}{$\triangleright$\enspace\texttt{Subdivision2({\mdseries\slshape M})\index{Subdivision2@\texttt{Subdivision2}!for IsMapOnSurface}
\label{Subdivision2:for IsMapOnSurface}
}\hfill{\scriptsize (operation)}}\\
\textbf{\indent Returns:\ }
Su2 



 Returns the two-dimensional subdivision of \mbox{\texttt{\mdseries\slshape M}}. }

 
\begin{Verbatim}[commandchars=!@|,fontsize=\small,frame=single,label=Example]
  !gapprompt@gap>| !gapinput@SchlafliSymbol(Subdivision2(Cube(3)));|
  [ 3, [ 4, 6 ] ]
\end{Verbatim}
 

\subsection{\textcolor{Chapter }{BarycentricSubdivision (for IsMapOnSurface)}}
\logpage{[ 14, 4, 6 ]}\nobreak
\hyperdef{L}{X81B9A11C788E3BDF}{}
{\noindent\textcolor{FuncColor}{$\triangleright$\enspace\texttt{BarycentricSubdivision({\mdseries\slshape M})\index{BarycentricSubdivision@\texttt{BarycentricSubdivision}!for IsMapOnSurface}
\label{BarycentricSubdivision:for IsMapOnSurface}
}\hfill{\scriptsize (operation)}}\\
\textbf{\indent Returns:\ }
barycentric{\textunderscore}subdivision 



 Gives the barycentric subdivision of \mbox{\texttt{\mdseries\slshape M}}. }

 
\begin{Verbatim}[commandchars=!@|,fontsize=\small,frame=single,label=Example]
  !gapprompt@gap>| !gapinput@m:=BarycentricSubdivision(Cube(3));;|
  !gapprompt@gap>| !gapinput@SchlafliSymbol(m);NumberOfFacets(m);|
  [ 3, [ 4, 6, 8 ] ]
  48
\end{Verbatim}
 

\subsection{\textcolor{Chapter }{Leapfrog (for IsMapOnSurface)}}
\logpage{[ 14, 4, 7 ]}\nobreak
\hyperdef{L}{X7A3AD3EA8594B99B}{}
{\noindent\textcolor{FuncColor}{$\triangleright$\enspace\texttt{Leapfrog({\mdseries\slshape M})\index{Leapfrog@\texttt{Leapfrog}!for IsMapOnSurface}
\label{Leapfrog:for IsMapOnSurface}
}\hfill{\scriptsize (operation)}}\\
\textbf{\indent Returns:\ }
leapfrog 



 Gives the result of performing the leapfrog operation on a map on a surface }

 
\begin{Verbatim}[commandchars=!@|,fontsize=\small,frame=single,label=Example]
  !gapprompt@gap>| !gapinput@Leapfrog(Dodecahedron());|
  3-maniplex with 360 flags
  !gapprompt@gap>| !gapinput@SchlafliSymbol(last);|
  [ [ 5, 6 ], 3 ]
\end{Verbatim}
 

\subsection{\textcolor{Chapter }{CombinatorialMap (for IsMapOnSurface)}}
\logpage{[ 14, 4, 8 ]}\nobreak
\hyperdef{L}{X79B68D5E7D7B3E0D}{}
{\noindent\textcolor{FuncColor}{$\triangleright$\enspace\texttt{CombinatorialMap({\mdseries\slshape M})\index{CombinatorialMap@\texttt{CombinatorialMap}!for IsMapOnSurface}
\label{CombinatorialMap:for IsMapOnSurface}
}\hfill{\scriptsize (operation)}}\\
\textbf{\indent Returns:\ }
combinatorial{\textunderscore}map 



 Gives the result of combinatorial operation on a map; this is equivalent to
taking the dual of the barycentric subdivision. }

 
\begin{Verbatim}[commandchars=!@|,fontsize=\small,frame=single,label=Example]
  !gapprompt@gap>| !gapinput@NumberOfEdges(Cube(3));|
  12
  !gapprompt@gap>| !gapinput@NumberOfEdges(CombinatorialMap(Cube(3)));|
  72
\end{Verbatim}
 

\subsection{\textcolor{Chapter }{Angle (for IsMapOnSurface)}}
\logpage{[ 14, 4, 9 ]}\nobreak
\hyperdef{L}{X7B3BACF57D7513DB}{}
{\noindent\textcolor{FuncColor}{$\triangleright$\enspace\texttt{Angle({\mdseries\slshape M})\index{Angle@\texttt{Angle}!for IsMapOnSurface}
\label{Angle:for IsMapOnSurface}
}\hfill{\scriptsize (operation)}}\\
\textbf{\indent Returns:\ }
angle{\textunderscore}map 



 Returns the angle map of a map. This is equivalent to taking the dual of the
medial. }

 
\begin{Verbatim}[commandchars=!@|,fontsize=\small,frame=single,label=Example]
  !gapprompt@gap>| !gapinput@NumberOfEdges(ToroidalMap44([3,0]));|
  18
  !gapprompt@gap>| !gapinput@NumberOfEdges(Angle(ToroidalMap44([3,0])));|
  36
\end{Verbatim}
 

\subsection{\textcolor{Chapter }{Gothic (for IsMapOnSurface)}}
\logpage{[ 14, 4, 10 ]}\nobreak
\hyperdef{L}{X8711070F7D6E5A54}{}
{\noindent\textcolor{FuncColor}{$\triangleright$\enspace\texttt{Gothic({\mdseries\slshape M})\index{Gothic@\texttt{Gothic}!for IsMapOnSurface}
\label{Gothic:for IsMapOnSurface}
}\hfill{\scriptsize (operation)}}\\
\textbf{\indent Returns:\ }
gothic 



 Returns the result of performing the gothic operation to a map. This is the
same as taking the dual of the medial of the truncation of the map. }

 
\begin{Verbatim}[commandchars=!@|,fontsize=\small,frame=single,label=Example]
  !gapprompt@gap>| !gapinput@m:=AbstractRegularPolytope([ 3, 6 ], "(r0 r1 r2)^6");;|
  !gapprompt@gap>| !gapinput@NumberOfEdges(m); NumberOfEdges(Gothic(m));|
  27
  162
\end{Verbatim}
 }

 
\section{\textcolor{Chapter }{Conway polyhedron operator notation}}\label{Chapter_Maps_On_Surfaces_Section_Conway_polyhedron_operator_notation}
\logpage{[ 14, 5, 0 ]}
\hyperdef{L}{X805E812282DDDFD7}{}
{
  We include here operators from Wikipedia that are not included above. 
\begin{itemize}
\item  \texttt{MapJoin}: Creates quadrilateral faces by placing a node in each face, and then the set
of edges are formed by the nodes corresponding to incident vertex-face pairs.
This is another name for \texttt{Angle}. 
\item  \texttt{Ambo}: This is another name for \texttt{Medial}. 
\end{itemize}
 Another excellent source for information on these types of operations is \texttt{https://antitile.readthedocs.io/en/latest/conway.html}. Additional functions are described below. 

\subsection{\textcolor{Chapter }{Reflection (for IsManiplex)}}
\logpage{[ 14, 5, 1 ]}\nobreak
\hyperdef{L}{X83324AFF8119E1E5}{}
{\noindent\textcolor{FuncColor}{$\triangleright$\enspace\texttt{Reflection({\mdseries\slshape M})\index{Reflection@\texttt{Reflection}!for IsManiplex}
\label{Reflection:for IsManiplex}
}\hfill{\scriptsize (operation)}}\\
\textbf{\indent Returns:\ }
reflection 



 Reverses the orientation of a maniplex. }

 
\begin{Verbatim}[commandchars=!@|,fontsize=\small,frame=single,label=Example]
  !gapprompt@gap>| !gapinput@Gyro(Dual(m))=Reflection(Gyro(Reflection(m)));|
  true
  !gapprompt@gap>| !gapinput@Reflection(m)=EnantiomorphicForm(m);|
  true
  !gapprompt@gap>| !gapinput@Reflection(Truncation(m))=Truncation(EnantiomorphicForm(m));|
  true
\end{Verbatim}
 

\subsection{\textcolor{Chapter }{Kis (for IsMapOnSurface)}}
\logpage{[ 14, 5, 2 ]}\nobreak
\hyperdef{L}{X78A66D6D8305AE37}{}
{\noindent\textcolor{FuncColor}{$\triangleright$\enspace\texttt{Kis({\mdseries\slshape M})\index{Kis@\texttt{Kis}!for IsMapOnSurface}
\label{Kis:for IsMapOnSurface}
}\hfill{\scriptsize (operation)}}\\
\textbf{\indent Returns:\ }
kis 



 Returns the Kis of the map, which erects a pyramid over each of the faces. }

 
\begin{Verbatim}[commandchars=!@|,fontsize=\small,frame=single,label=Example]
  !gapprompt@gap>| !gapinput@Kis(Cube(3));|
  3-maniplex with 144 flags
  !gapprompt@gap>| !gapinput@SchlafliSymbol(last);|
  [ 3, [ 4, 6 ] ]
\end{Verbatim}
 

\subsection{\textcolor{Chapter }{Needle (for IsMapOnSurface)}}
\logpage{[ 14, 5, 3 ]}\nobreak
\hyperdef{L}{X7854540981C5DEFF}{}
{\noindent\textcolor{FuncColor}{$\triangleright$\enspace\texttt{Needle({\mdseries\slshape M})\index{Needle@\texttt{Needle}!for IsMapOnSurface}
\label{Needle:for IsMapOnSurface}
}\hfill{\scriptsize (operation)}}\\
\textbf{\indent Returns:\ }
needle 



 Performs the needle operation to the map: edges connect adjacent face centers,
and face centers to incident vertices. }

 
\begin{Verbatim}[commandchars=!@|,fontsize=\small,frame=single,label=Example]
  !gapprompt@gap>| !gapinput@SchlafliSymbol(Needle(Cube(3)));|
  [ 3, [ 3, 8 ] ]
\end{Verbatim}
 

\subsection{\textcolor{Chapter }{Zip (for IsMapOnSurface)}}
\logpage{[ 14, 5, 4 ]}\nobreak
\hyperdef{L}{X79B10B59858AC7DD}{}
{\noindent\textcolor{FuncColor}{$\triangleright$\enspace\texttt{Zip({\mdseries\slshape M})\index{Zip@\texttt{Zip}!for IsMapOnSurface}
\label{Zip:for IsMapOnSurface}
}\hfill{\scriptsize (operation)}}\\
\textbf{\indent Returns:\ }
zip 



 Returns the zip of the map. }

 
\begin{Verbatim}[commandchars=!@|,fontsize=\small,frame=single,label=Example]
  !gapprompt@gap>| !gapinput@Zip(Cube(3))=TruncatedOctahedron();|
  true
\end{Verbatim}
 

\subsection{\textcolor{Chapter }{Ortho (for IsMapOnSurface)}}
\logpage{[ 14, 5, 5 ]}\nobreak
\hyperdef{L}{X8057E86C80D2CDF3}{}
{\noindent\textcolor{FuncColor}{$\triangleright$\enspace\texttt{Ortho({\mdseries\slshape M})\index{Ortho@\texttt{Ortho}!for IsMapOnSurface}
\label{Ortho:for IsMapOnSurface}
}\hfill{\scriptsize (operation)}}\\
\textbf{\indent Returns:\ }
ortho 



 Returns the ortho of the map (this is the same as applying the join twice.). }

 
\begin{Verbatim}[commandchars=!@|,fontsize=\small,frame=single,label=Example]
  !gapprompt@gap>| !gapinput@SchlafliSymbol(Ortho(Cube(3)));|
  [ 4, [ 3, 4 ] ]
\end{Verbatim}
 

\subsection{\textcolor{Chapter }{Expand (for IsMapOnSurface)}}
\logpage{[ 14, 5, 6 ]}\nobreak
\hyperdef{L}{X78991DD77E84C4B0}{}
{\noindent\textcolor{FuncColor}{$\triangleright$\enspace\texttt{Expand({\mdseries\slshape M})\index{Expand@\texttt{Expand}!for IsMapOnSurface}
\label{Expand:for IsMapOnSurface}
}\hfill{\scriptsize (operation)}}\\
\textbf{\indent Returns:\ }
expand 



 Returns the expand of the map (this is the same as applying the ambo operation
twice.). }

 
\begin{Verbatim}[commandchars=!@|,fontsize=\small,frame=single,label=Example]
  !gapprompt@gap>| !gapinput@Expand(Cube(3))=SmallRhombicuboctahedron();|
  true
\end{Verbatim}
 

\subsection{\textcolor{Chapter }{Gyro (for IsMapOnSurface)}}
\logpage{[ 14, 5, 7 ]}\nobreak
\hyperdef{L}{X861A270578877B3E}{}
{\noindent\textcolor{FuncColor}{$\triangleright$\enspace\texttt{Gyro({\mdseries\slshape M})\index{Gyro@\texttt{Gyro}!for IsMapOnSurface}
\label{Gyro:for IsMapOnSurface}
}\hfill{\scriptsize (operation)}}\\
\textbf{\indent Returns:\ }
gyro 



 Returns the gyro of the map. }

 
\begin{Verbatim}[commandchars=!@|,fontsize=\small,frame=single,label=Example]
  !gapprompt@gap>| !gapinput@Gyro(Dual(Cube(3)))=Gyro(Cube(3));|
  true
\end{Verbatim}
 

\subsection{\textcolor{Chapter }{Meta (for IsMapOnSurface)}}
\logpage{[ 14, 5, 8 ]}\nobreak
\hyperdef{L}{X82F5D51786A870C2}{}
{\noindent\textcolor{FuncColor}{$\triangleright$\enspace\texttt{Meta({\mdseries\slshape M})\index{Meta@\texttt{Meta}!for IsMapOnSurface}
\label{Meta:for IsMapOnSurface}
}\hfill{\scriptsize (operation)}}\\
\textbf{\indent Returns:\ }
meta 



 Constructs the meta of the given map. (This is the same as applying first the
join, and then the kis operation to the map). }

 
\begin{Verbatim}[commandchars=!@|,fontsize=\small,frame=single,label=Example]
  !gapprompt@gap>| !gapinput@Size(Cube(3))=NumberOfFacets(Meta(Cube(3)));|
  true
\end{Verbatim}
 

\subsection{\textcolor{Chapter }{Bevel (for IsMapOnSurface)}}
\logpage{[ 14, 5, 9 ]}\nobreak
\hyperdef{L}{X861B355F7C050517}{}
{\noindent\textcolor{FuncColor}{$\triangleright$\enspace\texttt{Bevel({\mdseries\slshape M})\index{Bevel@\texttt{Bevel}!for IsMapOnSurface}
\label{Bevel:for IsMapOnSurface}
}\hfill{\scriptsize (operation)}}\\
\textbf{\indent Returns:\ }
bevel 



 Constructs the bevel of a given map. (This is the same as truncating the ambo
of a map.) }

 
\begin{Verbatim}[commandchars=!@|,fontsize=\small,frame=single,label=Example]
  !gapprompt@gap>| !gapinput@CombinatorialMap(Cube(3))=Bevel(Cube(3));|
  true
\end{Verbatim}
 }

 
\section{\textcolor{Chapter }{Extended operations}}\label{Chapter_Maps_On_Surfaces_Section_Extended_operations}
\logpage{[ 14, 6, 0 ]}
\hyperdef{L}{X8506DF867A281C52}{}
{
  A number of these were introduced by George Hart. 

\subsection{\textcolor{Chapter }{Subdivide (for IsMapOnSurface)}}
\logpage{[ 14, 6, 1 ]}\nobreak
\hyperdef{L}{X7A95E5A883DC8FD9}{}
{\noindent\textcolor{FuncColor}{$\triangleright$\enspace\texttt{Subdivide({\mdseries\slshape M})\index{Subdivide@\texttt{Subdivide}!for IsMapOnSurface}
\label{Subdivide:for IsMapOnSurface}
}\hfill{\scriptsize (operation)}}\\
\textbf{\indent Returns:\ }
u 



 Returns the subdivide (u) of \mbox{\texttt{\mdseries\slshape M}}. }

 
\begin{Verbatim}[commandchars=!@|,fontsize=\small,frame=single,label=Example]
  !gapprompt@gap>| !gapinput@Chamfer(Dual(Cube(3)))=Dual(Subdivide(Cube(3)));|
  true
  !gapprompt@gap>| !gapinput@SchlafliSymbol(Subdivide(Cube(3)));|
  [ [ 3, 4 ], [ 3, 6 ] ]
\end{Verbatim}
 

\subsection{\textcolor{Chapter }{Propeller (for IsMapOnSurface)}}
\logpage{[ 14, 6, 2 ]}\nobreak
\hyperdef{L}{X7C0D8A6C80DADA50}{}
{\noindent\textcolor{FuncColor}{$\triangleright$\enspace\texttt{Propeller({\mdseries\slshape M})\index{Propeller@\texttt{Propeller}!for IsMapOnSurface}
\label{Propeller:for IsMapOnSurface}
}\hfill{\scriptsize (operation)}}\\
\textbf{\indent Returns:\ }
propeller 



 Constructs the propeller of the map. }

 
\begin{Verbatim}[commandchars=!@|,fontsize=\small,frame=single,label=Example]
  !gapprompt@gap>| !gapinput@Dual(Propeller(Cube(3)))=Propeller(Dual(Cube(3)));|
  true
  !gapprompt@gap>| !gapinput@Dual(Propeller(Dual(Cube(3))))=Propeller(Cube(3));|
  true
\end{Verbatim}
 

\subsection{\textcolor{Chapter }{Loft (for IsMapOnSurface)}}
\logpage{[ 14, 6, 3 ]}\nobreak
\hyperdef{L}{X847513C879F0A048}{}
{\noindent\textcolor{FuncColor}{$\triangleright$\enspace\texttt{Loft({\mdseries\slshape M})\index{Loft@\texttt{Loft}!for IsMapOnSurface}
\label{Loft:for IsMapOnSurface}
}\hfill{\scriptsize (operation)}}\\
\textbf{\indent Returns:\ }
loft 



 Constructs the loft of the map. }

 
\begin{Verbatim}[commandchars=!@|,fontsize=\small,frame=single,label=Example]
  !gapprompt@gap>| !gapinput@NumberOfFacets(Loft(Cube(3)));|
  30
  !gapprompt@gap>| !gapinput@SchlafliSymbol(Loft(Cube(3)));|
  [ 4, [ 3, 6 ] ]
\end{Verbatim}
 

\subsection{\textcolor{Chapter }{Quinto (for IsMapOnSurface)}}
\logpage{[ 14, 6, 4 ]}\nobreak
\hyperdef{L}{X7EE97C1484C3B167}{}
{\noindent\textcolor{FuncColor}{$\triangleright$\enspace\texttt{Quinto({\mdseries\slshape M})\index{Quinto@\texttt{Quinto}!for IsMapOnSurface}
\label{Quinto:for IsMapOnSurface}
}\hfill{\scriptsize (operation)}}\\
\textbf{\indent Returns:\ }
quinto 



 Constructs the quinto of the map. }

 
\begin{Verbatim}[commandchars=!@|,fontsize=\small,frame=single,label=Example]
  !gapprompt@gap>| !gapinput@SchlafliSymbol(Quinto(Cube(3)));|
  [ [ 4, 5 ], [ 3, 4 ] ]
\end{Verbatim}
 

\subsection{\textcolor{Chapter }{JoinLace (for IsMapOnSurface)}}
\logpage{[ 14, 6, 5 ]}\nobreak
\hyperdef{L}{X7EEA28B57BE65A08}{}
{\noindent\textcolor{FuncColor}{$\triangleright$\enspace\texttt{JoinLace({\mdseries\slshape M})\index{JoinLace@\texttt{JoinLace}!for IsMapOnSurface}
\label{JoinLace:for IsMapOnSurface}
}\hfill{\scriptsize (operation)}}\\
\textbf{\indent Returns:\ }
join-lace 



 Constructs the join-lace of the map. }

 
\begin{Verbatim}[commandchars=!@|,fontsize=\small,frame=single,label=Example]
  !gapprompt@gap>| !gapinput@SchlafliSymbol(JoinLace(Cube(3)));|
  [ [ 3, 4 ], [ 4, 6 ] ]
\end{Verbatim}
 

\subsection{\textcolor{Chapter }{Lace (for IsMapOnSurface)}}
\logpage{[ 14, 6, 6 ]}\nobreak
\hyperdef{L}{X784615747EFE7DB1}{}
{\noindent\textcolor{FuncColor}{$\triangleright$\enspace\texttt{Lace({\mdseries\slshape M})\index{Lace@\texttt{Lace}!for IsMapOnSurface}
\label{Lace:for IsMapOnSurface}
}\hfill{\scriptsize (operation)}}\\
\textbf{\indent Returns:\ }
lace 



 Constructs the lace of the map. }

 
\begin{Verbatim}[commandchars=!@|,fontsize=\small,frame=single,label=Example]
  !gapprompt@gap>| !gapinput@SchlafliSymbol(Lace(Cube(3)));|
  [ [ 3, 4 ], [ 4, 9 ] ]
\end{Verbatim}
 

\subsection{\textcolor{Chapter }{Stake (for IsMapOnSurface)}}
\logpage{[ 14, 6, 7 ]}\nobreak
\hyperdef{L}{X7E2A1B6380D9FF22}{}
{\noindent\textcolor{FuncColor}{$\triangleright$\enspace\texttt{Stake({\mdseries\slshape M})\index{Stake@\texttt{Stake}!for IsMapOnSurface}
\label{Stake:for IsMapOnSurface}
}\hfill{\scriptsize (operation)}}\\
\textbf{\indent Returns:\ }
stake 



 Constructs the stake of the map. }

 
\begin{Verbatim}[commandchars=!@|,fontsize=\small,frame=single,label=Example]
  !gapprompt@gap>| !gapinput@SchlafliSymbol(Stake(Cube(3)));|
  [ [ 3, 4 ], [ 3, 4, 9 ] ]
\end{Verbatim}
 

\subsection{\textcolor{Chapter }{Whirl (for IsMapOnSurface)}}
\logpage{[ 14, 6, 8 ]}\nobreak
\hyperdef{L}{X83A031F67C29740B}{}
{\noindent\textcolor{FuncColor}{$\triangleright$\enspace\texttt{Whirl({\mdseries\slshape M})\index{Whirl@\texttt{Whirl}!for IsMapOnSurface}
\label{Whirl:for IsMapOnSurface}
}\hfill{\scriptsize (operation)}}\\
\textbf{\indent Returns:\ }
whirl 



 Constructs the whirl of the map. }

 
\begin{Verbatim}[commandchars=!@|,fontsize=\small,frame=single,label=Example]
  !gapprompt@gap>| !gapinput@SchlafliSymbol(Whirl(Cube(3)));|
  [ [ 4, 6 ], 3 ]              ^
  !gapprompt@gap>| !gapinput@SchlafliSymbol(Whirl(Icosahedron()));|
  [ [ 3, 6 ], [ 3, 5 ] ]
\end{Verbatim}
 

\subsection{\textcolor{Chapter }{Volute (for IsMapOnSurface)}}
\logpage{[ 14, 6, 9 ]}\nobreak
\hyperdef{L}{X7B16C9818423A481}{}
{\noindent\textcolor{FuncColor}{$\triangleright$\enspace\texttt{Volute({\mdseries\slshape M})\index{Volute@\texttt{Volute}!for IsMapOnSurface}
\label{Volute:for IsMapOnSurface}
}\hfill{\scriptsize (operation)}}\\
\textbf{\indent Returns:\ }
volute 



 Constructs the volute of the map. This is equivalent to \texttt{Dual(Whirl(Dual(M)))}. }

 
\begin{Verbatim}[commandchars=!@|,fontsize=\small,frame=single,label=Example]
  !gapprompt@gap>| !gapinput@SchlafliSymbol(Volute(Cube(3)));|
  [ [ 3, 4 ], [ 3, 6 ] ]
  !gapprompt@gap>| !gapinput@SchlafliSymbol(Volute(Dual(Cube(3))));|
  [ 3, [ 4, 6 ] ]
\end{Verbatim}
 

\subsection{\textcolor{Chapter }{JoinKisKis (for IsMapOnSurface)}}
\logpage{[ 14, 6, 10 ]}\nobreak
\hyperdef{L}{X86520B207A4CC8ED}{}
{\noindent\textcolor{FuncColor}{$\triangleright$\enspace\texttt{JoinKisKis({\mdseries\slshape M})\index{JoinKisKis@\texttt{JoinKisKis}!for IsMapOnSurface}
\label{JoinKisKis:for IsMapOnSurface}
}\hfill{\scriptsize (operation)}}\\
\textbf{\indent Returns:\ }
joinkiskis 



 Constructs the join-kis-kis of the map. }

 
\begin{Verbatim}[commandchars=!@|,fontsize=\small,frame=single,label=Example]
  !gapprompt@gap>| !gapinput@SchlafliSymbol(JoinKisKis(Cube(3)));|
  [ [ 3, 4 ], [ 3, 8, 9 ] ]
\end{Verbatim}
 

\subsection{\textcolor{Chapter }{Cross (for IsMapOnSurface)}}
\logpage{[ 14, 6, 11 ]}\nobreak
\hyperdef{L}{X7B141EA9836E1759}{}
{\noindent\textcolor{FuncColor}{$\triangleright$\enspace\texttt{Cross({\mdseries\slshape M})\index{Cross@\texttt{Cross}!for IsMapOnSurface}
\label{Cross:for IsMapOnSurface}
}\hfill{\scriptsize (operation)}}\\
\textbf{\indent Returns:\ }
cross 



 Constructs the cross of the map. }

 
\begin{Verbatim}[commandchars=!@|,fontsize=\small,frame=single,label=Example]
  !gapprompt@gap>| !gapinput@SchlafliSymbol(Cross(Cube(3)));|
  [ [ 3, 4 ], [ 4, 6 ] ]
\end{Verbatim}
 }

 }

   
\chapter{\textcolor{Chapter }{Utility Functions}}\label{Chapter_Utility_Functions}
\logpage{[ 15, 0, 0 ]}
\hyperdef{L}{X810FFB1C8035C8BE}{}
{
  
\section{\textcolor{Chapter }{System}}\label{Chapter_Utility_Functions_Section_System}
\logpage{[ 15, 1, 0 ]}
\hyperdef{L}{X848461D08506A4A6}{}
{
  

\subsection{\textcolor{Chapter }{InfoRamp}}
\logpage{[ 15, 1, 1 ]}\nobreak
\hyperdef{L}{X863ECFAD85CFFCB0}{}
{\noindent\textcolor{FuncColor}{$\triangleright$\enspace\texttt{InfoRamp\index{InfoRamp@\texttt{InfoRamp}}
\label{InfoRamp}
}\hfill{\scriptsize (info class)}}\\


 The InfoClass for the Ramp package. This is sort of an "information channel"
that functions can send updates to, and by default, users of Ramp will see
these messages. To add such a message to a function that you are writing for
Ramp, use \texttt{Info(InfoRamp, 1, "This is a message!");}. For example, if you have a function \texttt{f} with this line, then the user will see this: 
\begin{Verbatim}[commandchars=@|A,fontsize=\small,frame=single,label=Example]
  @gapprompt|gap>A @gapinput|f(3);;A
  #I This is a message!
\end{Verbatim}
 To turn off messages from this class, use \texttt{SetInfoLevel(InfoRamp, 0)}. }

 }

 
\section{\textcolor{Chapter }{Polytopes}}\label{Chapter_Utility_Functions_Section_Polytopes}
\logpage{[ 15, 2, 0 ]}
\hyperdef{L}{X7B933C4686727183}{}
{
  

\subsection{\textcolor{Chapter }{AbstractPolytope}}
\logpage{[ 15, 2, 1 ]}\nobreak
\hyperdef{L}{X7D63A6087BD7F5F8}{}
{\noindent\textcolor{FuncColor}{$\triangleright$\enspace\texttt{AbstractPolytope({\mdseries\slshape args})\index{AbstractPolytope@\texttt{AbstractPolytope}}
\label{AbstractPolytope}
}\hfill{\scriptsize (function)}}\\


 Calls \texttt{Maniplex(args)} and verifies whether the output is polytopal. If not, this throws an error.
Use \texttt{AbstractPolytopeNC} to assume that the output is polytopal and mark it as such. }

 
\begin{Verbatim}[commandchars=!@|,fontsize=\small,frame=single,label=Example]
  !gapprompt@gap>| !gapinput@AbstractPolytope(Group([ (1,2)(3,4)(5,6)(7,8)(9,10), (1,10)(2,3)(4,5)(6,7)(8,9) ]));|
  Pgon(5)
\end{Verbatim}
 

\subsection{\textcolor{Chapter }{AbstractRegularPolytope}}
\logpage{[ 15, 2, 2 ]}\nobreak
\hyperdef{L}{X7C2837FB8174C8DC}{}
{\noindent\textcolor{FuncColor}{$\triangleright$\enspace\texttt{AbstractRegularPolytope({\mdseries\slshape args})\index{AbstractRegularPolytope@\texttt{AbstractRegularPolytope}}
\label{AbstractRegularPolytope}
}\hfill{\scriptsize (function)}}\\


 Calls \texttt{ReflexibleManiplex(args)} and verifies whether the output is polytopal. If not, this throws an error.
Use \texttt{AbstractRegularPolytopeNC} to assume that the output is polytopal and mark it as such. Also available as \texttt{ARP(args)} and \texttt{ARPNC(args)}. }

 
\begin{Verbatim}[commandchars=!@|,fontsize=\small,frame=single,label=Example]
  !gapprompt@gap>| !gapinput@Pgon(5)=AbstractRegularPolytope(Group([(2,3)(4,5),(1,2)(3,4)]));|
  true
\end{Verbatim}
 

\subsection{\textcolor{Chapter }{AbstractRotaryPolytope}}
\logpage{[ 15, 2, 3 ]}\nobreak
\hyperdef{L}{X79AB055B7DDFB160}{}
{\noindent\textcolor{FuncColor}{$\triangleright$\enspace\texttt{AbstractRotaryPolytope({\mdseries\slshape args})\index{AbstractRotaryPolytope@\texttt{AbstractRotaryPolytope}}
\label{AbstractRotaryPolytope}
}\hfill{\scriptsize (function)}}\\


 Calls \texttt{RotaryManiplex(args)} and verifies whether the output is polytopal. If not, this throws an error.
Use \texttt{AbstractRotaryPolytopeNC} to assume that the output is polytopal and mark it as such. }

 
\begin{Verbatim}[commandchars=!@|,fontsize=\small,frame=single,label=Example]
  !gapprompt@gap>| !gapinput@M := AbstractRotaryPolytope(Group((1,2)(3,4), (1,4)(2,3)));|
  regular 3-polytope of type [ 2, 2 ] with 8 flags
  !gapprompt@gap>| !gapinput@M := AbstractRotaryPolytope(Group((1,2,3,4), (1,2)));|
  Error, The given group is not a String Rotation Group...
\end{Verbatim}
 }

 
\section{\textcolor{Chapter }{Permutations}}\label{Chapter_Utility_Functions_Section_Permutations}
\logpage{[ 15, 3, 0 ]}
\hyperdef{L}{X80F808307A2D5AB8}{}
{
  

\subsection{\textcolor{Chapter }{TranslatePerm}}
\logpage{[ 15, 3, 1 ]}\nobreak
\hyperdef{L}{X7CDF25897D4DBE57}{}
{\noindent\textcolor{FuncColor}{$\triangleright$\enspace\texttt{TranslatePerm({\mdseries\slshape perm, k})\index{TranslatePerm@\texttt{TranslatePerm}}
\label{TranslatePerm}
}\hfill{\scriptsize (function)}}\\


 Returns a new permutation obtained from \mbox{\texttt{\mdseries\slshape perm}} by adding k to each moved point. }

 
\begin{Verbatim}[commandchars=!@|,fontsize=\small,frame=single,label=Example]
  !gapprompt@gap>| !gapinput@TranslatePerm((1,2,3,4),5);|
  (6,7,8,9)
\end{Verbatim}
 

\subsection{\textcolor{Chapter }{MultPerm}}
\logpage{[ 15, 3, 2 ]}\nobreak
\hyperdef{L}{X83788BBF7EF52486}{}
{\noindent\textcolor{FuncColor}{$\triangleright$\enspace\texttt{MultPerm({\mdseries\slshape perm, multiplier, offset})\index{MultPerm@\texttt{MultPerm}}
\label{MultPerm}
}\hfill{\scriptsize (function)}}\\


 Multiplies together perm, TranslatePerm(perm, offset), TranslatePerm(perm,
offset*2), ..., with \mbox{\texttt{\mdseries\slshape multiplier}} terms, and returns the result. }

 
\begin{Verbatim}[commandchars=!@|,fontsize=\small,frame=single,label=Example]
  !gapprompt@gap>| !gapinput@MultPerm((1,2,3)(4,5,6),3,7);|
  (1,2,3)(4,5,6)(8,9,10)(11,12,13)(15,16,17)(18,19,20)
  !gapprompt@gap>| !gapinput@MultPerm((1,2,3,4),2,4);|
  (1,2,3,4)(5,6,7,8)
\end{Verbatim}
 

\subsection{\textcolor{Chapter }{InvolutionListList}}
\logpage{[ 15, 3, 3 ]}\nobreak
\hyperdef{L}{X789F611179773CEA}{}
{\noindent\textcolor{FuncColor}{$\triangleright$\enspace\texttt{InvolutionListList({\mdseries\slshape list1, list2})\index{InvolutionListList@\texttt{InvolutionListList}}
\label{InvolutionListList}
}\hfill{\scriptsize (function)}}\\
\textbf{\indent Returns:\ }
involution 



 Construction the involution (when possible) with entries \texttt{(list1[i],list2[i])}. }

 
\begin{Verbatim}[commandchars=!@|,fontsize=\small,frame=single,label=Example]
  !gapprompt@gap>| !gapinput@InvolutionListList([3,4,5],[6,7,8]);|
  (3,6)(4,7)(5,8)
\end{Verbatim}
 

\subsection{\textcolor{Chapter }{PermFromRange (for IsPerm, IsPerm, IsPerm)}}
\logpage{[ 15, 3, 4 ]}\nobreak
\hyperdef{L}{X8738B70283F40353}{}
{\noindent\textcolor{FuncColor}{$\triangleright$\enspace\texttt{PermFromRange({\mdseries\slshape perm1[, perm2], perm3})\index{PermFromRange@\texttt{PermFromRange}!for IsPerm, IsPerm, IsPerm}
\label{PermFromRange:for IsPerm, IsPerm, IsPerm}
}\hfill{\scriptsize (operation)}}\\
\textbf{\indent Returns:\ }
Permutation 



 Given three permutations, where \mbox{\texttt{\mdseries\slshape perm2}} and \mbox{\texttt{\mdseries\slshape perm3}} are translations of \mbox{\texttt{\mdseries\slshape perm1}}, forms the permutation that we would most likely denote by perm1 * perm2 *
... * perm3. Namely, if \mbox{\texttt{\mdseries\slshape perm2}} is a translation of \mbox{\texttt{\mdseries\slshape perm1}} by k, then we successively translate by k until we get \mbox{\texttt{\mdseries\slshape perm3}}, and then we multiply those permutations together. 

 When only two permutations are given, then \mbox{\texttt{\mdseries\slshape perm2}} is the smallest translation of perm1 such that \texttt{SmallestMovedPoint(perm2) {\textgreater} LargestMovedPoint(perm1)}. }

 
\begin{Verbatim}[commandchars=!@|,fontsize=\small,frame=single,label=Example]
  !gapprompt@gap>| !gapinput@PermFromRange((1,2), (9,10));|
  (1,2)(3,4)(5,6)(7,8)(9,10)
  !gapprompt@gap>| !gapinput@PermFromRange((1,3), (13,15));|
  (1,3)(4,6)(7,9)(10,12)(13,15)
  !gapprompt@gap>| !gapinput@PermFromRange((2,3,4), (8,9,10));|
  (2,3,4)(5,6,7)(8,9,10)
  !gapprompt@gap>| !gapinput@PermFromRange((1,2), (101,102), (601,602));|
  (1,2)(101,102)(201,202)(301,302)(401,402)(501,502)(601,602)
\end{Verbatim}
 }

 
\section{\textcolor{Chapter }{Words on relations}}\label{Chapter_Utility_Functions_Section_Words_on_relations}
\logpage{[ 15, 4, 0 ]}
\hyperdef{L}{X8375E2157FA5BA5B}{}
{
  

\subsection{\textcolor{Chapter }{ParseStringCRels}}
\logpage{[ 15, 4, 1 ]}\nobreak
\hyperdef{L}{X7E84DB4E8746344A}{}
{\noindent\textcolor{FuncColor}{$\triangleright$\enspace\texttt{ParseStringCRels({\mdseries\slshape rels, g})\index{ParseStringCRels@\texttt{ParseStringCRels}}
\label{ParseStringCRels}
}\hfill{\scriptsize (function)}}\\
\textbf{\indent Returns:\ }
a list of relators 



 This helper function is used in several maniplex constructors. Given a string \mbox{\texttt{\mdseries\slshape rels}} that represents relations in an sggi, and an sggi g, returns a list of
elements in the free group of g represented by \mbox{\texttt{\mdseries\slshape rels}}. These can then be used to form a quotient of g. 
\begin{Verbatim}[commandchars=!@|,fontsize=\small,frame=single,label=Example]
  !gapprompt@gap>| !gapinput@g := AutomorphismGroup(CubicTiling(2));;|
  !gapprompt@gap>| !gapinput@rels := "(r0 r1 r2 r1)^6";;|
  !gapprompt@gap>| !gapinput@newrels := ParseStringCRels(rels, g);|
  [ (r0*r1*r2*r1)^6 ]
  !gapprompt@gap>| !gapinput@newrels[1] in FreeGroupOfFpGroup(g);|
  true
  !gapprompt@gap>| !gapinput@g2 := FactorGroupFpGroupByRels(g, newrels);|
  <fp group on the generators [ r0, r1, r2 ]>
\end{Verbatim}
 For convenience, you may use z1, z2, etc and h1, h2, etc in relations, where
zj means r0 (r1 r2)\texttt{\symbol{94}}j (the "j-zigzag" word) and hj means r0
(r1 r2)\texttt{\symbol{94}}j-1 r1 (the "j-hole" word). }

 

\subsection{\textcolor{Chapter }{ParseRotGpRels}}
\logpage{[ 15, 4, 2 ]}\nobreak
\hyperdef{L}{X79C6D76380B9F364}{}
{\noindent\textcolor{FuncColor}{$\triangleright$\enspace\texttt{ParseRotGpRels({\mdseries\slshape rels, g})\index{ParseRotGpRels@\texttt{ParseRotGpRels}}
\label{ParseRotGpRels}
}\hfill{\scriptsize (function)}}\\


 This helper function is used in several maniplex constructors. It is analogous
to ParseStringCRels, but for rotation groups instead. }

 
\begin{Verbatim}[commandchars=!@|,fontsize=\small,frame=single,label=Example]
  !gapprompt@gap>| !gapinput@g := UniversalRotationGroup([4,4]);|
  <fp group of size infinity on the generators [ s1, s2 ]>
  !gapprompt@gap>| !gapinput@rels := "(s1 s2^-1)^6";;|
  !gapprompt@gap>| !gapinput@newrels := ParseRotGpRels(rels, g);|
  [ (s1*s2^-1)^6 ]
  !gapprompt@gap>| !gapinput@g2 := FactorGroupFpGroupByRels(g, newrels);|
  <fp group on the generators [ s1, s2 ]>
  !gapprompt@gap>| !gapinput@M := RotaryManiplex(g2);|
  3-maniplex with 288 flags
  !gapprompt@gap>| !gapinput@M = ToroidalMap44([6,0]);|
  true
\end{Verbatim}
 

\subsection{\textcolor{Chapter }{StandardizeSggi}}
\logpage{[ 15, 4, 3 ]}\nobreak
\hyperdef{L}{X85C42B4986C205D0}{}
{\noindent\textcolor{FuncColor}{$\triangleright$\enspace\texttt{StandardizeSggi({\mdseries\slshape g})\index{StandardizeSggi@\texttt{StandardizeSggi}}
\label{StandardizeSggi}
}\hfill{\scriptsize (function)}}\\
\textbf{\indent Returns:\ }
IsSggi 



 Takes an sggi, and returns an isomorphic sggi that is a quotient of the
universal sggi of the appropriate rank. }

 
\begin{Verbatim}[commandchars=!@|,fontsize=\small,frame=single,label=Example]
  !gapprompt@gap>| !gapinput@f := FreeGroup("x","y","z");|
  <free group on the generators [ x, y, z ]>
  !gapprompt@gap>| !gapinput@AssignGeneratorVariables(f);|
  #I  Assigned the global variables [ x, y, z ]
  !gapprompt@gap>| !gapinput@g := f / [x^2, y^2, z^2, (x*z)^2, (x*y)^4, (y*z)^4, (x*y*z)^6];|
  <fp group on the generators [ x, y, z ]>
  !gapprompt@gap>| !gapinput@IsSggi(g);|
  true
  !gapprompt@gap>| !gapinput@g2 := StandardizeSggi(g);|
  <fp group on the generators [ r0, r1, r2 ]>
  !gapprompt@gap>| !gapinput@ReflexibleManiplex(g) = ReflexibleManiplex(g2);|
  true
\end{Verbatim}
 

\subsection{\textcolor{Chapter }{AddOrAppend}}
\logpage{[ 15, 4, 4 ]}\nobreak
\hyperdef{L}{X829B5DA779342A28}{}
{\noindent\textcolor{FuncColor}{$\triangleright$\enspace\texttt{AddOrAppend({\mdseries\slshape L, x})\index{AddOrAppend@\texttt{AddOrAppend}}
\label{AddOrAppend}
}\hfill{\scriptsize (function)}}\\


 Given a list \mbox{\texttt{\mdseries\slshape L}} and an object \mbox{\texttt{\mdseries\slshape x}}, this calls Append(L, x) if x is a list; otherwise it calls Add(L, x). Note
that since strings are internally represented as lists, AddOrAppend(L, "foo")
will append the characters 'f', 'o', 'o'. 
\begin{Verbatim}[commandchars=!@|,fontsize=\small,frame=single,label=Example]
  !gapprompt@gap>| !gapinput@L := [1, 2, 3];;|
  !gapprompt@gap>| !gapinput@AddOrAppend(L, 4);|
  !gapprompt@gap>| !gapinput@L;|
  [1, 2, 3, 4]
  !gapprompt@gap>| !gapinput@AddOrAppend(L, [5, 6]);|
  !gapprompt@gap>| !gapinput@L;|
  [1, 2, 3, 4, 5, 6];
\end{Verbatim}
 }

 

\subsection{\textcolor{Chapter }{WrappedPosetOperation}}
\logpage{[ 15, 4, 5 ]}\nobreak
\hyperdef{L}{X7C4A431283A32F67}{}
{\noindent\textcolor{FuncColor}{$\triangleright$\enspace\texttt{WrappedPosetOperation({\mdseries\slshape posetOp})\index{WrappedPosetOperation@\texttt{WrappedPosetOperation}}
\label{WrappedPosetOperation}
}\hfill{\scriptsize (function)}}\\


 Given a poset operation, creates a bare-bones maniplex operation that
delegates to the poset operation. 
\begin{Verbatim}[commandchars=!@|,fontsize=\small,frame=single,label=Example]
  !gapprompt@gap>| !gapinput@myjoin := WrappedPosetOperation(JoinProduct);|
  function( arg... ) ... end
  !gapprompt@gap>| !gapinput@M := myjoin(Pgon(4), Vertex());|
  3-maniplex
  !gapprompt@gap>| !gapinput@M = Pyramid(4);|
  true
\end{Verbatim}
 Usually, you will want to eventually create a fuller-featured wrapper of the
poset operation -- one that can infer more information from its arguments. But
this method is a good way to quickly test whether a poset operation works on
maniplexes the way one expects. }

 

\subsection{\textcolor{Chapter }{MarkAsPolytopal (for IsManiplex)}}
\logpage{[ 15, 4, 6 ]}\nobreak
\hyperdef{L}{X86B884FA7C22BC92}{}
{\noindent\textcolor{FuncColor}{$\triangleright$\enspace\texttt{MarkAsPolytopal({\mdseries\slshape M})\index{MarkAsPolytopal@\texttt{MarkAsPolytopal}!for IsManiplex}
\label{MarkAsPolytopal:for IsManiplex}
}\hfill{\scriptsize (operation)}}\\


 Sets \texttt{IsPolytopal(M)} as true, and if necessary, changes \texttt{String(M)} to reflect this. }

 

\subsection{\textcolor{Chapter }{ReallyNaturalHomomorphismByNormalSubgroup (for IsGroup,IsGroup)}}
\logpage{[ 15, 4, 7 ]}\nobreak
\hyperdef{L}{X7F5C8A788040FFBF}{}
{\noindent\textcolor{FuncColor}{$\triangleright$\enspace\texttt{ReallyNaturalHomomorphismByNormalSubgroup({\mdseries\slshape G, N})\index{ReallyNaturalHomomorphismByNormalSubgroup@\texttt{Really}\-\texttt{Natural}\-\texttt{Homomorphism}\-\texttt{By}\-\texttt{Normal}\-\texttt{Subgroup}!for IsGroup,IsGroup}
\label{ReallyNaturalHomomorphismByNormalSubgroup:for IsGroup,IsGroup}
}\hfill{\scriptsize (operation)}}\\
\textbf{\indent Returns:\ }
quotient group with generators appropriately mapped 



 Image(NaturalHomomorphismByNormalSubgroup(G,N)) tries to make the quotient
efficient in terms of the number of generators, which is disastrous for
studying Sggis. This fixes that. }

 }

 }

   
\chapter{\textcolor{Chapter }{Synonyms for Commands}}\label{Chapter_Synonyms_for_Commands}
\logpage{[ 16, 0, 0 ]}
\hyperdef{L}{X8019D74C7EBB499C}{}
{
  Here we list, in alphabetical order, synonyms for common commands. 
\begin{itemize}
\item  \texttt{Ambo} for \texttt{Medial} (\textbf{RAMP: Medial for IsManiplex}) 
\item  \texttt{AreIncidentFaces} for \texttt{AreIncidentElements} (\textbf{RAMP: AreIncidentElements for IsObject,IsObject}) 
\item  \texttt{ARP} for \texttt{AbstractRegularPolytope} (\textbf{RAMP: AbstractRegularPolytope}) 
\item  \texttt{Faces} for \texttt{ElementsList} (\textbf{RAMP: ElementsList for IsPoset}) 
\item  \texttt{FacesList} for \texttt{ElementsList} (\textbf{RAMP: ElementsList for IsPoset}) 
\item  \texttt{Flags} for \texttt{MaximalChains} (\textbf{RAMP: MaximalChains for IsPoset}) 
\item  \texttt{FlagsList} for \texttt{MaximalChains} (\textbf{RAMP: MaximalChains for IsPoset}) 
\item  \texttt{IsDiamondCondition} for \texttt{IsP4} (\textbf{RAMP: IsP4 for IsPoset}) 
\item  \texttt{IsStronglyFlagConnected} for \texttt{IsP3} (\textbf{RAMP: IsP3 for IsPoset}) 
\item  \texttt{MapJoin} for \texttt{Angle} (\textbf{RAMP: Angle for IsMapOnSurface}) 
\item  \texttt{MonodromyGroup} for \texttt{ConnectionGroup} (\textbf{RAMP: ConnectionGroup for IsPremaniplex}) 
\item  \texttt{NumberOfFlags} for \texttt{Size} (\textbf{RAMP: Size for IsManiplex}) 
\item  \texttt{PetrieDual} for \texttt{Petrial} (\textbf{RAMP: Petrial for IsManiplex}) 
\item  \texttt{RankPosetFaces} for \texttt{RankPosetElements} (\textbf{RAMP: RankPosetElements for IsPoset}) 
\item  \texttt{RefMan} for \texttt{ReflexibleManiplex} (\textbf{RAMP: ReflexibleManiplex}) 
\end{itemize}
 }

   
\chapter{\textcolor{Chapter }{Graphs for Premaniplexes}}\label{Chapter_Graphs_for_Premaniplexes}
\logpage{[ 17, 0, 0 ]}
\hyperdef{L}{X832EB60880155740}{}
{
  
\section{\textcolor{Chapter }{Constructors of Premaniplexes}}\label{Chapter_Graphs_for_Premaniplexes_Section_Constructors_of_Premaniplexes}
\logpage{[ 17, 1, 0 ]}
\hyperdef{L}{X79636D1B82B8AF46}{}
{
  

\subsection{\textcolor{Chapter }{Premaniplex (for IsGroup)}}
\logpage{[ 17, 1, 1 ]}\nobreak
\hyperdef{L}{X7B81F23579DC1CC4}{}
{\noindent\textcolor{FuncColor}{$\triangleright$\enspace\texttt{Premaniplex({\mdseries\slshape group})\index{Premaniplex@\texttt{Premaniplex}!for IsGroup}
\label{Premaniplex:for IsGroup}
}\hfill{\scriptsize (operation)}}\\
\textbf{\indent Returns:\ }
\texttt{IsPremaniplex}. 



 Given a group we return the premaniplex with that group as its connection
group. This function first checks whether \mbox{\texttt{\mdseries\slshape group}} is an Sggi. Use \texttt{PremaniplexNC} to bypass that check. }

 Here we build a premaniplex with 3 flags. 
\begin{Verbatim}[commandchars=!@|,fontsize=\small,frame=single,label=Example]
  !gapprompt@gap>| !gapinput@g:=Group((1,2),(2,3),(1,2));;|
  !gapprompt@gap>| !gapinput@Premaniplex(g);|
  Premaniplex of rank 3 with 3 flags
\end{Verbatim}
 

\subsection{\textcolor{Chapter }{Premaniplex (for IsEdgeLabeledGraph)}}
\logpage{[ 17, 1, 2 ]}\nobreak
\hyperdef{L}{X82F7D8227F616FD2}{}
{\noindent\textcolor{FuncColor}{$\triangleright$\enspace\texttt{Premaniplex({\mdseries\slshape edgelabeledgraph})\index{Premaniplex@\texttt{Premaniplex}!for IsEdgeLabeledGraph}
\label{Premaniplex:for IsEdgeLabeledGraph}
}\hfill{\scriptsize (operation)}}\\
\textbf{\indent Returns:\ }
\texttt{IsPremaniplex}. 



 Given an edge labeled graph we return the premaniplex with for that graph.
Note: We will assume (but not check) that the graph is a premaniplex, that is
to say, we are assumging each vertex is incident with one edge of each label. }

 Here we have a premaniplex with 2 flags. 
\begin{Verbatim}[commandchars=!@|,fontsize=\small,frame=single,label=Example]
  !gapprompt@gap>| !gapinput@gap> L:=[[[1,2],0], [[1,2],1], [[1],2], [[2],2]];;|
  !gapprompt@gap>| !gapinput@F:=EdgeLabeledGraphFromLabeledEdges(L);;|
  !gapprompt@gap>| !gapinput@Premaniplex(F);|
  Premaniplex of rank 3 with 2 flags
\end{Verbatim}
 

\subsection{\textcolor{Chapter }{STG1 (for IsInt)}}
\logpage{[ 17, 1, 3 ]}\nobreak
\hyperdef{L}{X7DB8D3047E0B5A7D}{}
{\noindent\textcolor{FuncColor}{$\triangleright$\enspace\texttt{STG1({\mdseries\slshape int})\index{STG1@\texttt{STG1}!for IsInt}
\label{STG1:for IsInt}
}\hfill{\scriptsize (operation)}}\\
\textbf{\indent Returns:\ }
premaniplex 



 Builds the 1 flag premaniplex of rank n Note See VOLTAGE OPERATIONS ON
MANIPLEXES }

 
\begin{Verbatim}[commandchars=!@|,fontsize=\small,frame=single,label=Example]
  !gapprompt@gap>| !gapinput@STG1(5);|
  Premaniplex of rank 5 with 1 flag
\end{Verbatim}
 

\subsection{\textcolor{Chapter }{STG2 (for IsInt,IsList)}}
\logpage{[ 17, 1, 4 ]}\nobreak
\hyperdef{L}{X7BCF47AC7876459A}{}
{\noindent\textcolor{FuncColor}{$\triangleright$\enspace\texttt{STG2({\mdseries\slshape int, list})\index{STG2@\texttt{STG2}!for IsInt,IsList}
\label{STG2:for IsInt,IsList}
}\hfill{\scriptsize (operation)}}\\
\textbf{\indent Returns:\ }
premaniplex 



 Builds the 2 flag premaniplex of rank n with semi-edges in I Note See VOLTAGE
OPERATIONS ON MANIPLEXES }

 
\begin{Verbatim}[commandchars=!@|,fontsize=\small,frame=single,label=Example]
  !gapprompt@gap>| !gapinput@STG2(5,[2,4]);|
  Premaniplex of rank 5 with 2 flags
\end{Verbatim}
 

\subsection{\textcolor{Chapter }{STG3 (for IsInt,IsInt)}}
\logpage{[ 17, 1, 5 ]}\nobreak
\hyperdef{L}{X7C2B1DB07CB38F59}{}
{\noindent\textcolor{FuncColor}{$\triangleright$\enspace\texttt{STG3({\mdseries\slshape int, int})\index{STG3@\texttt{STG3}!for IsInt,IsInt}
\label{STG3:for IsInt,IsInt}
}\hfill{\scriptsize (operation)}}\\
\textbf{\indent Returns:\ }
premaniplex 



 Builds the 3 flag premaniplex of rank n described on Page 11 of Symmetry Type
Graphs of Polytopes and Maniplexes. There are edges of label i-1 and i+1 are
parallel. }

 
\begin{Verbatim}[commandchars=!@|,fontsize=\small,frame=single,label=Example]
  !gapprompt@gap>| !gapinput@STG3(5,2);|
  Premaniplex of rank 5 with 3 flags
\end{Verbatim}
 

\subsection{\textcolor{Chapter }{STG3 (for IsInt,IsInt,IsInt)}}
\logpage{[ 17, 1, 6 ]}\nobreak
\hyperdef{L}{X7EFDB6887DEAFB4F}{}
{\noindent\textcolor{FuncColor}{$\triangleright$\enspace\texttt{STG3({\mdseries\slshape int, int, int})\index{STG3@\texttt{STG3}!for IsInt,IsInt,IsInt}
\label{STG3:for IsInt,IsInt,IsInt}
}\hfill{\scriptsize (operation)}}\\
\textbf{\indent Returns:\ }
premaniplex 



 Assumes j=i+1 and builds the 3 flag premaniplex of rank n described on Page 11
of Symmetry Type Graphs of Polytopes and Maniplexes. There are edges of label
i and j. }

 
\begin{Verbatim}[commandchars=!@|,fontsize=\small,frame=single,label=Example]
  !gapprompt@gap>| !gapinput@STG3(6,2,3);|
  Premaniplex of rank 6 with 3 flags
\end{Verbatim}
 

\subsection{\textcolor{Chapter }{FlagGraph (for IsPremaniplex)}}
\logpage{[ 17, 1, 7 ]}\nobreak
\hyperdef{L}{X863F65F6865859F1}{}
{\noindent\textcolor{FuncColor}{$\triangleright$\enspace\texttt{FlagGraph({\mdseries\slshape premaniplex})\index{FlagGraph@\texttt{FlagGraph}!for IsPremaniplex}
\label{FlagGraph:for IsPremaniplex}
}\hfill{\scriptsize (operation)}}\\
\textbf{\indent Returns:\ }
edgelabeledgraph 



 Returns the flag graph of a premaniplex }

 
\begin{Verbatim}[commandchars=!@|,fontsize=\small,frame=single,label=Example]
  !gapprompt@gap>| !gapinput@STG3(4,1);;|
   gap> FlagGraph(last);
  Edge labeled graph with 3 vertices, and edge labels [ 0, 1, 2, 3 ]
\end{Verbatim}
 

\subsection{\textcolor{Chapter }{LabeledDarts (for IsPremaniplex)}}
\logpage{[ 17, 1, 8 ]}\nobreak
\hyperdef{L}{X7E7E64817A09366F}{}
{\noindent\textcolor{FuncColor}{$\triangleright$\enspace\texttt{LabeledDarts({\mdseries\slshape p})\index{LabeledDarts@\texttt{LabeledDarts}!for IsPremaniplex}
\label{LabeledDarts:for IsPremaniplex}
}\hfill{\scriptsize (attribute)}}\\
\textbf{\indent Returns:\ }
list 



 Given a Premaniples p, returns the list of labeled darts from its flag graph. }

 
\begin{Verbatim}[commandchars=!@|,fontsize=\small,frame=single,label=Example]
  !gapprompt@gap>| !gapinput@P:=STG2(5,[2,4]);;|
  !gapprompt@gap>| !gapinput@LabeledDarts(P);|
  [ [ [ 1, 2 ], 0 ], [ [ 2, 1 ], 0 ], [ [ 1, 2 ], 1 ], [ [ 2, 1 ], 1 ], [ [ 1 ], 2 ], [ [ 1, 2 ], 3 ], [ [ 2, 1 ], 3 ], [ [ 1 ], 4 ], [ [ 2 ], 2 ], [ [ 2 ], 4 ] ]
\end{Verbatim}
 }

 }

   
\chapter{\textcolor{Chapter }{Voltage Graphs and Operations}}\label{Chapter_Voltage_Graphs_and_Operations}
\logpage{[ 18, 0, 0 ]}
\hyperdef{L}{X8150D23C7BD686A4}{}
{
  
\section{\textcolor{Chapter }{Voltage Operator}}\label{Chapter_Voltage_Graphs_and_Operations_Section_Voltage_Operator}
\logpage{[ 18, 1, 0 ]}
\hyperdef{L}{X7EDB00697FB48C33}{}
{
  

\subsection{\textcolor{Chapter }{VoltageOperator (for IsList, IsString,IsEdgeLabeledGraph)}}
\logpage{[ 18, 1, 1 ]}\nobreak
\hyperdef{L}{X8440AE1A7A45E3AD}{}
{\noindent\textcolor{FuncColor}{$\triangleright$\enspace\texttt{VoltageOperator({\mdseries\slshape etain, etaout, Xa})\index{VoltageOperator@\texttt{VoltageOperator}!for IsList, IsString,IsEdgeLabeledGraph}
\label{VoltageOperator:for IsList, IsString,IsEdgeLabeledGraph}
}\hfill{\scriptsize (operation)}}\\
\textbf{\indent Returns:\ }
IsManiplex 



 Returns the output of the voltage operator acting on Xa. Xa is a n-premaniplex
as an edge labeled graph, Y is a m-premaniplex. eta is a voltage assignment on
the darts of Y. etain is a list of all darts of Y. etaout is a string giving
words in the universal sggi of rank n, and the order of the words corresponds
to the order of the darts in etain. If Xa is given as a maniplex, the
operation is done to its flag graph. }

 

\subsection{\textcolor{Chapter }{VoltageOperator (for IsList, IsString,IsManiplex)}}
\logpage{[ 18, 1, 2 ]}\nobreak
\hyperdef{L}{X7817A3787920523B}{}
{\noindent\textcolor{FuncColor}{$\triangleright$\enspace\texttt{VoltageOperator({\mdseries\slshape arg1, arg2, arg3})\index{VoltageOperator@\texttt{VoltageOperator}!for IsList, IsString,IsManiplex}
\label{VoltageOperator:for IsList, IsString,IsManiplex}
}\hfill{\scriptsize (operation)}}\\


 

 }

 
\begin{Verbatim}[commandchars=!@|,fontsize=\small,frame=single,label=Example]
  The Petrial and the dual can be built using voltage operations
  Similarly for rank 3 other operations can be built this way.
  See VOLTAGE OPERATIONS ON MANIPLEXES by HUBARD, MOCHÁN, MONTERO
  !gapprompt@gap>| !gapinput@etain1:=[[[1],0],[[1],1],[[1],2],[[1],3]];;|
  !gapprompt@gap>| !gapinput@etain2:=[[[1],0],[[2],0],[[1],1],[[2],1],[[1,2],2]];;|
  !gapprompt@gap>| !gapinput@etain3:=[[[1],0],[[2],0],[[3],0],[[1],1],[[3],2],[[1,2],2],[[2,3],1]];;|
  !gapprompt@gap>| !gapinput@etaoutPetrial:="r0, r1 r3, r2, r3";;|
  !gapprompt@gap>| !gapinput@etaoutDual:="r3, r2, r1, r0";;|
  !gapprompt@gap>| !gapinput@etaoutMedial:="r1, r1, r0, r2, Id";;|
  !gapprompt@gap>| !gapinput@etaoutLeapfrog:="r1,r1,r2,r0,r0, , ";;|
  !gapprompt@gap>| !gapinput@etaoutTruncation:="r1, r1, r0, r2, r2,Id, Id";;|
  !gapprompt@gap>| !gapinput@Petrial(Cube(4)) =VoltageOperator(etain1, etaoutPetrial, Cube(4));|
  true
  !gapprompt@gap>| !gapinput@Dual(Cube(4)) = VoltageOperator(etain1, etaoutDual, Cube(4));|
  true
  !gapprompt@gap>| !gapinput@Medial(Dodecahedron()) = VoltageOperator(etain2, etaoutMedial, Dodecahedron());|
  true
  !gapprompt@gap>| !gapinput@Leapfrog(Simplex(3)) =  VoltageOperator(etain3, etaoutLeapfrog, Simplex(3));|
  true
  !gapprompt@gap>| !gapinput@Truncation(Prism(7)) = VoltageOperator(etain3, etaoutTruncation, Prism(7));|
  true
\end{Verbatim}
 

\subsection{\textcolor{Chapter }{AdmissiblePerms (for IsInt, IsList)}}
\logpage{[ 18, 1, 3 ]}\nobreak
\hyperdef{L}{X87B717357AD16360}{}
{\noindent\textcolor{FuncColor}{$\triangleright$\enspace\texttt{AdmissiblePerms({\mdseries\slshape n, I})\index{AdmissiblePerms@\texttt{AdmissiblePerms}!for IsInt, IsList}
\label{AdmissiblePerms:for IsInt, IsList}
}\hfill{\scriptsize (operation)}}\\
\textbf{\indent Returns:\ }
IsList 



 Returns a list of the admissible sequences that correspond to the flag orbits
for a Wythoffian of a rank n maniplex. The vertex in the fundamental region is
moved by ri for i in I. }

 
\begin{Verbatim}[commandchars=!@|,fontsize=\small,frame=single,label=Example]
  There will be three flag orbits in the truncation of a rank 3 maniplex, where truncation is a Wythoffican defined by I = [0,1]
  !gapprompt@gap>| !gapinput@AdmissiblePerms(3,[0,1]);|
   [ [ 0, 1, 2 ], [ 1, 0, 2 ], [ 1, 2, 0 ] ]
\end{Verbatim}
 

\subsection{\textcolor{Chapter }{WythoffSTG (for IsInt, IsList)}}
\logpage{[ 18, 1, 4 ]}\nobreak
\hyperdef{L}{X7B45168A7EB410C2}{}
{\noindent\textcolor{FuncColor}{$\triangleright$\enspace\texttt{WythoffSTG({\mdseries\slshape n, I})\index{WythoffSTG@\texttt{WythoffSTG}!for IsInt, IsList}
\label{WythoffSTG:for IsInt, IsList}
}\hfill{\scriptsize (operation)}}\\
\textbf{\indent Returns:\ }
IsList 



 Returns the symmetry type graph for a Wythoffian of rank n defined by a list
of indices. See, for instance, VOLTAGE OPERATIONS ON MANIPLEXES. }

 
\begin{Verbatim}[commandchars=!@|,fontsize=\small,frame=single,label=Example]
  Symmetry type graph of a medial operation
  !gapprompt@gap>| !gapinput@W:=WythoffSTG(3,[1]);|
  Edge labeled graph with 2 vertices, and edge labels [ 0, 1, 2 ]
  !gapprompt@gap>| !gapinput@LabeledEdges(W);|
  [ [ [ 1 ], 0 ], [ [ 1 ], 1 ], [ [ 1, 2 ], 2 ], [ [ 2 ], 0 ], [ [ 2 ], 1 ] ]
\end{Verbatim}
 

\subsection{\textcolor{Chapter }{WythoffLabeledEdges (for IsInt, IsList)}}
\logpage{[ 18, 1, 5 ]}\nobreak
\hyperdef{L}{X838E62987EA8CE63}{}
{\noindent\textcolor{FuncColor}{$\triangleright$\enspace\texttt{WythoffLabeledEdges({\mdseries\slshape n, I})\index{WythoffLabeledEdges@\texttt{WythoffLabeledEdges}!for IsInt, IsList}
\label{WythoffLabeledEdges:for IsInt, IsList}
}\hfill{\scriptsize (operation)}}\\
\textbf{\indent Returns:\ }
IsList 



 Returns the labeled edges of a possible symmetry type graph for a Wythoffian
of rank n defined by a list of indices. The actual graph is not returned, as
we require edge labeled graphs to have integer vertices in order to calculate
their connection groups. }

 
\begin{Verbatim}[commandchars=!@|,fontsize=\small,frame=single,label=Example]
  Labeled Edges of the Symmetry type graph of a medial operation
  !gapprompt@gap>| !gapinput@WythoffLabeledEdges(3,[1]);|
  [ [ [ [ 1, 0, 2 ] ], 0 ], [ [ [ 1, 0, 2 ] ], 1 ], [ [ [ 1, 2, 0 ] ], 0 ], [ [ [ 1, 2, 0 ] ], 1 ], [ [ [ 1, 2, 0 ], [ 1, 0, 2 ] ], 2 ] ]
\end{Verbatim}
 

\subsection{\textcolor{Chapter }{WythoffVoltageOperator (for IsInt, IsList, IsManiplex)}}
\logpage{[ 18, 1, 6 ]}\nobreak
\hyperdef{L}{X79B4314B7EAE8598}{}
{\noindent\textcolor{FuncColor}{$\triangleright$\enspace\texttt{WythoffVoltageOperator({\mdseries\slshape n, I, M})\index{WythoffVoltageOperator@\texttt{WythoffVoltageOperator}!for IsInt, IsList, IsManiplex}
\label{WythoffVoltageOperator:for IsInt, IsList, IsManiplex}
}\hfill{\scriptsize (operation)}}\\
\textbf{\indent Returns:\ }
IsList 



 Returns the maniplex built from a voltage operation given a Wythoffian }

 
\begin{Verbatim}[commandchars=!@|,fontsize=\small,frame=single,label=Example]
  Truncation built using voltages 
  !gapprompt@gap>| !gapinput@W:=WythoffVoltageOperator(3,[0,1],Dodecahedron());|
  3-maniplex with 360 flags
  !gapprompt@gap>| !gapinput@W=Truncation(Dodecahedron());|
  true
\end{Verbatim}
 

\subsection{\textcolor{Chapter }{VoltageGraph (for IsGroup,IsList,IsList)}}
\logpage{[ 18, 1, 7 ]}\nobreak
\hyperdef{L}{X80D0DC17799E3DBF}{}
{\noindent\textcolor{FuncColor}{$\triangleright$\enspace\texttt{VoltageGraph({\mdseries\slshape G, L, V})\index{VoltageGraph@\texttt{VoltageGraph}!for IsGroup,IsList,IsList}
\label{VoltageGraph:for IsGroup,IsList,IsList}
}\hfill{\scriptsize (operation)}}\\
\textbf{\indent Returns:\ }
IsVoltageGraph 



 Given an IsGroup \mbox{\texttt{\mdseries\slshape G}}, an IsList \mbox{\texttt{\mdseries\slshape L}}, and an IsList \mbox{\texttt{\mdseries\slshape V}}, \texttt{VoltageGraph(G,L,V)} will construct the voltage graph with voltages from G, labeled darts from L,
and voltages from V. }

 
\begin{Verbatim}[commandchars=!@|,fontsize=\small,frame=single,label=Example]
  !gapprompt@gap>| !gapinput@G:=ConnectionGroup(Cube(3));;|
  !gapprompt@gap>| !gapinput@L:=[ [[1],0], [[1],1], [[1,2],2], [[2],0], [[2],1]];;|
  !gapprompt@gap>| !gapinput@V:=[G.2, G.1, Identity(G), G.2, G.1];;|
  !gapprompt@gap>| !gapinput@VG:=VoltageGraph(G,L,V);|
  Voltage Graph with voltages from Group( [ (1,20)(2,13)(3,10)(4,45)(5,35)(6,7)(8,41)(9,28)(11,38)
  (12,24)(14,43)(15,34)(16,33)(17,19)(18,31)(21,39)(22,27)(23,26)(25,36)(29,32)(30,48)(37,47)
  (40,46)(42,44), (1,11)(2,32)(3,14)(4,25)(5,26)(6,27)(7,8)(9,43)(10,44)(12,29)(13,30)(15,39)(16,40)
  (17,41)(18,21)(19,22)(20,23)(24,48)(28,42)(31,47)(33,36)(34,37)(35,38)(45,46), (1,3)(2,7)
  (4,11)(5,12)(6,13)(8,18)(9,19)(10,20)(14,25)(15,26)(16,27)(17,28)(21,32)(22,33)(23,34)(24,35)
  (29,39)(30,40)(31,41)(36,43)(37,44)(38,45)(42,47)(46,48) ] )
\end{Verbatim}
 

\subsection{\textcolor{Chapter }{VoltageGraph (for IsGroup,IsPremaniplex,IsList)}}
\logpage{[ 18, 1, 8 ]}\nobreak
\hyperdef{L}{X83B940167FBDDB22}{}
{\noindent\textcolor{FuncColor}{$\triangleright$\enspace\texttt{VoltageGraph({\mdseries\slshape G, P, V})\index{VoltageGraph@\texttt{VoltageGraph}!for IsGroup,IsPremaniplex,IsList}
\label{VoltageGraph:for IsGroup,IsPremaniplex,IsList}
}\hfill{\scriptsize (operation)}}\\
\textbf{\indent Returns:\ }
IsVoltageGraph 



 Given an IsGroup \mbox{\texttt{\mdseries\slshape G}}, an IsPremaniplex \mbox{\texttt{\mdseries\slshape P}}, and an IsList \mbox{\texttt{\mdseries\slshape V}}, \texttt{VoltageGraph(G,P,V)} will construct the voltage graph with voltages from G, labeled darts from the
premaniplex P, and voltages from V. }

 
\begin{Verbatim}[commandchars=!@|,fontsize=\small,frame=single,label=Example]
  !gapprompt@gap>| !gapinput@G:=ConnectionGroup(Cube(3));;|
  !gapprompt@gap>| !gapinput@P:=STG2(3,[0,1]);|
  Premaniplex of rank 3 with 2 flags
  !gapprompt@gap>| !gapinput@L:=LabeledDarts(P);|
  [ [ [ 1 ], 0 ], [ [ 1 ], 1 ], [ [ 1, 2 ], 2 ], [ [ 2, 1 ], 2 ], [ [ 2 ], 0 ], [ [ 2 ], 1 ] ]
  !gapprompt@gap>| !gapinput@V:=[G.2, G.1, Identity(G), Identity(G), G.2, G.1];;|
  !gapprompt@gap>| !gapinput@VG:=VoltageGraph(G,P,V);|
  Voltage Graph with voltages from Group( [ (1,20)(2,13)(3,10)(4,45)(5,35)(6,7)(8,41)(9,28)(11,38)(12,24)(14,43)
  (15,34)(16,33)(17,19)(18,31)(21,39)(22,27)(23,26)(25,36)(29,32)(30,48)(37,47)(40,46)(42,44), 
  (1,11)(2,32)(3,14)(4,25)(5,26)(6,27)(7,8)(9,43)(10,44)(12,29)(13,30)(15,39)(16,40)(17,41)(18,21)(19,22)(20,23)
  (24,48)(28,42)(31,47)(33,36)(34,37)(35,38)(45,46), 
  (1,3)(2,7)(4,11)(5,12)(6,13)(8,18)(9,19)(10,20)(14,25)(15,26)(16,27)(17,28)(21,32)(22,33)(23,34)(24,35)(29,39)
  (30,40)(31,41)(36,43)(37,44)(38,45)(42,47)(46,48) ] )
\end{Verbatim}
 

\subsection{\textcolor{Chapter }{VoltageGraph (for IsGroup,IsPremaniplex)}}
\logpage{[ 18, 1, 9 ]}\nobreak
\hyperdef{L}{X81682EEF8674907A}{}
{\noindent\textcolor{FuncColor}{$\triangleright$\enspace\texttt{VoltageGraph({\mdseries\slshape G, P})\index{VoltageGraph@\texttt{VoltageGraph}!for IsGroup,IsPremaniplex}
\label{VoltageGraph:for IsGroup,IsPremaniplex}
}\hfill{\scriptsize (operation)}}\\
\textbf{\indent Returns:\ }
IsVoltageGraph 



 Given an IsGroup \mbox{\texttt{\mdseries\slshape G}}, and an IsPremaniplex \mbox{\texttt{\mdseries\slshape P}}, \texttt{VoltageGraph(G,P)} will construct the voltage graph with voltages from G, labeled darts from the
premaniplex P, and trivial voltages. }

 

\subsection{\textcolor{Chapter }{ChangeVoltage (for IsVoltageGraph,IsList, IsObject)}}
\logpage{[ 18, 1, 10 ]}\nobreak
\hyperdef{L}{X7FFBBED6838B11AD}{}
{\noindent\textcolor{FuncColor}{$\triangleright$\enspace\texttt{ChangeVoltage({\mdseries\slshape VG, ld, g})\index{ChangeVoltage@\texttt{ChangeVoltage}!for IsVoltageGraph,IsList, IsObject}
\label{ChangeVoltage:for IsVoltageGraph,IsList, IsObject}
}\hfill{\scriptsize (operation)}}\\


 Given an IsVoltageGraph \mbox{\texttt{\mdseries\slshape VG}}, an IsList \mbox{\texttt{\mdseries\slshape ld}}, and an IsObject \mbox{\texttt{\mdseries\slshape g}}, \texttt{ChangeVoltage(VG,ld,g)} will change the voltage for the one labeled dart ld to the group element g. }

 

\subsection{\textcolor{Chapter }{ChangeVoltage (for IsVoltageGraph,IsInt,IsInt, IsObject)}}
\logpage{[ 18, 1, 11 ]}\nobreak
\hyperdef{L}{X78C2B4FE8389CF98}{}
{\noindent\textcolor{FuncColor}{$\triangleright$\enspace\texttt{ChangeVoltage({\mdseries\slshape VG, lab, startvert, g})\index{ChangeVoltage@\texttt{ChangeVoltage}!for IsVoltageGraph,IsInt,IsInt, IsObject}
\label{ChangeVoltage:for IsVoltageGraph,IsInt,IsInt, IsObject}
}\hfill{\scriptsize (operation)}}\\


 Given an IsVoltageGraph \mbox{\texttt{\mdseries\slshape VG}}, an IsInt \mbox{\texttt{\mdseries\slshape lab}}, an IsInt \mbox{\texttt{\mdseries\slshape startvert}}, and an IsObject \mbox{\texttt{\mdseries\slshape g}}, \texttt{ChangeVoltage(VG,lab, startvert,g)} will change the voltage for the one labeled dart of label lab and start vertex
startvert to the group element g. }

 

\subsection{\textcolor{Chapter }{DerivedGraph (for IsVoltageGraph)}}
\logpage{[ 18, 1, 12 ]}\nobreak
\hyperdef{L}{X7BC3B94379122B29}{}
{\noindent\textcolor{FuncColor}{$\triangleright$\enspace\texttt{DerivedGraph({\mdseries\slshape VG})\index{DerivedGraph@\texttt{DerivedGraph}!for IsVoltageGraph}
\label{DerivedGraph:for IsVoltageGraph}
}\hfill{\scriptsize (attribute)}}\\
\textbf{\indent Returns:\ }
IsVoltageGraph 



 Given an IsVoltageGraph \mbox{\texttt{\mdseries\slshape VG}}, a \texttt{DerivedGraph(VG)} will construct the derived graph of the voltage graph VG. }

 

\subsection{\textcolor{Chapter }{VoltageOperator (for IsVoltageGraph, IsManiplex)}}
\logpage{[ 18, 1, 13 ]}\nobreak
\hyperdef{L}{X87B722CB84C389BB}{}
{\noindent\textcolor{FuncColor}{$\triangleright$\enspace\texttt{VoltageOperator({\mdseries\slshape VG, M})\index{VoltageOperator@\texttt{VoltageOperator}!for IsVoltageGraph, IsManiplex}
\label{VoltageOperator:for IsVoltageGraph, IsManiplex}
}\hfill{\scriptsize (operation)}}\\


 Given an IsVoltageGraph \mbox{\texttt{\mdseries\slshape VG}}, and an IsManiplex \mbox{\texttt{\mdseries\slshape M}}, \texttt{VoltageOperator(VG,M)} will return the voltage operator VG acting on M. }

 
\begin{Verbatim}[commandchars=!@|,fontsize=\small,frame=single,label=Example]
  !gapprompt@gap>| !gapinput@M:=Dodecahedron();;|
  !gapprompt@gap>| !gapinput@S:=STG2(3,[0,1]);|
  Premaniplex of rank 3 with 2 flags
  !gapprompt@gap>| !gapinput@C:=ConnectionGroup(M);;|
  !gapprompt@gap>| !gapinput@V:=VoltageGraph(C,S);;|
  !gapprompt@gap>| !gapinput@ChangeVoltage(V,0,1,C.2);;|
  !gapprompt@gap>| !gapinput@ChangeVoltage(V,0,2,C.2);;|
  !gapprompt@gap>| !gapinput@ChangeVoltage(V,1,1,C.1);;|
  !gapprompt@gap>| !gapinput@ChangeVoltage(V,1,2,C.3);;|
  !gapprompt@gap>| !gapinput@Medial(M) = VoltageOperator(V,M);|
  true
\end{Verbatim}
 

\subsection{\textcolor{Chapter }{VoltageOperator (for IsVoltageGraph,IsEdgeLabeledGraph)}}
\logpage{[ 18, 1, 14 ]}\nobreak
\hyperdef{L}{X7C446B798146B031}{}
{\noindent\textcolor{FuncColor}{$\triangleright$\enspace\texttt{VoltageOperator({\mdseries\slshape VG, ELG})\index{VoltageOperator@\texttt{VoltageOperator}!for IsVoltageGraph,IsEdgeLabeledGraph}
\label{VoltageOperator:for IsVoltageGraph,IsEdgeLabeledGraph}
}\hfill{\scriptsize (operation)}}\\


 Given an IsVoltageGraph \mbox{\texttt{\mdseries\slshape VG}}, and an IsEdgeLabeledGraph \mbox{\texttt{\mdseries\slshape ELM}}, \texttt{VoltageOperator(VG,M)} will return the voltage operator VG acting on ELM. }

 }

 }

 \def\bibname{References\logpage{[ "Bib", 0, 0 ]}
\hyperdef{L}{X7A6F98FD85F02BFE}{}
}

\bibliographystyle{alpha}
\bibliography{ramp.bib}

\addcontentsline{toc}{chapter}{References}

\def\indexname{Index\logpage{[ "Ind", 0, 0 ]}
\hyperdef{L}{X83A0356F839C696F}{}
}

\cleardoublepage
\phantomsection
\addcontentsline{toc}{chapter}{Index}


\printindex

\immediate\write\pagenrlog{["Ind", 0, 0], \arabic{page},}
\newpage
\immediate\write\pagenrlog{["End"], \arabic{page}];}
\immediate\closeout\pagenrlog
\end{document}
