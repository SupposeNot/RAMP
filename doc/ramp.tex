% generated by GAPDoc2LaTeX from XML source (Frank Luebeck)
\documentclass[a4paper,11pt]{report}

\usepackage[top=37mm,bottom=37mm,left=27mm,right=27mm]{geometry}
\sloppy
\pagestyle{myheadings}
\usepackage{amssymb}
\usepackage[utf8]{inputenc}
\usepackage{makeidx}
\makeindex
\usepackage{color}
\definecolor{FireBrick}{rgb}{0.5812,0.0074,0.0083}
\definecolor{RoyalBlue}{rgb}{0.0236,0.0894,0.6179}
\definecolor{RoyalGreen}{rgb}{0.0236,0.6179,0.0894}
\definecolor{RoyalRed}{rgb}{0.6179,0.0236,0.0894}
\definecolor{LightBlue}{rgb}{0.8544,0.9511,1.0000}
\definecolor{Black}{rgb}{0.0,0.0,0.0}

\definecolor{linkColor}{rgb}{0.0,0.0,0.554}
\definecolor{citeColor}{rgb}{0.0,0.0,0.554}
\definecolor{fileColor}{rgb}{0.0,0.0,0.554}
\definecolor{urlColor}{rgb}{0.0,0.0,0.554}
\definecolor{promptColor}{rgb}{0.0,0.0,0.589}
\definecolor{brkpromptColor}{rgb}{0.589,0.0,0.0}
\definecolor{gapinputColor}{rgb}{0.589,0.0,0.0}
\definecolor{gapoutputColor}{rgb}{0.0,0.0,0.0}

%%  for a long time these were red and blue by default,
%%  now black, but keep variables to overwrite
\definecolor{FuncColor}{rgb}{0.0,0.0,0.0}
%% strange name because of pdflatex bug:
\definecolor{Chapter }{rgb}{0.0,0.0,0.0}
\definecolor{DarkOlive}{rgb}{0.1047,0.2412,0.0064}


\usepackage{fancyvrb}

\usepackage{mathptmx,helvet}
\usepackage[T1]{fontenc}
\usepackage{textcomp}


\usepackage[
            pdftex=true,
            bookmarks=true,        
            a4paper=true,
            pdftitle={Written with GAPDoc},
            pdfcreator={LaTeX with hyperref package / GAPDoc},
            colorlinks=true,
            backref=page,
            breaklinks=true,
            linkcolor=linkColor,
            citecolor=citeColor,
            filecolor=fileColor,
            urlcolor=urlColor,
            pdfpagemode={UseNone}, 
           ]{hyperref}

\newcommand{\maintitlesize}{\fontsize{50}{55}\selectfont}

% write page numbers to a .pnr log file for online help
\newwrite\pagenrlog
\immediate\openout\pagenrlog =\jobname.pnr
\immediate\write\pagenrlog{PAGENRS := [}
\newcommand{\logpage}[1]{\protect\write\pagenrlog{#1, \thepage,}}
%% were never documented, give conflicts with some additional packages

\newcommand{\GAP}{\textsf{GAP}}

%% nicer description environments, allows long labels
\usepackage{enumitem}
\setdescription{style=nextline}

%% depth of toc
\setcounter{tocdepth}{1}





%% command for ColorPrompt style examples
\newcommand{\gapprompt}[1]{\color{promptColor}{\bfseries #1}}
\newcommand{\gapbrkprompt}[1]{\color{brkpromptColor}{\bfseries #1}}
\newcommand{\gapinput}[1]{\color{gapinputColor}{#1}}


\begin{document}

\logpage{[ 0, 0, 0 ]}
\begin{titlepage}
\mbox{}\vfill

\begin{center}{\maintitlesize \textbf{ RAMP \mbox{}}}\\
\vfill

\hypersetup{pdftitle= RAMP }
\markright{\scriptsize \mbox{}\hfill  RAMP  \hfill\mbox{}}
{\Huge \textbf{ The Research Assistant for Maniplexes and Polytopes \mbox{}}}\\
\vfill

{\Huge  0.5 \mbox{}}\\[1cm]
{ 21 July 2021 \mbox{}}\\[1cm]
\mbox{}\\[2cm]
{\Large \textbf{ Gabe Cunningham\\
    \mbox{}}}\\
{\Large \textbf{ Mark Mixer\\
  \mbox{}}}\\
{\Large \textbf{ Gordon Williams\\
  \mbox{}}}\\
\hypersetup{pdfauthor= Gabe Cunningham\\
    ;  Mark Mixer\\
  ;  Gordon Williams\\
  }
\end{center}\vfill

\mbox{}\\
{\mbox{}\\
\small \noindent \textbf{ Gabe Cunningham\\
    }  Email: \href{mailto://gabriel.cunningham@umb.edu} {\texttt{gabriel.cunningham@umb.edu}}\\
  Homepage: \href{http://www.gabrielcunningham.com} {\texttt{http://www.gabrielcunningham.com}}\\
  Address: \begin{minipage}[t]{8cm}\noindent
 Gabe Cunningham\\
 Department of Mathematics\\
 University of Massachusetts Boston\\
 100 William T. Morrissey Blvd.\\
 Boston MA 02125\\
 \end{minipage}
}\\
{\mbox{}\\
\small \noindent \textbf{ Mark Mixer\\
  }  Email: \href{mailto://mixerm@wit.edu} {\texttt{mixerm@wit.edu}}}\\
{\mbox{}\\
\small \noindent \textbf{ Gordon Williams\\
  }  Email: \href{mailto://giwilliams@alaska.edu} {\texttt{giwilliams@alaska.edu}}}\\
\end{titlepage}

\newpage\setcounter{page}{2}
{\small 
\section*{Copyright}
\logpage{[ 0, 0, 1 ]}
 \index{License} {\copyright} 1997-2021 by Gabe Cunningham, Mark Mixer, and Gordon Williams

 \textsf{RAMP} package is free software; you can redistribute it and/or modify it under the
terms of the \href{http://www.fsf.org/licenses/gpl.html} {GNU General Public License} as published by the Free Software Foundation; either version 2 of the License,
or (at your option) any later version. \mbox{}}\\[1cm]
{\small 
\section*{Acknowledgements}
\logpage{[ 0, 0, 2 ]}
 We appreciate very much all past and future comments, suggestions and
contributions to this package and its documentation provided by \textsf{GAP} users and developers. \mbox{}}\\[1cm]
\newpage

\def\contentsname{Contents\logpage{[ 0, 0, 3 ]}}

\tableofcontents
\newpage

     
\chapter{\textcolor{Chapter }{Graphs for Maniplexes}}\label{Chapter_Graphs_for_Maniplexes}
\logpage{[ 1, 0, 0 ]}
\hyperdef{L}{X84DE212D847339C0}{}
{
  
\section{\textcolor{Chapter }{Graphs for maniplexes functions}}\label{Chapter_Graphs_for_Maniplexes_Section_Graphs_for_maniplexes_functions}
\logpage{[ 1, 1, 0 ]}
\hyperdef{L}{X7CD3FD9D86C12B44}{}
{
  

\subsection{\textcolor{Chapter }{DirectedGraphFromListOfEdges (for IsList,IsList)}}
\logpage{[ 1, 1, 1 ]}\nobreak
\hyperdef{L}{X7FB3733D81739F2C}{}
{\noindent\textcolor{FuncColor}{$\triangleright$\enspace\texttt{DirectedGraphFromListOfEdges({\mdseries\slshape list, list})\index{DirectedGraphFromListOfEdges@\texttt{DirectedGraphFromListOfEdges}!for IsList,IsList}
\label{DirectedGraphFromListOfEdges:for IsList,IsList}
}\hfill{\scriptsize (operation)}}\\
\textbf{\indent Returns:\ }
\texttt{IsGraph}. Note this returns a directed graph. 



 Given a list of vertices and a list of directed-edges (represented as ordered
pairs), this outputs the directed graph with the appropriate vertex and
directed-edge set. }

 Here we have a directed cycle on 3 vertices. 
\begin{Verbatim}[commandchars=!@|,fontsize=\small,frame=single,label=Example]
  !gapprompt@gap>| !gapinput@g:= DirectedGraphFromListOfEdges([1,2,3],[[1,2],[2,3],[3,1]]);|
  rec( adjacencies := [ [ 2 ], [ 3 ], [ 1 ] ], group := Group(()), 
   isGraph := true, names := [ 1, 2, 3 ], order := 3, 
   representatives := [ 1, 2, 3 ], schreierVector := [ -1, -2, -3 ] )
\end{Verbatim}
 

\subsection{\textcolor{Chapter }{GraphFromListOfEdges (for IsList,IsList)}}
\logpage{[ 1, 1, 2 ]}\nobreak
\hyperdef{L}{X82810A0A7CF30BB4}{}
{\noindent\textcolor{FuncColor}{$\triangleright$\enspace\texttt{GraphFromListOfEdges({\mdseries\slshape list, list})\index{GraphFromListOfEdges@\texttt{GraphFromListOfEdges}!for IsList,IsList}
\label{GraphFromListOfEdges:for IsList,IsList}
}\hfill{\scriptsize (operation)}}\\
\textbf{\indent Returns:\ }
\texttt{IsGraph}. Note this returns an undirected graph. 



 Given a list of vertices and a list of (directed) edges (represented as
ordered pairs), this outputs the simple underlying graph with the appropriate
vertex and directed-edge set. }

 Here we have a simple complete graph on 4 vertices. 
\begin{Verbatim}[commandchars=!@|,fontsize=\small,frame=single,label=Example]
  !gapprompt@gap>| !gapinput@g:= GraphFromListOfEdges([1,2,3,4],[[1,2],[2,3],[3,1], [1,4], [2,4], [3,4]]);|
  rec( 
   adjacencies := [ [ 2, 3, 4 ], [ 1, 3, 4 ], [ 1, 2, 4 ], [ 1, 2, 3 ] ],
   group := Group(()), isGraph := true, isSimple := true, 
   names := [ 1, 2, 3, 4 ], order := 4, representatives := [ 1, 2, 3, 4 ]
     , schreierVector := [ -1, -2, -3, -4 ] )
\end{Verbatim}
 

\subsection{\textcolor{Chapter }{UnlabeledFlagGraph (for IsGroup)}}
\logpage{[ 1, 1, 3 ]}\nobreak
\hyperdef{L}{X86BE61C88420C6F6}{}
{\noindent\textcolor{FuncColor}{$\triangleright$\enspace\texttt{UnlabeledFlagGraph({\mdseries\slshape group})\index{UnlabeledFlagGraph@\texttt{UnlabeledFlagGraph}!for IsGroup}
\label{UnlabeledFlagGraph:for IsGroup}
}\hfill{\scriptsize (operation)}}\\
\textbf{\indent Returns:\ }
\texttt{IsGraph}. Note this returns an undirected graph. 



 Given a group (assumed to be the connection group of a maniplex), this outputs
the simple underlying flag graph. }

 Here we build the flag graph for the cube from its connection group. 
\begin{Verbatim}[commandchars=!@|,fontsize=\small,frame=single,label=Example]
  !gapprompt@gap>| !gapinput@g:= UnlabeledFlagGraph(ConnectionGroup(Cube(3)));|
  rec( 
   adjacencies := [ [ 3, 11, 20 ], [ 7, 13, 18 ], [ 1, 4, 10 ], 
       [ 3, 25, 34 ], [ 26, 28, 35 ], [ 7, 13, 41 ], [ 2, 6, 8 ], 
       [ 7, 27, 32 ], [ 28, 33, 35 ], [ 3, 20, 45 ], [ 1, 14, 23 ], 
       [ 15, 17, 24 ], [ 2, 6, 31 ], [ 11, 25, 44 ], [ 12, 45, 47 ], 
       [ 18, 28, 40 ], [ 12, 19, 27 ], [ 2, 16, 21 ], [ 17, 22, 24 ], 
       [ 1, 10, 38 ], [ 18, 32, 40 ], [ 19, 41, 48 ], [ 11, 35, 44 ], 
       [ 12, 19, 34 ], [ 4, 14, 37 ], [ 5, 38, 42 ], [ 8, 17, 30 ], 
       [ 5, 9, 16 ], [ 39, 41, 48 ], [ 27, 32, 47 ], [ 13, 33, 39 ], 
       [ 8, 21, 30 ], [ 9, 31, 46 ], [ 4, 24, 37 ], [ 5, 9, 23 ], 
       [ 43, 45, 47 ], [ 25, 34, 48 ], [ 20, 26, 43 ], [ 29, 31, 46 ], 
       [ 16, 21, 42 ], [ 6, 22, 29 ], [ 26, 40, 43 ], [ 36, 38, 42 ], 
       [ 14, 23, 46 ], [ 10, 15, 36 ], [ 33, 39, 44 ], [ 15, 30, 36 ], 
       [ 22, 29, 37 ] ], group := Group(()), isGraph := true, 
   isSimple := true, names := [ 1 .. 48 ], order := 48, 
   representatives := [ 1, 2, 3, 4, 5, 6, 7, 8, 9, 10, 11, 12, 13, 14, 
       15, 16, 17, 18, 19, 20, 21, 22, 23, 24, 25, 26, 27, 28, 29, 30, 
       31, 32, 33, 34, 35, 36, 37, 38, 39, 40, 41, 42, 43, 44, 45, 46, 
       47, 48 ], 
   schreierVector := [ -1, -2, -3, -4, -5, -6, -7, -8, -9, -10, -11, 
       -12, -13, -14, -15, -16, -17, -18, -19, -20, -21, -22, -23, -24, 
       -25, -26, -27, -28, -29, -30, -31, -32, -33, -34, -35, -36, -37, 
       -38, -39, -40, -41, -42, -43, -44, -45, -46, -47, -48 ] )
\end{Verbatim}
 This also works with a maniplex input. Here we build the flag graph for the
cube. 
\begin{Verbatim}[commandchars=!@|,fontsize=\small,frame=single,label=Example]
  !gapprompt@gap>| !gapinput@g:= UnlabeledFlagGraph(Cube(3));|
\end{Verbatim}
 

\subsection{\textcolor{Chapter }{FlagGraphWithLabels (for IsGroup)}}
\logpage{[ 1, 1, 4 ]}\nobreak
\hyperdef{L}{X791AEF1C7AA7138C}{}
{\noindent\textcolor{FuncColor}{$\triangleright$\enspace\texttt{FlagGraphWithLabels({\mdseries\slshape group})\index{FlagGraphWithLabels@\texttt{FlagGraphWithLabels}!for IsGroup}
\label{FlagGraphWithLabels:for IsGroup}
}\hfill{\scriptsize (operation)}}\\
\textbf{\indent Returns:\ }
a triple [\texttt{IsGraph}, \texttt{IsList}, \texttt{IsList}]. 



 Given a group (assumed to be the connection group of a maniplex), this outputs
a triple [graph,list,list]. The graph is the unlabeled flag graph of the
connection group. The first list gives the undirected edges in the flag
graphs. The second list gives the labels for these edges. }

 Here we again build the flag graph for the cube from its connection group, but
this time keep track of labels of the edges. 
\begin{Verbatim}[commandchars=!@|,fontsize=\small,frame=single,label=Example]
  !gapprompt@gap>| !gapinput@g:= FlagGraphWithLabels(ConnectionGroup(Cube(3)));|
  [ rec( 
       adjacencies := [ [ 3, 11, 20 ], [ 7, 13, 18 ], [ 1, 4, 10 ], 
           [ 3, 25, 34 ], [ 26, 28, 35 ], [ 7, 13, 41 ], [ 2, 6, 8 ], 
           [ 7, 27, 32 ], [ 28, 33, 35 ], [ 3, 20, 45 ], [ 1, 14, 23 ], 
           [ 15, 17, 24 ], [ 2, 6, 31 ], [ 11, 25, 44 ], [ 12, 45, 47 ], 
           [ 18, 28, 40 ], [ 12, 19, 27 ], [ 2, 16, 21 ], 
           [ 17, 22, 24 ], [ 1, 10, 38 ], [ 18, 32, 40 ], 
           [ 19, 41, 48 ], [ 11, 35, 44 ], [ 12, 19, 34 ], 
           [ 4, 14, 37 ], [ 5, 38, 42 ], [ 8, 17, 30 ], [ 5, 9, 16 ], 
           [ 39, 41, 48 ], [ 27, 32, 47 ], [ 13, 33, 39 ], 
           [ 8, 21, 30 ], [ 9, 31, 46 ], [ 4, 24, 37 ], [ 5, 9, 23 ], 
           [ 43, 45, 47 ], [ 25, 34, 48 ], [ 20, 26, 43 ], 
           [ 29, 31, 46 ], [ 16, 21, 42 ], [ 6, 22, 29 ], 
           [ 26, 40, 43 ], [ 36, 38, 42 ], [ 14, 23, 46 ], 
           [ 10, 15, 36 ], [ 33, 39, 44 ], [ 15, 30, 36 ], 
           [ 22, 29, 37 ] ], group := Group(()), isGraph := true, 
       isSimple := true, names := [ 1 .. 48 ], order := 48, 
       representatives := [ 1, 2, 3, 4, 5, 6, 7, 8, 9, 10, 11, 12, 13, 
           14, 15, 16, 17, 18, 19, 20, 21, 22, 23, 24, 25, 26, 27, 28, 
           29, 30, 31, 32, 33, 34, 35, 36, 37, 38, 39, 40, 41, 42, 43, 
           44, 45, 46, 47, 48 ], 
       schreierVector := [ -1, -2, -3, -4, -5, -6, -7, -8, -9, -10, -11, 
           -12, -13, -14, -15, -16, -17, -18, -19, -20, -21, -22, -23, 
           -24, -25, -26, -27, -28, -29, -30, -31, -32, -33, -34, -35, 
           -36, -37, -38, -39, -40, -41, -42, -43, -44, -45, -46, -47, 
           -48 ] ), 
   [ [ 1, 3 ], [ 1, 11 ], [ 1, 20 ], [ 2, 7 ], [ 2, 13 ], [ 2, 18 ], 
       [ 3, 4 ], [ 3, 10 ], [ 4, 25 ], [ 4, 34 ], [ 5, 26 ], [ 5, 28 ], 
       [ 5, 35 ], [ 6, 7 ], [ 6, 13 ], [ 6, 41 ], [ 7, 8 ], [ 8, 27 ], 
       [ 8, 32 ], [ 9, 28 ], [ 9, 33 ], [ 9, 35 ], [ 10, 20 ], 
       [ 10, 45 ], [ 11, 14 ], [ 11, 23 ], [ 12, 15 ], [ 12, 17 ], 
       [ 12, 24 ], [ 13, 31 ], [ 14, 25 ], [ 14, 44 ], [ 15, 45 ], 
       [ 15, 47 ], [ 16, 18 ], [ 16, 28 ], [ 16, 40 ], [ 17, 19 ], 
       [ 17, 27 ], [ 18, 21 ], [ 19, 22 ], [ 19, 24 ], [ 20, 38 ], 
       [ 21, 32 ], [ 21, 40 ], [ 22, 41 ], [ 22, 48 ], [ 23, 35 ], 
       [ 23, 44 ], [ 24, 34 ], [ 25, 37 ], [ 26, 38 ], [ 26, 42 ], 
       [ 27, 30 ], [ 29, 39 ], [ 29, 41 ], [ 29, 48 ], [ 30, 32 ], 
       [ 30, 47 ], [ 31, 33 ], [ 31, 39 ], [ 33, 46 ], [ 34, 37 ], 
       [ 36, 43 ], [ 36, 45 ], [ 36, 47 ], [ 37, 48 ], [ 38, 43 ], 
       [ 39, 46 ], [ 40, 42 ], [ 42, 43 ], [ 44, 46 ] ], 
   [ 3, 2, 1, 3, 1, 2, 2, 1, 3, 1, 2, 3, 1, 1, 3, 2, 2, 1, 3, 1, 2, 3, 
       3, 2, 3, 1, 2, 3, 1, 2, 2, 1, 1, 3, 1, 2, 3, 1, 2, 3, 2, 3, 2, 2, 
       1, 1, 3, 2, 3, 2, 1, 1, 3, 3, 2, 3, 1, 1, 2, 1, 3, 3, 3, 2, 3, 1, 
       2, 3, 1, 2, 1, 2 ] ]
\end{Verbatim}
 This also works with a maniplex input. Here we build the flag graph for the
cube. 
\begin{Verbatim}[commandchars=!@|,fontsize=\small,frame=single,label=Example]
  !gapprompt@gap>| !gapinput@g:= FlagGraphWithLabels(Cube(3));|
\end{Verbatim}
 

\subsection{\textcolor{Chapter }{LayerGraph (for IsGroup, IsInt, IsInt)}}
\logpage{[ 1, 1, 5 ]}\nobreak
\hyperdef{L}{X809815D08250B57F}{}
{\noindent\textcolor{FuncColor}{$\triangleright$\enspace\texttt{LayerGraph({\mdseries\slshape [group, int, int]})\index{LayerGraph@\texttt{LayerGraph}!for IsGroup, IsInt, IsInt}
\label{LayerGraph:for IsGroup, IsInt, IsInt}
}\hfill{\scriptsize (operation)}}\\
\textbf{\indent Returns:\ }
\texttt{IsGraph}. Note this returns an undirected graph. 



 Given a group (assumed to be the connection group of a maniplex), and two
integers, this outputs the simple underlying graph given by incidences of
faces of those ranks. Note: There are no warnings yet to make sure that i,j
are bounded by the rank. }

 Here we build the graph given by the 6 faces and 12 edges of a cube from its
connection group. 
\begin{Verbatim}[commandchars=!@|,fontsize=\small,frame=single,label=Example]
  !gapprompt@gap>| !gapinput@g:= LayerGraph(ConnectionGroup(Cube(3)),2,1);|
  rec( 
   adjacencies := [ [ 7, 10, 12, 17 ], [ 8, 10, 15, 18 ], 
       [ 7, 9, 13, 14 ], [ 8, 11, 13, 16 ], [ 9, 12, 16, 18 ], 
       [ 11, 14, 15, 17 ], [ 1, 3 ], [ 2, 4 ], [ 3, 5 ], [ 1, 2 ], 
       [ 4, 6 ], [ 1, 5 ], [ 3, 4 ], [ 3, 6 ], [ 2, 6 ], [ 4, 5 ], 
       [ 1, 6 ], [ 2, 5 ] ], group := Group(()), isGraph := true, 
   isSimple := true, names := [ 1 .. 18 ], order := 18, 
   representatives := [ 1, 2, 3, 4, 5, 6, 7, 8, 9, 10, 11, 12, 13, 14, 
       15, 16, 17, 18 ], 
   schreierVector := [ -1, -2, -3, -4, -5, -6, -7, -8, -9, -10, -11, 
       -12, -13, -14, -15, -16, -17, -18 ] )
\end{Verbatim}
 This also works with a maniplex input. Here we build the graph given by the 6
faces and 12 edges of a cube. 
\begin{Verbatim}[commandchars=!@|,fontsize=\small,frame=single,label=Example]
  !gapprompt@gap>| !gapinput@g:= LayerGraph(Cube(3),2,1);;|
\end{Verbatim}
 

\subsection{\textcolor{Chapter }{Skeleton (for IsManiplex)}}
\logpage{[ 1, 1, 6 ]}\nobreak
\hyperdef{L}{X7BB443CD832DAC12}{}
{\noindent\textcolor{FuncColor}{$\triangleright$\enspace\texttt{Skeleton({\mdseries\slshape maniplex})\index{Skeleton@\texttt{Skeleton}!for IsManiplex}
\label{Skeleton:for IsManiplex}
}\hfill{\scriptsize (operation)}}\\
\textbf{\indent Returns:\ }
\texttt{IsGraph}. Note this returns an undirected graph. 



 Given a maniplex, this outputs the 0-1 skeleton. The vertices are the 0-faces,
and the edges are the 1-faces. }

 Here we build the skeleton of the dodecahedron. 
\begin{Verbatim}[commandchars=!@|,fontsize=\small,frame=single,label=Example]
  !gapprompt@gap>| !gapinput@g:= Skeleton(Dodecahedron());;|
\end{Verbatim}
 

\subsection{\textcolor{Chapter }{CoSkeleton (for IsManiplex)}}
\logpage{[ 1, 1, 7 ]}\nobreak
\hyperdef{L}{X85F3D43E7D1BA8E5}{}
{\noindent\textcolor{FuncColor}{$\triangleright$\enspace\texttt{CoSkeleton({\mdseries\slshape maniplex})\index{CoSkeleton@\texttt{CoSkeleton}!for IsManiplex}
\label{CoSkeleton:for IsManiplex}
}\hfill{\scriptsize (operation)}}\\
\textbf{\indent Returns:\ }
\texttt{IsGraph}. Note this returns an undirected graph. 



 Given a maniplex, this outputs the $(n-1)$-$(n-2)$ skeleton, i.e., the 0-1 skeleton of the dual. The vertices are the $(n-1)$-faces, and the edges are the $(n-2)$-faces. }

 Here we build the co-skeleton of the dodecahedron and verify that it is the
skeleton of the icosahedron. 
\begin{Verbatim}[commandchars=!@|,fontsize=\small,frame=single,label=Example]
  !gapprompt@gap>| !gapinput@g:=CoSkeleton(Dodecahedron());;|
  !gapprompt@gap>| !gapinput@h:=Skeleton(Icosahedron());;|
  !gapprompt@gap>| !gapinput@g=h;|
  true
\end{Verbatim}
 

\subsection{\textcolor{Chapter }{Hasse (for IsManiplex)}}
\logpage{[ 1, 1, 8 ]}\nobreak
\hyperdef{L}{X7A0FB13C7EF68CBB}{}
{\noindent\textcolor{FuncColor}{$\triangleright$\enspace\texttt{Hasse({\mdseries\slshape group})\index{Hasse@\texttt{Hasse}!for IsManiplex}
\label{Hasse:for IsManiplex}
}\hfill{\scriptsize (operation)}}\\
\textbf{\indent Returns:\ }
\texttt{IsGraph}. Note this returns a directed graph. 



 Given a group, assumed to be the connection group of a maniplex, this outputs
the Hasse Diagram as a directed graph. Note: The unique minimal and maximal
face are assumed. }

 Here we build the Hasse Diagram of a 3-simplex from its representation as a
maniplex. 
\begin{Verbatim}[commandchars=!@|,fontsize=\small,frame=single,label=Example]
  !gapprompt@gap>| !gapinput@Hasse(Simplex(3));|
  rec( 
   adjacencies := [ [  ], [ 1 ], [ 1 ], [ 1 ], [ 1 ], [ 2, 4 ], 
       [ 2, 3 ], [ 3, 5 ], [ 2, 5 ], [ 4, 5 ], [ 3, 4 ], [ 6, 9, 10 ], 
       [ 6, 7, 11 ], [ 8, 10, 11 ], [ 7, 8, 9 ], [ 12, 13, 14, 15 ] ], 
   group := Group(()), isGraph := true, names := [ 1 .. 16 ], 
   order := 16, 
   representatives := [ 1, 2, 3, 4, 5, 6, 7, 8, 9, 10, 11, 12, 13, 14, 
       15, 16 ], 
   schreierVector := [ -1, -2, -3, -4, -5, -6, -7, -8, -9, -10, -11, 
       -12, -13, -14, -15, -16 ] )
\end{Verbatim}
 

\subsection{\textcolor{Chapter }{QuotientByLabel (for IsObject,IsList, IsList, IsList)}}
\logpage{[ 1, 1, 9 ]}\nobreak
\hyperdef{L}{X860AACCF860A3F66}{}
{\noindent\textcolor{FuncColor}{$\triangleright$\enspace\texttt{QuotientByLabel({\mdseries\slshape object, list, list, list})\index{QuotientByLabel@\texttt{QuotientByLabel}!for IsObject,IsList, IsList, IsList}
\label{QuotientByLabel:for IsObject,IsList, IsList, IsList}
}\hfill{\scriptsize (operation)}}\\
\textbf{\indent Returns:\ }
\texttt{IsGraph}. Note this returns an undirected graph. 



 Given a graph, its edges, and its edge labels, and a sublist of labels, this
creates the underlying simple graph of the quotient identifying vertices
connected by labels not in the sublist. }

 Here we start with the flag graph of the 3-cube (with edge labels 1,2,3), and
identify any vertices not connected by edge by edges of label 1. We can then
check that this new graph is bipartite. 
\begin{Verbatim}[commandchars=!@|,fontsize=\small,frame=single,label=Example]
  !gapprompt@gap>| !gapinput@P:=Cube(3);;|
  !gapprompt@gap>| !gapinput@f:=FlagGraphWithLabels(P);;|
  !gapprompt@gap>| !gapinput@g:=f[1];;|
  !gapprompt@gap>| !gapinput@ed:=f[2];;|
  !gapprompt@gap>| !gapinput@lab:=f[3];  #Note This triple is to be replace by a single object.|
  [ 3, 2, 1, 3, 1, 2, 1, 2, 3, 2, 1, 3, 2, 1, 1, 3, 2, 2, 3, 1, 3, 1, 2, 3, 2, 1, 1, 2, 2, 3, 1, 3, 1, 2, 
    3, 1, 2, 1, 3, 2, 2, 1, 2, 2, 3, 1, 1, 3, 1, 3, 3, 2, 1, 2, 1, 3, 3, 1, 3, 2, 2, 2, 2, 3, 3, 1, 3, 1, 1, 3, 2, 3 ]
  !gapprompt@gap>| !gapinput@Q:=QuotientByLabel(g,ed,lab,[1]);|
  rec( adjacencies := [ [ 5, 6, 8 ], [ 3, 4, 7 ], [ 2, 6, 8 ], [ 2, 5, 8 ], [ 1, 4, 7 ], [ 1, 3, 7 ], [ 2, 5, 6 ], [ 1, 3, 4 ] ], group := Group(()), isGraph := true, 
   isSimple := true, names := [ 1 .. 8 ], order := 8, representatives := [ 1, 2, 3, 4, 5, 6, 7, 8 ], schreierVector := [ -1, -2, -3, -4, -5, -6, -7, -8 ] )
  !gapprompt@gap>| !gapinput@IsBipartite(Q);|
  true
\end{Verbatim}
 

\subsection{\textcolor{Chapter }{EdgeLabeledGraphFromEdges (for IsList, IsList,IsList)}}
\logpage{[ 1, 1, 10 ]}\nobreak
\hyperdef{L}{X84FE2C317CFA6AFD}{}
{\noindent\textcolor{FuncColor}{$\triangleright$\enspace\texttt{EdgeLabeledGraphFromEdges({\mdseries\slshape list, list, list})\index{EdgeLabeledGraphFromEdges@\texttt{EdgeLabeledGraphFromEdges}!for IsList, IsList,IsList}
\label{EdgeLabeledGraphFromEdges:for IsList, IsList,IsList}
}\hfill{\scriptsize (operation)}}\\
\textbf{\indent Returns:\ }
\texttt{IsEdgeLabeledGraph}. 



 Given a list of vertices, a list of edges, and a list of edge labels, this
represents the edge labeled (multi)-graph with those parameters. Semi-edges
are represented by a singleton in the edge list. Loops are represented by
edges [i,i] }

 Here we have an edge labeled cycle graph with 6 vertices and edges alternating
in labels 0,1. 
\begin{Verbatim}[commandchars=!@|,fontsize=\small,frame=single,label=Example]
  V:=[1..6];;
  Edges:=[[1,2],[2,3],[3,4],[4,5],[5,6],[6,1]];;
  L:=[0,1,0,1,0,1];;
  gamma:=EdgeLabeledGraphFromEdges(V,Edges,L);
\end{Verbatim}
 

\subsection{\textcolor{Chapter }{FlagGraph (for IsGroup)}}
\logpage{[ 1, 1, 11 ]}\nobreak
\hyperdef{L}{X7A958E7C821DD8F8}{}
{\noindent\textcolor{FuncColor}{$\triangleright$\enspace\texttt{FlagGraph({\mdseries\slshape group})\index{FlagGraph@\texttt{FlagGraph}!for IsGroup}
\label{FlagGraph:for IsGroup}
}\hfill{\scriptsize (operation)}}\\
\textbf{\indent Returns:\ }
\texttt{IsEdgeLabeledGraph}. 



 Given group, assumed to be a connection group, output the labeled flag graph.
The input could also be a maniplex, then the connection group is calculated. }

 Here we have the flag graph of the 3-simplex from its connection group. 
\begin{Verbatim}[commandchars=!@|,fontsize=\small,frame=single,label=Example]
  C:=ConnectionGroup(Simplex(3));;
  gamma:=FlagGraph(C);
\end{Verbatim}
 

\subsection{\textcolor{Chapter }{UnlabeledSimpleGraph (for IsEdgeLabeledGraph)}}
\logpage{[ 1, 1, 12 ]}\nobreak
\hyperdef{L}{X82A8C3AD871EC6D7}{}
{\noindent\textcolor{FuncColor}{$\triangleright$\enspace\texttt{UnlabeledSimpleGraph({\mdseries\slshape edge-labeled-graph})\index{UnlabeledSimpleGraph@\texttt{UnlabeledSimpleGraph}!for IsEdgeLabeledGraph}
\label{UnlabeledSimpleGraph:for IsEdgeLabeledGraph}
}\hfill{\scriptsize (operation)}}\\
\textbf{\indent Returns:\ }
\texttt{IsGraph}. 



 Given an edge labeled (multi) graph, it returns the underlying simple graph,
with semi-edges, loops, and muliple-edges removed. }

 Here we have underlying simple graph for the flag graph of the cube. 
\begin{Verbatim}[commandchars=!@|,fontsize=\small,frame=single,label=Example]
  gamma:=UnlabeledSimpleGraph(FlagGraph(Cube(3)));
\end{Verbatim}
 

\subsection{\textcolor{Chapter }{EdgeLabelPreservingAutomorphismGroup (for IsEdgeLabeledGraph)}}
\logpage{[ 1, 1, 13 ]}\nobreak
\hyperdef{L}{X84263C6C81AB458C}{}
{\noindent\textcolor{FuncColor}{$\triangleright$\enspace\texttt{EdgeLabelPreservingAutomorphismGroup({\mdseries\slshape edge-labeled-graph})\index{EdgeLabelPreservingAutomorphismGroup@\texttt{Edge}\-\texttt{Label}\-\texttt{Preserving}\-\texttt{Automorphism}\-\texttt{Group}!for IsEdgeLabeledGraph}
\label{EdgeLabelPreservingAutomorphismGroup:for IsEdgeLabeledGraph}
}\hfill{\scriptsize (operation)}}\\
\textbf{\indent Returns:\ }
\texttt{IsGroup}. 



 Given an edge labeled (multi) graph, it returns automorphism group (preserving
the labels). Note, for now the labels are assumed to be [1..n]. Note This
tends to be very slow. I would like to look for a way to go back and forth
between flag automorphisms and poset automorphisms, as the latter are much
faster to compute. }

 Here we have the automorphism group of the flag graph of the cube. 
\begin{Verbatim}[commandchars=!@|,fontsize=\small,frame=single,label=Example]
  g:=EdgeLabelPreservingAutomorphismGroup(FlagGraph(Cube(3)));;
  Size(g);
\end{Verbatim}
 

\subsection{\textcolor{Chapter }{Simple (for IsEdgeLabeledGraph)}}
\logpage{[ 1, 1, 14 ]}\nobreak
\hyperdef{L}{X7D7BF5608431FED8}{}
{\noindent\textcolor{FuncColor}{$\triangleright$\enspace\texttt{Simple({\mdseries\slshape edge-labeled-graph})\index{Simple@\texttt{Simple}!for IsEdgeLabeledGraph}
\label{Simple:for IsEdgeLabeledGraph}
}\hfill{\scriptsize (operation)}}\\
\textbf{\indent Returns:\ }
\texttt{IsEdgeLabeledGraph }. 



 Given an edge labeled (multi) graph, it returns another edge labeled graph
where semi-edges, loops, and multiple edges are removed. Note only the "first"
edge label is retained if there are multiple edges. }

 

\subsection{\textcolor{Chapter }{ConnectedComponents (for IsEdgeLabeledGraph, IsList)}}
\logpage{[ 1, 1, 15 ]}\nobreak
\hyperdef{L}{X8159C96B8757873F}{}
{\noindent\textcolor{FuncColor}{$\triangleright$\enspace\texttt{ConnectedComponents({\mdseries\slshape edge-labeled-graph})\index{ConnectedComponents@\texttt{ConnectedComponents}!for IsEdgeLabeledGraph, IsList}
\label{ConnectedComponents:for IsEdgeLabeledGraph, IsList}
}\hfill{\scriptsize (operation)}}\\
\textbf{\indent Returns:\ }
\texttt{IsGraph}. 



 Given an edge labeled (multi) graph and a list of labels, it returns connected
components of the graph not using edges in the list of labels. Note if the
second argument is not used, it is assumed to be an empty list, and the
connected components of the original graph are returned. }

 Here we see that each connected component of the flag graph of the cube (which
has labels 1,2,3) where edges of label 2 are removed, is a 4 cycle. 
\begin{Verbatim}[commandchars=!@|,fontsize=\small,frame=single,label=Example]
  gamma:=ConnectedComponents(FlagGraph(Cube(3)),[2]);
\end{Verbatim}
 

\subsection{\textcolor{Chapter }{PRGraph (for IsGroup)}}
\logpage{[ 1, 1, 16 ]}\nobreak
\hyperdef{L}{X7A53CE8B820FD6A1}{}
{\noindent\textcolor{FuncColor}{$\triangleright$\enspace\texttt{PRGraph({\mdseries\slshape group})\index{PRGraph@\texttt{PRGraph}!for IsGroup}
\label{PRGraph:for IsGroup}
}\hfill{\scriptsize (operation)}}\\
\textbf{\indent Returns:\ }
\texttt{IsEdgeLabeledGraph }. 



 Given a group, it returns the permutation representation graph for that group.
When the group is a string C-group this is also called a CPR graph. The labels
of the edges are [1...r] where r is the number of generators of the group. }

 Here we see the CPR graph of the automorphism group of a cube (acting on its 8
vertices). 
\begin{Verbatim}[commandchars=!@|,fontsize=\small,frame=single,label=Example]
  G:=AutomorphismGroup(Cube(3));
  H:=Group(G.2,G.3);
  phi:=FactorCosetAction(G,H);
  G2:=Range(phi);
  gamma:=PRGraph(G2);
\end{Verbatim}
 

\subsection{\textcolor{Chapter }{CPRGraphFromGroups (for IsGroup,IsGroup)}}
\logpage{[ 1, 1, 17 ]}\nobreak
\hyperdef{L}{X85BBB22F8196DC0B}{}
{\noindent\textcolor{FuncColor}{$\triangleright$\enspace\texttt{CPRGraphFromGroups({\mdseries\slshape group, subgroup})\index{CPRGraphFromGroups@\texttt{CPRGraphFromGroups}!for IsGroup,IsGroup}
\label{CPRGraphFromGroups:for IsGroup,IsGroup}
}\hfill{\scriptsize (operation)}}\\
\textbf{\indent Returns:\ }
\texttt{IsEdgeLabeledGraph}. 



 Given a group and a subgroup. Returns the graph of the action of the first
group on cosets of the subgroup. }

 }

 }

   
\chapter{\textcolor{Chapter }{Basics}}\label{Chapter_Basics}
\logpage{[ 2, 0, 0 ]}
\hyperdef{L}{X868F7BAB7AC2EEBC}{}
{
  
\section{\textcolor{Chapter }{Constructors}}\label{Chapter_Basics_Section_Constructors}
\logpage{[ 2, 1, 0 ]}
\hyperdef{L}{X86EC0F0A78ECBC10}{}
{
  
\subsection{\textcolor{Chapter }{UniversalSggi}}\label{UniversalSggi}
\logpage{[ 2, 1, 1 ]}
\hyperdef{L}{X7AE843CE7878951F}{}
{
\noindent\textcolor{FuncColor}{$\triangleright$\enspace\texttt{UniversalSggi({\mdseries\slshape n})\index{UniversalSggi@\texttt{UniversalSggi}!for IsInt}
\label{UniversalSggi:for IsInt}
}\hfill{\scriptsize (operation)}}\\
\noindent\textcolor{FuncColor}{$\triangleright$\enspace\texttt{UniversalSggi({\mdseries\slshape sym})\index{UniversalSggi@\texttt{UniversalSggi}!for IsList}
\label{UniversalSggi:for IsList}
}\hfill{\scriptsize (operation)}}\\


 In the first form, returns the universal Coxeter Group of rank n. In the
second form, returns the Coxeter Group with Schlafli symbol sym. }

 

\subsection{\textcolor{Chapter }{Sggi (for IsList, IsList)}}
\logpage{[ 2, 1, 2 ]}\nobreak
\hyperdef{L}{X841E3F0F84465DCA}{}
{\noindent\textcolor{FuncColor}{$\triangleright$\enspace\texttt{Sggi({\mdseries\slshape symbol, relations})\index{Sggi@\texttt{Sggi}!for IsList, IsList}
\label{Sggi:for IsList, IsList}
}\hfill{\scriptsize (operation)}}\\


 Returns the sggi defined by the given Schlafli symbol and with the given
relations. The relations can be given by a list of Tietze words or as a string
of relators or relations that involve r0 etc. }

 

\subsection{\textcolor{Chapter }{ReflexibleManiplex (for IsGroup)}}
\logpage{[ 2, 1, 3 ]}\nobreak
\hyperdef{L}{X8494D0598240BD67}{}
{\noindent\textcolor{FuncColor}{$\triangleright$\enspace\texttt{ReflexibleManiplex({\mdseries\slshape g})\index{ReflexibleManiplex@\texttt{ReflexibleManiplex}!for IsGroup}
\label{ReflexibleManiplex:for IsGroup}
}\hfill{\scriptsize (operation)}}\\


 Given a group g (which should be a string C-group), returns the reflexible
maniplex with that automorphism group, where the privileged generators are
those returned by GeneratorsOfGroup(g). }

 

\subsection{\textcolor{Chapter }{ReflexibleManiplex (for IsList)}}
\logpage{[ 2, 1, 4 ]}\nobreak
\hyperdef{L}{X872407AC7E91F409}{}
{\noindent\textcolor{FuncColor}{$\triangleright$\enspace\texttt{ReflexibleManiplex({\mdseries\slshape sym})\index{ReflexibleManiplex@\texttt{ReflexibleManiplex}!for IsList}
\label{ReflexibleManiplex:for IsList}
}\hfill{\scriptsize (operation)}}\\


 Returns the universal reflexible maniplex (in fact, regular polytope) with
Schlafli symbol \mbox{\texttt{\mdseries\slshape sym}}. }

 

\subsection{\textcolor{Chapter }{ReflexibleManiplex (for IsList, IsList)}}
\logpage{[ 2, 1, 5 ]}\nobreak
\hyperdef{L}{X7A50E3AA7F5A9E55}{}
{\noindent\textcolor{FuncColor}{$\triangleright$\enspace\texttt{ReflexibleManiplex({\mdseries\slshape symbol, relations})\index{ReflexibleManiplex@\texttt{ReflexibleManiplex}!for IsList, IsList}
\label{ReflexibleManiplex:for IsList, IsList}
}\hfill{\scriptsize (operation)}}\\


 Returns the reflexible maniplex with the given Schlafli symbol and with the
given relations. The formatting of the relations is quite flexible. All of the
following work: 
\begin{Verbatim}[commandchars=!@|,fontsize=\small,frame=single,label=Example]
  q := ReflexibleManiplex([4,3,4], "(r0 r1 r2)^3, (r1 r2 r3)^3");
  q := ReflexibleManiplex([4,3,4], "(r0 r1 r2)^3 = (r1 r2 r3)^3 = 1");
  p := ReflexibleManiplex([infinity], "r0 r1 r0 = r1 r0 r1");
\end{Verbatim}
 If the option set{\textunderscore}schlafli is set, then we set the Schlafli
symbol to the one given. This may not be the correct Schlafli symbol, since
the relations may cause a collapse, so this should only be used if you know
that the Schlafli symbol is correct. }

 

\subsection{\textcolor{Chapter }{ReflexibleManiplex (for IsString)}}
\logpage{[ 2, 1, 6 ]}\nobreak
\hyperdef{L}{X8209E4587808FE97}{}
{\noindent\textcolor{FuncColor}{$\triangleright$\enspace\texttt{ReflexibleManiplex({\mdseries\slshape name})\index{ReflexibleManiplex@\texttt{ReflexibleManiplex}!for IsString}
\label{ReflexibleManiplex:for IsString}
}\hfill{\scriptsize (operation)}}\\


 Returns the regular polytope with the given symbolic name. Examples:
ReflexibleManiplex("\texttt{\symbol{123}}3,3,3\texttt{\symbol{125}}");
ReflexibleManiplex("\texttt{\symbol{123}}4,3\texttt{\symbol{125}}{\textunderscore}3");
If the option set{\textunderscore}schlafli is set, then we set the Schlafli
symbol to the one given. This may not be the correct Schlafli symbol, since
the relations may cause a collapse, so this should only be used if you know
that the Schlafli symbol is correct. }

 

\subsection{\textcolor{Chapter }{Maniplex (for IsGroup)}}
\logpage{[ 2, 1, 7 ]}\nobreak
\hyperdef{L}{X84CE00BB87E71848}{}
{\noindent\textcolor{FuncColor}{$\triangleright$\enspace\texttt{Maniplex({\mdseries\slshape G})\index{Maniplex@\texttt{Maniplex}!for IsGroup}
\label{Maniplex:for IsGroup}
}\hfill{\scriptsize (operation)}}\\


 Returns a maniplex with connection group \mbox{\texttt{\mdseries\slshape G}}, where \mbox{\texttt{\mdseries\slshape G}} is assumed to be a permutation group on the flags. }

 

\subsection{\textcolor{Chapter }{Maniplex (for IsReflexibleManiplex, IsGroup)}}
\logpage{[ 2, 1, 8 ]}\nobreak
\hyperdef{L}{X80B9127679FEBEAE}{}
{\noindent\textcolor{FuncColor}{$\triangleright$\enspace\texttt{Maniplex({\mdseries\slshape M, G})\index{Maniplex@\texttt{Maniplex}!for IsReflexibleManiplex, IsGroup}
\label{Maniplex:for IsReflexibleManiplex, IsGroup}
}\hfill{\scriptsize (operation)}}\\


 Given a reflexible maniplex \mbox{\texttt{\mdseries\slshape M}} and a subgroup \mbox{\texttt{\mdseries\slshape G}} of the flag-stabilizer of the base flag of \mbox{\texttt{\mdseries\slshape M}}, returns the maniplex \mbox{\texttt{\mdseries\slshape M/G}}. }

 

\subsection{\textcolor{Chapter }{Maniplex (for IsFunction, IsList)}}
\logpage{[ 2, 1, 9 ]}\nobreak
\hyperdef{L}{X81DA2F5185E38D67}{}
{\noindent\textcolor{FuncColor}{$\triangleright$\enspace\texttt{Maniplex({\mdseries\slshape F, inputs})\index{Maniplex@\texttt{Maniplex}!for IsFunction, IsList}
\label{Maniplex:for IsFunction, IsList}
}\hfill{\scriptsize (operation)}}\\


 Constructs a formal polytope, represented by an operation \mbox{\texttt{\mdseries\slshape F}} and a list of arguments \mbox{\texttt{\mdseries\slshape inputs}}. }

 

\subsection{\textcolor{Chapter }{Maniplex (for IsPoset)}}
\logpage{[ 2, 1, 10 ]}\nobreak
\hyperdef{L}{X864C043380FB58D4}{}
{\noindent\textcolor{FuncColor}{$\triangleright$\enspace\texttt{Maniplex({\mdseries\slshape P})\index{Maniplex@\texttt{Maniplex}!for IsPoset}
\label{Maniplex:for IsPoset}
}\hfill{\scriptsize (operation)}}\\


 Returns a maniplex with poset \mbox{\texttt{\mdseries\slshape P}}. }

 

\subsection{\textcolor{Chapter }{IsPolytopal (for IsManiplex)}}
\logpage{[ 2, 1, 11 ]}\nobreak
\hyperdef{L}{X79DD94037E0FACD4}{}
{\noindent\textcolor{FuncColor}{$\triangleright$\enspace\texttt{IsPolytopal({\mdseries\slshape M})\index{IsPolytopal@\texttt{IsPolytopal}!for IsManiplex}
\label{IsPolytopal:for IsManiplex}
}\hfill{\scriptsize (property)}}\\
\textbf{\indent Returns:\ }
\texttt{true} or \texttt{false} 



 Returns whether the maniplex \mbox{\texttt{\mdseries\slshape M}} is a polytope. Currently only implemented for reflexible maniplexes. }

 }

 }

   
\chapter{\textcolor{Chapter }{Combinatorics and Structure}}\label{Chapter_Combinatorics_and_Structure}
\logpage{[ 3, 0, 0 ]}
\hyperdef{L}{X8153E7658330D710}{}
{
  
\section{\textcolor{Chapter }{Faces}}\label{Chapter_Combinatorics_and_Structure_Section_Faces}
\logpage{[ 3, 1, 0 ]}
\hyperdef{L}{X872AD1E785C7EB03}{}
{
  

\subsection{\textcolor{Chapter }{NumberOfIFaces (for IsManiplex, IsInt)}}
\logpage{[ 3, 1, 1 ]}\nobreak
\hyperdef{L}{X86085A967E396035}{}
{\noindent\textcolor{FuncColor}{$\triangleright$\enspace\texttt{NumberOfIFaces({\mdseries\slshape M, i})\index{NumberOfIFaces@\texttt{NumberOfIFaces}!for IsManiplex, IsInt}
\label{NumberOfIFaces:for IsManiplex, IsInt}
}\hfill{\scriptsize (operation)}}\\


 Returns The number of \mbox{\texttt{\mdseries\slshape i}}-faces of \mbox{\texttt{\mdseries\slshape M}}. }

 

\subsection{\textcolor{Chapter }{NumberOfVertices (for IsManiplex)}}
\logpage{[ 3, 1, 2 ]}\nobreak
\hyperdef{L}{X7C126944873D2461}{}
{\noindent\textcolor{FuncColor}{$\triangleright$\enspace\texttt{NumberOfVertices({\mdseries\slshape M})\index{NumberOfVertices@\texttt{NumberOfVertices}!for IsManiplex}
\label{NumberOfVertices:for IsManiplex}
}\hfill{\scriptsize (attribute)}}\\


 Returns the number of vertices of \mbox{\texttt{\mdseries\slshape M}}. }

 

\subsection{\textcolor{Chapter }{NumberOfEdges (for IsManiplex)}}
\logpage{[ 3, 1, 3 ]}\nobreak
\hyperdef{L}{X86CDC9B38792B038}{}
{\noindent\textcolor{FuncColor}{$\triangleright$\enspace\texttt{NumberOfEdges({\mdseries\slshape M})\index{NumberOfEdges@\texttt{NumberOfEdges}!for IsManiplex}
\label{NumberOfEdges:for IsManiplex}
}\hfill{\scriptsize (attribute)}}\\


 Returns the number of edges of \mbox{\texttt{\mdseries\slshape M}}. }

 

\subsection{\textcolor{Chapter }{NumberOfFacets (for IsManiplex)}}
\logpage{[ 3, 1, 4 ]}\nobreak
\hyperdef{L}{X7DE9F1057B309554}{}
{\noindent\textcolor{FuncColor}{$\triangleright$\enspace\texttt{NumberOfFacets({\mdseries\slshape M})\index{NumberOfFacets@\texttt{NumberOfFacets}!for IsManiplex}
\label{NumberOfFacets:for IsManiplex}
}\hfill{\scriptsize (attribute)}}\\


 Returns the number of facets of \mbox{\texttt{\mdseries\slshape M}}. }

 

\subsection{\textcolor{Chapter }{NumberOfRidges (for IsManiplex)}}
\logpage{[ 3, 1, 5 ]}\nobreak
\hyperdef{L}{X7C5446247FD63F28}{}
{\noindent\textcolor{FuncColor}{$\triangleright$\enspace\texttt{NumberOfRidges({\mdseries\slshape M})\index{NumberOfRidges@\texttt{NumberOfRidges}!for IsManiplex}
\label{NumberOfRidges:for IsManiplex}
}\hfill{\scriptsize (attribute)}}\\


 Returns the number of ridges ((n-2)-faces) of \mbox{\texttt{\mdseries\slshape M}}. }

 

\subsection{\textcolor{Chapter }{Fvector (for IsManiplex)}}
\logpage{[ 3, 1, 6 ]}\nobreak
\hyperdef{L}{X816A47B879737629}{}
{\noindent\textcolor{FuncColor}{$\triangleright$\enspace\texttt{Fvector({\mdseries\slshape M})\index{Fvector@\texttt{Fvector}!for IsManiplex}
\label{Fvector:for IsManiplex}
}\hfill{\scriptsize (attribute)}}\\


 Returns the f-vector of \mbox{\texttt{\mdseries\slshape M}}. }

 

\subsection{\textcolor{Chapter }{Section (for IsManiplex, IsInt, IsInt)}}
\logpage{[ 3, 1, 7 ]}\nobreak
\hyperdef{L}{X7CDDC432804AF30B}{}
{\noindent\textcolor{FuncColor}{$\triangleright$\enspace\texttt{Section({\mdseries\slshape M, j, i})\index{Section@\texttt{Section}!for IsManiplex, IsInt, IsInt}
\label{Section:for IsManiplex, IsInt, IsInt}
}\hfill{\scriptsize (operation)}}\\


 Returns the section F{\textunderscore}j / F{\textunderscore}i, where
F{\textunderscore}j is the j-face of the base flag of \mbox{\texttt{\mdseries\slshape M}} and F{\textunderscore}i is the i-face of the base flag. }

 

\subsection{\textcolor{Chapter }{Section (for IsManiplex, IsInt, IsInt, IsInt)}}
\logpage{[ 3, 1, 8 ]}\nobreak
\hyperdef{L}{X7F11B08F7D74DBF1}{}
{\noindent\textcolor{FuncColor}{$\triangleright$\enspace\texttt{Section({\mdseries\slshape M, j, i, k})\index{Section@\texttt{Section}!for IsManiplex, IsInt, IsInt, IsInt}
\label{Section:for IsManiplex, IsInt, IsInt, IsInt}
}\hfill{\scriptsize (operation)}}\\


 Returns the section F{\textunderscore}j / F{\textunderscore}i, where
F{\textunderscore}j is the j-face of flag number \mbox{\texttt{\mdseries\slshape k}} of \mbox{\texttt{\mdseries\slshape M}} and F{\textunderscore}i is the i-face of the same flag. }

 

\subsection{\textcolor{Chapter }{Sections (for IsManiplex, IsInt, IsInt)}}
\logpage{[ 3, 1, 9 ]}\nobreak
\hyperdef{L}{X80D7EF0E7A838F19}{}
{\noindent\textcolor{FuncColor}{$\triangleright$\enspace\texttt{Sections({\mdseries\slshape M, j, i})\index{Sections@\texttt{Sections}!for IsManiplex, IsInt, IsInt}
\label{Sections:for IsManiplex, IsInt, IsInt}
}\hfill{\scriptsize (operation)}}\\


 Returns all sections of type F{\textunderscore}j / F{\textunderscore}i, where
F{\textunderscore}j is a j-face and F{\textunderscore}i is an incident i-face. }

 

\subsection{\textcolor{Chapter }{Facets (for IsManiplex)}}
\logpage{[ 3, 1, 10 ]}\nobreak
\hyperdef{L}{X78D48A797A8A981C}{}
{\noindent\textcolor{FuncColor}{$\triangleright$\enspace\texttt{Facets({\mdseries\slshape M})\index{Facets@\texttt{Facets}!for IsManiplex}
\label{Facets:for IsManiplex}
}\hfill{\scriptsize (attribute)}}\\


 Returns the facet-types of \mbox{\texttt{\mdseries\slshape M}} (i.e. the maniplexes corresponding to the facets). }

 

\subsection{\textcolor{Chapter }{Facet (for IsManiplex, IsInt)}}
\logpage{[ 3, 1, 11 ]}\nobreak
\hyperdef{L}{X844C665A82993B2D}{}
{\noindent\textcolor{FuncColor}{$\triangleright$\enspace\texttt{Facet({\mdseries\slshape M, k})\index{Facet@\texttt{Facet}!for IsManiplex, IsInt}
\label{Facet:for IsManiplex, IsInt}
}\hfill{\scriptsize (operation)}}\\


 Returns the facet of \mbox{\texttt{\mdseries\slshape M}} that contains the flag number \mbox{\texttt{\mdseries\slshape k}} (that is, the maniplex corresponding to the facet). }

 

\subsection{\textcolor{Chapter }{Facet (for IsManiplex)}}
\logpage{[ 3, 1, 12 ]}\nobreak
\hyperdef{L}{X86F8A85C83C04F3B}{}
{\noindent\textcolor{FuncColor}{$\triangleright$\enspace\texttt{Facet({\mdseries\slshape M})\index{Facet@\texttt{Facet}!for IsManiplex}
\label{Facet:for IsManiplex}
}\hfill{\scriptsize (attribute)}}\\


 Returns the facet of \mbox{\texttt{\mdseries\slshape M}} that contains flag number 1 (that is, the maniplex corresponding to the
facet). }

 

\subsection{\textcolor{Chapter }{VertexFigures (for IsManiplex)}}
\logpage{[ 3, 1, 13 ]}\nobreak
\hyperdef{L}{X7FB4AE2779695BB4}{}
{\noindent\textcolor{FuncColor}{$\triangleright$\enspace\texttt{VertexFigures({\mdseries\slshape M})\index{VertexFigures@\texttt{VertexFigures}!for IsManiplex}
\label{VertexFigures:for IsManiplex}
}\hfill{\scriptsize (attribute)}}\\


 Returns the types of vertex-figures of \mbox{\texttt{\mdseries\slshape M}} (i.e. the maniplexes corresponding to the vertex-figures). }

 

\subsection{\textcolor{Chapter }{VertexFigure (for IsManiplex, IsInt)}}
\logpage{[ 3, 1, 14 ]}\nobreak
\hyperdef{L}{X78A494BA79FE90C5}{}
{\noindent\textcolor{FuncColor}{$\triangleright$\enspace\texttt{VertexFigure({\mdseries\slshape M, k})\index{VertexFigure@\texttt{VertexFigure}!for IsManiplex, IsInt}
\label{VertexFigure:for IsManiplex, IsInt}
}\hfill{\scriptsize (operation)}}\\


 Returns the vertex-figures of \mbox{\texttt{\mdseries\slshape M}} that contains flag number \mbox{\texttt{\mdseries\slshape k}}. }

 

\subsection{\textcolor{Chapter }{VertexFigure (for IsManiplex)}}
\logpage{[ 3, 1, 15 ]}\nobreak
\hyperdef{L}{X7B18EFD87B76AF0E}{}
{\noindent\textcolor{FuncColor}{$\triangleright$\enspace\texttt{VertexFigure({\mdseries\slshape M})\index{VertexFigure@\texttt{VertexFigure}!for IsManiplex}
\label{VertexFigure:for IsManiplex}
}\hfill{\scriptsize (attribute)}}\\


 Returns the vertex-figures of \mbox{\texttt{\mdseries\slshape M}} that contains the base flag. }

 }

 
\section{\textcolor{Chapter }{Flatness}}\label{Chapter_Combinatorics_and_Structure_Section_Flatness}
\logpage{[ 3, 2, 0 ]}
\hyperdef{L}{X877E841E7FBCEAAE}{}
{
  
\subsection{\textcolor{Chapter }{IsFlat}}\label{IsFlat}
\logpage{[ 3, 2, 1 ]}
\hyperdef{L}{X7D54E1CF809FEB0E}{}
{
\noindent\textcolor{FuncColor}{$\triangleright$\enspace\texttt{IsFlat({\mdseries\slshape M})\index{IsFlat@\texttt{IsFlat}!for IsManiplex}
\label{IsFlat:for IsManiplex}
}\hfill{\scriptsize (property)}}\\
\noindent\textcolor{FuncColor}{$\triangleright$\enspace\texttt{IsFlat({\mdseries\slshape M, i, j})\index{IsFlat@\texttt{IsFlat}!for IsManiplex, IsInt, IsInt}
\label{IsFlat:for IsManiplex, IsInt, IsInt}
}\hfill{\scriptsize (operation)}}\\
\textbf{\indent Returns:\ }
\texttt{true} or \texttt{false} 



 In the first form, returns true if every vertex of the maniplex \mbox{\texttt{\mdseries\slshape M}} is incident to every facet. In the second form, returns true if every i-face
of the maniplex \mbox{\texttt{\mdseries\slshape M}} is incident to every j-face. }

 }

 
\section{\textcolor{Chapter }{Schlafli symbol}}\label{Chapter_Combinatorics_and_Structure_Section_Schlafli_symbol}
\logpage{[ 3, 3, 0 ]}
\hyperdef{L}{X8052D0A68676A82C}{}
{
  

\subsection{\textcolor{Chapter }{SchlafliSymbol (for IsManiplex)}}
\logpage{[ 3, 3, 1 ]}\nobreak
\hyperdef{L}{X7FA0B161838A442A}{}
{\noindent\textcolor{FuncColor}{$\triangleright$\enspace\texttt{SchlafliSymbol({\mdseries\slshape M})\index{SchlafliSymbol@\texttt{SchlafliSymbol}!for IsManiplex}
\label{SchlafliSymbol:for IsManiplex}
}\hfill{\scriptsize (attribute)}}\\


 Returns the Schlafli symbol of the maniplex \mbox{\texttt{\mdseries\slshape M}}. Each entry is either an integer or a set of integers, where entry number i
shows the polygons that we obtain as sections of (i+1)-faces over (i-2)-faces. }

 

\subsection{\textcolor{Chapter }{PseudoSchlafliSymbol (for IsManiplex)}}
\logpage{[ 3, 3, 2 ]}\nobreak
\hyperdef{L}{X78447BE8877668EA}{}
{\noindent\textcolor{FuncColor}{$\triangleright$\enspace\texttt{PseudoSchlafliSymbol({\mdseries\slshape M})\index{PseudoSchlafliSymbol@\texttt{PseudoSchlafliSymbol}!for IsManiplex}
\label{PseudoSchlafliSymbol:for IsManiplex}
}\hfill{\scriptsize (attribute)}}\\


 Sometimes when we make a maniplex, we know that the Schlafli symbol must be a
quotient of some symbol. This most frequently happens because we start with a
maniplex with a given Schlafli symbol and then take a quotient of it. In this
case, we store the given Schlafli symbol and call it a \emph{pseudo-Schlafli symbol} of \mbox{\texttt{\mdseries\slshape M}}. Note that whenever we compute the actual Schlafli symbol of \mbox{\texttt{\mdseries\slshape M}}, we update the pseudo-Schlafli symbol to match. }

 

\subsection{\textcolor{Chapter }{IsEquivelar (for IsManiplex)}}
\logpage{[ 3, 3, 3 ]}\nobreak
\hyperdef{L}{X7E9CCE9D794DC285}{}
{\noindent\textcolor{FuncColor}{$\triangleright$\enspace\texttt{IsEquivelar({\mdseries\slshape M})\index{IsEquivelar@\texttt{IsEquivelar}!for IsManiplex}
\label{IsEquivelar:for IsManiplex}
}\hfill{\scriptsize (property)}}\\
\textbf{\indent Returns:\ }
the the maniplex \mbox{\texttt{\mdseries\slshape M}} is equivelar; i.e., whether its Schlafli Symbol consists of integers at each
position (no lists). 



 

 }

 

\subsection{\textcolor{Chapter }{IsDegenerate (for IsManiplex)}}
\logpage{[ 3, 3, 4 ]}\nobreak
\hyperdef{L}{X8483BE798579B521}{}
{\noindent\textcolor{FuncColor}{$\triangleright$\enspace\texttt{IsDegenerate({\mdseries\slshape M})\index{IsDegenerate@\texttt{IsDegenerate}!for IsManiplex}
\label{IsDegenerate:for IsManiplex}
}\hfill{\scriptsize (property)}}\\
\textbf{\indent Returns:\ }
\texttt{true} or \texttt{false} 



 Returns whether the maniplex \mbox{\texttt{\mdseries\slshape M}} has any sections that are digons. We may eventually want to include maniplexes
with even smaller sections. }

 

\subsection{\textcolor{Chapter }{IsTight (for IsManiplex)}}
\logpage{[ 3, 3, 5 ]}\nobreak
\hyperdef{L}{X8088FFB2851334AB}{}
{\noindent\textcolor{FuncColor}{$\triangleright$\enspace\texttt{IsTight({\mdseries\slshape P})\index{IsTight@\texttt{IsTight}!for IsManiplex}
\label{IsTight:for IsManiplex}
}\hfill{\scriptsize (property)}}\\
\textbf{\indent Returns:\ }
\texttt{true} or \texttt{false} 



 Returns whether the polytope \mbox{\texttt{\mdseries\slshape P}} is tight, meaning that it has a Schlafli symbol
\texttt{\symbol{123}}k{\textunderscore}1, ...,
k{\textunderscore}\texttt{\symbol{123}}n-1\texttt{\symbol{125}}\texttt{\symbol{125}}
and has 2 k{\textunderscore}1 ...
k{\textunderscore}\texttt{\symbol{123}}n-1\texttt{\symbol{125}} flags, which
is the minimum possible. This property doesn't make any sense for
non-polytopal maniplexes, which aren't constrained by this lower bound. }

 }

 
\section{\textcolor{Chapter }{Basics}}\label{Chapter_Combinatorics_and_Structure_Section_Basics}
\logpage{[ 3, 4, 0 ]}
\hyperdef{L}{X868F7BAB7AC2EEBC}{}
{
  

\subsection{\textcolor{Chapter }{Size (for IsManiplex)}}
\logpage{[ 3, 4, 1 ]}\nobreak
\hyperdef{L}{X7C2ADC507F19E2E3}{}
{\noindent\textcolor{FuncColor}{$\triangleright$\enspace\texttt{Size({\mdseries\slshape M})\index{Size@\texttt{Size}!for IsManiplex}
\label{Size:for IsManiplex}
}\hfill{\scriptsize (attribute)}}\\


 Returns the number of flags of the maniplex \mbox{\texttt{\mdseries\slshape M}}. Synonym: \texttt{NumberOfFlags}. }

 

\subsection{\textcolor{Chapter }{RankManiplex (for IsManiplex)}}
\logpage{[ 3, 4, 2 ]}\nobreak
\hyperdef{L}{X79C21A557E80B021}{}
{\noindent\textcolor{FuncColor}{$\triangleright$\enspace\texttt{RankManiplex({\mdseries\slshape M})\index{RankManiplex@\texttt{RankManiplex}!for IsManiplex}
\label{RankManiplex:for IsManiplex}
}\hfill{\scriptsize (attribute)}}\\


 Returns the rank of the maniplex \mbox{\texttt{\mdseries\slshape M}}. }

 }

 
\section{\textcolor{Chapter }{Zigzags and holes}}\label{Chapter_Combinatorics_and_Structure_Section_Zigzags_and_holes}
\logpage{[ 3, 5, 0 ]}
\hyperdef{L}{X780FC00678185849}{}
{
  

\subsection{\textcolor{Chapter }{ZigzagLength (for IsManiplex, IsInt)}}
\logpage{[ 3, 5, 1 ]}\nobreak
\hyperdef{L}{X7D0AAD597CDE5697}{}
{\noindent\textcolor{FuncColor}{$\triangleright$\enspace\texttt{ZigzagLength({\mdseries\slshape M, j})\index{ZigzagLength@\texttt{ZigzagLength}!for IsManiplex, IsInt}
\label{ZigzagLength:for IsManiplex, IsInt}
}\hfill{\scriptsize (operation)}}\\
\textbf{\indent Returns:\ }
The lengths of \mbox{\texttt{\mdseries\slshape j}}-zigzags of the 3-maniplex \mbox{\texttt{\mdseries\slshape M}}. This corresponds to the lengths of orbits under r0 (r1
r2)\texttt{\symbol{94}}j. 



 

 }

 

\subsection{\textcolor{Chapter }{ZigzagVector (for IsManiplex)}}
\logpage{[ 3, 5, 2 ]}\nobreak
\hyperdef{L}{X7FC09F34870E6CDE}{}
{\noindent\textcolor{FuncColor}{$\triangleright$\enspace\texttt{ZigzagVector({\mdseries\slshape M})\index{ZigzagVector@\texttt{ZigzagVector}!for IsManiplex}
\label{ZigzagVector:for IsManiplex}
}\hfill{\scriptsize (attribute)}}\\
\textbf{\indent Returns:\ }
The lengths of all zigzags of the 3-maniplex \mbox{\texttt{\mdseries\slshape M}}. A rank 3 maniplex of type \texttt{\symbol{123}}p, q\texttt{\symbol{125}} has
Floor(q/2) distinct zigzag lengths because the j-zigzags are the same as the
(q-j)-zigzags. 



 

 }

 

\subsection{\textcolor{Chapter }{PetrieLength (for IsManiplex)}}
\logpage{[ 3, 5, 3 ]}\nobreak
\hyperdef{L}{X7C4A38D780F6AECD}{}
{\noindent\textcolor{FuncColor}{$\triangleright$\enspace\texttt{PetrieLength({\mdseries\slshape M})\index{PetrieLength@\texttt{PetrieLength}!for IsManiplex}
\label{PetrieLength:for IsManiplex}
}\hfill{\scriptsize (attribute)}}\\
\textbf{\indent Returns:\ }
The length of the petrie polygons of the maniplex \mbox{\texttt{\mdseries\slshape M}}. 



 

 }

 

\subsection{\textcolor{Chapter }{HoleLength (for IsManiplex, IsInt)}}
\logpage{[ 3, 5, 4 ]}\nobreak
\hyperdef{L}{X80FF41567B259CC1}{}
{\noindent\textcolor{FuncColor}{$\triangleright$\enspace\texttt{HoleLength({\mdseries\slshape M, j})\index{HoleLength@\texttt{HoleLength}!for IsManiplex, IsInt}
\label{HoleLength:for IsManiplex, IsInt}
}\hfill{\scriptsize (operation)}}\\
\textbf{\indent Returns:\ }
The lengths of \mbox{\texttt{\mdseries\slshape j}}-holes of the 3-maniplex \mbox{\texttt{\mdseries\slshape M}}. This corresponds to the lengths of orbits under r0 (r1
r2)\texttt{\symbol{94}}(j-1) r2. 



 

 }

 

\subsection{\textcolor{Chapter }{HoleVector (for IsManiplex)}}
\logpage{[ 3, 5, 5 ]}\nobreak
\hyperdef{L}{X7EE84DCF7EFC02C6}{}
{\noindent\textcolor{FuncColor}{$\triangleright$\enspace\texttt{HoleVector({\mdseries\slshape M})\index{HoleVector@\texttt{HoleVector}!for IsManiplex}
\label{HoleVector:for IsManiplex}
}\hfill{\scriptsize (attribute)}}\\
\textbf{\indent Returns:\ }
The lengths of all zigzags of the 3-maniplex \mbox{\texttt{\mdseries\slshape M}}. A rank 3 maniplex of type \texttt{\symbol{123}}p, q\texttt{\symbol{125}} has
Floor(q/2) distinct zigzag lengths because the j-zigzags are the same as the
(q-j)-zigzags. 



 

 }

 }

 }

   
\chapter{\textcolor{Chapter }{Actions}}\label{Chapter_Actions}
\logpage{[ 4, 0, 0 ]}
\hyperdef{L}{X833C5DA683E4EA15}{}
{
  
\section{\textcolor{Chapter }{Automorphism group acting on faces and chains}}\label{Chapter_Actions_Section_Automorphism_group_acting_on_faces_and_chains}
\logpage{[ 4, 1, 0 ]}
\hyperdef{L}{X84F0885E8664F785}{}
{
  

\subsection{\textcolor{Chapter }{AutomorphismGroupOnChains (for IsManiplex, IsCollection)}}
\logpage{[ 4, 1, 1 ]}\nobreak
\hyperdef{L}{X84B5DAED82D403DE}{}
{\noindent\textcolor{FuncColor}{$\triangleright$\enspace\texttt{AutomorphismGroupOnChains({\mdseries\slshape M, I})\index{AutomorphismGroupOnChains@\texttt{AutomorphismGroupOnChains}!for IsManiplex, IsCollection}
\label{AutomorphismGroupOnChains:for IsManiplex, IsCollection}
}\hfill{\scriptsize (operation)}}\\


 Returns a permutation group, representing the action of AutomorphismGroup(\mbox{\texttt{\mdseries\slshape M}}) on the chains of \mbox{\texttt{\mdseries\slshape M}} of type \mbox{\texttt{\mdseries\slshape I}}. }

 

\subsection{\textcolor{Chapter }{AutomorphismGroupOnIFaces (for IsManiplex, IsInt)}}
\logpage{[ 4, 1, 2 ]}\nobreak
\hyperdef{L}{X8549DB2585A989E4}{}
{\noindent\textcolor{FuncColor}{$\triangleright$\enspace\texttt{AutomorphismGroupOnIFaces({\mdseries\slshape M, i})\index{AutomorphismGroupOnIFaces@\texttt{AutomorphismGroupOnIFaces}!for IsManiplex, IsInt}
\label{AutomorphismGroupOnIFaces:for IsManiplex, IsInt}
}\hfill{\scriptsize (operation)}}\\


 Returns a permutation group, representing the action of AutomorphismGroup(\mbox{\texttt{\mdseries\slshape M}}) on the \mbox{\texttt{\mdseries\slshape i}}-faces of \mbox{\texttt{\mdseries\slshape M}}. }

 

\subsection{\textcolor{Chapter }{AutomorphismGroupOnVertices (for IsManiplex)}}
\logpage{[ 4, 1, 3 ]}\nobreak
\hyperdef{L}{X81972F5187B9C2AF}{}
{\noindent\textcolor{FuncColor}{$\triangleright$\enspace\texttt{AutomorphismGroupOnVertices({\mdseries\slshape M})\index{AutomorphismGroupOnVertices@\texttt{AutomorphismGroupOnVertices}!for IsManiplex}
\label{AutomorphismGroupOnVertices:for IsManiplex}
}\hfill{\scriptsize (attribute)}}\\


 Returns a permutation group, representing the action of AutomorphismGroup(\mbox{\texttt{\mdseries\slshape M}}) on the vertices of \mbox{\texttt{\mdseries\slshape M}}. }

 

\subsection{\textcolor{Chapter }{AutomorphismGroupOnEdges (for IsManiplex)}}
\logpage{[ 4, 1, 4 ]}\nobreak
\hyperdef{L}{X86CBC88A8718D7E4}{}
{\noindent\textcolor{FuncColor}{$\triangleright$\enspace\texttt{AutomorphismGroupOnEdges({\mdseries\slshape M})\index{AutomorphismGroupOnEdges@\texttt{AutomorphismGroupOnEdges}!for IsManiplex}
\label{AutomorphismGroupOnEdges:for IsManiplex}
}\hfill{\scriptsize (attribute)}}\\


 Returns a permutation group, representing the action of AutomorphismGroup(\mbox{\texttt{\mdseries\slshape M}}) on the edges of \mbox{\texttt{\mdseries\slshape M}}. }

 

\subsection{\textcolor{Chapter }{AutomorphismGroupOnFacets (for IsManiplex)}}
\logpage{[ 4, 1, 5 ]}\nobreak
\hyperdef{L}{X82726AF27C1E00D6}{}
{\noindent\textcolor{FuncColor}{$\triangleright$\enspace\texttt{AutomorphismGroupOnFacets({\mdseries\slshape M})\index{AutomorphismGroupOnFacets@\texttt{AutomorphismGroupOnFacets}!for IsManiplex}
\label{AutomorphismGroupOnFacets:for IsManiplex}
}\hfill{\scriptsize (attribute)}}\\


 Returns a permutation group, representing the action of AutomorphismGroup(\mbox{\texttt{\mdseries\slshape M}}) on the facets of \mbox{\texttt{\mdseries\slshape M}}. }

 }

 
\section{\textcolor{Chapter }{Number of orbits and transitivity}}\label{Chapter_Actions_Section_Number_of_orbits_and_transitivity}
\logpage{[ 4, 2, 0 ]}
\hyperdef{L}{X7A07E55F7D617911}{}
{
  

\subsection{\textcolor{Chapter }{NumberOfChainOrbits (for IsManiplex, IsCollection)}}
\logpage{[ 4, 2, 1 ]}\nobreak
\hyperdef{L}{X8423C2647CFF8C2E}{}
{\noindent\textcolor{FuncColor}{$\triangleright$\enspace\texttt{NumberOfChainOrbits({\mdseries\slshape M, I})\index{NumberOfChainOrbits@\texttt{NumberOfChainOrbits}!for IsManiplex, IsCollection}
\label{NumberOfChainOrbits:for IsManiplex, IsCollection}
}\hfill{\scriptsize (operation)}}\\


 Returns the number of orbits of chains of type \mbox{\texttt{\mdseries\slshape I}} under the action of AutomorphismGroup(\mbox{\texttt{\mdseries\slshape M}}). }

 

\subsection{\textcolor{Chapter }{NumberOfIFaceOrbits (for IsManiplex, IsInt)}}
\logpage{[ 4, 2, 2 ]}\nobreak
\hyperdef{L}{X82D959517BC24461}{}
{\noindent\textcolor{FuncColor}{$\triangleright$\enspace\texttt{NumberOfIFaceOrbits({\mdseries\slshape M, i})\index{NumberOfIFaceOrbits@\texttt{NumberOfIFaceOrbits}!for IsManiplex, IsInt}
\label{NumberOfIFaceOrbits:for IsManiplex, IsInt}
}\hfill{\scriptsize (operation)}}\\


 Returns the number of orbits of \mbox{\texttt{\mdseries\slshape i}}-faces under the action of AutomorphismGroup(\mbox{\texttt{\mdseries\slshape M}}). }

 

\subsection{\textcolor{Chapter }{NumberOfVertexOrbits (for IsManiplex)}}
\logpage{[ 4, 2, 3 ]}\nobreak
\hyperdef{L}{X86A6A073804A5780}{}
{\noindent\textcolor{FuncColor}{$\triangleright$\enspace\texttt{NumberOfVertexOrbits({\mdseries\slshape M})\index{NumberOfVertexOrbits@\texttt{NumberOfVertexOrbits}!for IsManiplex}
\label{NumberOfVertexOrbits:for IsManiplex}
}\hfill{\scriptsize (attribute)}}\\


 Returns the number of orbits of vertices under the action of
AutomorphismGroup(\mbox{\texttt{\mdseries\slshape M}}). }

 

\subsection{\textcolor{Chapter }{NumberOfEdgeOrbits (for IsManiplex)}}
\logpage{[ 4, 2, 4 ]}\nobreak
\hyperdef{L}{X7A41AD2B7A606FB2}{}
{\noindent\textcolor{FuncColor}{$\triangleright$\enspace\texttt{NumberOfEdgeOrbits({\mdseries\slshape M})\index{NumberOfEdgeOrbits@\texttt{NumberOfEdgeOrbits}!for IsManiplex}
\label{NumberOfEdgeOrbits:for IsManiplex}
}\hfill{\scriptsize (attribute)}}\\


 Returns the number of orbits of edges under the action of AutomorphismGroup(\mbox{\texttt{\mdseries\slshape M}}). }

 

\subsection{\textcolor{Chapter }{NumberOfFacetOrbits (for IsManiplex)}}
\logpage{[ 4, 2, 5 ]}\nobreak
\hyperdef{L}{X81C6FD5078C637E6}{}
{\noindent\textcolor{FuncColor}{$\triangleright$\enspace\texttt{NumberOfFacetOrbits({\mdseries\slshape M})\index{NumberOfFacetOrbits@\texttt{NumberOfFacetOrbits}!for IsManiplex}
\label{NumberOfFacetOrbits:for IsManiplex}
}\hfill{\scriptsize (attribute)}}\\


 Returns the number of orbits of facets under the action of AutomorphismGroup(\mbox{\texttt{\mdseries\slshape M}}). }

 

\subsection{\textcolor{Chapter }{IsChainTransitive (for IsManiplex, IsCollection)}}
\logpage{[ 4, 2, 6 ]}\nobreak
\hyperdef{L}{X7A2326997FB2CAA8}{}
{\noindent\textcolor{FuncColor}{$\triangleright$\enspace\texttt{IsChainTransitive({\mdseries\slshape M, I})\index{IsChainTransitive@\texttt{IsChainTransitive}!for IsManiplex, IsCollection}
\label{IsChainTransitive:for IsManiplex, IsCollection}
}\hfill{\scriptsize (operation)}}\\


 Returns whether the action of AutomorphismGroup(\mbox{\texttt{\mdseries\slshape M}}) on chains of type \mbox{\texttt{\mdseries\slshape I}} is transitive. }

 

\subsection{\textcolor{Chapter }{IsIFaceTransitive (for IsManiplex, IsInt)}}
\logpage{[ 4, 2, 7 ]}\nobreak
\hyperdef{L}{X80F9303F7A272F20}{}
{\noindent\textcolor{FuncColor}{$\triangleright$\enspace\texttt{IsIFaceTransitive({\mdseries\slshape M, i})\index{IsIFaceTransitive@\texttt{IsIFaceTransitive}!for IsManiplex, IsInt}
\label{IsIFaceTransitive:for IsManiplex, IsInt}
}\hfill{\scriptsize (operation)}}\\


 Returns whether the action of AutomorphismGroup(\mbox{\texttt{\mdseries\slshape M}}) on \mbox{\texttt{\mdseries\slshape i}}-faces is transitive. }

 

\subsection{\textcolor{Chapter }{IsVertexTransitive (for IsManiplex)}}
\logpage{[ 4, 2, 8 ]}\nobreak
\hyperdef{L}{X7E29A16E7B724836}{}
{\noindent\textcolor{FuncColor}{$\triangleright$\enspace\texttt{IsVertexTransitive({\mdseries\slshape M})\index{IsVertexTransitive@\texttt{IsVertexTransitive}!for IsManiplex}
\label{IsVertexTransitive:for IsManiplex}
}\hfill{\scriptsize (property)}}\\
\textbf{\indent Returns:\ }
\texttt{true} or \texttt{false} 



 Returns whether the action of AutomorphismGroup(\mbox{\texttt{\mdseries\slshape M}}) on vertices is transitive. }

 

\subsection{\textcolor{Chapter }{IsEdgeTransitive (for IsManiplex)}}
\logpage{[ 4, 2, 9 ]}\nobreak
\hyperdef{L}{X7CA8A3A17CB62B51}{}
{\noindent\textcolor{FuncColor}{$\triangleright$\enspace\texttt{IsEdgeTransitive({\mdseries\slshape M})\index{IsEdgeTransitive@\texttt{IsEdgeTransitive}!for IsManiplex}
\label{IsEdgeTransitive:for IsManiplex}
}\hfill{\scriptsize (property)}}\\
\textbf{\indent Returns:\ }
\texttt{true} or \texttt{false} 



 Returns whether the action of AutomorphismGroup(\mbox{\texttt{\mdseries\slshape M}}) on edges is transitive. }

 

\subsection{\textcolor{Chapter }{IsFacetTransitive (for IsManiplex)}}
\logpage{[ 4, 2, 10 ]}\nobreak
\hyperdef{L}{X85436C5C84C4031E}{}
{\noindent\textcolor{FuncColor}{$\triangleright$\enspace\texttt{IsFacetTransitive({\mdseries\slshape M})\index{IsFacetTransitive@\texttt{IsFacetTransitive}!for IsManiplex}
\label{IsFacetTransitive:for IsManiplex}
}\hfill{\scriptsize (property)}}\\
\textbf{\indent Returns:\ }
\texttt{true} or \texttt{false} 



 Returns whether the action of AutomorphismGroup(\mbox{\texttt{\mdseries\slshape M}}) on facets is transitive. }

 

\subsection{\textcolor{Chapter }{IsFullyTransitive (for IsManiplex)}}
\logpage{[ 4, 2, 11 ]}\nobreak
\hyperdef{L}{X7B898E19857B356F}{}
{\noindent\textcolor{FuncColor}{$\triangleright$\enspace\texttt{IsFullyTransitive({\mdseries\slshape M})\index{IsFullyTransitive@\texttt{IsFullyTransitive}!for IsManiplex}
\label{IsFullyTransitive:for IsManiplex}
}\hfill{\scriptsize (property)}}\\
\textbf{\indent Returns:\ }
\texttt{true} or \texttt{false} 



 Returns whether the action of AutomorphismGroup(\mbox{\texttt{\mdseries\slshape M}}) on i-faces is transitive for every i. }

 }

 
\section{\textcolor{Chapter }{Flag orbits}}\label{Chapter_Actions_Section_Flag_orbits}
\logpage{[ 4, 3, 0 ]}
\hyperdef{L}{X7BE5F0217862DBEF}{}
{
  

\subsection{\textcolor{Chapter }{SymmetryTypeGraph (for IsManiplex)}}
\logpage{[ 4, 3, 1 ]}\nobreak
\hyperdef{L}{X856ED1E7866152A0}{}
{\noindent\textcolor{FuncColor}{$\triangleright$\enspace\texttt{SymmetryTypeGraph({\mdseries\slshape M})\index{SymmetryTypeGraph@\texttt{SymmetryTypeGraph}!for IsManiplex}
\label{SymmetryTypeGraph:for IsManiplex}
}\hfill{\scriptsize (attribute)}}\\


 Returns the Symmetry Type Graph of the maniplex \mbox{\texttt{\mdseries\slshape M}}, encoded as a permutation group on Rank(\mbox{\texttt{\mdseries\slshape M}}) generators. }

 

\subsection{\textcolor{Chapter }{NumberOfFlagOrbits (for IsManiplex)}}
\logpage{[ 4, 3, 2 ]}\nobreak
\hyperdef{L}{X78060E3A788D2D3C}{}
{\noindent\textcolor{FuncColor}{$\triangleright$\enspace\texttt{NumberOfFlagOrbits({\mdseries\slshape M})\index{NumberOfFlagOrbits@\texttt{NumberOfFlagOrbits}!for IsManiplex}
\label{NumberOfFlagOrbits:for IsManiplex}
}\hfill{\scriptsize (attribute)}}\\


 Returns the number of orbits of the automorphism group of \mbox{\texttt{\mdseries\slshape M}} on its flags. }

 

\subsection{\textcolor{Chapter }{FlagOrbitRepresentatives (for IsManiplex)}}
\logpage{[ 4, 3, 3 ]}\nobreak
\hyperdef{L}{X852C679D8338AB53}{}
{\noindent\textcolor{FuncColor}{$\triangleright$\enspace\texttt{FlagOrbitRepresentatives({\mdseries\slshape M})\index{FlagOrbitRepresentatives@\texttt{FlagOrbitRepresentatives}!for IsManiplex}
\label{FlagOrbitRepresentatives:for IsManiplex}
}\hfill{\scriptsize (attribute)}}\\


 Returns one flag from each orbit under the action of AutomorphismGroup(\mbox{\texttt{\mdseries\slshape M}}). }

 

\subsection{\textcolor{Chapter }{IsReflexible (for IsManiplex)}}
\logpage{[ 4, 3, 4 ]}\nobreak
\hyperdef{L}{X7E4E689F86171C78}{}
{\noindent\textcolor{FuncColor}{$\triangleright$\enspace\texttt{IsReflexible({\mdseries\slshape M})\index{IsReflexible@\texttt{IsReflexible}!for IsManiplex}
\label{IsReflexible:for IsManiplex}
}\hfill{\scriptsize (property)}}\\
\textbf{\indent Returns:\ }
Whether the maniplex \mbox{\texttt{\mdseries\slshape M}} is reflexible (has one flag orbit). 



 

 }

 

\subsection{\textcolor{Chapter }{IsChiral (for IsManiplex)}}
\logpage{[ 4, 3, 5 ]}\nobreak
\hyperdef{L}{X7D74CD997C7FF39D}{}
{\noindent\textcolor{FuncColor}{$\triangleright$\enspace\texttt{IsChiral({\mdseries\slshape M})\index{IsChiral@\texttt{IsChiral}!for IsManiplex}
\label{IsChiral:for IsManiplex}
}\hfill{\scriptsize (property)}}\\
\textbf{\indent Returns:\ }
Whether the maniplex \mbox{\texttt{\mdseries\slshape M}} is chrial. 



 

 }

 

\subsection{\textcolor{Chapter }{IsRotary (for IsManiplex)}}
\logpage{[ 4, 3, 6 ]}\nobreak
\hyperdef{L}{X7DACFE5483893523}{}
{\noindent\textcolor{FuncColor}{$\triangleright$\enspace\texttt{IsRotary({\mdseries\slshape M})\index{IsRotary@\texttt{IsRotary}!for IsManiplex}
\label{IsRotary:for IsManiplex}
}\hfill{\scriptsize (property)}}\\
\textbf{\indent Returns:\ }
Whether the maniplex \mbox{\texttt{\mdseries\slshape M}} is rotary; i.e., whether it is either reflexible or chiral. 



 

 }

 }

 
\section{\textcolor{Chapter }{Faithfulness}}\label{Chapter_Actions_Section_Faithfulness}
\logpage{[ 4, 4, 0 ]}
\hyperdef{L}{X83355D207F38F997}{}
{
  

\subsection{\textcolor{Chapter }{IsVertexFaithful (for IsReflexibleManiplex)}}
\logpage{[ 4, 4, 1 ]}\nobreak
\hyperdef{L}{X785DE0A9842AAEA1}{}
{\noindent\textcolor{FuncColor}{$\triangleright$\enspace\texttt{IsVertexFaithful({\mdseries\slshape M})\index{IsVertexFaithful@\texttt{IsVertexFaithful}!for IsReflexibleManiplex}
\label{IsVertexFaithful:for IsReflexibleManiplex}
}\hfill{\scriptsize (property)}}\\
\textbf{\indent Returns:\ }
\texttt{true} or \texttt{false} 



 Returns whether the reflexible maniplex \mbox{\texttt{\mdseries\slshape M}} is vertex-faithful; i.e., whether the action of the automorphism group on the
vertices is faithful. }

 

\subsection{\textcolor{Chapter }{IsFacetFaithful (for IsReflexibleManiplex)}}
\logpage{[ 4, 4, 2 ]}\nobreak
\hyperdef{L}{X78CDF48E7CDED1AB}{}
{\noindent\textcolor{FuncColor}{$\triangleright$\enspace\texttt{IsFacetFaithful({\mdseries\slshape M})\index{IsFacetFaithful@\texttt{IsFacetFaithful}!for IsReflexibleManiplex}
\label{IsFacetFaithful:for IsReflexibleManiplex}
}\hfill{\scriptsize (property)}}\\
\textbf{\indent Returns:\ }
\texttt{true} or \texttt{false} 



 Returns whether the reflexible maniplex \mbox{\texttt{\mdseries\slshape M}} is facet-faithful; i.e., whether the action of the automorphism group on the
facets is faithful. }

 

\subsection{\textcolor{Chapter }{MaxVertexFaithfulQuotient (for IsReflexibleManiplex)}}
\logpage{[ 4, 4, 3 ]}\nobreak
\hyperdef{L}{X856543F87E52DDC6}{}
{\noindent\textcolor{FuncColor}{$\triangleright$\enspace\texttt{MaxVertexFaithfulQuotient({\mdseries\slshape M})\index{MaxVertexFaithfulQuotient@\texttt{MaxVertexFaithfulQuotient}!for IsReflexibleManiplex}
\label{MaxVertexFaithfulQuotient:for IsReflexibleManiplex}
}\hfill{\scriptsize (operation)}}\\


 Returns the maximal vertex-faithful reflexible maniplex covered by \mbox{\texttt{\mdseries\slshape M}}. }

 }

 }

   
\chapter{\textcolor{Chapter }{Constructions}}\label{Chapter_Constructions}
\logpage{[ 5, 0, 0 ]}
\hyperdef{L}{X836F56D77AA04554}{}
{
  
\section{\textcolor{Chapter }{Extensions, amalgamations, and quotients}}\label{Chapter_Constructions_Section_Extensions_amalgamations_and_quotients}
\logpage{[ 5, 1, 0 ]}
\hyperdef{L}{X871C2D73829F1FC1}{}
{
  

\subsection{\textcolor{Chapter }{UniversalPolytope (for IsInt)}}
\logpage{[ 5, 1, 1 ]}\nobreak
\hyperdef{L}{X808AEA5281D52911}{}
{\noindent\textcolor{FuncColor}{$\triangleright$\enspace\texttt{UniversalPolytope({\mdseries\slshape n})\index{UniversalPolytope@\texttt{UniversalPolytope}!for IsInt}
\label{UniversalPolytope:for IsInt}
}\hfill{\scriptsize (operation)}}\\


 Returns the universal polytope of rank \mbox{\texttt{\mdseries\slshape n}}. }

 

\subsection{\textcolor{Chapter }{FlatRegularPolyhedron (for IsInt, IsInt, IsInt, IsInt)}}
\logpage{[ 5, 1, 2 ]}\nobreak
\hyperdef{L}{X800E07C686CF3EA7}{}
{\noindent\textcolor{FuncColor}{$\triangleright$\enspace\texttt{FlatRegularPolyhedron({\mdseries\slshape p, q, i, j})\index{FlatRegularPolyhedron@\texttt{FlatRegularPolyhedron}!for IsInt, IsInt, IsInt, IsInt}
\label{FlatRegularPolyhedron:for IsInt, IsInt, IsInt, IsInt}
}\hfill{\scriptsize (operation)}}\\


 Returns the flat regular polyhedron with automorphism group [p, q] / (r2 r1 r0
r1 = (r0 r1)\texttt{\symbol{94}}i (r1 r2)\texttt{\symbol{94}}j). This function
does not currently validate the inputs to make sure that the output makes
sense. }

 

\subsection{\textcolor{Chapter }{UniversalExtension (for IsManiplex)}}
\logpage{[ 5, 1, 3 ]}\nobreak
\hyperdef{L}{X8582D0CF7EA1CB34}{}
{\noindent\textcolor{FuncColor}{$\triangleright$\enspace\texttt{UniversalExtension({\mdseries\slshape M})\index{UniversalExtension@\texttt{UniversalExtension}!for IsManiplex}
\label{UniversalExtension:for IsManiplex}
}\hfill{\scriptsize (operation)}}\\


 Returns the universal extension of \mbox{\texttt{\mdseries\slshape M}}, i.e. the maniplex with facets isomorphic to \mbox{\texttt{\mdseries\slshape M}} that covers all other maniplexes with facets isomorphic to \mbox{\texttt{\mdseries\slshape M}}. Currently only defined for reflexible maniplexes. }

 

\subsection{\textcolor{Chapter }{UniversalExtension (for IsManiplex, IsInt)}}
\logpage{[ 5, 1, 4 ]}\nobreak
\hyperdef{L}{X78B4610E7DE67E32}{}
{\noindent\textcolor{FuncColor}{$\triangleright$\enspace\texttt{UniversalExtension({\mdseries\slshape M, k})\index{UniversalExtension@\texttt{UniversalExtension}!for IsManiplex, IsInt}
\label{UniversalExtension:for IsManiplex, IsInt}
}\hfill{\scriptsize (operation)}}\\


 Returns the universal extension of \mbox{\texttt{\mdseries\slshape M}} with last entry of Schlafli symbol \mbox{\texttt{\mdseries\slshape k}}. Currently only defined for reflexible maniplexes. }

 

\subsection{\textcolor{Chapter }{TrivialExtension (for IsManiplex)}}
\logpage{[ 5, 1, 5 ]}\nobreak
\hyperdef{L}{X84BB6B21853173FD}{}
{\noindent\textcolor{FuncColor}{$\triangleright$\enspace\texttt{TrivialExtension({\mdseries\slshape M})\index{TrivialExtension@\texttt{TrivialExtension}!for IsManiplex}
\label{TrivialExtension:for IsManiplex}
}\hfill{\scriptsize (operation)}}\\


 Returns the trivial extension of \mbox{\texttt{\mdseries\slshape M}}, also known as \texttt{\symbol{123}}\mbox{\texttt{\mdseries\slshape M/}}, 2\texttt{\symbol{125}}. }

 

\subsection{\textcolor{Chapter }{FlatExtension (for IsManiplex, IsInt)}}
\logpage{[ 5, 1, 6 ]}\nobreak
\hyperdef{L}{X7A5853E278A83D74}{}
{\noindent\textcolor{FuncColor}{$\triangleright$\enspace\texttt{FlatExtension({\mdseries\slshape M, k})\index{FlatExtension@\texttt{FlatExtension}!for IsManiplex, IsInt}
\label{FlatExtension:for IsManiplex, IsInt}
}\hfill{\scriptsize (operation)}}\\


 Returns the flat extension of \mbox{\texttt{\mdseries\slshape M}} with last entry of Schlafli symbol \mbox{\texttt{\mdseries\slshape k}}. (As defined in "Flat Extensions of Abstract Polytopes".) Currently only
defined for reflexible maniplexes. }

 

\subsection{\textcolor{Chapter }{Amalgamate (for IsManiplex, IsManiplex)}}
\logpage{[ 5, 1, 7 ]}\nobreak
\hyperdef{L}{X790F22D47CD222EB}{}
{\noindent\textcolor{FuncColor}{$\triangleright$\enspace\texttt{Amalgamate({\mdseries\slshape M1, M2})\index{Amalgamate@\texttt{Amalgamate}!for IsManiplex, IsManiplex}
\label{Amalgamate:for IsManiplex, IsManiplex}
}\hfill{\scriptsize (operation)}}\\


 Returns the amalgamation of \mbox{\texttt{\mdseries\slshape M1}} and \mbox{\texttt{\mdseries\slshape M2}}. Implicitly assumes that \mbox{\texttt{\mdseries\slshape M1}} and \mbox{\texttt{\mdseries\slshape M2}} are compatible. Currently only defined for reflexible maniplexes. }

 

\subsection{\textcolor{Chapter }{Medial (for IsManiplex)}}
\logpage{[ 5, 1, 8 ]}\nobreak
\hyperdef{L}{X840BC19484E0E9CC}{}
{\noindent\textcolor{FuncColor}{$\triangleright$\enspace\texttt{Medial({\mdseries\slshape M})\index{Medial@\texttt{Medial}!for IsManiplex}
\label{Medial:for IsManiplex}
}\hfill{\scriptsize (operation)}}\\


 Given a 3-maniplex \mbox{\texttt{\mdseries\slshape M}}, returns its medial. }

 }

 
\section{\textcolor{Chapter }{Duality}}\label{Chapter_Constructions_Section_Duality}
\logpage{[ 5, 2, 0 ]}
\hyperdef{L}{X87FD993F7F6F2FAB}{}
{
  

\subsection{\textcolor{Chapter }{Dual (for IsManiplex)}}
\logpage{[ 5, 2, 1 ]}\nobreak
\hyperdef{L}{X7D62BD3E7F941F5B}{}
{\noindent\textcolor{FuncColor}{$\triangleright$\enspace\texttt{Dual({\mdseries\slshape M})\index{Dual@\texttt{Dual}!for IsManiplex}
\label{Dual:for IsManiplex}
}\hfill{\scriptsize (operation)}}\\
\textbf{\indent Returns:\ }
The maniplex that is dual to \mbox{\texttt{\mdseries\slshape M}}. 



 

 }

 

\subsection{\textcolor{Chapter }{IsSelfDual (for IsManiplex)}}
\logpage{[ 5, 2, 2 ]}\nobreak
\hyperdef{L}{X7FF8B96C83DA60CB}{}
{\noindent\textcolor{FuncColor}{$\triangleright$\enspace\texttt{IsSelfDual({\mdseries\slshape P})\index{IsSelfDual@\texttt{IsSelfDual}!for IsManiplex}
\label{IsSelfDual:for IsManiplex}
}\hfill{\scriptsize (property)}}\\
\textbf{\indent Returns:\ }
Whether this polytope is isomorphic to its dual. 



 Also works for IsPoset objects. }

 

\subsection{\textcolor{Chapter }{Petrial (for IsManiplex)}}
\logpage{[ 5, 2, 3 ]}\nobreak
\hyperdef{L}{X7E6D0732862D6BA3}{}
{\noindent\textcolor{FuncColor}{$\triangleright$\enspace\texttt{Petrial({\mdseries\slshape P})\index{Petrial@\texttt{Petrial}!for IsManiplex}
\label{Petrial:for IsManiplex}
}\hfill{\scriptsize (attribute)}}\\
\textbf{\indent Returns:\ }
The Petrial (Petrie dual) of \mbox{\texttt{\mdseries\slshape P}}. Note that this is not necessarily a polytope. 



 

 }

 

\subsection{\textcolor{Chapter }{IsSelfPetrial (for IsManiplex)}}
\logpage{[ 5, 2, 4 ]}\nobreak
\hyperdef{L}{X78FDB0047E3388A5}{}
{\noindent\textcolor{FuncColor}{$\triangleright$\enspace\texttt{IsSelfPetrial({\mdseries\slshape P})\index{IsSelfPetrial@\texttt{IsSelfPetrial}!for IsManiplex}
\label{IsSelfPetrial:for IsManiplex}
}\hfill{\scriptsize (property)}}\\
\textbf{\indent Returns:\ }
Whether this polytope is isomorphic to its Petrial. 



 

 }

 

\subsection{\textcolor{Chapter }{DirectDerivates (for IsManiplex)}}
\logpage{[ 5, 2, 5 ]}\nobreak
\hyperdef{L}{X7F8BED777CD6B80F}{}
{\noindent\textcolor{FuncColor}{$\triangleright$\enspace\texttt{DirectDerivates({\mdseries\slshape M})\index{DirectDerivates@\texttt{DirectDerivates}!for IsManiplex}
\label{DirectDerivates:for IsManiplex}
}\hfill{\scriptsize (operation)}}\\


 Returns a list of the \emph{direct derivates} of \mbox{\texttt{\mdseries\slshape M}}, which are the images of M under duality and Petriality. If the option
'polytopal' is set, then only returns those direct derivates that are
polytopal. }

 }

 
\section{\textcolor{Chapter }{Products}}\label{Chapter_Constructions_Section_Products}
\logpage{[ 5, 3, 0 ]}
\hyperdef{L}{X86CE352C7851221F}{}
{
  

\subsection{\textcolor{Chapter }{PyramidOver (for IsManiplex)}}
\logpage{[ 5, 3, 1 ]}\nobreak
\hyperdef{L}{X80A4156A8423CAF8}{}
{\noindent\textcolor{FuncColor}{$\triangleright$\enspace\texttt{PyramidOver({\mdseries\slshape M})\index{PyramidOver@\texttt{PyramidOver}!for IsManiplex}
\label{PyramidOver:for IsManiplex}
}\hfill{\scriptsize (operation)}}\\


 Returns the pyramid over \mbox{\texttt{\mdseries\slshape M}}. }

 

\subsection{\textcolor{Chapter }{Pyramid (for IsInt)}}
\logpage{[ 5, 3, 2 ]}\nobreak
\hyperdef{L}{X80EEFF7F7FE1B752}{}
{\noindent\textcolor{FuncColor}{$\triangleright$\enspace\texttt{Pyramid({\mdseries\slshape k})\index{Pyramid@\texttt{Pyramid}!for IsInt}
\label{Pyramid:for IsInt}
}\hfill{\scriptsize (operation)}}\\


 Returns the pyramid over a \mbox{\texttt{\mdseries\slshape k}}-gon. }

 

\subsection{\textcolor{Chapter }{PrismOver (for IsManiplex)}}
\logpage{[ 5, 3, 3 ]}\nobreak
\hyperdef{L}{X7C7D6F558335CFD0}{}
{\noindent\textcolor{FuncColor}{$\triangleright$\enspace\texttt{PrismOver({\mdseries\slshape M})\index{PrismOver@\texttt{PrismOver}!for IsManiplex}
\label{PrismOver:for IsManiplex}
}\hfill{\scriptsize (operation)}}\\


 Returns the prism over \mbox{\texttt{\mdseries\slshape M}}. }

 

\subsection{\textcolor{Chapter }{Prism (for IsInt)}}
\logpage{[ 5, 3, 4 ]}\nobreak
\hyperdef{L}{X86E5C03286E992CA}{}
{\noindent\textcolor{FuncColor}{$\triangleright$\enspace\texttt{Prism({\mdseries\slshape k})\index{Prism@\texttt{Prism}!for IsInt}
\label{Prism:for IsInt}
}\hfill{\scriptsize (operation)}}\\


 Returns the prism over a \mbox{\texttt{\mdseries\slshape k}}-gon. }

 }

 }

   
\chapter{\textcolor{Chapter }{Databases}}\label{Chapter_Databases}
\logpage{[ 6, 0, 0 ]}
\hyperdef{L}{X7EB183C3780A475B}{}
{
  
\section{\textcolor{Chapter }{Regular polyhedra}}\label{Chapter_Databases_Section_Regular_polyhedra}
\logpage{[ 6, 1, 0 ]}
\hyperdef{L}{X8062E376879531A7}{}
{
  

\subsection{\textcolor{Chapter }{DegeneratePolyhedra}}
\logpage{[ 6, 1, 1 ]}\nobreak
\hyperdef{L}{X7B2285CD79A7DF27}{}
{\noindent\textcolor{FuncColor}{$\triangleright$\enspace\texttt{DegeneratePolyhedra({\mdseries\slshape sizerange})\index{DegeneratePolyhedra@\texttt{DegeneratePolyhedra}}
\label{DegeneratePolyhedra}
}\hfill{\scriptsize (function)}}\\


 Returns all degenerate polyhedra (of type \texttt{\symbol{123}}2,
q\texttt{\symbol{125}} and \texttt{\symbol{123}}p, 2\texttt{\symbol{125}})
with sizes in \mbox{\texttt{\mdseries\slshape sizerange}}. Also accepts a single integer \emph{maxsize} as input to indicate a sizerange of [1..maxsize]. }

 

\subsection{\textcolor{Chapter }{FlatRegularPolyhedra}}
\logpage{[ 6, 1, 2 ]}\nobreak
\hyperdef{L}{X86869B8A80564ACF}{}
{\noindent\textcolor{FuncColor}{$\triangleright$\enspace\texttt{FlatRegularPolyhedra({\mdseries\slshape sizerange})\index{FlatRegularPolyhedra@\texttt{FlatRegularPolyhedra}}
\label{FlatRegularPolyhedra}
}\hfill{\scriptsize (function)}}\\


 Returns all nondegenerate flat regular polyhedra with sizes in \mbox{\texttt{\mdseries\slshape sizerange}}. Also accepts a single integer \emph{maxsize} as input to indicate a sizerange of [1..maxsize]. Currently supports a maxsize
of 4000 or less. }

 

\subsection{\textcolor{Chapter }{RegularToroidalPolyhedra44}}
\logpage{[ 6, 1, 3 ]}\nobreak
\hyperdef{L}{X8229CA838473FF2B}{}
{\noindent\textcolor{FuncColor}{$\triangleright$\enspace\texttt{RegularToroidalPolyhedra44({\mdseries\slshape sizerange})\index{RegularToroidalPolyhedra44@\texttt{RegularToroidalPolyhedra44}}
\label{RegularToroidalPolyhedra44}
}\hfill{\scriptsize (function)}}\\


 Returns all regular toroidal polyhedra of type
\texttt{\symbol{123}}4,4\texttt{\symbol{125}} with sizes in \mbox{\texttt{\mdseries\slshape sizerange}}. Also accepts a single integer \emph{maxsize} as input to indicate a sizerange of [1..maxsize]. }

 

\subsection{\textcolor{Chapter }{RegularToroidalPolyhedra36}}
\logpage{[ 6, 1, 4 ]}\nobreak
\hyperdef{L}{X783D35FD79C25457}{}
{\noindent\textcolor{FuncColor}{$\triangleright$\enspace\texttt{RegularToroidalPolyhedra36({\mdseries\slshape sizerange})\index{RegularToroidalPolyhedra36@\texttt{RegularToroidalPolyhedra36}}
\label{RegularToroidalPolyhedra36}
}\hfill{\scriptsize (function)}}\\


 Returns all regular toroidal polyhedra of type
\texttt{\symbol{123}}3,6\texttt{\symbol{125}} with sizes in \mbox{\texttt{\mdseries\slshape sizerange}}. Also accepts a single integer \emph{maxsize} as input to indicate a sizerange of [1..maxsize]. }

 

\subsection{\textcolor{Chapter }{SmallRegularPolyhedra}}
\logpage{[ 6, 1, 5 ]}\nobreak
\hyperdef{L}{X7BD5DABD833B9700}{}
{\noindent\textcolor{FuncColor}{$\triangleright$\enspace\texttt{SmallRegularPolyhedra({\mdseries\slshape sizerange})\index{SmallRegularPolyhedra@\texttt{SmallRegularPolyhedra}}
\label{SmallRegularPolyhedra}
}\hfill{\scriptsize (function)}}\\


 Returns all regular polyhedra with sizes in \mbox{\texttt{\mdseries\slshape sizerange}} flags. Currently supports a maxsize of 4000 or less. You can also set options
"nondegenerate" and "nonflat". 
\begin{Verbatim}[commandchars=!@|,fontsize=\small,frame=single,label=Example]
  L1 := SmallRegularPolyhedra(500);;
  L2 := SmallRegularPolyhedra(1000 : nondegenerate);;
  L3 := SmallRegularPolyhedra(2000 : nondegenerate, nonflat);;
\end{Verbatim}
 }

 

\subsection{\textcolor{Chapter }{SmallRegular4Polytopes}}
\logpage{[ 6, 1, 6 ]}\nobreak
\hyperdef{L}{X83C76844861B7B5A}{}
{\noindent\textcolor{FuncColor}{$\triangleright$\enspace\texttt{SmallRegular4Polytopes({\mdseries\slshape sizerange})\index{SmallRegular4Polytopes@\texttt{SmallRegular4Polytopes}}
\label{SmallRegular4Polytopes}
}\hfill{\scriptsize (function)}}\\


 Returns all regular 4-polytopes with sizes in \mbox{\texttt{\mdseries\slshape sizerange}} flags. Currently supports a maxsize of 4000 or less. }

 

\subsection{\textcolor{Chapter }{SmallChiralPolyhedra}}
\logpage{[ 6, 1, 7 ]}\nobreak
\hyperdef{L}{X7C5BEAD87A858EE7}{}
{\noindent\textcolor{FuncColor}{$\triangleright$\enspace\texttt{SmallChiralPolyhedra({\mdseries\slshape sizerange})\index{SmallChiralPolyhedra@\texttt{SmallChiralPolyhedra}}
\label{SmallChiralPolyhedra}
}\hfill{\scriptsize (function)}}\\


 Returns all chiral polyhedra with sizes in \mbox{\texttt{\mdseries\slshape sizerange}} flags. Currently supports a maxsize of 4000 or less. }

 

\subsection{\textcolor{Chapter }{SmallChiral4Polytopes}}
\logpage{[ 6, 1, 8 ]}\nobreak
\hyperdef{L}{X85ED37A97CE7F77C}{}
{\noindent\textcolor{FuncColor}{$\triangleright$\enspace\texttt{SmallChiral4Polytopes({\mdseries\slshape sizerange})\index{SmallChiral4Polytopes@\texttt{SmallChiral4Polytopes}}
\label{SmallChiral4Polytopes}
}\hfill{\scriptsize (function)}}\\


 Returns all chiral 4-polytopes with sizes in \mbox{\texttt{\mdseries\slshape sizerange}} flags. Currently supports a maxsize of 4000 or less. }

 }

 }

   
\chapter{\textcolor{Chapter }{Families of Polytopes}}\label{Chapter_Families_of_Polytopes}
\logpage{[ 7, 0, 0 ]}
\hyperdef{L}{X7BF24A5D7B3386D7}{}
{
  
\section{\textcolor{Chapter }{Classical Polytopes}}\label{Chapter_Families_of_Polytopes_Section_Classical_Polytopes}
\logpage{[ 7, 1, 0 ]}
\hyperdef{L}{X7A281A1C7DCB2D96}{}
{
  

\subsection{\textcolor{Chapter }{Vertex}}
\logpage{[ 7, 1, 1 ]}\nobreak
\hyperdef{L}{X868FA75B794AE1AA}{}
{\noindent\textcolor{FuncColor}{$\triangleright$\enspace\texttt{Vertex({\mdseries\slshape })\index{Vertex@\texttt{Vertex}}
\label{Vertex}
}\hfill{\scriptsize (operation)}}\\


 Returns the universal 0-polytope. }

 

\subsection{\textcolor{Chapter }{Edge}}
\logpage{[ 7, 1, 2 ]}\nobreak
\hyperdef{L}{X7DA6B54B7F300B92}{}
{\noindent\textcolor{FuncColor}{$\triangleright$\enspace\texttt{Edge({\mdseries\slshape })\index{Edge@\texttt{Edge}}
\label{Edge}
}\hfill{\scriptsize (operation)}}\\


 Returns the universal 1-polytope. }

 

\subsection{\textcolor{Chapter }{Pgon (for IsInt)}}
\logpage{[ 7, 1, 3 ]}\nobreak
\hyperdef{L}{X8436B8097852EF9B}{}
{\noindent\textcolor{FuncColor}{$\triangleright$\enspace\texttt{Pgon({\mdseries\slshape p})\index{Pgon@\texttt{Pgon}!for IsInt}
\label{Pgon:for IsInt}
}\hfill{\scriptsize (operation)}}\\


 Returns the p-gon. }

 

\subsection{\textcolor{Chapter }{Cube (for IsInt)}}
\logpage{[ 7, 1, 4 ]}\nobreak
\hyperdef{L}{X8306D0C17A6BDDCE}{}
{\noindent\textcolor{FuncColor}{$\triangleright$\enspace\texttt{Cube({\mdseries\slshape n})\index{Cube@\texttt{Cube}!for IsInt}
\label{Cube:for IsInt}
}\hfill{\scriptsize (operation)}}\\


 Returns the n-cube. }

 

\subsection{\textcolor{Chapter }{HemiCube (for IsInt)}}
\logpage{[ 7, 1, 5 ]}\nobreak
\hyperdef{L}{X828C02F4861F0CD3}{}
{\noindent\textcolor{FuncColor}{$\triangleright$\enspace\texttt{HemiCube({\mdseries\slshape n})\index{HemiCube@\texttt{HemiCube}!for IsInt}
\label{HemiCube:for IsInt}
}\hfill{\scriptsize (operation)}}\\


 Returns the n-hemi-cube. }

 

\subsection{\textcolor{Chapter }{CrossPolytope (for IsInt)}}
\logpage{[ 7, 1, 6 ]}\nobreak
\hyperdef{L}{X78DC5BA486C3288C}{}
{\noindent\textcolor{FuncColor}{$\triangleright$\enspace\texttt{CrossPolytope({\mdseries\slshape n})\index{CrossPolytope@\texttt{CrossPolytope}!for IsInt}
\label{CrossPolytope:for IsInt}
}\hfill{\scriptsize (operation)}}\\


 Returns the n-cross-polytope. }

 

\subsection{\textcolor{Chapter }{HemiCrossPolytope (for IsInt)}}
\logpage{[ 7, 1, 7 ]}\nobreak
\hyperdef{L}{X7B0D84B87F00A73F}{}
{\noindent\textcolor{FuncColor}{$\triangleright$\enspace\texttt{HemiCrossPolytope({\mdseries\slshape n})\index{HemiCrossPolytope@\texttt{HemiCrossPolytope}!for IsInt}
\label{HemiCrossPolytope:for IsInt}
}\hfill{\scriptsize (operation)}}\\


 Returns the n-hemi-cross-polytope. }

 

\subsection{\textcolor{Chapter }{Simplex (for IsInt)}}
\logpage{[ 7, 1, 8 ]}\nobreak
\hyperdef{L}{X82C75D12838D3FD0}{}
{\noindent\textcolor{FuncColor}{$\triangleright$\enspace\texttt{Simplex({\mdseries\slshape n})\index{Simplex@\texttt{Simplex}!for IsInt}
\label{Simplex:for IsInt}
}\hfill{\scriptsize (operation)}}\\


 Returns the n-simplex. }

 

\subsection{\textcolor{Chapter }{CubicTiling (for IsInt)}}
\logpage{[ 7, 1, 9 ]}\nobreak
\hyperdef{L}{X7CCDE7817EC0E5B5}{}
{\noindent\textcolor{FuncColor}{$\triangleright$\enspace\texttt{CubicTiling({\mdseries\slshape n})\index{CubicTiling@\texttt{CubicTiling}!for IsInt}
\label{CubicTiling:for IsInt}
}\hfill{\scriptsize (operation)}}\\


 Returns the rank n+1 polytope; the tiling of E\texttt{\symbol{94}}n by
n-cubes. }

 

\subsection{\textcolor{Chapter }{Dodecahedron}}
\logpage{[ 7, 1, 10 ]}\nobreak
\hyperdef{L}{X81A6D8FE876EB3BE}{}
{\noindent\textcolor{FuncColor}{$\triangleright$\enspace\texttt{Dodecahedron({\mdseries\slshape })\index{Dodecahedron@\texttt{Dodecahedron}}
\label{Dodecahedron}
}\hfill{\scriptsize (operation)}}\\


 Returns the dodecahedron, \texttt{\symbol{123}}5, 3\texttt{\symbol{125}}. }

 

\subsection{\textcolor{Chapter }{HemiDodecahedron}}
\logpage{[ 7, 1, 11 ]}\nobreak
\hyperdef{L}{X7A49325F782047CC}{}
{\noindent\textcolor{FuncColor}{$\triangleright$\enspace\texttt{HemiDodecahedron({\mdseries\slshape })\index{HemiDodecahedron@\texttt{HemiDodecahedron}}
\label{HemiDodecahedron}
}\hfill{\scriptsize (operation)}}\\


 Returns the hemi-dodecahedron, \texttt{\symbol{123}}5,
3\texttt{\symbol{125}}{\textunderscore}5. }

 

\subsection{\textcolor{Chapter }{Icosahedron}}
\logpage{[ 7, 1, 12 ]}\nobreak
\hyperdef{L}{X83E0EF8F7CCD6979}{}
{\noindent\textcolor{FuncColor}{$\triangleright$\enspace\texttt{Icosahedron({\mdseries\slshape })\index{Icosahedron@\texttt{Icosahedron}}
\label{Icosahedron}
}\hfill{\scriptsize (operation)}}\\


 Returns the icosahedron, \texttt{\symbol{123}}3, 5\texttt{\symbol{125}}. }

 

\subsection{\textcolor{Chapter }{HemiIcosahedron}}
\logpage{[ 7, 1, 13 ]}\nobreak
\hyperdef{L}{X7CBB4FC88050D25F}{}
{\noindent\textcolor{FuncColor}{$\triangleright$\enspace\texttt{HemiIcosahedron({\mdseries\slshape })\index{HemiIcosahedron@\texttt{HemiIcosahedron}}
\label{HemiIcosahedron}
}\hfill{\scriptsize (operation)}}\\


 Returns the hemi-icosahedron, \texttt{\symbol{123}}3,
5\texttt{\symbol{125}}{\textunderscore}5. }

 

\subsection{\textcolor{Chapter }{24Cell}}
\logpage{[ 7, 1, 14 ]}\nobreak
\hyperdef{L}{X7C7194D3826A9287}{}
{\noindent\textcolor{FuncColor}{$\triangleright$\enspace\texttt{24Cell({\mdseries\slshape })\index{24Cell@\texttt{24Cell}}
\label{24Cell}
}\hfill{\scriptsize (operation)}}\\


 Returns the 24-cell, \texttt{\symbol{123}}3, 4, 3\texttt{\symbol{125}}. }

 

\subsection{\textcolor{Chapter }{Hemi24Cell}}
\logpage{[ 7, 1, 15 ]}\nobreak
\hyperdef{L}{X863C4F1D85F017B3}{}
{\noindent\textcolor{FuncColor}{$\triangleright$\enspace\texttt{Hemi24Cell({\mdseries\slshape })\index{Hemi24Cell@\texttt{Hemi24Cell}}
\label{Hemi24Cell}
}\hfill{\scriptsize (operation)}}\\


 Returns the hemi-24-cell, \texttt{\symbol{123}}3, 4,
3\texttt{\symbol{125}}{\textunderscore}6. }

 

\subsection{\textcolor{Chapter }{120Cell}}
\logpage{[ 7, 1, 16 ]}\nobreak
\hyperdef{L}{X7A7A51CC878450C4}{}
{\noindent\textcolor{FuncColor}{$\triangleright$\enspace\texttt{120Cell({\mdseries\slshape })\index{120Cell@\texttt{120Cell}}
\label{120Cell}
}\hfill{\scriptsize (operation)}}\\


 Returns the 120-cell, \texttt{\symbol{123}}5, 3, 3\texttt{\symbol{125}}. }

 

\subsection{\textcolor{Chapter }{Hemi120Cell}}
\logpage{[ 7, 1, 17 ]}\nobreak
\hyperdef{L}{X80EBD6F28447F653}{}
{\noindent\textcolor{FuncColor}{$\triangleright$\enspace\texttt{Hemi120Cell({\mdseries\slshape })\index{Hemi120Cell@\texttt{Hemi120Cell}}
\label{Hemi120Cell}
}\hfill{\scriptsize (operation)}}\\


 Returns the hemi-120-cell, \texttt{\symbol{123}}5, 3,
3\texttt{\symbol{125}}{\textunderscore}15. }

 

\subsection{\textcolor{Chapter }{600Cell}}
\logpage{[ 7, 1, 18 ]}\nobreak
\hyperdef{L}{X82FCA8347D417FB6}{}
{\noindent\textcolor{FuncColor}{$\triangleright$\enspace\texttt{600Cell({\mdseries\slshape })\index{600Cell@\texttt{600Cell}}
\label{600Cell}
}\hfill{\scriptsize (operation)}}\\


 Returns the 600-cell, \texttt{\symbol{123}}3, 3, 5\texttt{\symbol{125}}. }

 

\subsection{\textcolor{Chapter }{Hemi600Cell}}
\logpage{[ 7, 1, 19 ]}\nobreak
\hyperdef{L}{X786D2F0A7BB97182}{}
{\noindent\textcolor{FuncColor}{$\triangleright$\enspace\texttt{Hemi600Cell({\mdseries\slshape })\index{Hemi600Cell@\texttt{Hemi600Cell}}
\label{Hemi600Cell}
}\hfill{\scriptsize (operation)}}\\


 Returns the hemi-600-cell, \texttt{\symbol{123}}3, 3,
5\texttt{\symbol{125}}{\textunderscore}15. }

 }

 
\section{\textcolor{Chapter }{Toroids}}\label{Chapter_Families_of_Polytopes_Section_Toroids}
\logpage{[ 7, 2, 0 ]}
\hyperdef{L}{X825A89737F01BFF1}{}
{
  

\subsection{\textcolor{Chapter }{CubicalToroid (for IsInt,IsInt,IsInt)}}
\logpage{[ 7, 2, 1 ]}\nobreak
\hyperdef{L}{X840968B17C45927A}{}
{\noindent\textcolor{FuncColor}{$\triangleright$\enspace\texttt{CubicalToroid({\mdseries\slshape s, k, n})\index{CubicalToroid@\texttt{CubicalToroid}!for IsInt,IsInt,IsInt}
\label{CubicalToroid:for IsInt,IsInt,IsInt}
}\hfill{\scriptsize (operation)}}\\
\textbf{\indent Returns:\ }
IsManiplex 



 Given IsInt triple \mbox{\texttt{\mdseries\slshape s, k, n}}, will return the regular toroid $\{4, 3^{n-2},4\}_{\vec s}$ where $\vec s=(s^k, 0^{n-k})$. }

 
\begin{Verbatim}[commandchars=!@|,fontsize=\small,frame=single,label=Example]
  !gapprompt@gap>| !gapinput@m44:=CubicalToroid(3,2,2);;|
  !gapprompt@gap>| !gapinput@m44=ToroidalMap44([3,3]);|
  true
\end{Verbatim}
 

\subsection{\textcolor{Chapter }{3343Toroid (for IsInt,IsInt)}}
\logpage{[ 7, 2, 2 ]}\nobreak
\hyperdef{L}{X870D878B843C9D5F}{}
{\noindent\textcolor{FuncColor}{$\triangleright$\enspace\texttt{3343Toroid({\mdseries\slshape s, k})\index{3343Toroid@\texttt{3343Toroid}!for IsInt,IsInt}
\label{3343Toroid:for IsInt,IsInt}
}\hfill{\scriptsize (operation)}}\\
\textbf{\indent Returns:\ }
IsManiplex 



 Given IsInt pair \mbox{\texttt{\mdseries\slshape s, k}}, will return the regular toroid $\{3,3,4,3\}_{\vec s}$ where $\vec s=(s^k, 0^{n-k})$. Note that $k$ must be 0 or 1. }

 

\subsection{\textcolor{Chapter }{24CellToroid (for IsInt,IsInt)}}
\logpage{[ 7, 2, 3 ]}\nobreak
\hyperdef{L}{X830F2B047B0C3C13}{}
{\noindent\textcolor{FuncColor}{$\triangleright$\enspace\texttt{24CellToroid({\mdseries\slshape s, k})\index{24CellToroid@\texttt{24CellToroid}!for IsInt,IsInt}
\label{24CellToroid:for IsInt,IsInt}
}\hfill{\scriptsize (operation)}}\\
\textbf{\indent Returns:\ }
IsManiplex 



 Given IsInt pair \mbox{\texttt{\mdseries\slshape s, k}}, will return the regular toroid $\{3,4,3,3\}_{\vec s}$ where $\vec s=(s^k, 0^{n-k})$. Note that $k$ must be 0 or 1. }

 }

 
\section{\textcolor{Chapter }{Uniform Polyhedra}}\label{Chapter_Families_of_Polytopes_Section_Uniform_Polyhedra}
\logpage{[ 7, 3, 0 ]}
\hyperdef{L}{X8708502981F8565A}{}
{
  

\subsection{\textcolor{Chapter }{TruncatedOctahedron}}
\logpage{[ 7, 3, 1 ]}\nobreak
\hyperdef{L}{X82251B0B7FF0867C}{}
{\noindent\textcolor{FuncColor}{$\triangleright$\enspace\texttt{TruncatedOctahedron({\mdseries\slshape })\index{TruncatedOctahedron@\texttt{TruncatedOctahedron}}
\label{TruncatedOctahedron}
}\hfill{\scriptsize (operation)}}\\
\textbf{\indent Returns:\ }
maniplex 



 Constructs the truncated octahedron. }

 

\subsection{\textcolor{Chapter }{TruncatedCube}}
\logpage{[ 7, 3, 2 ]}\nobreak
\hyperdef{L}{X83E8732F7B5723CA}{}
{\noindent\textcolor{FuncColor}{$\triangleright$\enspace\texttt{TruncatedCube({\mdseries\slshape })\index{TruncatedCube@\texttt{TruncatedCube}}
\label{TruncatedCube}
}\hfill{\scriptsize (operation)}}\\
\textbf{\indent Returns:\ }
maniplex 



 Constructs the truncated octahedron. }

 

\subsection{\textcolor{Chapter }{Icosadodecahedron}}
\logpage{[ 7, 3, 3 ]}\nobreak
\hyperdef{L}{X7C262C2B7B88190E}{}
{\noindent\textcolor{FuncColor}{$\triangleright$\enspace\texttt{Icosadodecahedron({\mdseries\slshape })\index{Icosadodecahedron@\texttt{Icosadodecahedron}}
\label{Icosadodecahedron}
}\hfill{\scriptsize (operation)}}\\
\textbf{\indent Returns:\ }
maniplex 



 Constructs the icosadodecahedron. }

 

\subsection{\textcolor{Chapter }{TruncatedIcosahedron}}
\logpage{[ 7, 3, 4 ]}\nobreak
\hyperdef{L}{X81B089967C16B074}{}
{\noindent\textcolor{FuncColor}{$\triangleright$\enspace\texttt{TruncatedIcosahedron({\mdseries\slshape })\index{TruncatedIcosahedron@\texttt{TruncatedIcosahedron}}
\label{TruncatedIcosahedron}
}\hfill{\scriptsize (operation)}}\\
\textbf{\indent Returns:\ }
maniplex 



 Constructs the truncated icosahedron. }

 

\subsection{\textcolor{Chapter }{SmallRhombicuboctahedron}}
\logpage{[ 7, 3, 5 ]}\nobreak
\hyperdef{L}{X838FD696802FEFD8}{}
{\noindent\textcolor{FuncColor}{$\triangleright$\enspace\texttt{SmallRhombicuboctahedron({\mdseries\slshape })\index{SmallRhombicuboctahedron@\texttt{SmallRhombicuboctahedron}}
\label{SmallRhombicuboctahedron}
}\hfill{\scriptsize (operation)}}\\
\textbf{\indent Returns:\ }
maniplex 



 Constructs the small rhombicuboctahedron. }

 

\subsection{\textcolor{Chapter }{Pseudorhombicuboctahedron}}
\logpage{[ 7, 3, 6 ]}\nobreak
\hyperdef{L}{X84853A907A3EF1C5}{}
{\noindent\textcolor{FuncColor}{$\triangleright$\enspace\texttt{Pseudorhombicuboctahedron({\mdseries\slshape })\index{Pseudorhombicuboctahedron@\texttt{Pseudorhombicuboctahedron}}
\label{Pseudorhombicuboctahedron}
}\hfill{\scriptsize (operation)}}\\
\textbf{\indent Returns:\ }
maniplex 



 Constructs the pseudorhombicuboctahedron. }

 

\subsection{\textcolor{Chapter }{SnubCube}}
\logpage{[ 7, 3, 7 ]}\nobreak
\hyperdef{L}{X7EA8AA0C84A43469}{}
{\noindent\textcolor{FuncColor}{$\triangleright$\enspace\texttt{SnubCube({\mdseries\slshape })\index{SnubCube@\texttt{SnubCube}}
\label{SnubCube}
}\hfill{\scriptsize (operation)}}\\
\textbf{\indent Returns:\ }
maniplex 



 Constructs the snub cube. }

 

\subsection{\textcolor{Chapter }{SmallRhombicosidodecahedron}}
\logpage{[ 7, 3, 8 ]}\nobreak
\hyperdef{L}{X7A2FC231845204F5}{}
{\noindent\textcolor{FuncColor}{$\triangleright$\enspace\texttt{SmallRhombicosidodecahedron({\mdseries\slshape })\index{SmallRhombicosidodecahedron@\texttt{SmallRhombicosidodecahedron}}
\label{SmallRhombicosidodecahedron}
}\hfill{\scriptsize (operation)}}\\
\textbf{\indent Returns:\ }
maniplex 



 Constructs the small rhombicosidodecahedron. }

 

\subsection{\textcolor{Chapter }{GreatRhombicosidodecahedron}}
\logpage{[ 7, 3, 9 ]}\nobreak
\hyperdef{L}{X87A29A6B8375339F}{}
{\noindent\textcolor{FuncColor}{$\triangleright$\enspace\texttt{GreatRhombicosidodecahedron({\mdseries\slshape })\index{GreatRhombicosidodecahedron@\texttt{GreatRhombicosidodecahedron}}
\label{GreatRhombicosidodecahedron}
}\hfill{\scriptsize (operation)}}\\
\textbf{\indent Returns:\ }
maniplex 



 Constructs the great rhombicosidodecahedron. }

 

\subsection{\textcolor{Chapter }{SnubDodecahedron}}
\logpage{[ 7, 3, 10 ]}\nobreak
\hyperdef{L}{X874B90BD82EA1192}{}
{\noindent\textcolor{FuncColor}{$\triangleright$\enspace\texttt{SnubDodecahedron({\mdseries\slshape })\index{SnubDodecahedron@\texttt{SnubDodecahedron}}
\label{SnubDodecahedron}
}\hfill{\scriptsize (operation)}}\\
\textbf{\indent Returns:\ }
maniplex 



 Constructs the small snub dodecahedron. }

 

\subsection{\textcolor{Chapter }{TruncatedDodecahedron}}
\logpage{[ 7, 3, 11 ]}\nobreak
\hyperdef{L}{X79A2DEE48126C748}{}
{\noindent\textcolor{FuncColor}{$\triangleright$\enspace\texttt{TruncatedDodecahedron({\mdseries\slshape })\index{TruncatedDodecahedron@\texttt{TruncatedDodecahedron}}
\label{TruncatedDodecahedron}
}\hfill{\scriptsize (operation)}}\\
\textbf{\indent Returns:\ }
maniplex 



 Constructs the truncated dodecahedron. }

 

\subsection{\textcolor{Chapter }{GreatRhombicuboctahedron}}
\logpage{[ 7, 3, 12 ]}\nobreak
\hyperdef{L}{X7D08FB508708D8B2}{}
{\noindent\textcolor{FuncColor}{$\triangleright$\enspace\texttt{GreatRhombicuboctahedron({\mdseries\slshape })\index{GreatRhombicuboctahedron@\texttt{GreatRhombicuboctahedron}}
\label{GreatRhombicuboctahedron}
}\hfill{\scriptsize (operation)}}\\
\textbf{\indent Returns:\ }
maniplex 



 Constructs the great rhombicuboctahedron. }

 }

 }

   
\chapter{\textcolor{Chapter }{Groups}}\label{Chapter_Groups}
\logpage{[ 8, 0, 0 ]}
\hyperdef{L}{X8716635F7951801B}{}
{
  
\section{\textcolor{Chapter }{Groups}}\label{Chapter_Groups_Section_Groups}
\logpage{[ 8, 1, 0 ]}
\hyperdef{L}{X8716635F7951801B}{}
{
  

\subsection{\textcolor{Chapter }{AutomorphismGroup (for IsManiplex)}}
\logpage{[ 8, 1, 1 ]}\nobreak
\hyperdef{L}{X78746F0385216D57}{}
{\noindent\textcolor{FuncColor}{$\triangleright$\enspace\texttt{AutomorphismGroup({\mdseries\slshape M})\index{AutomorphismGroup@\texttt{AutomorphismGroup}!for IsManiplex}
\label{AutomorphismGroup:for IsManiplex}
}\hfill{\scriptsize (attribute)}}\\


 Returns the automorphism group of \mbox{\texttt{\mdseries\slshape M}}. This group is not guaranteed to be in any particular form. }

 

\subsection{\textcolor{Chapter }{AutomorphismGroupFpGroup (for IsManiplex)}}
\logpage{[ 8, 1, 2 ]}\nobreak
\hyperdef{L}{X837641127998923F}{}
{\noindent\textcolor{FuncColor}{$\triangleright$\enspace\texttt{AutomorphismGroupFpGroup({\mdseries\slshape M})\index{AutomorphismGroupFpGroup@\texttt{AutomorphismGroupFpGroup}!for IsManiplex}
\label{AutomorphismGroupFpGroup:for IsManiplex}
}\hfill{\scriptsize (attribute)}}\\


 Returns the automorphism group of \mbox{\texttt{\mdseries\slshape M}} as a finitely presented group. }

 

\subsection{\textcolor{Chapter }{AutomorphismGroupPermGroup (for IsManiplex)}}
\logpage{[ 8, 1, 3 ]}\nobreak
\hyperdef{L}{X7D6651587DF3790A}{}
{\noindent\textcolor{FuncColor}{$\triangleright$\enspace\texttt{AutomorphismGroupPermGroup({\mdseries\slshape M})\index{AutomorphismGroupPermGroup@\texttt{AutomorphismGroupPermGroup}!for IsManiplex}
\label{AutomorphismGroupPermGroup:for IsManiplex}
}\hfill{\scriptsize (attribute)}}\\


 Returns the automorphism group of \mbox{\texttt{\mdseries\slshape M}} as a permutation group. }

 

\subsection{\textcolor{Chapter }{AutomorphismGroupOnFlags (for IsManiplex)}}
\logpage{[ 8, 1, 4 ]}\nobreak
\hyperdef{L}{X810F4F4A864B1B39}{}
{\noindent\textcolor{FuncColor}{$\triangleright$\enspace\texttt{AutomorphismGroupOnFlags({\mdseries\slshape M})\index{AutomorphismGroupOnFlags@\texttt{AutomorphismGroupOnFlags}!for IsManiplex}
\label{AutomorphismGroupOnFlags:for IsManiplex}
}\hfill{\scriptsize (attribute)}}\\


 Returns the automorphism group of \mbox{\texttt{\mdseries\slshape M}} as a permutation group action on the flags of \mbox{\texttt{\mdseries\slshape M}}. }

 

\subsection{\textcolor{Chapter }{ConnectionGroup (for IsManiplex)}}
\logpage{[ 8, 1, 5 ]}\nobreak
\hyperdef{L}{X7815931C7E926F4F}{}
{\noindent\textcolor{FuncColor}{$\triangleright$\enspace\texttt{ConnectionGroup({\mdseries\slshape M})\index{ConnectionGroup@\texttt{ConnectionGroup}!for IsManiplex}
\label{ConnectionGroup:for IsManiplex}
}\hfill{\scriptsize (attribute)}}\\


 Returns the connection group of \mbox{\texttt{\mdseries\slshape M}} as a permutation group. We may eventually allow other types of connection
groups. Synonym: MonodromyGroup }

 

\subsection{\textcolor{Chapter }{EvenConnectionGroup (for IsManiplex)}}
\logpage{[ 8, 1, 6 ]}\nobreak
\hyperdef{L}{X7CE8CFB183A4FFEE}{}
{\noindent\textcolor{FuncColor}{$\triangleright$\enspace\texttt{EvenConnectionGroup({\mdseries\slshape M})\index{EvenConnectionGroup@\texttt{EvenConnectionGroup}!for IsManiplex}
\label{EvenConnectionGroup:for IsManiplex}
}\hfill{\scriptsize (attribute)}}\\


 Returns the even-word subgroup of the connection group of \mbox{\texttt{\mdseries\slshape M}} as a permutation group. }

 

\subsection{\textcolor{Chapter }{RotationGroup (for IsManiplex)}}
\logpage{[ 8, 1, 7 ]}\nobreak
\hyperdef{L}{X7D194510834667C8}{}
{\noindent\textcolor{FuncColor}{$\triangleright$\enspace\texttt{RotationGroup({\mdseries\slshape M})\index{RotationGroup@\texttt{RotationGroup}!for IsManiplex}
\label{RotationGroup:for IsManiplex}
}\hfill{\scriptsize (attribute)}}\\


 Returns the rotation group of \mbox{\texttt{\mdseries\slshape M}}. This group is not guaranteed to be in any particular form. }

 

\subsection{\textcolor{Chapter }{ChiralityGroup (for IsRotaryManiplex)}}
\logpage{[ 8, 1, 8 ]}\nobreak
\hyperdef{L}{X7FDCA212782DC0E3}{}
{\noindent\textcolor{FuncColor}{$\triangleright$\enspace\texttt{ChiralityGroup({\mdseries\slshape M})\index{ChiralityGroup@\texttt{ChiralityGroup}!for IsRotaryManiplex}
\label{ChiralityGroup:for IsRotaryManiplex}
}\hfill{\scriptsize (attribute)}}\\


 Returns the chirality group of the rotary maniplex \mbox{\texttt{\mdseries\slshape M}}. This is the kernel of the group epimorphism from the rotation group of \mbox{\texttt{\mdseries\slshape M}} to the rotation group of its maximal reflexible quotient. In particular, the
chirality group is trivial if and only if \mbox{\texttt{\mdseries\slshape M}} is reflexible. }

 

\subsection{\textcolor{Chapter }{ExtraRelators (for IsReflexibleManiplex)}}
\logpage{[ 8, 1, 9 ]}\nobreak
\hyperdef{L}{X7D2CE6707B452C0D}{}
{\noindent\textcolor{FuncColor}{$\triangleright$\enspace\texttt{ExtraRelators({\mdseries\slshape M})\index{ExtraRelators@\texttt{ExtraRelators}!for IsReflexibleManiplex}
\label{ExtraRelators:for IsReflexibleManiplex}
}\hfill{\scriptsize (attribute)}}\\


 For a reflexible maniplex \mbox{\texttt{\mdseries\slshape M}}, returns the relators needed to define its automorphism group as a quotient
of the string Coxeter group given by its Schlafli symbol. Not particularly
robust at the moment. }

 

\subsection{\textcolor{Chapter }{ExtraRotRelators (for IsRotaryManiplex)}}
\logpage{[ 8, 1, 10 ]}\nobreak
\hyperdef{L}{X797752A88078CF22}{}
{\noindent\textcolor{FuncColor}{$\triangleright$\enspace\texttt{ExtraRotRelators({\mdseries\slshape M})\index{ExtraRotRelators@\texttt{ExtraRotRelators}!for IsRotaryManiplex}
\label{ExtraRotRelators:for IsRotaryManiplex}
}\hfill{\scriptsize (attribute)}}\\


 For a reflexible maniplex \mbox{\texttt{\mdseries\slshape M}}, returns the relators needed to define its rotation group as a quotient of
the rotation group of a string Coxeter group given by its Schlafli symbol. Not
particularly robust at the moment. }

 

\subsection{\textcolor{Chapter }{IsStringC (for IsGroup)}}
\logpage{[ 8, 1, 11 ]}\nobreak
\hyperdef{L}{X835FA8527E2442F4}{}
{\noindent\textcolor{FuncColor}{$\triangleright$\enspace\texttt{IsStringC({\mdseries\slshape G})\index{IsStringC@\texttt{IsStringC}!for IsGroup}
\label{IsStringC:for IsGroup}
}\hfill{\scriptsize (operation)}}\\


 For an sggi \mbox{\texttt{\mdseries\slshape G}}, returns whether the group is a string C group. It does not check whether \mbox{\texttt{\mdseries\slshape G}} is an sggi. }

 

\subsection{\textcolor{Chapter }{IsStringCPlus (for IsGroup)}}
\logpage{[ 8, 1, 12 ]}\nobreak
\hyperdef{L}{X8445AFAB7CF3F319}{}
{\noindent\textcolor{FuncColor}{$\triangleright$\enspace\texttt{IsStringCPlus({\mdseries\slshape G})\index{IsStringCPlus@\texttt{IsStringCPlus}!for IsGroup}
\label{IsStringCPlus:for IsGroup}
}\hfill{\scriptsize (operation)}}\\


 For a "string rotation group" \mbox{\texttt{\mdseries\slshape G}}, returns whether the group is a string C+ group. It does not check whether \mbox{\texttt{\mdseries\slshape G}} is a string rotation group. }

 }

 }

   
\chapter{\textcolor{Chapter }{Mixing of Maniplexes}}\label{Chapter_Mixing_of_Maniplexes}
\logpage{[ 9, 0, 0 ]}
\hyperdef{L}{X7CB85C4182EF7F87}{}
{
  
\section{\textcolor{Chapter }{Mixing of Maniplexes functions}}\label{Chapter_Mixing_of_Maniplexes_Section_Mixing_of_Maniplexes_functions}
\logpage{[ 9, 1, 0 ]}
\hyperdef{L}{X7EE6BE9E7BB852DC}{}
{
  

\subsection{\textcolor{Chapter }{Mix (for IsPermGroup, IsPermGroup)}}
\logpage{[ 9, 1, 1 ]}\nobreak
\hyperdef{L}{X7AE60EBC7BA2392A}{}
{\noindent\textcolor{FuncColor}{$\triangleright$\enspace\texttt{Mix({\mdseries\slshape permgroup, permgroup})\index{Mix@\texttt{Mix}!for IsPermGroup, IsPermGroup}
\label{Mix:for IsPermGroup, IsPermGroup}
}\hfill{\scriptsize (operation)}}\\
\textbf{\indent Returns:\ }
\texttt{IsGroup }. 



 Given two (permutation) groups returns the mix of those groups. Note, also
works with FPgroups. }

 Here we build the mix of the connection groups of a 3-cube and an edge. 
\begin{Verbatim}[commandchars=!@|,fontsize=\small,frame=single,label=Example]
  !gapprompt@gap>| !gapinput@g1:=ConnectionGroup(Cube(3));|
  <permutation group with 3 generators>
  !gapprompt@gap>| !gapinput@g2:=ConnectionGroup(Edge());|
  Group([ (1,2) ])
  !gapprompt@gap>| !gapinput@Mix(g1,g2);|
  <permutation group with 3 generators>
\end{Verbatim}
 

\subsection{\textcolor{Chapter }{Mix (for IsFpGroup, IsFpGroup)}}
\logpage{[ 9, 1, 2 ]}\nobreak
\hyperdef{L}{X79E9662F7D4043A1}{}
{\noindent\textcolor{FuncColor}{$\triangleright$\enspace\texttt{Mix({\mdseries\slshape fpgroup, fpgroup})\index{Mix@\texttt{Mix}!for IsFpGroup, IsFpGroup}
\label{Mix:for IsFpGroup, IsFpGroup}
}\hfill{\scriptsize (operation)}}\\


 Returns the Mix of two Finitely Presented groups gp and gq. }

 

\subsection{\textcolor{Chapter }{Mix (for IsReflexibleManiplex, IsReflexibleManiplex)}}
\logpage{[ 9, 1, 3 ]}\nobreak
\hyperdef{L}{X8441A4DC7C59EDB5}{}
{\noindent\textcolor{FuncColor}{$\triangleright$\enspace\texttt{Mix({\mdseries\slshape maniplex, maniplex})\index{Mix@\texttt{Mix}!for IsReflexibleManiplex, IsReflexibleManiplex}
\label{Mix:for IsReflexibleManiplex, IsReflexibleManiplex}
}\hfill{\scriptsize (operation)}}\\
\textbf{\indent Returns:\ }
\texttt{IsReflexibleManiplex }. 



 Given maniplexes returns the IsReflexibleManiplex from the mix of their
connection groups }

 

\subsection{\textcolor{Chapter }{Comix (for IsFpGroup, IsFpGroup)}}
\logpage{[ 9, 1, 4 ]}\nobreak
\hyperdef{L}{X8688A51E816A5176}{}
{\noindent\textcolor{FuncColor}{$\triangleright$\enspace\texttt{Comix({\mdseries\slshape fpgroup, fpgroup})\index{Comix@\texttt{Comix}!for IsFpGroup, IsFpGroup}
\label{Comix:for IsFpGroup, IsFpGroup}
}\hfill{\scriptsize (operation)}}\\


 Returns the comix of two Finitely Presented groups gp and gq. }

 

\subsection{\textcolor{Chapter }{Comix (for IsReflexibleManiplex, IsReflexibleManiplex)}}
\logpage{[ 9, 1, 5 ]}\nobreak
\hyperdef{L}{X8287C9F684CA66F5}{}
{\noindent\textcolor{FuncColor}{$\triangleright$\enspace\texttt{Comix({\mdseries\slshape maniplex, maniplex})\index{Comix@\texttt{Comix}!for IsReflexibleManiplex, IsReflexibleManiplex}
\label{Comix:for IsReflexibleManiplex, IsReflexibleManiplex}
}\hfill{\scriptsize (operation)}}\\
\textbf{\indent Returns:\ }
\texttt{IsReflexibleManiplex }. 



 Given maniplexes returns the IsReflexibleManiplex from the comix of their
connection groups }

 

\subsection{\textcolor{Chapter }{CtoL (for IsInt,IsInt,IsInt,IsInt)}}
\logpage{[ 9, 1, 6 ]}\nobreak
\hyperdef{L}{X7CD7AB1C7DF21CFE}{}
{\noindent\textcolor{FuncColor}{$\triangleright$\enspace\texttt{CtoL({\mdseries\slshape int, int, int, int})\index{CtoL@\texttt{CtoL}!for IsInt,IsInt,IsInt,IsInt}
\label{CtoL:for IsInt,IsInt,IsInt,IsInt}
}\hfill{\scriptsize (operation)}}\\
\textbf{\indent Returns:\ }
\texttt{IsInteger }. 



 CtoL Returns an integer between 1 and N*M associated with the pair [a,b]. LtoC
Returns an ordered pair [a,b] associated with the integer between 1 and N*M. a
should range between 1 and N, and b should range between 1 and M N is how many
columns (x coordinates), M is how many rows (y coordinates) in a matrix
Functions are inverses. }

 

\subsection{\textcolor{Chapter }{FlagMix (for IsManiplex, IsManiplex)}}
\logpage{[ 9, 1, 7 ]}\nobreak
\hyperdef{L}{X853A36497F3AF41E}{}
{\noindent\textcolor{FuncColor}{$\triangleright$\enspace\texttt{FlagMix({\mdseries\slshape permgroup, permgroup})\index{FlagMix@\texttt{FlagMix}!for IsManiplex, IsManiplex}
\label{FlagMix:for IsManiplex, IsManiplex}
}\hfill{\scriptsize (operation)}}\\
\textbf{\indent Returns:\ }
\texttt{IsManiplex }. 



 Given two (permutation) groups gp, gg this returns the maniplex of the "flag"
mix of two maniplexes with connection groups gp and gq. }

 }

 }

   
\chapter{\textcolor{Chapter }{Properties}}\label{Chapter_Properties}
\logpage{[ 10, 0, 0 ]}
\hyperdef{L}{X871597447BB998A1}{}
{
  
\section{\textcolor{Chapter }{Orientability}}\label{Chapter_Properties_Section_Orientability}
\logpage{[ 10, 1, 0 ]}
\hyperdef{L}{X861E4BAD800B2785}{}
{
  

\subsection{\textcolor{Chapter }{IsOrientable (for IsManiplex)}}
\logpage{[ 10, 1, 1 ]}\nobreak
\hyperdef{L}{X7DCAF9D27F36EBD3}{}
{\noindent\textcolor{FuncColor}{$\triangleright$\enspace\texttt{IsOrientable({\mdseries\slshape M})\index{IsOrientable@\texttt{IsOrientable}!for IsManiplex}
\label{IsOrientable:for IsManiplex}
}\hfill{\scriptsize (property)}}\\
\textbf{\indent Returns:\ }
\texttt{true} or \texttt{false} 



 A maniplex is orientable if its flag graph is bipartite. }

 

\subsection{\textcolor{Chapter }{IsIOrientable (for IsManiplex, IsList)}}
\logpage{[ 10, 1, 2 ]}\nobreak
\hyperdef{L}{X79F849DD7F778616}{}
{\noindent\textcolor{FuncColor}{$\triangleright$\enspace\texttt{IsIOrientable({\mdseries\slshape M, I})\index{IsIOrientable@\texttt{IsIOrientable}!for IsManiplex, IsList}
\label{IsIOrientable:for IsManiplex, IsList}
}\hfill{\scriptsize (operation)}}\\


 For a subset I of \texttt{\symbol{123}}0, ..., n-1\texttt{\symbol{125}}, a
maniplex is I-orientable if every closed path in its flag graph contains an
even number of edges with colors in I. }

 

\subsection{\textcolor{Chapter }{IsVertexBipartite (for IsManiplex)}}
\logpage{[ 10, 1, 3 ]}\nobreak
\hyperdef{L}{X877609AB7ABD24AC}{}
{\noindent\textcolor{FuncColor}{$\triangleright$\enspace\texttt{IsVertexBipartite({\mdseries\slshape M})\index{IsVertexBipartite@\texttt{IsVertexBipartite}!for IsManiplex}
\label{IsVertexBipartite:for IsManiplex}
}\hfill{\scriptsize (property)}}\\
\textbf{\indent Returns:\ }
\texttt{true} or \texttt{false} 



 A maniplex is vertex-bipartite if its 1-skeleton is bipartite. This is
equivalent to being I-orientable for I =
\texttt{\symbol{123}}0\texttt{\symbol{125}}. }

 

\subsection{\textcolor{Chapter }{IsFacetBipartite (for IsManiplex)}}
\logpage{[ 10, 1, 4 ]}\nobreak
\hyperdef{L}{X7BD9E8A87D6E8FAB}{}
{\noindent\textcolor{FuncColor}{$\triangleright$\enspace\texttt{IsFacetBipartite({\mdseries\slshape M})\index{IsFacetBipartite@\texttt{IsFacetBipartite}!for IsManiplex}
\label{IsFacetBipartite:for IsManiplex}
}\hfill{\scriptsize (property)}}\\
\textbf{\indent Returns:\ }
\texttt{true} or \texttt{false} 



 A maniplex is facet-bipartite if the 1-skeleton of its dual is bipartite. This
is equivalent to being I-orientable for I =
\texttt{\symbol{123}}n-1\texttt{\symbol{125}}. }

 

\subsection{\textcolor{Chapter }{OrientableCover (for IsManiplex)}}
\logpage{[ 10, 1, 5 ]}\nobreak
\hyperdef{L}{X7BAA22327D298902}{}
{\noindent\textcolor{FuncColor}{$\triangleright$\enspace\texttt{OrientableCover({\mdseries\slshape M})\index{OrientableCover@\texttt{OrientableCover}!for IsManiplex}
\label{OrientableCover:for IsManiplex}
}\hfill{\scriptsize (attribute)}}\\


 Returns the minimal \emph{orientable cover} of the maniplex \mbox{\texttt{\mdseries\slshape M}}. }

 

\subsection{\textcolor{Chapter }{IOrientableCover (for IsManiplex, IsList)}}
\logpage{[ 10, 1, 6 ]}\nobreak
\hyperdef{L}{X81EF6062840E85AF}{}
{\noindent\textcolor{FuncColor}{$\triangleright$\enspace\texttt{IOrientableCover({\mdseries\slshape M})\index{IOrientableCover@\texttt{IOrientableCover}!for IsManiplex, IsList}
\label{IOrientableCover:for IsManiplex, IsList}
}\hfill{\scriptsize (operation)}}\\


 Returns the minimal \emph{I-orientable cover} of the maniplex \mbox{\texttt{\mdseries\slshape M}}. }

 }

 }

   
\chapter{\textcolor{Chapter }{Posets}}\label{Chapter_Posets}
\logpage{[ 11, 0, 0 ]}
\hyperdef{L}{X79540DAB85902432}{}
{
  I'm in the process of reconciling all of this, but there are going to be a
number of ways to \texttt{define} a poset: 
\begin{itemize}
\item  As an \texttt{IsPosetOfFlags}, where the underlying description is an ordered list of length $n+2$. Each of the $n+2$ list elements is a list of faces, and the assumption is that these are the
faces of rank $i-2$, where $i$ is the index in the master list (e.g., \texttt{l[1][1]} would usually correspond to the unique $-1$ face of a polytope -- and there won't be an \texttt{l[1][2]}). Each face is then a list of the flags incident with that face. 
\item  As an \texttt{IsPosetOfIndices}, where the underlying description is a binary relation on a set of indices,
which correspond to labels for the elements of the poset. 
\item  If the poset is known to be atomic, then by a description of the faces in
terms of the atoms... usually we'll just need the list of the elements of
maximal rank, from which all other elements may be obtained. 
\item  As an \texttt{IsPosetOfElements}, where the elements could be anything, and we have a known function
determining the partial order on the elements. 
\end{itemize}
 Usually, we assume that the poset will have a natural rank function on it. 
\section{\textcolor{Chapter }{Poset constructors}}\label{Chapter_Posets_Section_Poset_constructors}
\logpage{[ 11, 1, 0 ]}
\hyperdef{L}{X87ED175B85A4C5B7}{}
{
  

\subsection{\textcolor{Chapter }{PosetFromFaceListOfFlags (for IsList)}}
\logpage{[ 11, 1, 1 ]}\nobreak
\hyperdef{L}{X7D495B407CDFAEA2}{}
{\noindent\textcolor{FuncColor}{$\triangleright$\enspace\texttt{PosetFromFaceListOfFlags({\mdseries\slshape list})\index{PosetFromFaceListOfFlags@\texttt{PosetFromFaceListOfFlags}!for IsList}
\label{PosetFromFaceListOfFlags:for IsList}
}\hfill{\scriptsize (operation)}}\\
\textbf{\indent Returns:\ }
\texttt{IsPosetOfFlags}. Note that the function is INTENTIONALLY agnostic about whether it is being
given full poset or not. 



 Given a \mbox{\texttt{\mdseries\slshape list}} of lists of faces in increasing rank, where each face is described by the
incident flags, gives you a IsPosetOfFlags object back. Note that if you don't
include faces or ranks, this function doesn't know about about them! 

 \emph{Notes:} I'm rethinking this a little bit. Since we \emph{already} know that the poset admits a description in terms of faces described by
incident flags, then we have a set with a natural rank function, and all
maximal chains must be the same length. I think I should probably take
advantage of that a little more. Will rewrite the code to take advantage of
the assumptions that IsP1 and IsP2 are true. I'll try not to break things. }

 Here we have a poset using the \texttt{IsPosetOfFlags} description for the triangle. 
\begin{Verbatim}[commandchars=!@|,fontsize=\small,frame=single,label=Example]
  !gapprompt@gap>| !gapinput@poset:=PosetFromFaceListOfFlags([[[1,2,3,4,5,6]],[[1,2],[3,6],[4,5]],[[1,4],[2,3],[5,6]],[[1,2,3,4,5,6]]]);|
  A poset using the IsPosetOfFlags representation with 8 faces.
  !gapprompt@gap>| !gapinput@FaceListOfPoset(poset);|
  [ [ [ 1, 2, 3, 4, 5, 6 ] ], [ [ 1, 2 ], [ 3, 6 ], [ 4, 5 ] ], [ [ 1, 4 ], [ 2, 3 ], [ 5, 6 ] ], [ [ 1, 2, 3, 4, 5, 6 ] ] ]
\end{Verbatim}
 

\subsection{\textcolor{Chapter }{PosetFromConnectionGroup (for IsPermGroup)}}
\logpage{[ 11, 1, 2 ]}\nobreak
\hyperdef{L}{X8720C4757EE12082}{}
{\noindent\textcolor{FuncColor}{$\triangleright$\enspace\texttt{PosetFromConnectionGroup({\mdseries\slshape g})\index{PosetFromConnectionGroup@\texttt{PosetFromConnectionGroup}!for IsPermGroup}
\label{PosetFromConnectionGroup:for IsPermGroup}
}\hfill{\scriptsize (operation)}}\\
\textbf{\indent Returns:\ }
\texttt{IsPosetOfFlags} with \texttt{IsP1}=true. 



 Given a group, returns a poset with an internal representation as a list of
faces ordered by rank, where each face is represented as a list of the flags
it contains. Note that this function includes the minimal (empty) face and the
maximal face of the maniplex. Note that the $i$-faces correspond to the $i+1$ item in the list because of how GAP indexes lists. }

 
\begin{Verbatim}[commandchars=!@|,fontsize=\small,frame=single,label=Example]
  !gapprompt@gap>| !gapinput@g:=Group([(1,4)(2,3)(5,6),(1,2)(3,6)(4,5)]);|
  Group([ (1,4)(2,3)(5,6), (1,2)(3,6)(4,5) ])
  !gapprompt@gap>| !gapinput@PosetFromConnectionGroup(g);|
  A poset using the IsPosetOfFlags representation with 8 faces.
\end{Verbatim}
 

\subsection{\textcolor{Chapter }{PosetFromManiplex (for IsManiplex)}}
\logpage{[ 11, 1, 3 ]}\nobreak
\hyperdef{L}{X7C4F3D307C5A7E7A}{}
{\noindent\textcolor{FuncColor}{$\triangleright$\enspace\texttt{PosetFromManiplex({\mdseries\slshape mani})\index{PosetFromManiplex@\texttt{PosetFromManiplex}!for IsManiplex}
\label{PosetFromManiplex:for IsManiplex}
}\hfill{\scriptsize (operation)}}\\
\textbf{\indent Returns:\ }
\texttt{IsPosetOfFlags} 



 Given a maniplex, returns a poset of the maniplex with an internal
representation as a list of faces ordered by rank, where each face is
represented as a list of the flags it contains. Note that this function does
include the minimal (empty) face and the maximal face of the maniplex. Note
that the $i$-faces correspond to the $i+1$ item in the list because of how GAP indexes lists. }

 
\begin{Verbatim}[commandchars=!@|,fontsize=\small,frame=single,label=Example]
  !gapprompt@gap>| !gapinput@p:=HemiCube(3);|
  Regular 3-polytope of type [ 4, 3 ] with 24 flags
  !gapprompt@gap>| !gapinput@PosetFromManiplex(p);|
  A poset using the IsPosetOfFlags representation with 15 faces.
\end{Verbatim}
 

\subsection{\textcolor{Chapter }{PosetFromPartialOrder (for IsBinaryRelation)}}
\logpage{[ 11, 1, 4 ]}\nobreak
\hyperdef{L}{X7ECC2FBC7AF3A960}{}
{\noindent\textcolor{FuncColor}{$\triangleright$\enspace\texttt{PosetFromPartialOrder({\mdseries\slshape partialOrder})\index{PosetFromPartialOrder@\texttt{PosetFromPartialOrder}!for IsBinaryRelation}
\label{PosetFromPartialOrder:for IsBinaryRelation}
}\hfill{\scriptsize (operation)}}\\
\textbf{\indent Returns:\ }
\texttt{IsPosetOfIndices} 



 Given a partial order on a finite set of size $n$, this function will create a partial order on \texttt{[1..n]}. }

 
\begin{Verbatim}[commandchars=!@|,fontsize=\small,frame=single,label=Example]
  !gapprompt@gap>| !gapinput@l:=List([[1,1],[1,2],[1,3],[1,4],[2,4],[2,2],[3,3],[4,4]],x->Tuple(x));|
  !gapprompt@gap>| !gapinput@r:=BinaryRelationByElements(Domain([1..4]), l);|
  <general mapping: Domain([ 1 .. 4 ]) -> Domain([ 1 .. 4 ]) >
  !gapprompt@gap>| !gapinput@poset:=PosetFromPartialOrder(r);|
  A poset using the IsPosetOfIndices representation 
  !gapprompt@gap>| !gapinput@h:=HasseDiagramBinaryRelation(PartialOrder(poset));|
  <general mapping: Domain([ 1 .. 4 ]) -> Domain([ 1 .. 4 ]) >
  !gapprompt@gap>| !gapinput@Successors(h);|
  [ [ 2, 3 ], [ 4 ], [  ], [  ] ]
\end{Verbatim}
 Note that what we've accomplished here is the poset containing the elements 1,
2, 3, 4 with partial order determined by whether the first element divides the
second. The essential information about the poset can be obtained from the
Hasse diagram. 

\subsection{\textcolor{Chapter }{PosetFromElements (for IsList,IsFunction)}}
\logpage{[ 11, 1, 5 ]}\nobreak
\hyperdef{L}{X79AEF6FF8699020E}{}
{\noindent\textcolor{FuncColor}{$\triangleright$\enspace\texttt{PosetFromElements({\mdseries\slshape list{\textunderscore}of{\textunderscore}faces, func})\index{PosetFromElements@\texttt{PosetFromElements}!for IsList,IsFunction}
\label{PosetFromElements:for IsList,IsFunction}
}\hfill{\scriptsize (operation)}}\\
\textbf{\indent Returns:\ }
\texttt{IsPosetOfElements} 



 This is for gathering elements with a known ordering \mbox{\texttt{\mdseries\slshape func}} on two variables into a poset. Also note, the expectation is that \mbox{\texttt{\mdseries\slshape func}} behaves similarly to IsSubset, i.e., \mbox{\texttt{\mdseries\slshape func}} (x,y)=true means $y$ is less than $x$ in the order. }

 
\begin{Verbatim}[commandchars=!@|,fontsize=\small,frame=single,label=Example]
  !gapprompt@gap>| !gapinput@ g:=SymmetricGroup(3);|
  Sym( [ 1 .. 3 ] )
  !gapprompt@gap>| !gapinput@asg:=AllSubgroups(g);|
  [ Group(()), Group([ (2,3) ]), Group([ (1,2) ]), Group([ (1,3) ]), Group([ (1,2,3) ]),   Group([ (1,2,3), (2,3) ]) ]
  !gapprompt@gap>| !gapinput@poset:=PosetFromElements(asg,IsSubgroup);|
  A poset on 6 elements using the IsPosetOfIndices representation.
  !gapprompt@gap>| !gapinput@HasseDiagramBinaryRelation(PartialOrder(poset));|
  <general mapping: Domain([ 1 .. 6 ]) -> Domain([ 1 .. 6 ]) >
  !gapprompt@gap>| !gapinput@Successors(last);|
  [ [ 2, 3, 4, 5 ], [ 6 ], [ 6 ], [ 6 ], [ 6 ], [  ] ]
  !gapprompt@gap>| !gapinput@List( ElementsList(poset){[2,6]}, ElementObject);|
  [ Group([ (2,3) ]), Group([ (1,2,3), (2,3) ]) ]
\end{Verbatim}
 
\subsection{\textcolor{Chapter }{Helper functions for special partial orders}}\label{Helper_functions}
\logpage{[ 11, 1, 6 ]}
\hyperdef{L}{X82DB1F987C68B392}{}
{
\noindent\textcolor{FuncColor}{$\triangleright$\enspace\texttt{PairCompareFlagsList({\mdseries\slshape list1, list2})\index{PairCompareFlagsList@\texttt{PairCompareFlagsList}!for IsList,IsList}
\label{PairCompareFlagsList:for IsList,IsList}
}\hfill{\scriptsize (operation)}}\\
\noindent\textcolor{FuncColor}{$\triangleright$\enspace\texttt{PairCompareAtomsList({\mdseries\slshape list1, list2})\index{PairCompareAtomsList@\texttt{PairCompareAtomsList}!for IsList,IsList}
\label{PairCompareAtomsList:for IsList,IsList}
}\hfill{\scriptsize (operation)}}\\
\textbf{\indent Returns:\ }
\texttt{true} or \texttt{false} 



 The functions PairCompareFlagsList and PairCompareAtomsList are used in poset
construction. Function assumes \mbox{\texttt{\mdseries\slshape list1}} and \mbox{\texttt{\mdseries\slshape list2}} are of the form [\texttt{listOfFlags},\texttt{i}] where \texttt{listOfFlags} is a list of flags in the face and \texttt{i} is the rank of the face. Allows comparison of HasFlagList elements. Function
assumes \mbox{\texttt{\mdseries\slshape list1}} and \mbox{\texttt{\mdseries\slshape list2}} are of the form \texttt{[listOfAtoms,int]} where \texttt{listOfAtoms} is a list of flags in the face and \texttt{int} is the rank of the face. Allows comparison of HasAtomList elements. }

 

\subsection{\textcolor{Chapter }{DualPoset (for IsPoset)}}
\logpage{[ 11, 1, 7 ]}\nobreak
\hyperdef{L}{X878F761878BCB6B9}{}
{\noindent\textcolor{FuncColor}{$\triangleright$\enspace\texttt{DualPoset({\mdseries\slshape poset})\index{DualPoset@\texttt{DualPoset}!for IsPoset}
\label{DualPoset:for IsPoset}
}\hfill{\scriptsize (operation)}}\\
\textbf{\indent Returns:\ }
dual 



 Given a \mbox{\texttt{\mdseries\slshape poset}}, will construct a poset isomorphic to the dual of \mbox{\texttt{\mdseries\slshape poset}}. }

 
\begin{Verbatim}[commandchars=!@|,fontsize=\small,frame=single,label=Example]
  !gapprompt@gap>| !gapinput@p:=PosetFromManiplex(Cube(3));; c:=PosetFromManiplex(CrossPolytope(3));;|
  !gapprompt@gap>| !gapinput@IsIsomorphicPoset(DualPoset(DualPoset(p)),p);|
  true
  !gapprompt@gap>| !gapinput@IsIsomorphicPoset(DualPoset(p),c);|
  true
  !gapprompt@gap>| !gapinput@IsIsomorphicPoset(DualPoset(p),p);|
  false
\end{Verbatim}
 }

 
\section{\textcolor{Chapter }{Poset attributes}}\label{Chapter_Posets_Section_Poset_attributes}
\logpage{[ 11, 2, 0 ]}
\hyperdef{L}{X859F4CD47E3311AD}{}
{
  Posets have many properties we might be interested in. Here's a few. 

\subsection{\textcolor{Chapter }{MaximalChains (for IsPoset)}}
\logpage{[ 11, 2, 1 ]}\nobreak
\hyperdef{L}{X81E0832F7B92DB7A}{}
{\noindent\textcolor{FuncColor}{$\triangleright$\enspace\texttt{MaximalChains({\mdseries\slshape poset})\index{MaximalChains@\texttt{MaximalChains}!for IsPoset}
\label{MaximalChains:for IsPoset}
}\hfill{\scriptsize (attribute)}}\\


 Gives the list of maximal chains in a poset in terms of the elements of the
poset. Synonyms are \texttt{FlagsList} and \texttt{Flags}. Tends to work faster (sometimes significantly) if the poset \texttt{HasPartialOrder}. }

 Synonym is \texttt{FlagsList}. 
\begin{Verbatim}[commandchars=!@|,fontsize=\small,frame=single,label=Example]
  !gapprompt@gap>| !gapinput@poset:=PosetFromManiplex(HemiCube(3));|
  A poset using the IsPosetOfFlags representation.
  !gapprompt@gap>| !gapinput@MaximalChains(poset)[1];|
  [ An element of a poset made of flags, An element of a poset made of flags, 
    An element of a poset made of flags, An element of a poset made of flags, 
    An element of a poset made of flags ]
  !gapprompt@gap>| !gapinput@List(last,x->RankInPoset(x,poset));|
  [ -1, 0, 1, 2, 3 ]
\end{Verbatim}
 

\subsection{\textcolor{Chapter }{RankPoset (for IsPoset)}}
\logpage{[ 11, 2, 2 ]}\nobreak
\hyperdef{L}{X7C9B4AA77C73A761}{}
{\noindent\textcolor{FuncColor}{$\triangleright$\enspace\texttt{RankPoset({\mdseries\slshape poset})\index{RankPoset@\texttt{RankPoset}!for IsPoset}
\label{RankPoset:for IsPoset}
}\hfill{\scriptsize (attribute)}}\\


 If the poset \texttt{IsP1}, ranks are assumed to run from $-1$ to $n$, and function will return $n$. If \texttt{IsP1(poset)=false}, ranks are assumed to run from 1 to $n$. In RAMP, at least currently, we are assuming that graded/ranked posets are
bounded. Note that in general what you \emph{actually} want to do is call \texttt{Rank(poset)}. The reason is that \texttt{Rank} will calculate the \texttt{RankPoset} if it isn't set, and then set and store the value in the poset. }

 

\subsection{\textcolor{Chapter }{ElementsList (for IsPoset)}}
\logpage{[ 11, 2, 3 ]}\nobreak
\label{elements}
\hyperdef{L}{X8448C7067B0EF49D}{}
{\noindent\textcolor{FuncColor}{$\triangleright$\enspace\texttt{ElementsList({\mdseries\slshape poset})\index{ElementsList@\texttt{ElementsList}!for IsPoset}
\label{ElementsList:for IsPoset}
}\hfill{\scriptsize (attribute)}}\\


 Will recover the list of faces of the poset, format may depend on \emph{type} of representation of \texttt{poset}. 
\begin{itemize}
\item  We also have \texttt{FacesList} and \texttt{Faces} as synonyms for this command. 
\end{itemize}
 }

 

\subsection{\textcolor{Chapter }{OrderingFunction (for IsPoset)}}
\logpage{[ 11, 2, 4 ]}\nobreak
\hyperdef{L}{X7C5580EC83A4955B}{}
{\noindent\textcolor{FuncColor}{$\triangleright$\enspace\texttt{OrderingFunction({\mdseries\slshape poset})\index{OrderingFunction@\texttt{OrderingFunction}!for IsPoset}
\label{OrderingFunction:for IsPoset}
}\hfill{\scriptsize (attribute)}}\\


 \texttt{OrderingFunction} is an attribute of a poset which stores a function for ordering elements. }

 
\begin{Verbatim}[commandchars=!@|,fontsize=\small,frame=single,label=Example]
  !gapprompt@gap>| !gapinput@p:=PosetFromManiplex(Cube(2));;|
  !gapprompt@gap>| !gapinput@p3:=PosetFromElements(RankedFaceListOfPoset(p),PairCompareFlagsList);;|
  !gapprompt@gap>| !gapinput@f3:=FacesList(p3);;|
  !gapprompt@gap>| !gapinput@OrderingFunction(p3)(ElementObject(f3[2]),ElementObject(f3[1]));|
  true
  !gapprompt@gap>| !gapinput@OrderingFunction(p3)(ElementObject(f3[1]),ElementObject(f3[2]));|
  false
\end{Verbatim}
 

\subsection{\textcolor{Chapter }{IsFlaggable (for IsPoset)}}
\logpage{[ 11, 2, 5 ]}\nobreak
\hyperdef{L}{X842B0F7E81670077}{}
{\noindent\textcolor{FuncColor}{$\triangleright$\enspace\texttt{IsFlaggable({\mdseries\slshape poset})\index{IsFlaggable@\texttt{IsFlaggable}!for IsPoset}
\label{IsFlaggable:for IsPoset}
}\hfill{\scriptsize (property)}}\\
\textbf{\indent Returns:\ }
\texttt{true} or \texttt{false} 



 Checks or creates the value of the attribute \texttt{IsFlaggable} for an \texttt{IsPoset}. Point here is to see if the structure of the poset is sufficient to
determine the flag graph. For IsPosetOfFlags this is another way of saying
that the intersection of the faces (thought of as collections of flags)
containing a flag is that selfsame flag. (Might be equivalent to
prepolytopal... but Gabe was tired and Gordon hasn't bothered to think about
it yet.) Now also works with generic poset element types (not just \texttt{IsPosetOfFlags}). }

 

\subsection{\textcolor{Chapter }{IsAtomic (for IsPoset)}}
\logpage{[ 11, 2, 6 ]}\nobreak
\hyperdef{L}{X816460188169E650}{}
{\noindent\textcolor{FuncColor}{$\triangleright$\enspace\texttt{IsAtomic({\mdseries\slshape poset})\index{IsAtomic@\texttt{IsAtomic}!for IsPoset}
\label{IsAtomic:for IsPoset}
}\hfill{\scriptsize (property)}}\\
\textbf{\indent Returns:\ }
\texttt{true} or \texttt{false} 



 This checks whether or not the faces of an IsP1 poset may be described
uniquely in terms of the posets atoms. 

 Note: At some point this will have to be renamed, but I've forgotten the right
terminology. }

 
\begin{Verbatim}[commandchars=!@|,fontsize=\small,frame=single,label=Example]
  !gapprompt@gap>| !gapinput@po:=BinaryRelationOnPoints([[2,3],[4,5],[4,5],[6],[6],[]]);;|
  !gapprompt@gap>| !gapinput@po:=ReflexiveClosureBinaryRelation(TransitiveClosureBinaryRelation(po));;|
  !gapprompt@gap>| !gapinput@p:=PosetFromPartialOrder(po);; IsAtomic(p);|
  false
  !gapprompt@gap>| !gapinput@p2:=PosetFromManiplex(Cube(3));; IsAtomic(p2);|
  true
\end{Verbatim}
 

\subsection{\textcolor{Chapter }{PartialOrder (for IsPoset)}}
\logpage{[ 11, 2, 7 ]}\nobreak
\hyperdef{L}{X85A9CAA682FE56EB}{}
{\noindent\textcolor{FuncColor}{$\triangleright$\enspace\texttt{PartialOrder({\mdseries\slshape poset})\index{PartialOrder@\texttt{PartialOrder}!for IsPoset}
\label{PartialOrder:for IsPoset}
}\hfill{\scriptsize (attribute)}}\\
\textbf{\indent Returns:\ }
\texttt{partial order} 



 HasPartialOrder Checks if \mbox{\texttt{\mdseries\slshape poset}} has a declared partial order (binary relation). SetPartialOrder assigns a
partial order to the \mbox{\texttt{\mdseries\slshape poset}}. In many cases, PartialOrder is able to compute one from structural
information. }

 

\subsection{\textcolor{Chapter }{ListIsP1Poset (for IsList)}}
\logpage{[ 11, 2, 8 ]}\nobreak
\hyperdef{L}{X8455987F7E9A88FF}{}
{\noindent\textcolor{FuncColor}{$\triangleright$\enspace\texttt{ListIsP1Poset({\mdseries\slshape list})\index{ListIsP1Poset@\texttt{ListIsP1Poset}!for IsList}
\label{ListIsP1Poset:for IsList}
}\hfill{\scriptsize (operation)}}\\
\textbf{\indent Returns:\ }
\texttt{true} or \texttt{false} 



 Given \mbox{\texttt{\mdseries\slshape list}}, comprised of sublists of faces ordered by rank, each face listing the flags
on the face, this function will tell you if the list corresponds to a P1 poset
or not. }

 

\subsection{\textcolor{Chapter }{IsP1 (for IsPoset)}}
\logpage{[ 11, 2, 9 ]}\nobreak
\hyperdef{L}{X81269ACB7EF202C4}{}
{\noindent\textcolor{FuncColor}{$\triangleright$\enspace\texttt{IsP1({\mdseries\slshape poset})\index{IsP1@\texttt{IsP1}!for IsPoset}
\label{IsP1:for IsPoset}
}\hfill{\scriptsize (property)}}\\
\textbf{\indent Returns:\ }
\texttt{true} or \texttt{false} 



 Determines whether a poset has property P1 from ARP. }

 
\begin{Verbatim}[commandchars=!@|,fontsize=\small,frame=single,label=Example]
  !gapprompt@gap>| !gapinput@p:=PosetFromElements(AllSubgroups(AlternatingGroup(4)),IsSubgroup);|
  A poset using the IsPosetOfIndices representation 
  !gapprompt@gap>| !gapinput@IsP1(p);|
  true
  !gapprompt@gap>| !gapinput@p2:=PosetFromFaceListOfFlags([[[1],[2]],[[1,2]]]);|
  A poset using the IsPosetOfFlags representation with 3 faces.
  !gapprompt@gap>| !gapinput@IsP1(p2);|
  false
\end{Verbatim}
 

\subsection{\textcolor{Chapter }{IsP2 (for IsPoset)}}
\logpage{[ 11, 2, 10 ]}\nobreak
\hyperdef{L}{X7AFA3C077FD9583A}{}
{\noindent\textcolor{FuncColor}{$\triangleright$\enspace\texttt{IsP2({\mdseries\slshape poset})\index{IsP2@\texttt{IsP2}!for IsPoset}
\label{IsP2:for IsPoset}
}\hfill{\scriptsize (property)}}\\
\textbf{\indent Returns:\ }
\texttt{true} or \texttt{false} 



 Determines whether a poset has property P2 from ARP. }

 
\begin{Verbatim}[commandchars=!@|,fontsize=\small,frame=single,label=Example]
  !gapprompt@gap>| !gapinput@poset:=PosetFromManiplex(HemiCube(3)); |
  !gapprompt@gap>| !gapinput@IsP2(poset);|
  true
\end{Verbatim}
 Another nice example 
\begin{Verbatim}[commandchars=!@|,fontsize=\small,frame=single,label=Example]
  !gapprompt@gap>| !gapinput@g:=AlternatingGroup(4);; a:=AllSubgroups(g);; poset:=PosetFromElements(a,IsSubgroup);|
  A poset using the IsPosetOfIndices representation 
  !gapprompt@gap>| !gapinput@IsP2(poset);|
  false
\end{Verbatim}
 

\subsection{\textcolor{Chapter }{IsP3 (for IsPoset)}}
\logpage{[ 11, 2, 11 ]}\nobreak
\hyperdef{L}{X8723519F84529E4C}{}
{\noindent\textcolor{FuncColor}{$\triangleright$\enspace\texttt{IsP3({\mdseries\slshape poset})\index{IsP3@\texttt{IsP3}!for IsPoset}
\label{IsP3:for IsPoset}
}\hfill{\scriptsize (property)}}\\
\textbf{\indent Returns:\ }
\texttt{true} or \texttt{false} 



 Determines whether a poset is strongly flag connected (property P3' from ARP).
May also be called with command \texttt{IsStronglyFlagConnected}. If you are not working with a pre-polytope, expect this to take a LONG time. }

 Helper for IsP3 

\subsection{\textcolor{Chapter }{IsFlagConnected (for IsPoset)}}
\logpage{[ 11, 2, 12 ]}\nobreak
\hyperdef{L}{X85F554347C5AEF30}{}
{\noindent\textcolor{FuncColor}{$\triangleright$\enspace\texttt{IsFlagConnected({\mdseries\slshape poset})\index{IsFlagConnected@\texttt{IsFlagConnected}!for IsPoset}
\label{IsFlagConnected:for IsPoset}
}\hfill{\scriptsize (property)}}\\
\textbf{\indent Returns:\ }
\texttt{true} or \texttt{false} 



 Determines whether a poset is flag connected. }

 

\subsection{\textcolor{Chapter }{IsP4 (for IsPoset)}}
\logpage{[ 11, 2, 13 ]}\nobreak
\hyperdef{L}{X80F461FA7D8FEDC7}{}
{\noindent\textcolor{FuncColor}{$\triangleright$\enspace\texttt{IsP4({\mdseries\slshape poset})\index{IsP4@\texttt{IsP4}!for IsPoset}
\label{IsP4:for IsPoset}
}\hfill{\scriptsize (property)}}\\
\textbf{\indent Returns:\ }
\texttt{true} or \texttt{false} 



 Determines whether a poset satisfies the diamond condition. May also be
invoked using \texttt{IsDiamondCondition}. }

 

\subsection{\textcolor{Chapter }{IsPolytope (for IsPoset)}}
\logpage{[ 11, 2, 14 ]}\nobreak
\hyperdef{L}{X83D63725789E745D}{}
{\noindent\textcolor{FuncColor}{$\triangleright$\enspace\texttt{IsPolytope({\mdseries\slshape poset})\index{IsPolytope@\texttt{IsPolytope}!for IsPoset}
\label{IsPolytope:for IsPoset}
}\hfill{\scriptsize (property)}}\\
\textbf{\indent Returns:\ }
\texttt{true} or \texttt{false} 



 Determines whether a poset is an abstract polytope. }

 
\begin{Verbatim}[commandchars=!@|,fontsize=\small,frame=single,label=Example]
  !gapprompt@gap>| !gapinput@poset:=PosetFromManiplex(Cube(3));|
  A poset using the IsPosetOfFlags representation with 28 faces.
  !gapprompt@gap>| !gapinput@IsPolytope(poset);|
  true
  !gapprompt@gap>| !gapinput@KnownPropertiesOfObject(poset);|
  [ "IsP1", "IsP2", "IsP3", "IsP4", "IsPolytope" ]
  !gapprompt@gap>| !gapinput@poset2:=PosetFromElements(AllSubgroups(AlternatingGroup(4)),IsSubgroup);|
  A poset using the IsPosetOfIndices representation 
  !gapprompt@gap>| !gapinput@IsPolytope(poset2);|
  false
  !gapprompt@gap>| !gapinput@KnownPropertiesOfObject(poset2);|
  [ "IsP1", "IsP2", "IsPolytope" ]
\end{Verbatim}
 

\subsection{\textcolor{Chapter }{IsPrePolytope (for IsPoset)}}
\logpage{[ 11, 2, 15 ]}\nobreak
\hyperdef{L}{X8085287683CCF81B}{}
{\noindent\textcolor{FuncColor}{$\triangleright$\enspace\texttt{IsPrePolytope({\mdseries\slshape poset})\index{IsPrePolytope@\texttt{IsPrePolytope}!for IsPoset}
\label{IsPrePolytope:for IsPoset}
}\hfill{\scriptsize (property)}}\\
\textbf{\indent Returns:\ }
\texttt{true} or \texttt{false} 



 Determines whether a poset is an abstract pre-polytope. }

 }

 
\section{\textcolor{Chapter }{Working with posets}}\label{Chapter_Posets_Section_Working_with_posets}
\logpage{[ 11, 3, 0 ]}
\hyperdef{L}{X8163385F7B822934}{}
{
  

\subsection{\textcolor{Chapter }{IsIsomorphicPoset (for IsPoset,IsPoset)}}
\logpage{[ 11, 3, 1 ]}\nobreak
\hyperdef{L}{X80A2785B7D079FD1}{}
{\noindent\textcolor{FuncColor}{$\triangleright$\enspace\texttt{IsIsomorphicPoset({\mdseries\slshape poset1, poset2})\index{IsIsomorphicPoset@\texttt{IsIsomorphicPoset}!for IsPoset,IsPoset}
\label{IsIsomorphicPoset:for IsPoset,IsPoset}
}\hfill{\scriptsize (operation)}}\\
\textbf{\indent Returns:\ }
\texttt{true} or \texttt{false} \texttt{true} or \texttt{false} 



 Determines whether \mbox{\texttt{\mdseries\slshape poset1}} and \mbox{\texttt{\mdseries\slshape poset2}} are equal by Determines whether \mbox{\texttt{\mdseries\slshape poset1}} and \mbox{\texttt{\mdseries\slshape poset2}} are isomorphic by checking to see if their Hasse diagrams are isomorphic. }

 
\begin{Verbatim}[commandchars=!@|,fontsize=\small,frame=single,label=Example]
  !gapprompt@gap>| !gapinput@ IsIsomorphicPoset( PosetFromManiplex( PyramidOver( Cube(3) ) ),  PosetFromManiplex( PrismOver (PyramidOver( Cube(2) ) ) ) );|
  false
  !gapprompt@gap>| !gapinput@ IsIsomorphicPoset( PosetFromManiplex( PyramidOver( Cube(3) ) ), PosetFromManiplex( PyramidOver( PrismOver( Cube(2) ) ) ) );|
  true
\end{Verbatim}
 

\subsection{\textcolor{Chapter }{PosetIsomorphism (for IsPoset,IsPoset)}}
\logpage{[ 11, 3, 2 ]}\nobreak
\hyperdef{L}{X8340F2597BD837A5}{}
{\noindent\textcolor{FuncColor}{$\triangleright$\enspace\texttt{PosetIsomorphism({\mdseries\slshape poset1, poset2})\index{PosetIsomorphism@\texttt{PosetIsomorphism}!for IsPoset,IsPoset}
\label{PosetIsomorphism:for IsPoset,IsPoset}
}\hfill{\scriptsize (operation)}}\\
\textbf{\indent Returns:\ }
map on face indices 



 When \mbox{\texttt{\mdseries\slshape poset1}} and \mbox{\texttt{\mdseries\slshape poset2}} are isomorphic, will give you a map from the faces of \mbox{\texttt{\mdseries\slshape poset1}} to the faces of \mbox{\texttt{\mdseries\slshape poset2}}. }

 

\subsection{\textcolor{Chapter }{FlagsAsListOfFacesFromPoset (for IsPoset)}}
\logpage{[ 11, 3, 3 ]}\nobreak
\hyperdef{L}{X7B9029A4868E7138}{}
{\noindent\textcolor{FuncColor}{$\triangleright$\enspace\texttt{FlagsAsListOfFacesFromPoset({\mdseries\slshape poset})\index{FlagsAsListOfFacesFromPoset@\texttt{FlagsAsListOfFacesFromPoset}!for IsPoset}
\label{FlagsAsListOfFacesFromPoset:for IsPoset}
}\hfill{\scriptsize (operation)}}\\
\textbf{\indent Returns:\ }
\texttt{IsList} 



 Given a \mbox{\texttt{\mdseries\slshape poset}}, this will give you a version of the list of flags in terms of the proper
faces described in the \mbox{\texttt{\mdseries\slshape poset}}. Note that the flag list does not include the minimal face or the maximal
face if the poset IsP2; i.e., this gives a list of flags where each face is
described in terms of its (enumerated) list of incident flags. }

 

\subsection{\textcolor{Chapter }{RankedFaceListOfPoset (for IsPoset)}}
\logpage{[ 11, 3, 4 ]}\nobreak
\hyperdef{L}{X7A80223E83EF52E8}{}
{\noindent\textcolor{FuncColor}{$\triangleright$\enspace\texttt{RankedFaceListOfPoset({\mdseries\slshape IsPosetOfFlags})\index{RankedFaceListOfPoset@\texttt{RankedFaceListOfPoset}!for IsPoset}
\label{RankedFaceListOfPoset:for IsPoset}
}\hfill{\scriptsize (operation)}}\\
\textbf{\indent Returns:\ }
\texttt{list} 



 Gives a list of [\mbox{\texttt{\mdseries\slshape face}},\mbox{\texttt{\mdseries\slshape rank}}] pairs for all the faces of \mbox{\texttt{\mdseries\slshape poset}}. Assumptions here are that faces are lists of incident flags. }

 

\subsection{\textcolor{Chapter }{AdjacentFlag (for IsPosetOfFlags,IsList,IsInt)}}
\logpage{[ 11, 3, 5 ]}\nobreak
\hyperdef{L}{X7ACE88EC803EDE07}{}
{\noindent\textcolor{FuncColor}{$\triangleright$\enspace\texttt{AdjacentFlag({\mdseries\slshape poset, flag, i})\index{AdjacentFlag@\texttt{AdjacentFlag}!for IsPosetOfFlags,IsList,IsInt}
\label{AdjacentFlag:for IsPosetOfFlags,IsList,IsInt}
}\hfill{\scriptsize (operation)}}\\
\textbf{\indent Returns:\ }
\texttt{flag(s)} 



 Given a poset, a flag, and a rank, this function will give you the \mbox{\texttt{\mdseries\slshape i}}-adjacent flag. Note that adjacencies are listed from ranks 0 to one less than
the dimension. You can replace \mbox{\texttt{\mdseries\slshape flag}} with the integer corresponding to that flag. Appending \texttt{true} to the arguments will give the position of the flag instead of its description
from \texttt{FlagsAsListOfFacesFromPoset}. }

 

\subsection{\textcolor{Chapter }{AdjacentFlags (for IsPoset,IsList,IsInt)}}
\logpage{[ 11, 3, 6 ]}\nobreak
\hyperdef{L}{X7C94E895794BA41F}{}
{\noindent\textcolor{FuncColor}{$\triangleright$\enspace\texttt{AdjacentFlags({\mdseries\slshape poset, flagaslistoffaces, adjacencyrank})\index{AdjacentFlags@\texttt{AdjacentFlags}!for IsPoset,IsList,IsInt}
\label{AdjacentFlags:for IsPoset,IsList,IsInt}
}\hfill{\scriptsize (operation)}}\\


 If your poset isn't P4, there may be multiple adjacent maximal chains at a
given rank. This function handles that case. May substitute \texttt{IsInt} for \texttt{flagaslistoffaces} corresponding to position of \texttt{flag} in list of maximal chains. }

 

\subsection{\textcolor{Chapter }{EqualChains (for IsList,IsList)}}
\logpage{[ 11, 3, 7 ]}\nobreak
\hyperdef{L}{X7E1C59DF835343FE}{}
{\noindent\textcolor{FuncColor}{$\triangleright$\enspace\texttt{EqualChains({\mdseries\slshape flag1, flag2})\index{EqualChains@\texttt{EqualChains}!for IsList,IsList}
\label{EqualChains:for IsList,IsList}
}\hfill{\scriptsize (operation)}}\\


 Determines whether two chains are equal. }

 

\subsection{\textcolor{Chapter }{ConnectionGeneratorOfPoset (for IsPoset,IsInt)}}
\logpage{[ 11, 3, 8 ]}\nobreak
\hyperdef{L}{X84E3B63C8639771B}{}
{\noindent\textcolor{FuncColor}{$\triangleright$\enspace\texttt{ConnectionGeneratorOfPoset({\mdseries\slshape poset, i})\index{ConnectionGeneratorOfPoset@\texttt{ConnectionGeneratorOfPoset}!for IsPoset,IsInt}
\label{ConnectionGeneratorOfPoset:for IsPoset,IsInt}
}\hfill{\scriptsize (operation)}}\\
\textbf{\indent Returns:\ }
A permutation on the flags. 



 Given a \mbox{\texttt{\mdseries\slshape poset}} and an integer $i$, this function will give you the associated permutation for the rank $i$-connection. }

 

\subsection{\textcolor{Chapter }{ConnectionGroup (for IsPoset)}}
\logpage{[ 11, 3, 9 ]}\nobreak
\hyperdef{L}{X859C651184ED9424}{}
{\noindent\textcolor{FuncColor}{$\triangleright$\enspace\texttt{ConnectionGroup({\mdseries\slshape poset})\index{ConnectionGroup@\texttt{ConnectionGroup}!for IsPoset}
\label{ConnectionGroup:for IsPoset}
}\hfill{\scriptsize (attribute)}}\\
\textbf{\indent Returns:\ }
\texttt{IsPermGroup} 



 Given a \mbox{\texttt{\mdseries\slshape poset}} that is \texttt{IsPrePolytope}, this function will give you the connection group. }

 

\subsection{\textcolor{Chapter }{AutomorphismGroup (for IsPoset)}}
\logpage{[ 11, 3, 10 ]}\nobreak
\hyperdef{L}{X7C6813D27F0BD0F8}{}
{\noindent\textcolor{FuncColor}{$\triangleright$\enspace\texttt{AutomorphismGroup({\mdseries\slshape poset})\index{AutomorphismGroup@\texttt{AutomorphismGroup}!for IsPoset}
\label{AutomorphismGroup:for IsPoset}
}\hfill{\scriptsize (attribute)}}\\


 Given a \mbox{\texttt{\mdseries\slshape poset}}, gives the automorphism group of the poset as an action on the maximal
chains. }

 

\subsection{\textcolor{Chapter }{AutomorphismGroupOnElements (for IsPoset)}}
\logpage{[ 11, 3, 11 ]}\nobreak
\hyperdef{L}{X80C2783D7FD60E56}{}
{\noindent\textcolor{FuncColor}{$\triangleright$\enspace\texttt{AutomorphismGroupOnElements({\mdseries\slshape poset})\index{AutomorphismGroupOnElements@\texttt{AutomorphismGroupOnElements}!for IsPoset}
\label{AutomorphismGroupOnElements:for IsPoset}
}\hfill{\scriptsize (attribute)}}\\


 Given a \mbox{\texttt{\mdseries\slshape poset}}, gives the automorphism group of the poset as an action on the elements. }

 

\subsection{\textcolor{Chapter }{FaceListOfPoset (for IsPoset)}}
\logpage{[ 11, 3, 12 ]}\nobreak
\hyperdef{L}{X79F1837C835A5EFF}{}
{\noindent\textcolor{FuncColor}{$\triangleright$\enspace\texttt{FaceListOfPoset({\mdseries\slshape poset})\index{FaceListOfPoset@\texttt{FaceListOfPoset}!for IsPoset}
\label{FaceListOfPoset:for IsPoset}
}\hfill{\scriptsize (operation)}}\\
\textbf{\indent Returns:\ }
\texttt{list} 



 Gives a list of faces collected into lists ordered by increasing rank.
Suitable as input for \texttt{PosetFromFaceListOfFlags}. Argument must be IsPosetOfFlags. }

 

\subsection{\textcolor{Chapter }{RankPosetElements (for IsPoset)}}
\logpage{[ 11, 3, 13 ]}\nobreak
\hyperdef{L}{X7980D74F7D8C1B95}{}
{\noindent\textcolor{FuncColor}{$\triangleright$\enspace\texttt{RankPosetElements({\mdseries\slshape poset})\index{RankPosetElements@\texttt{RankPosetElements}!for IsPoset}
\label{RankPosetElements:for IsPoset}
}\hfill{\scriptsize (operation)}}\\


 Assigns to each face of a poset (when possible) the rank of the element in the
poset. }

 

\subsection{\textcolor{Chapter }{FacesByRankOfPoset (for IsPoset)}}
\logpage{[ 11, 3, 14 ]}\nobreak
\hyperdef{L}{X82CBC68480F222C7}{}
{\noindent\textcolor{FuncColor}{$\triangleright$\enspace\texttt{FacesByRankOfPoset({\mdseries\slshape poset})\index{FacesByRankOfPoset@\texttt{FacesByRankOfPoset}!for IsPoset}
\label{FacesByRankOfPoset:for IsPoset}
}\hfill{\scriptsize (operation)}}\\
\textbf{\indent Returns:\ }
\texttt{list} 



 Gives lists of faces ordered by rank. Also sets the rank for each of the
faces. }

 

\subsection{\textcolor{Chapter }{HasseDiagramOfPoset (for IsPoset)}}
\logpage{[ 11, 3, 15 ]}\nobreak
\hyperdef{L}{X7AE5050E7F6182C8}{}
{\noindent\textcolor{FuncColor}{$\triangleright$\enspace\texttt{HasseDiagramOfPoset({\mdseries\slshape poset})\index{HasseDiagramOfPoset@\texttt{HasseDiagramOfPoset}!for IsPoset}
\label{HasseDiagramOfPoset:for IsPoset}
}\hfill{\scriptsize (operation)}}\\
\textbf{\indent Returns:\ }
directed graph 



 

 }

 

\subsection{\textcolor{Chapter }{AsPosetOfAtoms (for IsPoset)}}
\logpage{[ 11, 3, 16 ]}\nobreak
\hyperdef{L}{X83ACE1CB7DE65C44}{}
{\noindent\textcolor{FuncColor}{$\triangleright$\enspace\texttt{AsPosetOfAtoms({\mdseries\slshape poset})\index{AsPosetOfAtoms@\texttt{AsPosetOfAtoms}!for IsPoset}
\label{AsPosetOfAtoms:for IsPoset}
}\hfill{\scriptsize (operation)}}\\
\textbf{\indent Returns:\ }
posetFromAtoms 



 If an IsP1 poset admits a description of its elements in terms of its atoms,
this function will construct an isomorphic poset whose faces are described
using PosetFromAtomList. }

 
\begin{Verbatim}[commandchars=!@|,fontsize=\small,frame=single,label=Example]
  !gapprompt@gap>| !gapinput@poset:=PosetFromManiplex(Cube(2));;|
  !gapprompt@gap>| !gapinput@p2:=AsPosetOfAtoms(poset);|
  A poset on 10 elements using the IsPosetOfIndices representation.
  !gapprompt@gap>| !gapinput@IsIsomorphicPoset(poset,p2);|
  true
\end{Verbatim}
 }

 
\section{\textcolor{Chapter }{Elements of posets, also known as faces.}}\label{Chapter_Posets_Section_Elements_of_posets_also_known_as_faces}
\logpage{[ 11, 4, 0 ]}
\hyperdef{L}{X86FBAB3E8056F479}{}
{
  

\subsection{\textcolor{Chapter }{RanksInPosets (for IsPosetElement)}}
\logpage{[ 11, 4, 1 ]}\nobreak
\hyperdef{L}{X7D43D66E82CC96DF}{}
{\noindent\textcolor{FuncColor}{$\triangleright$\enspace\texttt{RanksInPosets({\mdseries\slshape posetelement})\index{RanksInPosets@\texttt{RanksInPosets}!for IsPosetElement}
\label{RanksInPosets:for IsPosetElement}
}\hfill{\scriptsize (attribute)}}\\
\textbf{\indent Returns:\ }
list 



 Gives the \texttt{list} of posets \mbox{\texttt{\mdseries\slshape posetelement}} is in, and the corresponding rank (if available) as a list of ordered pairs of
the form \texttt{[poset,rank]}. \#! Note that this attribute is mutable, so if you modify it you may break
things. }

 

\subsection{\textcolor{Chapter }{AddRanksInPosets (for IsPosetElement,IsPoset,IsInt)}}
\logpage{[ 11, 4, 2 ]}\nobreak
\hyperdef{L}{X8399E8CD8022DF5F}{}
{\noindent\textcolor{FuncColor}{$\triangleright$\enspace\texttt{AddRanksInPosets({\mdseries\slshape posetelement, poset, int})\index{AddRanksInPosets@\texttt{AddRanksInPosets}!for IsPosetElement,IsPoset,IsInt}
\label{AddRanksInPosets:for IsPosetElement,IsPoset,IsInt}
}\hfill{\scriptsize (operation)}}\\
\textbf{\indent Returns:\ }
null 



 Adds an entry in the list of RanksInPosets for \mbox{\texttt{\mdseries\slshape posetelement}} corresponding to \mbox{\texttt{\mdseries\slshape poset}} with assigned rank \mbox{\texttt{\mdseries\slshape int}}. }

 

\subsection{\textcolor{Chapter }{FlagList (for IsPosetElement)}}
\logpage{[ 11, 4, 3 ]}\nobreak
\hyperdef{L}{X819B1AEF84E08B4D}{}
{\noindent\textcolor{FuncColor}{$\triangleright$\enspace\texttt{FlagList({\mdseries\slshape posetelement, \texttt{\symbol{123}}face\texttt{\symbol{125}}})\index{FlagList@\texttt{FlagList}!for IsPosetElement}
\label{FlagList:for IsPosetElement}
}\hfill{\scriptsize (attribute)}}\\
\textbf{\indent Returns:\ }
\texttt{list} 



 Description of \mbox{\texttt{\mdseries\slshape posetelement}} n as a list of incident flags (when present). }

 

\subsection{\textcolor{Chapter }{AtomList (for IsPosetElement)}}
\logpage{[ 11, 4, 4 ]}\nobreak
\hyperdef{L}{X878F991183754744}{}
{\noindent\textcolor{FuncColor}{$\triangleright$\enspace\texttt{AtomList({\mdseries\slshape posetelement, \texttt{\symbol{123}}face\texttt{\symbol{125}}})\index{AtomList@\texttt{AtomList}!for IsPosetElement}
\label{AtomList:for IsPosetElement}
}\hfill{\scriptsize (attribute)}}\\
\textbf{\indent Returns:\ }
\texttt{list} 



 Description of \mbox{\texttt{\mdseries\slshape posetelement}} n as a list of atoms (when present). }

 }

 
\section{\textcolor{Chapter }{Element Constructors}}\label{Chapter_Posets_Section_Element_Constructors}
\logpage{[ 11, 5, 0 ]}
\hyperdef{L}{X85445BE07F161F88}{}
{
  

\subsection{\textcolor{Chapter }{PosetElementWithOrder (for IsObject,IsFunction)}}
\logpage{[ 11, 5, 1 ]}\nobreak
\hyperdef{L}{X81CE0D9C872BE42B}{}
{\noindent\textcolor{FuncColor}{$\triangleright$\enspace\texttt{PosetElementWithOrder({\mdseries\slshape obj, func})\index{PosetElementWithOrder@\texttt{PosetElementWithOrder}!for IsObject,IsFunction}
\label{PosetElementWithOrder:for IsObject,IsFunction}
}\hfill{\scriptsize (operation)}}\\
\textbf{\indent Returns:\ }
\texttt{IsFace} 



 Creates a \texttt{face} with \mbox{\texttt{\mdseries\slshape obj}} and ordering function \texttt{func}. }

 

\subsection{\textcolor{Chapter }{PosetElementFromListOfFlags (for IsList,IsPoset,IsInt)}}
\logpage{[ 11, 5, 2 ]}\nobreak
\hyperdef{L}{X79608DCB8749DEDC}{}
{\noindent\textcolor{FuncColor}{$\triangleright$\enspace\texttt{PosetElementFromListOfFlags({\mdseries\slshape list, poset, n})\index{PosetElementFromListOfFlags@\texttt{PosetElementFromListOfFlags}!for IsList,IsPoset,IsInt}
\label{PosetElementFromListOfFlags:for IsList,IsPoset,IsInt}
}\hfill{\scriptsize (operation)}}\\
\textbf{\indent Returns:\ }
\texttt{IsPosetElement} 



 This is used to create a face of rank \mbox{\texttt{\mdseries\slshape n}} from a \mbox{\texttt{\mdseries\slshape list}} of flags of \mbox{\texttt{\mdseries\slshape poset}}. }

 

\subsection{\textcolor{Chapter }{PosetElementFromAtomList (for IsList)}}
\logpage{[ 11, 5, 3 ]}\nobreak
\hyperdef{L}{X794F736784C735B0}{}
{\noindent\textcolor{FuncColor}{$\triangleright$\enspace\texttt{PosetElementFromAtomList({\mdseries\slshape list})\index{PosetElementFromAtomList@\texttt{PosetElementFromAtomList}!for IsList}
\label{PosetElementFromAtomList:for IsList}
}\hfill{\scriptsize (operation)}}\\
\textbf{\indent Returns:\ }
\texttt{IsFace} 



 Creates a \texttt{face} with \mbox{\texttt{\mdseries\slshape list}} of atoms. If you wish to assign ranks or membership in a poset, you must do
this separately. }

 

\subsection{\textcolor{Chapter }{PosetElementFromIndex (for IsObject)}}
\logpage{[ 11, 5, 4 ]}\nobreak
\hyperdef{L}{X7D7FDF76792A6C04}{}
{\noindent\textcolor{FuncColor}{$\triangleright$\enspace\texttt{PosetElementFromIndex({\mdseries\slshape obj})\index{PosetElementFromIndex@\texttt{PosetElementFromIndex}!for IsObject}
\label{PosetElementFromIndex:for IsObject}
}\hfill{\scriptsize (operation)}}\\
\textbf{\indent Returns:\ }
\texttt{IsFace} 



 Creates a \texttt{face} with index \mbox{\texttt{\mdseries\slshape obj}} at rank \mbox{\texttt{\mdseries\slshape n}}. }

 

\subsection{\textcolor{Chapter }{PosetElementWithPartialOrder (for IsObject, IsBinaryRelation)}}
\logpage{[ 11, 5, 5 ]}\nobreak
\hyperdef{L}{X8141EB127AA1FFE1}{}
{\noindent\textcolor{FuncColor}{$\triangleright$\enspace\texttt{PosetElementWithPartialOrder({\mdseries\slshape obj, order})\index{PosetElementWithPartialOrder@\texttt{PosetElementWithPartialOrder}!for IsObject, IsBinaryRelation}
\label{PosetElementWithPartialOrder:for IsObject, IsBinaryRelation}
}\hfill{\scriptsize (operation)}}\\
\textbf{\indent Returns:\ }
\texttt{IsFace} 



 Creates a \texttt{face} with index \mbox{\texttt{\mdseries\slshape obj}} and BinaryRelation \mbox{\texttt{\mdseries\slshape order}} on \mbox{\texttt{\mdseries\slshape obj}}. Function does not check to make sure \mbox{\texttt{\mdseries\slshape order}} has \mbox{\texttt{\mdseries\slshape obj}} in its domain. }

 }

 
\section{\textcolor{Chapter }{Element operations}}\label{Chapter_Posets_Section_Element_operations}
\logpage{[ 11, 6, 0 ]}
\hyperdef{L}{X874169E2825654E9}{}
{
  

\subsection{\textcolor{Chapter }{RankInPoset (for IsPosetElement,IsPoset)}}
\logpage{[ 11, 6, 1 ]}\nobreak
\hyperdef{L}{X78A47D5C8513F4E2}{}
{\noindent\textcolor{FuncColor}{$\triangleright$\enspace\texttt{RankInPoset({\mdseries\slshape [face, poset]})\index{RankInPoset@\texttt{RankInPoset}!for IsPosetElement,IsPoset}
\label{RankInPoset:for IsPosetElement,IsPoset}
}\hfill{\scriptsize (operation)}}\\
\textbf{\indent Returns:\ }
\texttt{IsInt} 



 Given an element \mbox{\texttt{\mdseries\slshape face}} and a poset \mbox{\texttt{\mdseries\slshape poset}} to which it belongs, will give you the rank of \mbox{\texttt{\mdseries\slshape face}} in \mbox{\texttt{\mdseries\slshape poset}}. }

 

\subsection{\textcolor{Chapter }{IsSubface (for IsFace,IsFace,IsPoset)}}
\logpage{[ 11, 6, 2 ]}\nobreak
\hyperdef{L}{X798D28CC83E97F05}{}
{\noindent\textcolor{FuncColor}{$\triangleright$\enspace\texttt{IsSubface({\mdseries\slshape [face1, face2, poset]})\index{IsSubface@\texttt{IsSubface}!for IsFace,IsFace,IsPoset}
\label{IsSubface:for IsFace,IsFace,IsPoset}
}\hfill{\scriptsize (operation)}}\\
\textbf{\indent Returns:\ }
\texttt{true} or \texttt{false} 



 \mbox{\texttt{\mdseries\slshape face1}} and \mbox{\texttt{\mdseries\slshape face2}} are IsFace or IsPosetElement. IsSubface will check to see if \mbox{\texttt{\mdseries\slshape face2}} is a subface of \mbox{\texttt{\mdseries\slshape face1}} in \mbox{\texttt{\mdseries\slshape poset}}. You may drop the argument \mbox{\texttt{\mdseries\slshape poset}} if the faces only belong to one poset in common. Warning: if the elements are
made up of atoms, then IsSubface doesn't need to know what poset you are
working with. }

 

\subsection{\textcolor{Chapter }{IsEqualFaces (for IsFace, IsFace, IsPoset)}}
\logpage{[ 11, 6, 3 ]}\nobreak
\hyperdef{L}{X855B0DBE82C67770}{}
{\noindent\textcolor{FuncColor}{$\triangleright$\enspace\texttt{IsEqualFaces({\mdseries\slshape arg1, arg2, arg3})\index{IsEqualFaces@\texttt{IsEqualFaces}!for IsFace, IsFace, IsPoset}
\label{IsEqualFaces:for IsFace, IsFace, IsPoset}
}\hfill{\scriptsize (operation)}}\\


 Determines whether two faces are equal in a poset. Note that \texttt{\texttt{\symbol{92}}=} tests whether they are the identical object or not. }

 

\subsection{\textcolor{Chapter }{AreIncidentElements (for IsObject,IsObject)}}
\logpage{[ 11, 6, 4 ]}\nobreak
\hyperdef{L}{X84B2B67A805CD31D}{}
{\noindent\textcolor{FuncColor}{$\triangleright$\enspace\texttt{AreIncidentElements({\mdseries\slshape object1, object2})\index{AreIncidentElements@\texttt{AreIncidentElements}!for IsObject,IsObject}
\label{AreIncidentElements:for IsObject,IsObject}
}\hfill{\scriptsize (operation)}}\\
\textbf{\indent Returns:\ }
\texttt{true} or \texttt{false} 



 Given two poset elements, will tell you if they are incident. 
\begin{itemize}
\item  Synonym function: \texttt{AreIncidentFaces}. 
\end{itemize}
 }

 }

 }

   
\chapter{\textcolor{Chapter }{Products of Posets and Digraphs}}\label{Chapter_Products_of_Posets_and_Digraphs}
\logpage{[ 12, 0, 0 ]}
\hyperdef{L}{X85B4AB907E78ADD1}{}
{
  This uses the work of Gleason and Hubard. 
\section{\textcolor{Chapter }{Construction methods}}\label{Chapter_Products_of_Posets_and_Digraphs_Section_Construction_methods}
\logpage{[ 12, 1, 0 ]}
\hyperdef{L}{X7B9C137879BB5529}{}
{
  Anyone know how to link stuff? 

\subsection{\textcolor{Chapter }{JoinProduct (for IsPoset,IsPoset)}}
\logpage{[ 12, 1, 1 ]}\nobreak
\hyperdef{L}{X87B984FB860BDF3F}{}
{\noindent\textcolor{FuncColor}{$\triangleright$\enspace\texttt{JoinProduct({\mdseries\slshape poset1, poset2})\index{JoinProduct@\texttt{JoinProduct}!for IsPoset,IsPoset}
\label{JoinProduct:for IsPoset,IsPoset}
}\hfill{\scriptsize (operation)}}\\
\textbf{\indent Returns:\ }
poset 



 Given two posets, this forms the join product. If given two partial orders,
returns the join product of the partial orders. }

 
\begin{Verbatim}[commandchars=!@|,fontsize=\small,frame=single,label=Example]
  !gapprompt@gap>| !gapinput@p:=PosetFromManiplex(Cube(2));|
  A poset
  !gapprompt@gap>| !gapinput@rel:=BinaryRelationOnPoints([[1,2],[2]]);|
  Binary Relation on 2 points
  !gapprompt@gap>| !gapinput@p1:=PosetFromPartialOrder(rel);|
  A poset using the IsPosetOfIndices representation
  !gapprompt@gap>| !gapinput@j:=JoinProduct(p,p1);|
  A poset using the IsPosetOfIndices representation
  !gapprompt@gap>| !gapinput@IsIsomorphicPoset(j,PosetFromManiplex(PyramidOver(Cube(2))));|
  true
\end{Verbatim}
 

\subsection{\textcolor{Chapter }{CartesianProduct (for IsPoset,IsPoset)}}
\logpage{[ 12, 1, 2 ]}\nobreak
\hyperdef{L}{X80F7AC5385F572D2}{}
{\noindent\textcolor{FuncColor}{$\triangleright$\enspace\texttt{CartesianProduct({\mdseries\slshape polytope1, polytope2})\index{CartesianProduct@\texttt{CartesianProduct}!for IsPoset,IsPoset}
\label{CartesianProduct:for IsPoset,IsPoset}
}\hfill{\scriptsize (operation)}}\\
\textbf{\indent Returns:\ }
polytope 



 Given two polytopes, forms the cartesian product of the polytopes. Should also
work if you give it any two posets. }

 
\begin{Verbatim}[commandchars=!@|,fontsize=\small,frame=single,label=Example]
  !gapprompt@gap>| !gapinput@p1:=PosetFromManiplex(Edge());|
  A poset
  !gapprompt@gap>| !gapinput@p2:=PosetFromManiplex(Simplex(2));|
  A poset
  !gapprompt@gap>| !gapinput@c:=CartesianProduct(p1,p2);|
  A poset using the IsPosetOfIndices representation
  !gapprompt@gap>| !gapinput@IsIsomorphicPoset(c,PosetFromManiplex(PrismOver(Simplex(2))));|
  true
\end{Verbatim}
 

\subsection{\textcolor{Chapter }{DirectSumOfPosets (for IsPoset,IsPoset)}}
\logpage{[ 12, 1, 3 ]}\nobreak
\hyperdef{L}{X8124A8217B62E4D9}{}
{\noindent\textcolor{FuncColor}{$\triangleright$\enspace\texttt{DirectSumOfPosets({\mdseries\slshape polytope1, polytope2})\index{DirectSumOfPosets@\texttt{DirectSumOfPosets}!for IsPoset,IsPoset}
\label{DirectSumOfPosets:for IsPoset,IsPoset}
}\hfill{\scriptsize (operation)}}\\
\textbf{\indent Returns:\ }
polytope 



 Given two polytopes, forms the direct sum of the polytopes. }

 
\begin{Verbatim}[commandchars=!@|,fontsize=\small,frame=single,label=Example]
  !gapprompt@gap>| !gapinput@p1:=PosetFromManiplex(Cube(2));;p2:=PosetFromManiplex(Edge());;|
  !gapprompt@gap>| !gapinput@ds:=DirectSumOfPosets(p1,p2);|
  A poset using the IsPosetOfIndices representation.
  !gapprompt@gap>| !gapinput@IsIsomorphicPoset(ds,PosetFromManiplex(CrossPolytope(3)));|
  true
\end{Verbatim}
 

\subsection{\textcolor{Chapter }{TopologicalProduct (for IsPoset,IsPoset)}}
\logpage{[ 12, 1, 4 ]}\nobreak
\hyperdef{L}{X7D5144DE79A8E5BF}{}
{\noindent\textcolor{FuncColor}{$\triangleright$\enspace\texttt{TopologicalProduct({\mdseries\slshape polytope1, polytope2})\index{TopologicalProduct@\texttt{TopologicalProduct}!for IsPoset,IsPoset}
\label{TopologicalProduct:for IsPoset,IsPoset}
}\hfill{\scriptsize (operation)}}\\
\textbf{\indent Returns:\ }
polytope 



 Given two polytopes, forms the topological product of the polytopes. }

 Here we demonstrate that the topological product (as expected) when taking the
product of a triangle with itself gives us the torus $\{4,4\}_{(3,0)}$ with 72 flags. 
\begin{Verbatim}[commandchars=!@|,fontsize=\small,frame=single,label=Example]
  !gapprompt@gap>| !gapinput@p:=PosetFromManiplex(Pgon(3));|
  A poset using the IsPosetOfFlags representation.
  !gapprompt@gap>| !gapinput@tp:=TopologicalProduct(p,p);|
  A poset using the IsPosetOfIndices representation.
  !gapprompt@gap>| !gapinput@s0 := (5,6);;|
  !gapprompt@gap>| !gapinput@s1 := (1,2)(3,5)(4,6);;|
  !gapprompt@gap>| !gapinput@s2 := (2,3);;|
  !gapprompt@gap>| !gapinput@poly := Group([s0,s1,s2]);;|
  !gapprompt@gap>| !gapinput@torus:=PosetFromManiplex(ReflexibleManiplex(poly));|
  A poset using the IsPosetOfFlags representation.
  !gapprompt@gap>| !gapinput@IsIsomorphicPoset(p,tp);|
  false
  !gapprompt@gap>| !gapinput@IsIsomorphicPoset(torus,tp);|
  true
\end{Verbatim}
 

\subsection{\textcolor{Chapter }{Antiprism (for IsPoset)}}
\logpage{[ 12, 1, 5 ]}\nobreak
\hyperdef{L}{X78FB044D7B9F5DB6}{}
{\noindent\textcolor{FuncColor}{$\triangleright$\enspace\texttt{Antiprism({\mdseries\slshape polytope})\index{Antiprism@\texttt{Antiprism}!for IsPoset}
\label{Antiprism:for IsPoset}
}\hfill{\scriptsize (operation)}}\\
\textbf{\indent Returns:\ }
poset 



 Given a \mbox{\texttt{\mdseries\slshape polytope}} (actually, should work for any poset), will return the antiprism of the \mbox{\texttt{\mdseries\slshape polytope}} (poset). }

 
\begin{Verbatim}[commandchars=!@|,fontsize=\small,frame=single,label=Example]
  !gapprompt@gap>| !gapinput@p:=PosetFromManiplex(Pgon(3));;|
  !gapprompt@gap>| !gapinput@a:=Antiprism(p);;|
  !gapprompt@gap>| !gapinput@IsIsomorphicPoset(a,PosetFromManiplex(CrossPolytope(3)));|
  true
  !gapprompt@gap>| !gapinput@p:=PosetFromManiplex(Pgon(4));;a:=Antiprism(p);;|
  !gapprompt@gap>| !gapinput@d:=DualPoset(p);;ad:=Antiprism(d);;|
  !gapprompt@gap>| !gapinput@IsIsomorphicPoset(a,ad);|
  true
\end{Verbatim}
 }

 }

   
\chapter{\textcolor{Chapter }{Comparing maniplexes}}\label{Chapter_Comparing_maniplexes}
\logpage{[ 13, 0, 0 ]}
\hyperdef{L}{X7A3CC7F9873E02BF}{}
{
  
\section{\textcolor{Chapter }{Quotients and covers}}\label{Chapter_Comparing_maniplexes_Section_Quotients_and_covers}
\logpage{[ 13, 1, 0 ]}
\hyperdef{L}{X7CD5138B85A97590}{}
{
  

\subsection{\textcolor{Chapter }{IsQuotient (for IsManiplex, IsManiplex)}}
\logpage{[ 13, 1, 1 ]}\nobreak
\hyperdef{L}{X7B4CC6757C463F20}{}
{\noindent\textcolor{FuncColor}{$\triangleright$\enspace\texttt{IsQuotient({\mdseries\slshape M1, M2})\index{IsQuotient@\texttt{IsQuotient}!for IsManiplex, IsManiplex}
\label{IsQuotient:for IsManiplex, IsManiplex}
}\hfill{\scriptsize (operation)}}\\


 Returns whether \mbox{\texttt{\mdseries\slshape M1}} is a quotient of \mbox{\texttt{\mdseries\slshape M2}}. }

 

\subsection{\textcolor{Chapter }{IsCover (for IsManiplex, IsManiplex)}}
\logpage{[ 13, 1, 2 ]}\nobreak
\hyperdef{L}{X845044F47ACF897C}{}
{\noindent\textcolor{FuncColor}{$\triangleright$\enspace\texttt{IsCover({\mdseries\slshape M1, M2})\index{IsCover@\texttt{IsCover}!for IsManiplex, IsManiplex}
\label{IsCover:for IsManiplex, IsManiplex}
}\hfill{\scriptsize (operation)}}\\


 Returns whether \mbox{\texttt{\mdseries\slshape M1}} is a cover of \mbox{\texttt{\mdseries\slshape M2}}. }

 

\subsection{\textcolor{Chapter }{IsIsomorphicManiplex (for IsManiplex, IsManiplex)}}
\logpage{[ 13, 1, 3 ]}\nobreak
\hyperdef{L}{X7B1006697C002741}{}
{\noindent\textcolor{FuncColor}{$\triangleright$\enspace\texttt{IsIsomorphicManiplex({\mdseries\slshape M1, M2})\index{IsIsomorphicManiplex@\texttt{IsIsomorphicManiplex}!for IsManiplex, IsManiplex}
\label{IsIsomorphicManiplex:for IsManiplex, IsManiplex}
}\hfill{\scriptsize (operation)}}\\


 Returns whether \mbox{\texttt{\mdseries\slshape M1}} is isomorphic to \mbox{\texttt{\mdseries\slshape M2}}. }

 

\subsection{\textcolor{Chapter }{SmallestRegularCover (for IsManiplex)}}
\logpage{[ 13, 1, 4 ]}\nobreak
\hyperdef{L}{X879C71E784C41B77}{}
{\noindent\textcolor{FuncColor}{$\triangleright$\enspace\texttt{SmallestRegularCover({\mdseries\slshape M})\index{SmallestRegularCover@\texttt{SmallestRegularCover}!for IsManiplex}
\label{SmallestRegularCover:for IsManiplex}
}\hfill{\scriptsize (attribute)}}\\


 Returns the smallest regular cover of \mbox{\texttt{\mdseries\slshape M}}, which is the maniplex whose automorphism group is the connection group of \mbox{\texttt{\mdseries\slshape M}}. }

 

\subsection{\textcolor{Chapter }{QuotientManiplex (for IsReflexibleManiplex, IsString)}}
\logpage{[ 13, 1, 5 ]}\nobreak
\hyperdef{L}{X85BE14877927DB85}{}
{\noindent\textcolor{FuncColor}{$\triangleright$\enspace\texttt{QuotientManiplex({\mdseries\slshape M, relStr})\index{QuotientManiplex@\texttt{QuotientManiplex}!for IsReflexibleManiplex, IsString}
\label{QuotientManiplex:for IsReflexibleManiplex, IsString}
}\hfill{\scriptsize (operation)}}\\


 Given a reflexible maniplex \mbox{\texttt{\mdseries\slshape M}}, generates the subgroup S of AutomorphismGroup(\mbox{\texttt{\mdseries\slshape M}}) given by relStr, and returns the quotient maniplex M / S. For example,
QuotientManiplex(CubicTiling(2), "(r0 r1 r2 r1)\texttt{\symbol{94}}5, (r1 r0
r1 r2)\texttt{\symbol{94}}2") returns the toroidal map
\texttt{\symbol{123}}4,4\texttt{\symbol{125}}{\textunderscore}\texttt{\symbol{123}}(5,0),(0,2)\texttt{\symbol{125}}.
You can also input this as CubicTiling(2) / "(r0 r1 r2
r1)\texttt{\symbol{94}}5, (r1 r0 r1 r2)\texttt{\symbol{94}}2". }

 

\subsection{\textcolor{Chapter }{ReflexibleQuotientManiplex (for IsManiplex, IsList)}}
\logpage{[ 13, 1, 6 ]}\nobreak
\hyperdef{L}{X87DCBB9E78E383BB}{}
{\noindent\textcolor{FuncColor}{$\triangleright$\enspace\texttt{ReflexibleQuotientManiplex({\mdseries\slshape M, rels})\index{ReflexibleQuotientManiplex@\texttt{ReflexibleQuotientManiplex}!for IsManiplex, IsList}
\label{ReflexibleQuotientManiplex:for IsManiplex, IsList}
}\hfill{\scriptsize (operation)}}\\


 Given a reflexible maniplex \mbox{\texttt{\mdseries\slshape M}}, generates the normal closure N of the subgroup S of AutomorphismGroup(\mbox{\texttt{\mdseries\slshape M}}) given by relStr, and returns the quotient maniplex M / N, which will be
reflexible. For example, QuotientManiplex(CubicTiling(2), "(r0 r1 r2
r1)\texttt{\symbol{94}}5, (r1 r0 r1 r2)\texttt{\symbol{94}}2") returns the
toroidal map
\texttt{\symbol{123}}4,4\texttt{\symbol{125}}{\textunderscore}\texttt{\symbol{123}}(1,0)\texttt{\symbol{125}},
because the normal closure of the group generated by (r0 r1 r2
r1)\texttt{\symbol{94}}5 and (r1 r0 r1 r2)\texttt{\symbol{94}}2 is the group
generated by r0 r1 r2 r1 and r1 r0 r1 r2. }

 }

 }

   
\chapter{\textcolor{Chapter }{ramp automatic generated documentation}}\label{Chapter_ramp_automatic_generated_documentation}
\logpage{[ 14, 0, 0 ]}
\hyperdef{L}{X7CFFADAB8780AAA2}{}
{
  
\section{\textcolor{Chapter }{ramp automatic generated documentation of methods}}\label{Chapter_ramp_automatic_generated_documentation_Section_ramp_automatic_generated_documentation_of_methods}
\logpage{[ 14, 1, 0 ]}
\hyperdef{L}{X7C78352D81F80B4B}{}
{
  

\subsection{\textcolor{Chapter }{UniversalRotationGroup (for IsInt)}}
\logpage{[ 14, 1, 1 ]}\nobreak
\hyperdef{L}{X8056E3557E06EFF6}{}
{\noindent\textcolor{FuncColor}{$\triangleright$\enspace\texttt{UniversalRotationGroup({\mdseries\slshape n})\index{UniversalRotationGroup@\texttt{UniversalRotationGroup}!for IsInt}
\label{UniversalRotationGroup:for IsInt}
}\hfill{\scriptsize (operation)}}\\


 Returns the rotation subgroup of the universal Coxeter Group of rank n. }

 

\subsection{\textcolor{Chapter }{UniversalRotationGroup (for IsList)}}
\logpage{[ 14, 1, 2 ]}\nobreak
\hyperdef{L}{X7D6042317A76FDC3}{}
{\noindent\textcolor{FuncColor}{$\triangleright$\enspace\texttt{UniversalRotationGroup({\mdseries\slshape sym})\index{UniversalRotationGroup@\texttt{UniversalRotationGroup}!for IsList}
\label{UniversalRotationGroup:for IsList}
}\hfill{\scriptsize (operation)}}\\


 Returns the rotation subgroup of the Coxeter Group with Schlafli symbol sym. }

 

\subsection{\textcolor{Chapter }{RotaryManiplex (for IsGroup)}}
\logpage{[ 14, 1, 3 ]}\nobreak
\hyperdef{L}{X84662FEC86F981A0}{}
{\noindent\textcolor{FuncColor}{$\triangleright$\enspace\texttt{RotaryManiplex({\mdseries\slshape g})\index{RotaryManiplex@\texttt{RotaryManiplex}!for IsGroup}
\label{RotaryManiplex:for IsGroup}
}\hfill{\scriptsize (operation)}}\\


 Given a group g (which should be a string rotation group), returns the rotary
maniplex with that rotation group, where the privileged generators are those
returned by GeneratorsOfGroup(g). }

 

\subsection{\textcolor{Chapter }{RotaryManiplex (for IsList)}}
\logpage{[ 14, 1, 4 ]}\nobreak
\hyperdef{L}{X85D4214883208895}{}
{\noindent\textcolor{FuncColor}{$\triangleright$\enspace\texttt{RotaryManiplex({\mdseries\slshape sym})\index{RotaryManiplex@\texttt{RotaryManiplex}!for IsList}
\label{RotaryManiplex:for IsList}
}\hfill{\scriptsize (operation)}}\\


 Returns the universal rotary maniplex (in fact, regular polytope) with
Schlafli symbol \mbox{\texttt{\mdseries\slshape sym}}. }

 

\subsection{\textcolor{Chapter }{RotaryManiplex (for IsList, IsList)}}
\logpage{[ 14, 1, 5 ]}\nobreak
\hyperdef{L}{X7C4F288287BD75D6}{}
{\noindent\textcolor{FuncColor}{$\triangleright$\enspace\texttt{RotaryManiplex({\mdseries\slshape symbol, relations})\index{RotaryManiplex@\texttt{RotaryManiplex}!for IsList, IsList}
\label{RotaryManiplex:for IsList, IsList}
}\hfill{\scriptsize (operation)}}\\


 Returns the rotary maniplex with the given Schlafli symbol and with the given
relations. The relations are given by a string that refers to the generators
s1, s2, etc. For example: 
\begin{Verbatim}[commandchars=!@|,fontsize=\small,frame=single,label=Example]
  !gapprompt@gap>| !gapinput@M := RotaryManiplex([4,4], "(s2^-1 s1)^6");;|
\end{Verbatim}
 If the option set{\textunderscore}schlafli is set, then we set the Schlafli
symbol to the one given. This may not be the correct Schlafli symbol, since
the relations may cause a collapse, so this should only be used if you know
that the Schlafli symbol is correct. }

 

\subsection{\textcolor{Chapter }{EnantiomorphicForm (for IsRotaryManiplex)}}
\logpage{[ 14, 1, 6 ]}\nobreak
\hyperdef{L}{X7E64776382984E5A}{}
{\noindent\textcolor{FuncColor}{$\triangleright$\enspace\texttt{EnantiomorphicForm({\mdseries\slshape M})\index{EnantiomorphicForm@\texttt{EnantiomorphicForm}!for IsRotaryManiplex}
\label{EnantiomorphicForm:for IsRotaryManiplex}
}\hfill{\scriptsize (operation)}}\\


 The \emph{enantiomorphic form} of a rotary maniplex is the same maniplex, but where we choose the new base
flag to be one of the flags that is adjacent to the original base flag. If M
is reflexible, then this choice has no effect. Otherwise, if M is chiral, then
the enantiomorphic form gives us a different presentation for the rotation
group. }

 

\subsection{\textcolor{Chapter }{DatabaseString (for IsManiplex)}}
\logpage{[ 14, 1, 7 ]}\nobreak
\hyperdef{L}{X7B89A32E7D94769F}{}
{\noindent\textcolor{FuncColor}{$\triangleright$\enspace\texttt{DatabaseString({\mdseries\slshape M})\index{DatabaseString@\texttt{DatabaseString}!for IsManiplex}
\label{DatabaseString:for IsManiplex}
}\hfill{\scriptsize (operation)}}\\
\textbf{\indent Returns:\ }
String 



 Given a maniplex \mbox{\texttt{\mdseries\slshape M}}, returns a string representation of \mbox{\texttt{\mdseries\slshape M}} suitable for saving in a database for later retrieval. }

 

\subsection{\textcolor{Chapter }{ManiplexFromDatabaseString (for IsString)}}
\logpage{[ 14, 1, 8 ]}\nobreak
\hyperdef{L}{X84FC221B80847F84}{}
{\noindent\textcolor{FuncColor}{$\triangleright$\enspace\texttt{ManiplexFromDatabaseString({\mdseries\slshape maniplexString})\index{ManiplexFromDatabaseString@\texttt{ManiplexFromDatabaseString}!for IsString}
\label{ManiplexFromDatabaseString:for IsString}
}\hfill{\scriptsize (operation)}}\\
\textbf{\indent Returns:\ }
IsManiplex 



 Given a string \mbox{\texttt{\mdseries\slshape maniplexString}}, representing a maniplex stored in a database, returns the maniplex that is
represented. }

 

\subsection{\textcolor{Chapter }{Cuboctahedron}}
\logpage{[ 14, 1, 9 ]}\nobreak
\hyperdef{L}{X7D1C3FE3780376C5}{}
{\noindent\textcolor{FuncColor}{$\triangleright$\enspace\texttt{Cuboctahedron({\mdseries\slshape })\index{Cuboctahedron@\texttt{Cuboctahedron}}
\label{Cuboctahedron}
}\hfill{\scriptsize (operation)}}\\
\textbf{\indent Returns:\ }
maniplex 



 Constructs the cuboctahedron. }

 

\subsection{\textcolor{Chapter }{TruncatedTetrahedron}}
\logpage{[ 14, 1, 10 ]}\nobreak
\hyperdef{L}{X7914BBCD85B92572}{}
{\noindent\textcolor{FuncColor}{$\triangleright$\enspace\texttt{TruncatedTetrahedron({\mdseries\slshape })\index{TruncatedTetrahedron@\texttt{TruncatedTetrahedron}}
\label{TruncatedTetrahedron}
}\hfill{\scriptsize (operation)}}\\
\textbf{\indent Returns:\ }
maniplex 



 Constructs the truncated tetrahedron. }

 }

 }

   
\chapter{\textcolor{Chapter }{Utility functions}}\label{Chapter_Utility_functions}
\logpage{[ 15, 0, 0 ]}
\hyperdef{L}{X810FFB1C8035C8BE}{}
{
  
\section{\textcolor{Chapter }{Utility functions}}\label{Chapter_Utility_functions_Section_Utility_functions}
\logpage{[ 15, 1, 0 ]}
\hyperdef{L}{X810FFB1C8035C8BE}{}
{
  

\subsection{\textcolor{Chapter }{AbstractPolytope}}
\logpage{[ 15, 1, 1 ]}\nobreak
\hyperdef{L}{X7D63A6087BD7F5F8}{}
{\noindent\textcolor{FuncColor}{$\triangleright$\enspace\texttt{AbstractPolytope({\mdseries\slshape args})\index{AbstractPolytope@\texttt{AbstractPolytope}}
\label{AbstractPolytope}
}\hfill{\scriptsize (function)}}\\


 Calls \texttt{Maniplex(args)} and marks the output as polytopal. }

 

\subsection{\textcolor{Chapter }{AbstractRegularPolytope}}
\logpage{[ 15, 1, 2 ]}\nobreak
\hyperdef{L}{X7C2837FB8174C8DC}{}
{\noindent\textcolor{FuncColor}{$\triangleright$\enspace\texttt{AbstractRegularPolytope({\mdseries\slshape args})\index{AbstractRegularPolytope@\texttt{AbstractRegularPolytope}}
\label{AbstractRegularPolytope}
}\hfill{\scriptsize (function)}}\\


 Calls \texttt{ReflexibleManiplex(args)} and marks the output as polytopal. Also available as \texttt{ARP(args)}. }

 

\subsection{\textcolor{Chapter }{AbstractRotaryPolytope}}
\logpage{[ 15, 1, 3 ]}\nobreak
\hyperdef{L}{X79AB055B7DDFB160}{}
{\noindent\textcolor{FuncColor}{$\triangleright$\enspace\texttt{AbstractRotaryPolytope({\mdseries\slshape args})\index{AbstractRotaryPolytope@\texttt{AbstractRotaryPolytope}}
\label{AbstractRotaryPolytope}
}\hfill{\scriptsize (function)}}\\


 Calls \texttt{RotaryManiplex(args)} and marks the output as polytopal. }

 

\subsection{\textcolor{Chapter }{TranslatePerm}}
\logpage{[ 15, 1, 4 ]}\nobreak
\hyperdef{L}{X7CDF25897D4DBE57}{}
{\noindent\textcolor{FuncColor}{$\triangleright$\enspace\texttt{TranslatePerm({\mdseries\slshape perm, k})\index{TranslatePerm@\texttt{TranslatePerm}}
\label{TranslatePerm}
}\hfill{\scriptsize (function)}}\\


 Returns a new permutation obtained from \mbox{\texttt{\mdseries\slshape perm}} by adding k to each moved point. }

 

\subsection{\textcolor{Chapter }{MultPerm}}
\logpage{[ 15, 1, 5 ]}\nobreak
\hyperdef{L}{X83788BBF7EF52486}{}
{\noindent\textcolor{FuncColor}{$\triangleright$\enspace\texttt{MultPerm({\mdseries\slshape perm, multiplier, offset})\index{MultPerm@\texttt{MultPerm}}
\label{MultPerm}
}\hfill{\scriptsize (function)}}\\


 Multiplies together perm, TranslatePerm(perm, offset), TranslatePerm(perm,
offset*2), ..., with \mbox{\texttt{\mdseries\slshape multiplier}} terms, and returns the result. }

 

\subsection{\textcolor{Chapter }{ParseStringCRels}}
\logpage{[ 15, 1, 6 ]}\nobreak
\hyperdef{L}{X7E84DB4E8746344A}{}
{\noindent\textcolor{FuncColor}{$\triangleright$\enspace\texttt{ParseStringCRels({\mdseries\slshape rels, g})\index{ParseStringCRels@\texttt{ParseStringCRels}}
\label{ParseStringCRels}
}\hfill{\scriptsize (function)}}\\


 This helper function is used in several maniplex constructors. Given a string \mbox{\texttt{\mdseries\slshape rels}} that represents relations in an sggi, and an sggi g, returns a list of
elements of g represented by \mbox{\texttt{\mdseries\slshape rels}}. 
\begin{Verbatim}[commandchars=!@|,fontsize=\small,frame=single,label=Example]
  g := AutomorphismGroup(CubicTiling(2));;
  rels := "(r0 r1 r2 r1)^6";;
  newrels := ParseStringCRels(rels, g);
  [ (r0*r1*r2*r1)^6 ]
\end{Verbatim}
 For convenience, you may use z1, z2, etc and h1, h2, etc in relations, where
zj means r0 (r1 r2)\texttt{\symbol{94}}j and hj means r0 (r1
r2)\texttt{\symbol{94}}j-1 r1. (That is, the former is the word corresponding
to j-zigzags of a polyhedron, and the latter corresponds to j-holes.) }

 

\subsection{\textcolor{Chapter }{ParseRotGpRels}}
\logpage{[ 15, 1, 7 ]}\nobreak
\hyperdef{L}{X79C6D76380B9F364}{}
{\noindent\textcolor{FuncColor}{$\triangleright$\enspace\texttt{ParseRotGpRels({\mdseries\slshape rels, g})\index{ParseRotGpRels@\texttt{ParseRotGpRels}}
\label{ParseRotGpRels}
}\hfill{\scriptsize (function)}}\\


 This helper function is used in several maniplex constructors. It is analogous
to ParseStringCRels, but for rotation groups instead. }

 

\subsection{\textcolor{Chapter }{AddOrAppend}}
\logpage{[ 15, 1, 8 ]}\nobreak
\hyperdef{L}{X829B5DA779342A28}{}
{\noindent\textcolor{FuncColor}{$\triangleright$\enspace\texttt{AddOrAppend({\mdseries\slshape L, x})\index{AddOrAppend@\texttt{AddOrAppend}}
\label{AddOrAppend}
}\hfill{\scriptsize (function)}}\\


 Given a list \mbox{\texttt{\mdseries\slshape L}} and an object \mbox{\texttt{\mdseries\slshape x}}, this calls Append(L, x) if x is a list; otherwise it calls Add(L, x). Note
that since strings are internally represented as lists, AddOrAppend(L, "foo")
will append the characters 'f', 'o', 'o'. }

 }

 }

 \def\indexname{Index\logpage{[ "Ind", 0, 0 ]}
\hyperdef{L}{X83A0356F839C696F}{}
}

\cleardoublepage
\phantomsection
\addcontentsline{toc}{chapter}{Index}


\printindex

\immediate\write\pagenrlog{["Ind", 0, 0], \arabic{page},}
\immediate\write\pagenrlog{["Ind", 0, 0], \arabic{page},}
\newpage
\immediate\write\pagenrlog{["End"], \arabic{page}];}
\immediate\closeout\pagenrlog
\end{document}
