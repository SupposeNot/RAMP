% generated by GAPDoc2LaTeX from XML source (Frank Luebeck)
\documentclass[a4paper,11pt]{report}

\usepackage[top=37mm,bottom=37mm,left=27mm,right=27mm]{geometry}
\sloppy
\pagestyle{myheadings}
\usepackage{amssymb}
\usepackage[utf8]{inputenc}
\usepackage{makeidx}
\makeindex
\usepackage{color}
\definecolor{FireBrick}{rgb}{0.5812,0.0074,0.0083}
\definecolor{RoyalBlue}{rgb}{0.0236,0.0894,0.6179}
\definecolor{RoyalGreen}{rgb}{0.0236,0.6179,0.0894}
\definecolor{RoyalRed}{rgb}{0.6179,0.0236,0.0894}
\definecolor{LightBlue}{rgb}{0.8544,0.9511,1.0000}
\definecolor{Black}{rgb}{0.0,0.0,0.0}

\definecolor{linkColor}{rgb}{0.0,0.0,0.554}
\definecolor{citeColor}{rgb}{0.0,0.0,0.554}
\definecolor{fileColor}{rgb}{0.0,0.0,0.554}
\definecolor{urlColor}{rgb}{0.0,0.0,0.554}
\definecolor{promptColor}{rgb}{0.0,0.0,0.589}
\definecolor{brkpromptColor}{rgb}{0.589,0.0,0.0}
\definecolor{gapinputColor}{rgb}{0.589,0.0,0.0}
\definecolor{gapoutputColor}{rgb}{0.0,0.0,0.0}

%%  for a long time these were red and blue by default,
%%  now black, but keep variables to overwrite
\definecolor{FuncColor}{rgb}{0.0,0.0,0.0}
%% strange name because of pdflatex bug:
\definecolor{Chapter }{rgb}{0.0,0.0,0.0}
\definecolor{DarkOlive}{rgb}{0.1047,0.2412,0.0064}


\usepackage{fancyvrb}

\usepackage{mathptmx,helvet}
\usepackage[T1]{fontenc}
\usepackage{textcomp}


\usepackage[
            pdftex=true,
            bookmarks=true,        
            a4paper=true,
            pdftitle={Written with GAPDoc},
            pdfcreator={LaTeX with hyperref package / GAPDoc},
            colorlinks=true,
            backref=page,
            breaklinks=true,
            linkcolor=linkColor,
            citecolor=citeColor,
            filecolor=fileColor,
            urlcolor=urlColor,
            pdfpagemode={UseNone}, 
           ]{hyperref}

\newcommand{\maintitlesize}{\fontsize{50}{55}\selectfont}

% write page numbers to a .pnr log file for online help
\newwrite\pagenrlog
\immediate\openout\pagenrlog =\jobname.pnr
\immediate\write\pagenrlog{PAGENRS := [}
\newcommand{\logpage}[1]{\protect\write\pagenrlog{#1, \thepage,}}
%% were never documented, give conflicts with some additional packages

\newcommand{\GAP}{\textsf{GAP}}

%% nicer description environments, allows long labels
\usepackage{enumitem}
\setdescription{style=nextline}

%% depth of toc
\setcounter{tocdepth}{1}





%% command for ColorPrompt style examples
\newcommand{\gapprompt}[1]{\color{promptColor}{\bfseries #1}}
\newcommand{\gapbrkprompt}[1]{\color{brkpromptColor}{\bfseries #1}}
\newcommand{\gapinput}[1]{\color{gapinputColor}{#1}}


\begin{document}

\logpage{[ 0, 0, 0 ]}
\begin{titlepage}
\mbox{}\vfill

\begin{center}{\maintitlesize \textbf{ RAMP \mbox{}}}\\
\vfill

\hypersetup{pdftitle= RAMP }
\markright{\scriptsize \mbox{}\hfill  RAMP  \hfill\mbox{}}
{\Huge \textbf{ The Research Assistant for Maniplexes and Polytopes \mbox{}}}\\
\vfill

{\Huge  0.3 \mbox{}}\\[1cm]
{ 28 July 2020 \mbox{}}\\[1cm]
\mbox{}\\[2cm]
{\Large \textbf{ Gabe Cunningham\\
    \mbox{}}}\\
\hypersetup{pdfauthor= Gabe Cunningham\\
    }
\end{center}\vfill

\mbox{}\\
{\mbox{}\\
\small \noindent \textbf{ Gabe Cunningham\\
    }  Email: \href{mailto://gabriel.cunningham@umb.edu} {\texttt{gabriel.cunningham@umb.edu}}\\
  Homepage: \href{http://www.gabrielcunningham.com} {\texttt{http://www.gabrielcunningham.com}}\\
  Address: \begin{minipage}[t]{8cm}\noindent
 Gabe Cunningham\\
 Department of Mathematics\\
 University of Massachusetts Boston\\
 100 William T. Morrissey Blvd.\\
 Boston MA 02125\\
 \end{minipage}
}\\
\end{titlepage}

\newpage\setcounter{page}{2}
{\small 
\section*{Copyright}
\logpage{[ 0, 0, 1 ]}
 \index{License} {\copyright} 1997-2020 by Gabe Cunningham

 \textsf{RAMP} package is free software; you can redistribute it and/or modify it under the
terms of the \href{http://www.fsf.org/licenses/gpl.html} {GNU General Public License} as published by the Free Software Foundation; either version 2 of the License,
or (at your option) any later version. \mbox{}}\\[1cm]
{\small 
\section*{Acknowledgements}
\logpage{[ 0, 0, 2 ]}
 We appreciate very much all past and future comments, suggestions and
contributions to this package and its documentation provided by \textsf{GAP} users and developers. \mbox{}}\\[1cm]
\newpage

\def\contentsname{Contents\logpage{[ 0, 0, 3 ]}}

\tableofcontents
\newpage

     
\chapter{\textcolor{Chapter }{Combinatorics and Structure}}\label{Chapter_Combinatorics_and_Structure}
\logpage{[ 1, 0, 0 ]}
\hyperdef{L}{X8153E7658330D710}{}
{
  
\section{\textcolor{Chapter }{Faces}}\label{Chapter_Combinatorics_and_Structure_Section_Faces}
\logpage{[ 1, 1, 0 ]}
\hyperdef{L}{X872AD1E785C7EB03}{}
{
  

\subsection{\textcolor{Chapter }{NumberOfIFaces (for IsAbstractPolytope, IsInt)}}
\logpage{[ 1, 1, 1 ]}\nobreak
\hyperdef{L}{X7C9767CF83DF2D4E}{}
{\noindent\textcolor{FuncColor}{$\triangleright$\enspace\texttt{NumberOfIFaces({\mdseries\slshape P, i})\index{NumberOfIFaces@\texttt{NumberOfIFaces}!for IsAbstractPolytope, IsInt}
\label{NumberOfIFaces:for IsAbstractPolytope, IsInt}
}\hfill{\scriptsize (operation)}}\\
\textbf{\indent Returns:\ }
The number of \mbox{\texttt{\mdseries\slshape i}}-faces that \mbox{\texttt{\mdseries\slshape P}} has. 



 

 }

 

\subsection{\textcolor{Chapter }{NumberOfVertices (for IsAbstractPolytope)}}
\logpage{[ 1, 1, 2 ]}\nobreak
\hyperdef{L}{X79F0778B813DF3F5}{}
{\noindent\textcolor{FuncColor}{$\triangleright$\enspace\texttt{NumberOfVertices({\mdseries\slshape P})\index{NumberOfVertices@\texttt{NumberOfVertices}!for IsAbstractPolytope}
\label{NumberOfVertices:for IsAbstractPolytope}
}\hfill{\scriptsize (attribute)}}\\


 

 }

 

\subsection{\textcolor{Chapter }{NumberOfEdges (for IsAbstractPolytope)}}
\logpage{[ 1, 1, 3 ]}\nobreak
\hyperdef{L}{X7F9A1F897CA0DB2E}{}
{\noindent\textcolor{FuncColor}{$\triangleright$\enspace\texttt{NumberOfEdges({\mdseries\slshape P})\index{NumberOfEdges@\texttt{NumberOfEdges}!for IsAbstractPolytope}
\label{NumberOfEdges:for IsAbstractPolytope}
}\hfill{\scriptsize (attribute)}}\\


 

 }

 

\subsection{\textcolor{Chapter }{NumberOfFacets (for IsAbstractPolytope)}}
\logpage{[ 1, 1, 4 ]}\nobreak
\hyperdef{L}{X8551AD7A7F965D6B}{}
{\noindent\textcolor{FuncColor}{$\triangleright$\enspace\texttt{NumberOfFacets({\mdseries\slshape P})\index{NumberOfFacets@\texttt{NumberOfFacets}!for IsAbstractPolytope}
\label{NumberOfFacets:for IsAbstractPolytope}
}\hfill{\scriptsize (attribute)}}\\


 

 }

 

\subsection{\textcolor{Chapter }{NumberOfRidges (for IsAbstractPolytope)}}
\logpage{[ 1, 1, 5 ]}\nobreak
\hyperdef{L}{X7B3106A17B70F717}{}
{\noindent\textcolor{FuncColor}{$\triangleright$\enspace\texttt{NumberOfRidges({\mdseries\slshape P})\index{NumberOfRidges@\texttt{NumberOfRidges}!for IsAbstractPolytope}
\label{NumberOfRidges:for IsAbstractPolytope}
}\hfill{\scriptsize (attribute)}}\\


 

 }

 

\subsection{\textcolor{Chapter }{Fvector (for IsAbstractPolytope)}}
\logpage{[ 1, 1, 6 ]}\nobreak
\hyperdef{L}{X7B44EFF77918F030}{}
{\noindent\textcolor{FuncColor}{$\triangleright$\enspace\texttt{Fvector({\mdseries\slshape P})\index{Fvector@\texttt{Fvector}!for IsAbstractPolytope}
\label{Fvector:for IsAbstractPolytope}
}\hfill{\scriptsize (attribute)}}\\


 

 }

 

\subsection{\textcolor{Chapter }{Facets (for IsAbstractPolytope)}}
\logpage{[ 1, 1, 7 ]}\nobreak
\hyperdef{L}{X8764C78A7A67D088}{}
{\noindent\textcolor{FuncColor}{$\triangleright$\enspace\texttt{Facets({\mdseries\slshape P})\index{Facets@\texttt{Facets}!for IsAbstractPolytope}
\label{Facets:for IsAbstractPolytope}
}\hfill{\scriptsize (attribute)}}\\


 Currently only works for regular polytopes. }

 

\subsection{\textcolor{Chapter }{VertexFigures (for IsAbstractPolytope)}}
\logpage{[ 1, 1, 8 ]}\nobreak
\hyperdef{L}{X80B10A8C825B30A2}{}
{\noindent\textcolor{FuncColor}{$\triangleright$\enspace\texttt{VertexFigures({\mdseries\slshape P})\index{VertexFigures@\texttt{VertexFigures}!for IsAbstractPolytope}
\label{VertexFigures:for IsAbstractPolytope}
}\hfill{\scriptsize (attribute)}}\\


 Currently only works for regular polytopes. }

 }

 }

   
\chapter{\textcolor{Chapter }{Families of Polytopes}}\label{Chapter_Families_of_Polytopes}
\logpage{[ 2, 0, 0 ]}
\hyperdef{L}{X7BF24A5D7B3386D7}{}
{
  
\section{\textcolor{Chapter }{Classical Polytopes}}\label{Chapter_Families_of_Polytopes_Section_Classical_Polytopes}
\logpage{[ 2, 1, 0 ]}
\hyperdef{L}{X7A281A1C7DCB2D96}{}
{
  

\subsection{\textcolor{Chapter }{Pgon (for IsInt)}}
\logpage{[ 2, 1, 1 ]}\nobreak
\hyperdef{L}{X8436B8097852EF9B}{}
{\noindent\textcolor{FuncColor}{$\triangleright$\enspace\texttt{Pgon({\mdseries\slshape p})\index{Pgon@\texttt{Pgon}!for IsInt}
\label{Pgon:for IsInt}
}\hfill{\scriptsize (operation)}}\\


 

 }

 

\subsection{\textcolor{Chapter }{Cube (for IsInt)}}
\logpage{[ 2, 1, 2 ]}\nobreak
\hyperdef{L}{X8306D0C17A6BDDCE}{}
{\noindent\textcolor{FuncColor}{$\triangleright$\enspace\texttt{Cube({\mdseries\slshape n})\index{Cube@\texttt{Cube}!for IsInt}
\label{Cube:for IsInt}
}\hfill{\scriptsize (operation)}}\\


 

 }

 

\subsection{\textcolor{Chapter }{HemiCube (for IsInt)}}
\logpage{[ 2, 1, 3 ]}\nobreak
\hyperdef{L}{X828C02F4861F0CD3}{}
{\noindent\textcolor{FuncColor}{$\triangleright$\enspace\texttt{HemiCube({\mdseries\slshape n})\index{HemiCube@\texttt{HemiCube}!for IsInt}
\label{HemiCube:for IsInt}
}\hfill{\scriptsize (operation)}}\\


 

 }

 

\subsection{\textcolor{Chapter }{CrossPolytope (for IsInt)}}
\logpage{[ 2, 1, 4 ]}\nobreak
\hyperdef{L}{X78DC5BA486C3288C}{}
{\noindent\textcolor{FuncColor}{$\triangleright$\enspace\texttt{CrossPolytope({\mdseries\slshape n})\index{CrossPolytope@\texttt{CrossPolytope}!for IsInt}
\label{CrossPolytope:for IsInt}
}\hfill{\scriptsize (operation)}}\\


 

 }

 

\subsection{\textcolor{Chapter }{HemiCrossPolytope (for IsInt)}}
\logpage{[ 2, 1, 5 ]}\nobreak
\hyperdef{L}{X7B0D84B87F00A73F}{}
{\noindent\textcolor{FuncColor}{$\triangleright$\enspace\texttt{HemiCrossPolytope({\mdseries\slshape n})\index{HemiCrossPolytope@\texttt{HemiCrossPolytope}!for IsInt}
\label{HemiCrossPolytope:for IsInt}
}\hfill{\scriptsize (operation)}}\\


 

 }

 

\subsection{\textcolor{Chapter }{Simplex (for IsInt)}}
\logpage{[ 2, 1, 6 ]}\nobreak
\hyperdef{L}{X82C75D12838D3FD0}{}
{\noindent\textcolor{FuncColor}{$\triangleright$\enspace\texttt{Simplex({\mdseries\slshape n})\index{Simplex@\texttt{Simplex}!for IsInt}
\label{Simplex:for IsInt}
}\hfill{\scriptsize (operation)}}\\


 

 }

 

\subsection{\textcolor{Chapter }{CubicTiling (for IsInt)}}
\logpage{[ 2, 1, 7 ]}\nobreak
\hyperdef{L}{X7CCDE7817EC0E5B5}{}
{\noindent\textcolor{FuncColor}{$\triangleright$\enspace\texttt{CubicTiling({\mdseries\slshape n})\index{CubicTiling@\texttt{CubicTiling}!for IsInt}
\label{CubicTiling:for IsInt}
}\hfill{\scriptsize (operation)}}\\


 

 }

 

 

\subsection{\textcolor{Chapter }{Dodecahedron}}
\logpage{[ 2, 1, 8 ]}\nobreak
\hyperdef{L}{X81A6D8FE876EB3BE}{}
{\noindent\textcolor{FuncColor}{$\triangleright$\enspace\texttt{Dodecahedron({\mdseries\slshape })\index{Dodecahedron@\texttt{Dodecahedron}}
\label{Dodecahedron}
}\hfill{\scriptsize (operation)}}\\


 

 }

 

 

\subsection{\textcolor{Chapter }{HemiDodecahedron}}
\logpage{[ 2, 1, 9 ]}\nobreak
\hyperdef{L}{X7A49325F782047CC}{}
{\noindent\textcolor{FuncColor}{$\triangleright$\enspace\texttt{HemiDodecahedron({\mdseries\slshape })\index{HemiDodecahedron@\texttt{HemiDodecahedron}}
\label{HemiDodecahedron}
}\hfill{\scriptsize (operation)}}\\


 

 }

 

 

\subsection{\textcolor{Chapter }{Icosahedron}}
\logpage{[ 2, 1, 10 ]}\nobreak
\hyperdef{L}{X83E0EF8F7CCD6979}{}
{\noindent\textcolor{FuncColor}{$\triangleright$\enspace\texttt{Icosahedron({\mdseries\slshape })\index{Icosahedron@\texttt{Icosahedron}}
\label{Icosahedron}
}\hfill{\scriptsize (operation)}}\\


 

 }

 

 

\subsection{\textcolor{Chapter }{HemiIcosahedron}}
\logpage{[ 2, 1, 11 ]}\nobreak
\hyperdef{L}{X7CBB4FC88050D25F}{}
{\noindent\textcolor{FuncColor}{$\triangleright$\enspace\texttt{HemiIcosahedron({\mdseries\slshape })\index{HemiIcosahedron@\texttt{HemiIcosahedron}}
\label{HemiIcosahedron}
}\hfill{\scriptsize (operation)}}\\


 

 }

 

 

\subsection{\textcolor{Chapter }{24Cell}}
\logpage{[ 2, 1, 12 ]}\nobreak
\hyperdef{L}{X7C7194D3826A9287}{}
{\noindent\textcolor{FuncColor}{$\triangleright$\enspace\texttt{24Cell({\mdseries\slshape })\index{24Cell@\texttt{24Cell}}
\label{24Cell}
}\hfill{\scriptsize (operation)}}\\


 

 }

 

 

\subsection{\textcolor{Chapter }{Hemi24Cell}}
\logpage{[ 2, 1, 13 ]}\nobreak
\hyperdef{L}{X863C4F1D85F017B3}{}
{\noindent\textcolor{FuncColor}{$\triangleright$\enspace\texttt{Hemi24Cell({\mdseries\slshape })\index{Hemi24Cell@\texttt{Hemi24Cell}}
\label{Hemi24Cell}
}\hfill{\scriptsize (operation)}}\\


 

 }

 

 

\subsection{\textcolor{Chapter }{120Cell}}
\logpage{[ 2, 1, 14 ]}\nobreak
\hyperdef{L}{X7A7A51CC878450C4}{}
{\noindent\textcolor{FuncColor}{$\triangleright$\enspace\texttt{120Cell({\mdseries\slshape })\index{120Cell@\texttt{120Cell}}
\label{120Cell}
}\hfill{\scriptsize (operation)}}\\


 

 }

 

 

\subsection{\textcolor{Chapter }{Hemi120Cell}}
\logpage{[ 2, 1, 15 ]}\nobreak
\hyperdef{L}{X80EBD6F28447F653}{}
{\noindent\textcolor{FuncColor}{$\triangleright$\enspace\texttt{Hemi120Cell({\mdseries\slshape })\index{Hemi120Cell@\texttt{Hemi120Cell}}
\label{Hemi120Cell}
}\hfill{\scriptsize (operation)}}\\


 

 }

 

 

\subsection{\textcolor{Chapter }{600Cell}}
\logpage{[ 2, 1, 16 ]}\nobreak
\hyperdef{L}{X82FCA8347D417FB6}{}
{\noindent\textcolor{FuncColor}{$\triangleright$\enspace\texttt{600Cell({\mdseries\slshape })\index{600Cell@\texttt{600Cell}}
\label{600Cell}
}\hfill{\scriptsize (operation)}}\\


 

 }

 

 

\subsection{\textcolor{Chapter }{Hemi600Cell}}
\logpage{[ 2, 1, 17 ]}\nobreak
\hyperdef{L}{X786D2F0A7BB97182}{}
{\noindent\textcolor{FuncColor}{$\triangleright$\enspace\texttt{Hemi600Cell({\mdseries\slshape })\index{Hemi600Cell@\texttt{Hemi600Cell}}
\label{Hemi600Cell}
}\hfill{\scriptsize (operation)}}\\


 

 }

 }

 }

   
\chapter{\textcolor{Chapter }{Properties}}\label{Chapter_Properties}
\logpage{[ 3, 0, 0 ]}
\hyperdef{L}{X871597447BB998A1}{}
{
  
\section{\textcolor{Chapter }{Orientability}}\label{Chapter_Properties_Section_Orientability}
\logpage{[ 3, 1, 0 ]}
\hyperdef{L}{X861E4BAD800B2785}{}
{
  

\subsection{\textcolor{Chapter }{IsOrientable (for IsAbstractPolytope)}}
\logpage{[ 3, 1, 1 ]}\nobreak
\hyperdef{L}{X7F5348C17B3A5F7C}{}
{\noindent\textcolor{FuncColor}{$\triangleright$\enspace\texttt{IsOrientable({\mdseries\slshape p})\index{IsOrientable@\texttt{IsOrientable}!for IsAbstractPolytope}
\label{IsOrientable:for IsAbstractPolytope}
}\hfill{\scriptsize (property)}}\\
\textbf{\indent Returns:\ }
\texttt{true} or \texttt{false} 



 A polytope is orientable if its flag graph is bipartite. Currently only
implemented for regular polytopes. }

 

\subsection{\textcolor{Chapter }{IsIOrientable (for IsAbstractPolytope, IsList)}}
\logpage{[ 3, 1, 2 ]}\nobreak
\hyperdef{L}{X79D34C437FFC914D}{}
{\noindent\textcolor{FuncColor}{$\triangleright$\enspace\texttt{IsIOrientable({\mdseries\slshape p, I})\index{IsIOrientable@\texttt{IsIOrientable}!for IsAbstractPolytope, IsList}
\label{IsIOrientable:for IsAbstractPolytope, IsList}
}\hfill{\scriptsize (operation)}}\\


 For a subset I of \texttt{\symbol{123}}0, ..., n-1\texttt{\symbol{125}}, a
polytope if I-orientable if every closed path in its flag graph contains an
even number of edges with colors in I. Currently only implemented for regular
polytopes. }

 

\subsection{\textcolor{Chapter }{IsVertexBipartite (for IsAbstractPolytope)}}
\logpage{[ 3, 1, 3 ]}\nobreak
\hyperdef{L}{X8217A16284ABCDFA}{}
{\noindent\textcolor{FuncColor}{$\triangleright$\enspace\texttt{IsVertexBipartite({\mdseries\slshape p})\index{IsVertexBipartite@\texttt{IsVertexBipartite}!for IsAbstractPolytope}
\label{IsVertexBipartite:for IsAbstractPolytope}
}\hfill{\scriptsize (property)}}\\
\textbf{\indent Returns:\ }
\texttt{true} or \texttt{false} 



 A polytope is vertex-bipartite if its 1-skeleton is bipartite. This is
equivalent to being I-orientable for I =
\texttt{\symbol{123}}0\texttt{\symbol{125}}. }

 

\subsection{\textcolor{Chapter }{IsFacetBipartite (for IsAbstractPolytope)}}
\logpage{[ 3, 1, 4 ]}\nobreak
\hyperdef{L}{X796744327B6E583F}{}
{\noindent\textcolor{FuncColor}{$\triangleright$\enspace\texttt{IsFacetBipartite({\mdseries\slshape p})\index{IsFacetBipartite@\texttt{IsFacetBipartite}!for IsAbstractPolytope}
\label{IsFacetBipartite:for IsAbstractPolytope}
}\hfill{\scriptsize (property)}}\\
\textbf{\indent Returns:\ }
\texttt{true} or \texttt{false} 



 A polytope is facet-bipartite if the 1-skeleton of its dual is bipartite. This
is equivalent to being I-orientable for I =
\texttt{\symbol{123}}n-1\texttt{\symbol{125}}. }

 }

 }

   
\chapter{\textcolor{Chapter }{Basics}}\label{Chapter_Basics}
\logpage{[ 4, 0, 0 ]}
\hyperdef{L}{X868F7BAB7AC2EEBC}{}
{
  
\section{\textcolor{Chapter }{Constructors}}\label{Chapter_Basics_Section_Constructors}
\logpage{[ 4, 1, 0 ]}
\hyperdef{L}{X86EC0F0A78ECBC10}{}
{
  
\subsection{\textcolor{Chapter }{UniversalSggi}}\label{UniversalSggi}
\logpage{[ 4, 1, 1 ]}
\hyperdef{L}{X7AE843CE7878951F}{}
{
\noindent\textcolor{FuncColor}{$\triangleright$\enspace\texttt{UniversalSggi({\mdseries\slshape n})\index{UniversalSggi@\texttt{UniversalSggi}!for IsInt}
\label{UniversalSggi:for IsInt}
}\hfill{\scriptsize (operation)}}\\
\noindent\textcolor{FuncColor}{$\triangleright$\enspace\texttt{UniversalSggi({\mdseries\slshape sym})\index{UniversalSggi@\texttt{UniversalSggi}!for IsList}
\label{UniversalSggi:for IsList}
}\hfill{\scriptsize (operation)}}\\


 In the first form, returns the universal Coxeter Group of rank n. In the
second form, returns the Coxeter Group with Schlafli symbol sym. }

 

\subsection{\textcolor{Chapter }{AbstractRegularPolytope (for IsGroup)}}
\logpage{[ 4, 1, 2 ]}\nobreak
\hyperdef{L}{X7F1E96C780B3DA49}{}
{\noindent\textcolor{FuncColor}{$\triangleright$\enspace\texttt{AbstractRegularPolytope({\mdseries\slshape g})\index{AbstractRegularPolytope@\texttt{AbstractRegularPolytope}!for IsGroup}
\label{AbstractRegularPolytope:for IsGroup}
}\hfill{\scriptsize (operation)}}\\


 Given a group g (which should be a string C-group), returns the abstract
regular polytope with that automorphism group, where the privileged generators
are those returned by GeneratorsOfGroup(g). }

 

\subsection{\textcolor{Chapter }{AbstractRegularPolytope (for IsList, IsString)}}
\logpage{[ 4, 1, 3 ]}\nobreak
\hyperdef{L}{X7E9DB77282B1DA64}{}
{\noindent\textcolor{FuncColor}{$\triangleright$\enspace\texttt{AbstractRegularPolytope({\mdseries\slshape symbol, relations})\index{AbstractRegularPolytope@\texttt{AbstractRegularPolytope}!for IsList, IsString}
\label{AbstractRegularPolytope:for IsList, IsString}
}\hfill{\scriptsize (operation)}}\\


 Returns an abstract regular polytope with the given Schlafli symbol and with
the given relations. The formatting of the relations is quite flexible. All of
the following work: 
\begin{Verbatim}[commandchars=!@|,fontsize=\small,frame=single,label=Example]
  q := AbstractRegularPolytope([4,3,4], "(r0 r1 r2)^3, (r1 r2 r3)^3");
  q := AbstractRegularPolytope([4,3,4], "(r0 r1 r2)^3 = (r1 r2 r3)^3 = 1");
  p := AbstractRegularPolytope([infinity], "r0 r1 r0 = r1 r0 r1");
\end{Verbatim}
 }

 

\subsection{\textcolor{Chapter }{AbstractRegularPolytope (for IsString)}}
\logpage{[ 4, 1, 4 ]}\nobreak
\hyperdef{L}{X7F9F0BB37F5E2D5B}{}
{\noindent\textcolor{FuncColor}{$\triangleright$\enspace\texttt{AbstractRegularPolytope({\mdseries\slshape name})\index{AbstractRegularPolytope@\texttt{AbstractRegularPolytope}!for IsString}
\label{AbstractRegularPolytope:for IsString}
}\hfill{\scriptsize (operation)}}\\


 Returns the regular polytope with the given symbolic name. Examples:
AbstractRegularPolytope("\texttt{\symbol{123}}3,3,3\texttt{\symbol{125}}");
AbstractRegularPolytope("\texttt{\symbol{123}}4,3\texttt{\symbol{125}}{\textunderscore}3"); }

 }

 }

 \def\indexname{Index\logpage{[ "Ind", 0, 0 ]}
\hyperdef{L}{X83A0356F839C696F}{}
}

\cleardoublepage
\phantomsection
\addcontentsline{toc}{chapter}{Index}


\printindex

\immediate\write\pagenrlog{["Ind", 0, 0], \arabic{page},}
\immediate\write\pagenrlog{["Ind", 0, 0], \arabic{page},}
\immediate\write\pagenrlog{["Ind", 0, 0], \arabic{page},}
\immediate\write\pagenrlog{["Ind", 0, 0], \arabic{page},}
\newpage
\immediate\write\pagenrlog{["End"], \arabic{page}];}
\immediate\closeout\pagenrlog
\end{document}
